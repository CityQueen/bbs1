\documentclass[11pt,twocolumn,oneside,openany,headings=optiontotoc,11pt,numbers=noenddot]{article}

\usepackage[a4paper]{geometry}
\usepackage[utf8]{inputenc}
\usepackage[T1]{fontenc}
\usepackage{lmodern}
\usepackage[ngerman]{babel}
\usepackage{ngerman}

\usepackage[onehalfspacing]{setspace}

\usepackage{fancyhdr}
\usepackage{fancybox}

\usepackage{rotating}
\usepackage{varwidth}


\usepackage{pdflscape}
\usepackage{graphicx}
\usepackage{graphbox}
\graphicspath{
	{Pics/PDFs/}
	{Pics/JPGs/}
	{Pics/PNGs/}
}
\usepackage{caption}
\usepackage{tabularx}
\usepackage{dashrule}
\usepackage{hhline}
\usepackage{multirow}
\usepackage{enumerate}
\usepackage[hidelinks]{hyperref}
\usepackage{listings}

\usepackage[table]{xcolor}
\usepackage{array}
\usepackage{enumitem,amssymb,amsmath}
\usepackage{interval}
\usepackage{stmaryrd}
\usepackage{polynom}
\usepackage{diagbox}
\usepackage{dashrule}
\usepackage{framed}
\usepackage{mdframed}
\usepackage{karnaugh-map}

\usepackage{blindtext}

\usepackage{eso-pic}

\usepackage{amssymb}
\usepackage{eurosym}
\pagestyle{headings}
\renewcommand{\headrulewidth}{0.2pt}
\renewcommand{\footrulewidth}{0.2pt}
\newcommand*{\underdownarrow}[2]{\ensuremath{\underset{\overset{\Big\downarrow}{#2}}{#1}}}
\setlength{\fboxsep}{5pt}

% Codestyle defined
\definecolor{codegreen}{rgb}{0,0.6,0}
\definecolor{codegray}{rgb}{0.5,0.5,0.5}
\definecolor{codepurple}{rgb}{0.58,0,0.82}
\definecolor{backcolour}{rgb}{0.95,0.95,0.92}
\definecolor{deepgreen}{rgb}{0,0.5,0}
\definecolor{darkblue}{rgb}{0,0,0.65}
\definecolor{mauve}{rgb}{0.40, 0.19,0.28}
\colorlet{exceptioncolour}{yellow!50!red}
\colorlet{commandcolour}{blue!60!black}
\colorlet{numpycolour}{blue!60!green}
\colorlet{specmethodcolour}{violet}

%Neue Spaltendefinition
\newcolumntype{L}[1]{>{\raggedright\let\newline\\\arraybackslash\hspace{0pt}}m{#1}}
\newcolumntype{M}[1]{>{\centering\arraybackslash}X}
\newcommand{\cmnt}[1]{\ignorespaces}
%Textausrichtung ändern
\newcommand\tabrotate[1]{\rotatebox{90}{\raggedright#1\hspace{\tabcolsep}}}

%Intervall-Konfig
\intervalconfig {
	soft open fences
}

%Bash
\lstdefinestyle{BashInputStyle}{
	language=bash,
	basicstyle=\small\sffamily,
	backgroundcolor=\color{backcolour},
	columns=fullflexible,
	backgroundcolor=\color{backcolour},
	breaklines=true,
}
%Java
\lstdefinestyle{JavaInputStyle}{
	language=Java,
	backgroundcolor=\color{backcolour},
	aboveskip=1mm,
	belowskip=1mm,
	showstringspaces=false,
	columns=flexible,
	basicstyle={\footnotesize\ttfamily},
	numberstyle={\tiny},
	numbers=none,
	keywordstyle=\color{purple},,
	commentstyle=\color{deepgreen},
	stringstyle=\color{blue},
	emph={out},
	emphstyle=\color{darkblue},
	emph={[2]rand},
	emphstyle=[2]\color{specmethodcolour},
	breaklines=true,
	breakatwhitespace=true,
	tabsize=2,
}
%Python
\lstdefinestyle{PythonInputStyle}{
	language=Python,
	alsoletter={1234567890},
	aboveskip=1ex,
	basicstyle=\footnotesize,
	breaklines=true,
	breakatwhitespace= true,
	backgroundcolor=\color{backcolour},
	commentstyle=\color{red},
	otherkeywords={\ , \}, \{, \&,\|},
	emph={and,break,class,continue,def,yield,del,elif,else,%
		except,exec,finally,for,from,global,if,import,in,%
		lambda,not,or,pass,print,raise,return,try,while,assert},
	emphstyle=\color{exceptioncolour},
	emph={[2]True,False,None,min},
	emphstyle=[2]\color{specmethodcolour},
	emph={[3]object,type,isinstance,copy,deepcopy,zip,enumerate,reversed,list,len,dict,tuple,xrange,append,execfile,real,imag,reduce,str,repr},
	emphstyle=[3]\color{commandcolour},
	emph={[4]ode, fsolve, sqrt, exp, sin, cos, arccos, pi,  array, norm, solve, dot, arange, , isscalar, max, sum, flatten, shape, reshape, find, any, all, abs, plot, linspace, legend, quad, polyval,polyfit, hstack, concatenate,vstack,column_stack,empty,zeros,ones,rand,vander,grid,pcolor,eig,eigs,eigvals,svd,qr,tan,det,logspace,roll,mean,cumsum,cumprod,diff,vectorize,lstsq,cla,eye,xlabel,ylabel,squeeze},
	emphstyle=[4]\color{numpycolour},
	emph={[5]__init__,__add__,__mul__,__div__,__sub__,__call__,__getitem__,__setitem__,__eq__,__ne__,__nonzero__,__rmul__,__radd__,__repr__,__str__,__get__,__truediv__,__pow__,__name__,__future__,__all__},
	emphstyle=[5]\color{specmethodcolour},
	emph={[6]assert,range,yield},
	emphstyle=[6]\color{specmethodcolour}\bfseries,
	emph={[7]Exception,NameError,IndexError,SyntaxError,TypeError,ValueError,OverflowError,ZeroDivisionError,KeyboardInterrupt},
	emphstyle=[7]\color{specmethodcolour}\bfseries,
	emph={[8]taster,send,sendMail,capture,check,noMsg,go,move,switch,humTem,ventilate,buzz},
	emphstyle=[8]\color{blue},
	keywordstyle=\color{blue}\bfseries,
	rulecolor=\color{black!40},
	showstringspaces=false,
	stringstyle=\color{deepgreen}
}

\lstset{literate=%
	{Ö}{{\"O}}1
	{Ä}{{\"A}}1
	{Ü}{{\"U}}1
	{ß}{{\ss}}1
	{ü}{{\"u}}1
	{ä}{{\"a}}1
	{ö}{{\"o}}1
}

% Neue Klassenarbeits-Umgebung
\newenvironment{worksheet}[3]
% Begin-Bereich
{
	\newpage
	\sffamily
	\setcounter{page}{1}
	\ClearShipoutPicture
	\AddToShipoutPicture{
		\put(55,761){{
				\mbox{\parbox{385\unitlength}{\tiny \color{codegray}BBS I Mainz, #1 \newline #2
						\newline #3
					}
				}
			}
		}
		\put(455,761){{
				\mbox{\hspace{0.3cm}\includegraphics[width=0.2\textwidth]{../../logo.jpg}}
			}
		}
	}
}
% End-Bereich
{
	\clearpage
	\ClearShipoutPicture
}

\setlength{\columnsep}{3em}
\setlength{\columnseprule}{0.5pt}

\geometry{left=2.00cm,right=2.00cm,top=3.00cm,bottom=1.00cm,includeheadfoot}
\pagenumbering{gobble}
\pagestyle{empty}

\begin{document}
	\begin{worksheet}{BS FI}{1. Lehrjahr, LF 4 - Einfache IT-Systeme}{Digitaltechnik - Schaltfunktionen}
		\section{Schaltfunktionen und Booleschen Funktionen}
		Wir wollen die Frage diskutieren, wie sich ein Rechner oder genauer die Elemente eines Rechners Verhalten.\\
		Dabei hängt die Art, wie der Output vom Input bestimmt wird, offensichtlich vom Aufbau des Rechners ab; ferner gilt, ein Rechner arbeitet \textit{deterministisch}, d.h. er reagiert in \textbf{eindeutiger Weise} auf einen bestimmten Input.\\
		Unter der Annahme, dass Input und  Output aus Daten bzw. Dualfolgen bestehen, lässt sich folgende Definition aufstellen:
		\begin{framed}
			\textbf{Definition D1.1}\\Sind \(n,m \in \mathbb{N}\). Dann heißt
			\(\mathcal{F}: B^n \rightarrow B^m\) \textit{\textbf{Schaltfunktion}}.
		\end{framed}
		\noindent
		Input für einen Rechner ist also ein Bit-\textit{n}-Tupel. Der Output einer solchen Schaltfunktion (\(\mathcal{F}\)) ist dann wieder ein Bit-\textit{m}-Tupel.\\
		Hat eine solche Funktion lediglich eine Bit-Ausgabe (also \(n = 1\)), so wird die Funktion mit \(\mathit{f}\) bezeichnet. Dieser Unterscheidung liegt ein wichtiger Spezialfall zugrunde, der wie folgt definiert ist:
		\begin{framed}
			\textbf{Definition D1.2}\\Eine Schaltfunktion \(\mathit{f}: B^n \rightarrow B\)
			heißt (\textit{n}-stellige) \textit{\textbf{Boolesche Funktion}}.
		\end{framed}
		Man merke sich also, eine Schaltfunktion \(\mathcal{F}(x_1,...,x_n)\) ist die \textit{n}-stellige Verkettung einzelner \textit{Boolescher Funktionen} \(\mathit{f}_i\).
		\subsection{Einschlägiger Index}
		Betrachten wir die nachfolgende Funktionstabelle zu \(f: B^3 \rightarrow B\)
		\begin{center}
			\begin{tabular}{|c|ccc|c|}
				\hline
				\textit{i} & \(x_1\) & \(x_2\) & \(x_3\) & \(f(x_1,x_2,x_3)\)\\
				\hline
				0 & 0 & 0 & 0 & 0\\
				\hline
				1 & 0 & 0 & 1 & 0\\
				\hline
				2 & 0 & 1 & 0 & 0\\
				\hline
				3 & 0 & 1 & 1 & 1\\
				\hline
				4 & 1 & 0 & 0 & 0\\
				\hline
				5 & 1 & 0 & 1 & 1\\
				\hline
				6 & 1 & 1 & 0 & 0\\
				\hline
				7 & 1 & 1 & 1 & 1\\
				\hline
			\end{tabular}
		\end{center}
		Mit \textit{i} bezeichnen wir die Zeilennummer und mit \(i_1 \ldots i_n\) bezeichnen wir die Ziffernfolge der Dualdarstellung von \textit{i}.
		\begin{framed}
			\textbf{Definition D1.3}\\ \textit{i} heißt \textit{einschlägiger} Index zu \(f\), falls \(f(i_1,\ldots,i_n) = 1\) ist.
		\end{framed}
		\subsection{Minterm}
		Für die Belegung der \textit{i}-ten Zeile dieser Funktionstabelle 
		\begin{framed}
			\textbf{Definition D1.4}\\ Sei \textit{i} ein Index von \(f: B^n \rightarrow B\) und (\(i_1\ldots i_n\)) die Dualdarstellung von \textit{i}. Dann heißt die Funktion \(m_i: B^n \rightarrow B\) definiert durch
			\[m_i(x_1,\ldots x_n) := x_{1}^{i_1}\cdot x_{2}^{i_2}\cdot\ldots\cdot x_{n}^{i_n}\]
			\textit{i-ter \textbf{Minterm}} von \(f\). Dabei gilt
			\[x_{j}^{i_j} = \begin{cases}x_j & i_j=1 \\ \overline x_j & i_j = 0\end{cases}\]
		\end{framed}
		\subsection*{Ihre Aufgabe} Bestimmen Sie die \underline{Minterme} für die oben angegebene Funktionstabelle.\\
		\textit{Hinweis}: Die \textit{einschlägigen} Indizes sind 3, 5 und 7.
		\subsection{Sum Of Products (SOP)} Mit Hilfe der oben genannten Minterme können wir eine Boolesche Funktion nun wie folgt beschreiben:
		\begin{framed}
			\textbf{Satz S1.1}\\ Jede Boolesche Funktion \(f: B^n \rightarrow B\) ist eindeutig darstellbar als Summer der Minterme ihrer einschlägigen Indizes. Ist \(I \subseteq \{0,\ldots , 2^n -1\}\) die Menge der einschlägigen Indizes von \(f\), so gilt
			\[f = \sum_{i\in I} m_i\]
		\end{framed}
		Die hier angegebene Darstellung heißt auch \textit{disjunktive Normalform} (DNF) einer Booleschen Funktion.
		\subsection*{Ihre Aufgabe} Geben Sie zu oben genannter Funktion die \underline{SOP} an.
		\subsection{Maxterme} Eine alternative Darstellung der eben erwähnten DNF benötigt ein anderes Konstrukt.
		\begin{framed}
			\textbf{Definition D1.5}\\Sei \textit{i} ein Index von \(f:B^n\rightarrow B\), und sei \(m_i\) der \textit{i}-te Minterm von \(f\). Dann heißt die Funktion \(M_i: B^n \rightarrow B\) definiert durch
			\[M_i(x_1,\ldots x_n) := \overline{m_i(x_1,\ldots x_n)}\]
			\textit{i}-ter \textit{Maxterm} von \(f\).
		\end{framed}
		Mit dieser Definition lässt sich in Anlehnung an die Definition der Minterme folgendes festhalten:\\ Ein Maxterm \(M_i\) nimmt genau dann den Wert 0 an, wenn das Argument \((x_1,\ldots x_n)\) die Dualdarstellung von \textit{i} ist.
		\subsection*{Ihre Aufgabe} Bestimmen Sie die \underline{Maxterme} für die oben angegebene Funktionstabelle.\\
		\textit{Hinweis}: Die \textit{nicht einschlägigen} Indizes sind 0, 1, 2, 4 und 6.
		\subsection{Product Of Sums (POS)}
		Mit diese Feststellung lässt sich für eine Boolesche Funktion folgendes sagen:
		\begin{framed}
			\textbf{Satz S1.2} Jede Boolesche Funktion \(f: B^n \rightarrow B\) ist eindeutig darstellbar als das Produkt der Maxterme ihrer nicht einschlägigen Indizes.
		\end{framed}
		Die hier genannte Darstellung wird auch \textit{konjunktive Normalform} (KNF) von \(f\) genannt.
		\subsection*{Ihre Aufgabe} Geben Sie zur oben dargestellten Funktion die \underline{POS} an.
		\subsection{Overview} Offensichtlich ist die DNF zu bevorzugen, wenn die Anzahl der einschlägigen Indizes kleiner ist als die Anzahl der nicht einschlägigen; Ansonsten verwenden Sie die KNF.\\
		\par\bigskip\noindent
		\tiny{\color{codegray}\textit{W. Oberschelp/G.Vossen} Rechneraufbau und Rechnerstrukturen 10. Auflage (11-21)}
	\end{worksheet}
\end{document}