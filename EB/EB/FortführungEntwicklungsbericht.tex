\documentclass[oneside,openany,headings=optiontotoc,11pt,numbers=noenddot]{article}

\usepackage[a4paper]{geometry}
\usepackage[utf8]{inputenc}
\usepackage[T1]{fontenc}
\usepackage{lmodern}
\usepackage[ngerman]{babel}
\usepackage{ngerman}

\usepackage[onehalfspacing]{setspace}

\usepackage{fancyhdr}
\usepackage{fancybox}

\usepackage{rotating}
\usepackage{varwidth}


\usepackage{pdflscape}
\usepackage{graphicx}
\usepackage{graphbox}
\graphicspath{
	{Pics/PDFs/}
	{Pics/JPGs/}
	{Pics/PNGs/}
}
\usepackage{caption}
\usepackage{tabularx}
\usepackage{dashrule}
\usepackage{hhline}
\usepackage{multirow}
\usepackage{enumerate}
\usepackage[hidelinks]{hyperref}
\usepackage{listings}

\usepackage[table]{xcolor}
\usepackage{array}
\usepackage{enumitem,amssymb,amsmath}
\usepackage{interval}
\usepackage{stmaryrd}
\usepackage{polynom}
\usepackage{diagbox}
\usepackage{dashrule}
\usepackage{framed}
\usepackage{mdframed}
\usepackage{karnaugh-map}

\usepackage{blindtext}

\usepackage{eso-pic}

\usepackage{amssymb}
\usepackage{eurosym}
\pagestyle{headings}
\renewcommand{\headrulewidth}{0.2pt}
\renewcommand{\footrulewidth}{0.2pt}
\newcommand*{\underdownarrow}[2]{\ensuremath{\underset{\overset{\Big\downarrow}{#2}}{#1}}}
\setlength{\fboxsep}{5pt}

% Codestyle defined
\definecolor{codegreen}{rgb}{0,0.6,0}
\definecolor{codegray}{rgb}{0.5,0.5,0.5}
\definecolor{codepurple}{rgb}{0.58,0,0.82}
\definecolor{backcolour}{rgb}{0.95,0.95,0.92}
\definecolor{deepgreen}{rgb}{0,0.5,0}
\definecolor{darkblue}{rgb}{0,0,0.65}
\definecolor{mauve}{rgb}{0.40, 0.19,0.28}
\colorlet{exceptioncolour}{yellow!50!red}
\colorlet{commandcolour}{blue!60!black}
\colorlet{numpycolour}{blue!60!green}
\colorlet{specmethodcolour}{violet}

%Neue Spaltendefinition
\newcolumntype{L}[1]{>{\raggedright\let\newline\\\arraybackslash\hspace{0pt}}m{#1}}
\newcolumntype{M}[1]{>{\centering\arraybackslash}X}
\newcommand{\cmnt}[1]{\ignorespaces}
%Textausrichtung ändern
\newcommand\tabrotate[1]{\rotatebox{90}{\raggedright#1\hspace{\tabcolsep}}}

%Intervall-Konfig
\intervalconfig {
	soft open fences
}

%Bash
\lstdefinestyle{BashInputStyle}{
	language=bash,
	basicstyle=\small\sffamily,
	backgroundcolor=\color{backcolour},
	columns=fullflexible,
	backgroundcolor=\color{backcolour},
	breaklines=true,
}
%Java
\lstdefinestyle{JavaInputStyle}{
	language=Java,
	backgroundcolor=\color{backcolour},
	aboveskip=1mm,
	belowskip=1mm,
	showstringspaces=false,
	columns=flexible,
	basicstyle={\footnotesize\ttfamily},
	numberstyle={\tiny},
	numbers=none,
	keywordstyle=\color{purple},,
	commentstyle=\color{deepgreen},
	stringstyle=\color{blue},
	emph={out},
	emphstyle=\color{darkblue},
	emph={[2]rand},
	emphstyle=[2]\color{specmethodcolour},
	breaklines=true,
	breakatwhitespace=true,
	tabsize=2,
}
%Python
\lstdefinestyle{PythonInputStyle}{
	language=Python,
	alsoletter={1234567890},
	aboveskip=1ex,
	basicstyle=\footnotesize,
	breaklines=true,
	breakatwhitespace= true,
	backgroundcolor=\color{backcolour},
	commentstyle=\color{red},
	otherkeywords={\ , \}, \{, \&,\|},
	emph={and,break,class,continue,def,yield,del,elif,else,%
		except,exec,finally,for,from,global,if,import,in,%
		lambda,not,or,pass,print,raise,return,try,while,assert},
	emphstyle=\color{exceptioncolour},
	emph={[2]True,False,None,min},
	emphstyle=[2]\color{specmethodcolour},
	emph={[3]object,type,isinstance,copy,deepcopy,zip,enumerate,reversed,list,len,dict,tuple,xrange,append,execfile,real,imag,reduce,str,repr},
	emphstyle=[3]\color{commandcolour},
	emph={[4]ode, fsolve, sqrt, exp, sin, cos, arccos, pi,  array, norm, solve, dot, arange, , isscalar, max, sum, flatten, shape, reshape, find, any, all, abs, plot, linspace, legend, quad, polyval,polyfit, hstack, concatenate,vstack,column_stack,empty,zeros,ones,rand,vander,grid,pcolor,eig,eigs,eigvals,svd,qr,tan,det,logspace,roll,mean,cumsum,cumprod,diff,vectorize,lstsq,cla,eye,xlabel,ylabel,squeeze},
	emphstyle=[4]\color{numpycolour},
	emph={[5]__init__,__add__,__mul__,__div__,__sub__,__call__,__getitem__,__setitem__,__eq__,__ne__,__nonzero__,__rmul__,__radd__,__repr__,__str__,__get__,__truediv__,__pow__,__name__,__future__,__all__},
	emphstyle=[5]\color{specmethodcolour},
	emph={[6]assert,range,yield},
	emphstyle=[6]\color{specmethodcolour}\bfseries,
	emph={[7]Exception,NameError,IndexError,SyntaxError,TypeError,ValueError,OverflowError,ZeroDivisionError,KeyboardInterrupt},
	emphstyle=[7]\color{specmethodcolour}\bfseries,
	emph={[8]taster,send,sendMail,capture,check,noMsg,go,move,switch,humTem,ventilate,buzz},
	emphstyle=[8]\color{blue},
	keywordstyle=\color{blue}\bfseries,
	rulecolor=\color{black!40},
	showstringspaces=false,
	stringstyle=\color{deepgreen}
}

\lstset{literate=%
	{Ö}{{\"O}}1
	{Ä}{{\"A}}1
	{Ü}{{\"U}}1
	{ß}{{\ss}}1
	{ü}{{\"u}}1
	{ä}{{\"a}}1
	{ö}{{\"o}}1
}

% Neue Klassenarbeits-Umgebung
\newenvironment{worksheet}[3]
% Begin-Bereich
{
	\newpage
	\sffamily
	\setcounter{page}{1}
	\ClearShipoutPicture
	\AddToShipoutPicture{
		\put(55,761){{
				\mbox{\parbox{385\unitlength}{\tiny \color{codegray}BBS I Mainz, #1 \newline #2
						\newline #3
					}
				}
			}
		}
		\put(455,761){{
				\mbox{\hspace{0.3cm}\includegraphics[width=0.2\textwidth]{../../logo.jpg}}
			}
		}
	}
}
% End-Bereich
{
	\clearpage
	\ClearShipoutPicture
}
\usepackage{titlesec}
\titlespacing*{\section}{0pt}{1.1\baselineskip}{0.1\baselineskip}

\geometry{left=2.50cm,right=2.50cm,top=3.00cm,bottom=1.00cm,includeheadfoot}
\pagestyle{plain}

\begin{document}				
	\begin{worksheet}{Fortführung Entwicklungsbericht}{StRef\grq{} Carolyn Wesp}{24. November bis 14. Januar 2019}
		\section*{\color{blue}{\ul{Wer war ich und wer bin ich jetzt? - Eine abschließende Selbstreflexion}}}
		\tiny{Geschrieben am: 17. Dezember 2018}\small\\
		\par\noindent
		Aufgrund diverser Rückmeldungen habe ich mich dazu entschieden, mein Verhalten, meine Einstellung und mein Selbstbild noch einmal genauer zu beleuchten um mir selbst bewusst zu werden: Welche Rolle übernehme ich als Lehrperson? Wie verstehe ich meine Aufgabe? Wie nehme ich mich selbst wahr, aber auch wie nehmen mich meine Schüler wahr?
		
		\section*{\color{blue}{\ul{Eigenständigkeit braucht Zeit}}}
		\tiny{Geschrieben am: 12. Dezember 2018}\small\\
		\par\noindent
		HBF seit Schuljahresbeginn viel durch selbstorganisiertes Lernen gelernt - Methode der eigenständigen Erarbeitung funktioniert hier schon besser\\
		\par\noindent
		Schüler wissen wie sie an Aufgaben herangehen und kennen die Situation - verlassen sich auf ihre Mitschüler und auf die Tatsache, dass ich Rede und Antwort stehen kann, wenn sie sich nicht selbst und auch nicht gegenseitig helfen können.\\
		\par\noindent
		BGY 17 hat zwar auch seit Schuljahresbeginn viel durch selbstständige Erarbeitung und Trial \& Error gelernt, kommt aber mit der Situation, die Arbeit selbstständig zu organisieren nicht so gut klar. Sind eher Einzelgänger und Arbeiten für sich. Tendiere dazu, wenn sie mit Hundeaugen ganz verzweifelt schauen, mit ihnen ein Beispiel zu besprechen.\\
		\par\noindent		
		Muss mich hier noch mehr zurücknehmen und versuchen weniger oft schwach zu werden.\\
		\par\noindent		
		$\lbrack$Update: 10.12.2018$\rbrack$ Schüler haben heute Auftrag erhalten selbstständig über den Datentyp String zu recherchieren und sich mit den Eigenschaften und Möglichkeiten auseinanderzusetzen\\
		\par\noindent		
		Unterschiedlich vorgegangen - Erklärvideos angeschaut, Wikipedia bzw Enzyclopädie Seiten gelesen -- gelesenes ausprobiert und stichwortartig zusammengefasst.\\
		
		\section*{\color{blue}{\ul{$\lbrack$Mathematik$\rbrack$ Pair and Share}}}
		\tiny{Geschrieben am: 04. Dezember 2018}\small\\
		\par\noindent
		Neues Konzept eingeführt: Die Schüler beraten sich zunächst gegenseitig, klären Fragen und formulieren übrig gebliebene Unklarheiten\\
		\par\noindent
		Sammeln der Fragen an der Tafel und Klärung im Plenum (Schülerlösung als Link)\\
		\par\noindent		
		Tafelbild gemeinsam befüllt\\
		\par\noindent		
		Gleiches Vorgehen für die Reihe geplant - mehr Zeit für Reflexion einplanen.\\
		
		\section*{\color{blue}{\ul{Können die Schüler selbstgesteuert Lernen?}}}
		\tiny{Geschrieben am: 01. Dezember 2018}\small\\
		\par\noindent
		$\lbrack$Mathematik$\rbrack$ Wochenplan und Skripts\\
		\par\noindent		
		$\lbrack$Informatik$\rbrack$ Skripts mit Übungsaufgaben\\
		
		\section*{\color{blue}{\ul{Präventive Maßnahme bei Störungen}}}
		\tiny{Geschrieben am: 22. November 2018}\small\\
		\par\noindent
		Alternativen zur Lautstärkeampel.\\
		\par\noindent		
		Eher auf positive Verstärkung gehen\\
		
		\section*{\color{blue}{\ul{$\lbrack$Mathematik$\rbrack$ Kann ein cheat-sheet die Schüler bei der Klausur unterstützen?}}}
		\tiny{Geschrieben am: 10. November 2018}\small\\
		\par\noindent
		Freitag Klausur (6. und 7. Stunde) geschrieben\\
		\par\noindent		
		In der Vorstunde (5. Stunde) nochmal Fragen geklärt - viele Schüler wollten die Wiederholungen und Antworten während der Klausur an der Tafel stehen lassen.\\
		\par\noindent		
		Hatte gedacht, die Klausur sei zu einfach, aber ich habe den Kurs scheinbar überschätzt...Sie haben die 90 Minuten durchgängig geschrieben
	\end{worksheet}
\end{document}