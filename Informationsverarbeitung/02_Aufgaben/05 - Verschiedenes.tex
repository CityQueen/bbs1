\documentclass[oneside,openany,headings=optiontotoc,11pt,numbers=noenddot]{scrreprt}

\usepackage[a4paper]{geometry}
\usepackage[utf8]{inputenc}
\usepackage[T1]{fontenc}
\usepackage{lmodern}
\usepackage[ngerman]{babel}
\usepackage{ngerman}

\usepackage[onehalfspacing]{setspace}

\usepackage{fancyhdr}
\usepackage{fancybox}

\usepackage{rotating}
\usepackage{varwidth}


\usepackage{pdflscape}
\usepackage{graphicx}
\usepackage{graphbox}
\graphicspath{
	{Pics/PDFs/}
	{Pics/JPGs/}
	{Pics/PNGs/}
}
\usepackage{caption}
\usepackage{tabularx}
\usepackage{dashrule}
\usepackage{hhline}
\usepackage{multirow}
\usepackage{enumerate}
\usepackage[hidelinks]{hyperref}
\usepackage{listings}

\usepackage[table]{xcolor}
\usepackage{array}
\usepackage{enumitem,amssymb,amsmath}
\usepackage{interval}
\usepackage{stmaryrd}
\usepackage{polynom}
\usepackage{diagbox}
\usepackage{dashrule}
\usepackage{framed}
\usepackage{mdframed}
\usepackage{karnaugh-map}

\usepackage{blindtext}

\usepackage{eso-pic}

\usepackage{amssymb}
\usepackage{eurosym}
\pagestyle{headings}
\renewcommand{\headrulewidth}{0.2pt}
\renewcommand{\footrulewidth}{0.2pt}
\newcommand*{\underdownarrow}[2]{\ensuremath{\underset{\overset{\Big\downarrow}{#2}}{#1}}}
\setlength{\fboxsep}{5pt}

% Codestyle defined
\definecolor{codegreen}{rgb}{0,0.6,0}
\definecolor{codegray}{rgb}{0.5,0.5,0.5}
\definecolor{codepurple}{rgb}{0.58,0,0.82}
\definecolor{backcolour}{rgb}{0.95,0.95,0.92}
\definecolor{deepgreen}{rgb}{0,0.5,0}
\definecolor{darkblue}{rgb}{0,0,0.65}
\definecolor{mauve}{rgb}{0.40, 0.19,0.28}
\colorlet{exceptioncolour}{yellow!50!red}
\colorlet{commandcolour}{blue!60!black}
\colorlet{numpycolour}{blue!60!green}
\colorlet{specmethodcolour}{violet}

%Neue Spaltendefinition
\newcolumntype{L}[1]{>{\raggedright\let\newline\\\arraybackslash\hspace{0pt}}m{#1}}
\newcolumntype{M}[1]{>{\centering\arraybackslash}X}
\newcommand{\cmnt}[1]{\ignorespaces}
%Textausrichtung ändern
\newcommand\tabrotate[1]{\rotatebox{90}{\raggedright#1\hspace{\tabcolsep}}}

%Intervall-Konfig
\intervalconfig {
	soft open fences
}

%Bash
\lstdefinestyle{BashInputStyle}{
	language=bash,
	basicstyle=\small\sffamily,
	backgroundcolor=\color{backcolour},
	columns=fullflexible,
	backgroundcolor=\color{backcolour},
	breaklines=true,
}
%Java
\lstdefinestyle{JavaInputStyle}{
	language=Java,
	backgroundcolor=\color{backcolour},
	aboveskip=1mm,
	belowskip=1mm,
	showstringspaces=false,
	columns=flexible,
	basicstyle={\footnotesize\ttfamily},
	numberstyle={\tiny},
	numbers=none,
	keywordstyle=\color{purple},,
	commentstyle=\color{deepgreen},
	stringstyle=\color{blue},
	emph={out},
	emphstyle=\color{darkblue},
	emph={[2]rand},
	emphstyle=[2]\color{specmethodcolour},
	breaklines=true,
	breakatwhitespace=true,
	tabsize=2,
}
%Python
\lstdefinestyle{PythonInputStyle}{
	language=Python,
	alsoletter={1234567890},
	aboveskip=1ex,
	basicstyle=\footnotesize,
	breaklines=true,
	breakatwhitespace= true,
	backgroundcolor=\color{backcolour},
	commentstyle=\color{red},
	otherkeywords={\ , \}, \{, \&,\|},
	emph={and,break,class,continue,def,yield,del,elif,else,%
		except,exec,finally,for,from,global,if,import,in,%
		lambda,not,or,pass,print,raise,return,try,while,assert},
	emphstyle=\color{exceptioncolour},
	emph={[2]True,False,None,min},
	emphstyle=[2]\color{specmethodcolour},
	emph={[3]object,type,isinstance,copy,deepcopy,zip,enumerate,reversed,list,len,dict,tuple,xrange,append,execfile,real,imag,reduce,str,repr},
	emphstyle=[3]\color{commandcolour},
	emph={[4]ode, fsolve, sqrt, exp, sin, cos, arccos, pi,  array, norm, solve, dot, arange, , isscalar, max, sum, flatten, shape, reshape, find, any, all, abs, plot, linspace, legend, quad, polyval,polyfit, hstack, concatenate,vstack,column_stack,empty,zeros,ones,rand,vander,grid,pcolor,eig,eigs,eigvals,svd,qr,tan,det,logspace,roll,mean,cumsum,cumprod,diff,vectorize,lstsq,cla,eye,xlabel,ylabel,squeeze},
	emphstyle=[4]\color{numpycolour},
	emph={[5]__init__,__add__,__mul__,__div__,__sub__,__call__,__getitem__,__setitem__,__eq__,__ne__,__nonzero__,__rmul__,__radd__,__repr__,__str__,__get__,__truediv__,__pow__,__name__,__future__,__all__},
	emphstyle=[5]\color{specmethodcolour},
	emph={[6]assert,range,yield},
	emphstyle=[6]\color{specmethodcolour}\bfseries,
	emph={[7]Exception,NameError,IndexError,SyntaxError,TypeError,ValueError,OverflowError,ZeroDivisionError,KeyboardInterrupt},
	emphstyle=[7]\color{specmethodcolour}\bfseries,
	emph={[8]taster,send,sendMail,capture,check,noMsg,go,move,switch,humTem,ventilate,buzz},
	emphstyle=[8]\color{blue},
	keywordstyle=\color{blue}\bfseries,
	rulecolor=\color{black!40},
	showstringspaces=false,
	stringstyle=\color{deepgreen}
}

\lstset{literate=%
	{Ö}{{\"O}}1
	{Ä}{{\"A}}1
	{Ü}{{\"U}}1
	{ß}{{\ss}}1
	{ü}{{\"u}}1
	{ä}{{\"a}}1
	{ö}{{\"o}}1
}

% Neue Klassenarbeits-Umgebung
\newenvironment{worksheet}[3]
% Begin-Bereich
{
	\newpage
	\sffamily
	\setcounter{page}{1}
	\ClearShipoutPicture
	\AddToShipoutPicture{
		\put(55,761){{
				\mbox{\parbox{385\unitlength}{\tiny \color{codegray}BBS I Mainz, #1 \newline #2
						\newline #3
					}
				}
			}
		}
		\put(455,761){{
				\mbox{\hspace{0.3cm}\includegraphics[width=0.2\textwidth]{../../logo.jpg}}
			}
		}
	}
}
% End-Bereich
{
	\clearpage
	\ClearShipoutPicture
}

\geometry{left=1.50cm,right=1.50cm,top=3.00cm,bottom=1.00cm,includeheadfoot}

\begin{document}
	\begin{worksheet}{Informationsverarbeitung}{Lernabschnitt: Strukturiert Programmieren}{Gemischte Aufgaben}
		\noindent
		\sffamily
		\begin{framed}
			\noindent
			\textbf{Aufgabe 29:} \underline{\grqq{}grafische Ausgabe\grqq{}}\\
			Schreiben Sie ein Programm, dass die Seitenlänge eines Quadrates erfragt und dann im Textmodus win Quadrat dieser Größe ausgibt.\\
			Gibt der Anwender beispielsweise 5 ein, so soll die Ausgabe wie folgt aussehen:\\
			\begin{tabular}{ccccc}
				* & * & * & * & *\\
				* & & & & *\\
				* & & & & *\\
				* & & & & *\\
				* & * & * & * & *\\
			\end{tabular}\\
			\par\noindent
			Schreiben Sie im Programm entsprechende Methoden und führen Sie eine Fehlerbehandlung durch.\\
			\par\noindent
			\textbf{Aufgabe 30} \underline{Geschachtelte Schleifen}\\
			Schreiben Sie ein Programm, das folgende Ausgabe erzeugt:\\
			1 abcdefg\\
			12 abcdef\\
			123 abcde\\
			1234 abcd\\
			12345 abc\\
			123456 ab\\
			1234567 a\\
			\par\noindent
			\textbf{Aufgabe 31} \underline{Ausgabe einer Ganzzahl im Wortlaut}\\
			Schreiben Sie ein Programm, das eine eingegebene Ganzzahl im Wortlaut ausgibt.\\
			Es reicht, wenn ihr Programm die Zahl 345 als \grqq{}Drei Vier Fünf\grqq{} ausgibt.\\
			Sorgen Sie dafür, dass alle auftretenden Sonderfälle korrekt behandelt werden.\\
			\par\noindent
			Eine interessante Erweiterung dieser Aufgabe besteht darin, die Zahl so auszugeben, wie sie tatsächlich ausgesprochen wird, in obigem Beispiel also \grqq{}Dreihundertfünfundvierzig\grqq{}.\\
			\par\noindent
			\textbf{Aufgabe 32} \underline{Sieb des Eratosthenes}\\
			Eine seit über 2000 Jahren bekannte Form der Primzahlbestimmung wird als \grqq{}Sieb des Eratosthenes\grqq{} bezeichnet.\\
			Angenommen Sie sollen alle Primzahlen im Bereich von 1 bis 1000 bestimmen. Gehen Sie dazu wie folgt vor:\\
			Erstellen Sie ein boolsches Array mit 1000 Elementen und initialisieren Sie alle Elemente auf \lstinline[style=JavaInputStyle]|true|.\\
			Ändern Sie das erste Element auf \lstinline[style=JavaInputStyle]|false|.\\
			Nun führen Sie in einer Schleife folgende Schritte aus:
			\begin{itemize}
				\item Suchen Sie das nächste Element, desse Wert \lstinline[style=JavaInputStyle]|true| ist. Falls kein weiteres Element diese Bedingung erfüllt, beenden Sie die Schleife.
				\item Ändern Sie den Wert aller Elemente, deren Position ein ganzzahliges Vielfaches der aktuellen Position ist, auf \lstinline[style=JavaInputStyle]|false|.\\
				Haben Sie beispielsweise das fünfte Element gefunden, so werden auf diese Weise die Elemente 10, 15, 20, \ldots , 990 und 995 ausgestrichen.
			\end{itemize}
			Geben Sie nun die Positionen aller Arrayelemente aus, deren Wert \lstinline[style=JavaInputStyle]|true| ist. Es handelt sich dabei ujm die Primzahlen zwischen 1 und 1000.\\
			\par\noindent
			Schreiben Sie ein Programm, das das \grqq{}Sieb des Eratosthenes\grqq{} realisiert und ermitteln Sie auf diese Weise alle Primzahlen zwischen 1 und 1000.\\
			\par\noindent
			\textbf{Aufgabe 33} \underline{deMorgan\grq{}sche Regeln}\\
			Die deMorgan\grqq{}schen Regeln gehören zu den wichtigsten Umformungsregeln logischer Ausdrücke. Tatsächlich werden Sie auch in der Praxis häufig benötigt und beim Programmieren sollte man sie sicher berherrschen können.\\
			Die Regeln beschreiben, wie UND- oder ODER-verknüpfte Ausdrücke invertiert werden können. Sie lauten:\\
			NICHT(a UND b) \(\Leftrightarrow\) (NICHT a) oder (NICHT b)\\
			NICHT(a ODER b) \(\Leftrightarrow\) (NICHT a) UND (NICHT b)\\
			Schreiben Sie ein Programm, das die Gültigkeit der deMorgan\grq{}schen Regeln mit Hilfe boolescher Variablen und logischer Operatoren empirisch überprüft.
		\end{framed}
	\end{worksheet}
\end{document}