\documentclass[11pt,oneside,openany,headings=optiontotoc,11pt,numbers=noenddot]{article}

\usepackage[a4paper]{geometry}
\usepackage[utf8]{inputenc}
\usepackage[T1]{fontenc}
\usepackage{lmodern}
\usepackage[ngerman]{babel}
\usepackage{ngerman}

\usepackage[onehalfspacing]{setspace}

\usepackage{fancyhdr}
\usepackage{fancybox}

\usepackage{rotating}
\usepackage{varwidth}


\usepackage{pdflscape}
\usepackage{graphicx}
\usepackage{graphbox}
\graphicspath{
	{Pics/PDFs/}
	{Pics/JPGs/}
	{Pics/PNGs/}
}
\usepackage{caption}
\usepackage{tabularx}
\usepackage{dashrule}
\usepackage{hhline}
\usepackage{multirow}
\usepackage{enumerate}
\usepackage[hidelinks]{hyperref}
\usepackage{listings}

\usepackage[table]{xcolor}
\usepackage{array}
\usepackage{enumitem,amssymb,amsmath}
\usepackage{interval}
\usepackage{stmaryrd}
\usepackage{polynom}
\usepackage{diagbox}
\usepackage{dashrule}
\usepackage{framed}
\usepackage{mdframed}
\usepackage{karnaugh-map}

\usepackage{blindtext}

\usepackage{eso-pic}

\usepackage{amssymb}
\usepackage{eurosym}
\pagestyle{headings}
\renewcommand{\headrulewidth}{0.2pt}
\renewcommand{\footrulewidth}{0.2pt}
\newcommand*{\underdownarrow}[2]{\ensuremath{\underset{\overset{\Big\downarrow}{#2}}{#1}}}
\setlength{\fboxsep}{5pt}

% Codestyle defined
\definecolor{codegreen}{rgb}{0,0.6,0}
\definecolor{codegray}{rgb}{0.5,0.5,0.5}
\definecolor{codepurple}{rgb}{0.58,0,0.82}
\definecolor{backcolour}{rgb}{0.95,0.95,0.92}
\definecolor{deepgreen}{rgb}{0,0.5,0}
\definecolor{darkblue}{rgb}{0,0,0.65}
\definecolor{mauve}{rgb}{0.40, 0.19,0.28}
\colorlet{exceptioncolour}{yellow!50!red}
\colorlet{commandcolour}{blue!60!black}
\colorlet{numpycolour}{blue!60!green}
\colorlet{specmethodcolour}{violet}

%Neue Spaltendefinition
\newcolumntype{L}[1]{>{\raggedright\let\newline\\\arraybackslash\hspace{0pt}}m{#1}}
\newcolumntype{M}[1]{>{\centering\arraybackslash}X}
\newcommand{\cmnt}[1]{\ignorespaces}
%Textausrichtung ändern
\newcommand\tabrotate[1]{\rotatebox{90}{\raggedright#1\hspace{\tabcolsep}}}

%Intervall-Konfig
\intervalconfig {
	soft open fences
}

%Bash
\lstdefinestyle{BashInputStyle}{
	language=bash,
	basicstyle=\small\sffamily,
	backgroundcolor=\color{backcolour},
	columns=fullflexible,
	backgroundcolor=\color{backcolour},
	breaklines=true,
}
%Java
\lstdefinestyle{JavaInputStyle}{
	language=Java,
	backgroundcolor=\color{backcolour},
	aboveskip=1mm,
	belowskip=1mm,
	showstringspaces=false,
	columns=flexible,
	basicstyle={\footnotesize\ttfamily},
	numberstyle={\tiny},
	numbers=none,
	keywordstyle=\color{purple},,
	commentstyle=\color{deepgreen},
	stringstyle=\color{blue},
	emph={out},
	emphstyle=\color{darkblue},
	emph={[2]rand},
	emphstyle=[2]\color{specmethodcolour},
	breaklines=true,
	breakatwhitespace=true,
	tabsize=2,
}
%Python
\lstdefinestyle{PythonInputStyle}{
	language=Python,
	alsoletter={1234567890},
	aboveskip=1ex,
	basicstyle=\footnotesize,
	breaklines=true,
	breakatwhitespace= true,
	backgroundcolor=\color{backcolour},
	commentstyle=\color{red},
	otherkeywords={\ , \}, \{, \&,\|},
	emph={and,break,class,continue,def,yield,del,elif,else,%
		except,exec,finally,for,from,global,if,import,in,%
		lambda,not,or,pass,print,raise,return,try,while,assert},
	emphstyle=\color{exceptioncolour},
	emph={[2]True,False,None,min},
	emphstyle=[2]\color{specmethodcolour},
	emph={[3]object,type,isinstance,copy,deepcopy,zip,enumerate,reversed,list,len,dict,tuple,xrange,append,execfile,real,imag,reduce,str,repr},
	emphstyle=[3]\color{commandcolour},
	emph={[4]ode, fsolve, sqrt, exp, sin, cos, arccos, pi,  array, norm, solve, dot, arange, , isscalar, max, sum, flatten, shape, reshape, find, any, all, abs, plot, linspace, legend, quad, polyval,polyfit, hstack, concatenate,vstack,column_stack,empty,zeros,ones,rand,vander,grid,pcolor,eig,eigs,eigvals,svd,qr,tan,det,logspace,roll,mean,cumsum,cumprod,diff,vectorize,lstsq,cla,eye,xlabel,ylabel,squeeze},
	emphstyle=[4]\color{numpycolour},
	emph={[5]__init__,__add__,__mul__,__div__,__sub__,__call__,__getitem__,__setitem__,__eq__,__ne__,__nonzero__,__rmul__,__radd__,__repr__,__str__,__get__,__truediv__,__pow__,__name__,__future__,__all__},
	emphstyle=[5]\color{specmethodcolour},
	emph={[6]assert,range,yield},
	emphstyle=[6]\color{specmethodcolour}\bfseries,
	emph={[7]Exception,NameError,IndexError,SyntaxError,TypeError,ValueError,OverflowError,ZeroDivisionError,KeyboardInterrupt},
	emphstyle=[7]\color{specmethodcolour}\bfseries,
	emph={[8]taster,send,sendMail,capture,check,noMsg,go,move,switch,humTem,ventilate,buzz},
	emphstyle=[8]\color{blue},
	keywordstyle=\color{blue}\bfseries,
	rulecolor=\color{black!40},
	showstringspaces=false,
	stringstyle=\color{deepgreen}
}

\lstset{literate=%
	{Ö}{{\"O}}1
	{Ä}{{\"A}}1
	{Ü}{{\"U}}1
	{ß}{{\ss}}1
	{ü}{{\"u}}1
	{ä}{{\"a}}1
	{ö}{{\"o}}1
}

% Neue Klassenarbeits-Umgebung
\newenvironment{worksheet}[3]
% Begin-Bereich
{
	\newpage
	\sffamily
	\setcounter{page}{1}
	\ClearShipoutPicture
	\AddToShipoutPicture{
		\put(55,761){{
				\mbox{\parbox{385\unitlength}{\tiny \color{codegray}BBS I Mainz, #1 \newline #2
						\newline #3
					}
				}
			}
		}
		\put(455,761){{
				\mbox{\hspace{0.3cm}\includegraphics[width=0.2\textwidth]{../../logo.jpg}}
			}
		}
	}
}
% End-Bereich
{
	\clearpage
	\ClearShipoutPicture
}

\setlength{\columnsep}{3em}
\setlength{\columnseprule}{0.5pt}

\geometry{left=1.50cm,right=1.50cm,top=3.00cm,bottom=1.00cm,includeheadfoot}
\pagestyle{plain}
\pagenumbering{arabic}

\begin{document}
	\begin{worksheet}{Informationsverarbeitung}{Lernabschnitt: Strukturiert programmieren}{Programmiersprachen}
		\setlength{\columnseprule}{0pt}
		\noindent
		Bevor wir mit der Programmierung und dem erlernen einer Programmiersprache anfangen, wollen wir uns zunächst mit diversen Begrifflichkeiten und allgemeinen Informationen zu Programmiersprachen auseinandersetzen.
		\section{Programmiersprachen}
		Als Programmiersprache bezeichnen wir eine Sprache zur Formulierung von Algorithmen und Datenstrukturen, welche auf einem Computer abgearbeitet werden.\\
		Im Gegensatz zu natürlichen Sprachen in der ein Wort mehrere Bedeutungen besitzen kann, ist in einer Programmiersprache eindeutig festgelegt, welche Zeichenfolge als Programm zugelassen sind und was diese Zeichenfolgen bewirken.\\
		Programmiersprachen bilden also die wichtigste Schnittstelle zwischen Benutzer und Computer.
		\subsection{Sprachübersetzer}
		Mi einem \textbf{Interpreter} wird ein Programm jedes Mal Befehl für Befehl in die für das System verständliche Form übersetzt und ausgeführt.\\
		Will man ein solches Programm auf einem anderen System ausführen, so benötigt man einen Interpreter, der das Programm auf diesem System ausführen kann.\\
		\par\noindent
		Mit einem \textbf{Compiler} hingegen wird das Programm \underline{einmal} für ein bestimmtes System übersetzt und kann dann nur auf diesem ausgeführt werden. Um das gleiche Programm auf einem anderen System ausführen zu können, muss es mit einem für das neue System vorhandenen Compiler erneut übersetzt werden.
		\subsection{Beschreibungsmittel für Programmiersprachen}
		Möchten wir uns über eine Programmiersprache unterhalten, gibt es \textbf{zwei} Mittel, die wir nutzen können.\\
		Spricht man von der \textbf{Syntax} einer Programmiersprache, so meint man die Menge aller Regeln, nach denen zulässige Sätze in dieser Sprache gebildet werden können. Sie beschreibt also die Struktur dieser Sprache und kann mit der Grammatik einer natürlichen Sprache verglichen werden.\\
		Die Syntax definiert also, welche Zeichenfolgen korrekt formulierte Programme der Sprache sind und welche nicht.\\
		\par\noindent
		Die \textbf{Semantik} hingegen ist die Lehre der inhaltlichen Bedeutung einer Sprache. Bei der Programmiersprache kann damit beschrieben werden, was während der Ausführung eines Programms bzw. eines Programmteiles geschieht.\\
		Bei der Beschreibung werden Wechselwirkungen berücksichtigt.
		\subsection{Klassen von Programmiersprachen}
		Bei Programmiersprachen unterscheidet man im Prinzip nach den zugrundeliegenden Paradigmen\footnote{Muster/Schema}.\\
		Die wichtigsten Bestandteile eines Paradigmas sind:
		\begin{itemize}
			\item das theoretisch-mathematische Modell, auf dem es basiert - also wie funktioniert es?
			\item der unterschiedliche Variablenbegriff mit den Operationen auf die Variablen
			\item die elementaren Programmbausteine und die Konstruktionsmechanismen, mit denen man sie zu Programmen zusammensetzt
		\end{itemize}
		\textbf{\underline{Imperative Programmiersprachen}}
		Ein Programm ist eine Folge von Anweisungen (z.B. schreibe in eine Variable einen Wert; Springe an die Stelle XY).\\
		\par\noindent
		\textbf{\underline{Funktionale Programmiersprachen}}\\
		Ein Programm ist eine Menge von Funktionen, die Eingabedaten auf Ausgabedaten abbildet.\\
		\par\noindent
		\textbf{\underline{Logische Programmiersprachen}}
		Das Sprachparadigma stützt sich auf die symbolische Logik. Ein Programm besteht also aus einer Menge von Anweisungen, die beschreiben, was über ein gewünschtes Ergebnis wahr ist.\\
		\par\noindent
		\textbf{\underline{Objektorientierte Programmiersprache}}
		Ein Programm besteht aus einer Menge von Objekten, die miteinander über Botschaften kommunizieren.
	\end{worksheet}
\end{document}