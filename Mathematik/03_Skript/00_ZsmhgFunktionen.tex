\documentclass[11pt,twocolumn,oneside,openany,headings=optiontotoc,11pt,numbers=noenddot]{article}

\usepackage[a4paper]{geometry}
\usepackage[utf8]{inputenc}
\usepackage[T1]{fontenc}
\usepackage{lmodern}
\usepackage[ngerman]{babel}
\usepackage{ngerman}

\usepackage[onehalfspacing]{setspace}

\usepackage{fancyhdr}
\usepackage{fancybox}

\usepackage{rotating}
\usepackage{varwidth}


\usepackage{pdflscape}
\usepackage{graphicx}
\usepackage{graphbox}
\graphicspath{
	{Pics/PDFs/}
	{Pics/JPGs/}
	{Pics/PNGs/}
}
\usepackage{caption}
\usepackage{tabularx}
\usepackage{dashrule}
\usepackage{hhline}
\usepackage{multirow}
\usepackage{enumerate}
\usepackage[hidelinks]{hyperref}
\usepackage{listings}

\usepackage[table]{xcolor}
\usepackage{array}
\usepackage{enumitem,amssymb,amsmath}
\usepackage{interval}
\usepackage{stmaryrd}
\usepackage{polynom}
\usepackage{diagbox}
\usepackage{dashrule}
\usepackage{framed}
\usepackage{mdframed}
\usepackage{karnaugh-map}

\usepackage{blindtext}

\usepackage{eso-pic}

\usepackage{amssymb}
\usepackage{eurosym}
\pagestyle{headings}
\renewcommand{\headrulewidth}{0.2pt}
\renewcommand{\footrulewidth}{0.2pt}
\newcommand*{\underdownarrow}[2]{\ensuremath{\underset{\overset{\Big\downarrow}{#2}}{#1}}}
\setlength{\fboxsep}{5pt}

% Codestyle defined
\definecolor{codegreen}{rgb}{0,0.6,0}
\definecolor{codegray}{rgb}{0.5,0.5,0.5}
\definecolor{codepurple}{rgb}{0.58,0,0.82}
\definecolor{backcolour}{rgb}{0.95,0.95,0.92}
\definecolor{deepgreen}{rgb}{0,0.5,0}
\definecolor{darkblue}{rgb}{0,0,0.65}
\definecolor{mauve}{rgb}{0.40, 0.19,0.28}
\colorlet{exceptioncolour}{yellow!50!red}
\colorlet{commandcolour}{blue!60!black}
\colorlet{numpycolour}{blue!60!green}
\colorlet{specmethodcolour}{violet}

%Neue Spaltendefinition
\newcolumntype{L}[1]{>{\raggedright\let\newline\\\arraybackslash\hspace{0pt}}m{#1}}
\newcolumntype{M}[1]{>{\centering\arraybackslash}X}
\newcommand{\cmnt}[1]{\ignorespaces}
%Textausrichtung ändern
\newcommand\tabrotate[1]{\rotatebox{90}{\raggedright#1\hspace{\tabcolsep}}}

%Intervall-Konfig
\intervalconfig {
	soft open fences
}

%Bash
\lstdefinestyle{BashInputStyle}{
	language=bash,
	basicstyle=\small\sffamily,
	backgroundcolor=\color{backcolour},
	columns=fullflexible,
	backgroundcolor=\color{backcolour},
	breaklines=true,
}
%Java
\lstdefinestyle{JavaInputStyle}{
	language=Java,
	backgroundcolor=\color{backcolour},
	aboveskip=1mm,
	belowskip=1mm,
	showstringspaces=false,
	columns=flexible,
	basicstyle={\footnotesize\ttfamily},
	numberstyle={\tiny},
	numbers=none,
	keywordstyle=\color{purple},,
	commentstyle=\color{deepgreen},
	stringstyle=\color{blue},
	emph={out},
	emphstyle=\color{darkblue},
	emph={[2]rand},
	emphstyle=[2]\color{specmethodcolour},
	breaklines=true,
	breakatwhitespace=true,
	tabsize=2,
}
%Python
\lstdefinestyle{PythonInputStyle}{
	language=Python,
	alsoletter={1234567890},
	aboveskip=1ex,
	basicstyle=\footnotesize,
	breaklines=true,
	breakatwhitespace= true,
	backgroundcolor=\color{backcolour},
	commentstyle=\color{red},
	otherkeywords={\ , \}, \{, \&,\|},
	emph={and,break,class,continue,def,yield,del,elif,else,%
		except,exec,finally,for,from,global,if,import,in,%
		lambda,not,or,pass,print,raise,return,try,while,assert},
	emphstyle=\color{exceptioncolour},
	emph={[2]True,False,None,min},
	emphstyle=[2]\color{specmethodcolour},
	emph={[3]object,type,isinstance,copy,deepcopy,zip,enumerate,reversed,list,len,dict,tuple,xrange,append,execfile,real,imag,reduce,str,repr},
	emphstyle=[3]\color{commandcolour},
	emph={[4]ode, fsolve, sqrt, exp, sin, cos, arccos, pi,  array, norm, solve, dot, arange, , isscalar, max, sum, flatten, shape, reshape, find, any, all, abs, plot, linspace, legend, quad, polyval,polyfit, hstack, concatenate,vstack,column_stack,empty,zeros,ones,rand,vander,grid,pcolor,eig,eigs,eigvals,svd,qr,tan,det,logspace,roll,mean,cumsum,cumprod,diff,vectorize,lstsq,cla,eye,xlabel,ylabel,squeeze},
	emphstyle=[4]\color{numpycolour},
	emph={[5]__init__,__add__,__mul__,__div__,__sub__,__call__,__getitem__,__setitem__,__eq__,__ne__,__nonzero__,__rmul__,__radd__,__repr__,__str__,__get__,__truediv__,__pow__,__name__,__future__,__all__},
	emphstyle=[5]\color{specmethodcolour},
	emph={[6]assert,range,yield},
	emphstyle=[6]\color{specmethodcolour}\bfseries,
	emph={[7]Exception,NameError,IndexError,SyntaxError,TypeError,ValueError,OverflowError,ZeroDivisionError,KeyboardInterrupt},
	emphstyle=[7]\color{specmethodcolour}\bfseries,
	emph={[8]taster,send,sendMail,capture,check,noMsg,go,move,switch,humTem,ventilate,buzz},
	emphstyle=[8]\color{blue},
	keywordstyle=\color{blue}\bfseries,
	rulecolor=\color{black!40},
	showstringspaces=false,
	stringstyle=\color{deepgreen}
}

\lstset{literate=%
	{Ö}{{\"O}}1
	{Ä}{{\"A}}1
	{Ü}{{\"U}}1
	{ß}{{\ss}}1
	{ü}{{\"u}}1
	{ä}{{\"a}}1
	{ö}{{\"o}}1
}

% Neue Klassenarbeits-Umgebung
\newenvironment{worksheet}[3]
% Begin-Bereich
{
	\newpage
	\sffamily
	\setcounter{page}{1}
	\ClearShipoutPicture
	\AddToShipoutPicture{
		\put(55,761){{
				\mbox{\parbox{385\unitlength}{\tiny \color{codegray}BBS I Mainz, #1 \newline #2
						\newline #3
					}
				}
			}
		}
		\put(455,761){{
				\mbox{\hspace{0.3cm}\includegraphics[width=0.2\textwidth]{../../logo.jpg}}
			}
		}
	}
}
% End-Bereich
{
	\clearpage
	\ClearShipoutPicture
}

\setlength{\columnsep}{3em}
\setlength{\columnseprule}{0.5pt}

\geometry{left=1.50cm,right=1.50cm,top=3.00cm,bottom=1.00cm,includeheadfoot}
\pagenumbering{arabic}
\pagestyle{plain}

\begin{document}
	\begin{worksheet}{}{Mathematik}{Abschnitt: Grundlagen Funktionen}
		\setcounter{section}{2}
		\setcounter{page}{10}
		\section{Begrifflichkeiten Funktionen}
		\subsection{Wertepaare} Häufig werden wir mit Wertepaaren (x-Wert, y-Wert) konfrontiert. Mehrere Wertepaare zusammengenommen können eine Entwicklung oder gewisse Abhängigkeiten darstellen.\\
		Diese können entweder in einer Tabelle oder als Punkte in Koordinatensystemen dargestellt werden.\\
		\par\bigskip\noindent
		\underline{Beispiel:} Wir betrachten zunächst einmal die Entwicklung der Instagram-Aktie im letzten Monat.\\
		Dabei gibt die \textbf{erste Komponente} des Wertepaares das Jahr an und die \textbf{zweite Komponente} den dazugehörigen Aktienkurs.\\
		\par\bigskip\noindent
		\begin{tabularx}{0.45\textwidth}{c|c}
			\textbf{Zeit} & \textbf{Kurs (in Euro)}\\
			(0 = 16.07.2018 & \\
			1 Einheit = 1 Tag) & \\
			\hline
			0 & 177,22\\
			\hline
			1 & 177,93\\
			\hline
			2 & 180,04\\
			\hline
			3 & 180,28\\
			\hline
			4 & 179,99\\
			\hline
		\end{tabularx}\\
		\par\noindent
		Als \underline{Punkte in einem Koordinatensystem} entspricht dies der nachfolgenden Darstellung:\\
		\includegraphics[width=0.45\textwidth]{../99_Bilder/fb.jpg}\\
		Ein weiteres \underline{Beispiel} kann der Lagerbestand von Bierkisten in einem Getränkemarkt sein.\\
		\par\bigskip\noindent
		\includegraphics[width=0.5\textwidth]{../99_Bilder/bier.jpg}
		\subsection{Wertepaare werden zum Graphen}
		Wie im oberen Diagramm unschwer zu erkennen, können Punktdiagramme unübersichtlich sein. Abhilfe kann das Verbinden der einzelnen Punkte durch Linien schaffen.\\
		\par\bigskip\noindent
		\includegraphics[width=0.5\textwidth]{../99_Bilder/bier1.jpg}\\
		\par\bigskip\noindent
		Für jede einzelne Verbindungslinie könnte man eine Entwicklung in einer Gleichung darstellen. Beim Lagerbestand wären das 5 verschiedene Gleichungen.\\
		Daher versucht eine glatte Trendlinie aufzustellen, so dass man die Entwicklung grob wiedergeben kann.\\
		\par\bigskip\noindent
		\includegraphics[width=0.5\textwidth]{../99_Bilder/bier2.jpg}\\
		\par\bigskip\noindent
		Diese Trendlinie kann durch die Gleichung
		\[y = -27x + 170\]
		angegeben werden.
		\subsection{Die Gleichung wird zum Graphen}
		Sind Sie mit der Aufgabe konfrontiert eine gegebene Gleichung in einem Koordinatensystem einzuzeichnen, ist der erste Schritt, entsprechende Wertepaare zu bestimmen. Dafür besetzt man das \(x\) der Gleichung mit einer Zahl und berechnet den Wert des Terms.\\
		Zur besseren Übersicht überträgt man diese Wertepaare in eine \textbf{\underline{Wertetabelle}}.\\
		\begin{tabularx}{0.45\textwidth}{|X|X|X|X|X|X|X|}
			\hline
			x & 0 & 1 & 2 & 3 & \(\ldots\) & \(\ldots\)\\
			\hline
			y & & & & & & \\
			\hline
		\end{tabularx}
		\subsubsection*{\underline{Ihre Aufgabe}} Berechnen Sie einige Wertepaare! Und übertragen diese in eine Wertetabelle.
		\begin{tabularx}{0.45\textwidth}{X}
			\(y = 4x + 12\)\\
			\(y = -3x - 21\)\\
			\(y = 2x - 5\)
		\end{tabularx}
		\par\bigskip\noindent
		Überträgt man die berechneten Wertepaare in ein Koordinatensystem und verbindet die einzelnen Punkte erhält man eine durchgezogene Linie. Diese repräsentiert im Prinzip nur die Aneinanderreihung von ganz vielen berechneten Wertepaaren.\\
		Diese durchgezogene Linie nennt man \underline{\textbf{Graph}}.
		\subsection{Modell vs. Realität}
		Wie Sie am Beispiel des Lagerbestands sicher bereits erkannt haben, gibt es Abweichungen zwischen dem realen Lagerbestand und der Trendlinie. Das ist darauf zurückzuführen, dass die Gleichung lediglich ein \underline{Modell} ist, das versucht, die Realität zu vereinfachen. Man kann sagen, dass ein Modell ein idealisiertes Abbild der Realität ist.\\
		\par\bigskip\noindent
		\begin{tabularx}{0.45\textwidth}{XX}
			\includegraphics[width=0.25\textwidth]{../99_Bilder/lager.jpg} & \includegraphics[width=0.25\textwidth]{../99_Bilder/bier2.jpg}
		\end{tabularx}
		\par\bigskip\noindent
		\underline{Vorteile des Modells:}
		\begin{itemize}
			\item[-] geradliniger Graph ohne Zacken
			\item[-] Analyse der Entwicklung möglich
			\item[-] Darstellung der Entwicklung mittels einer Gleichung
			\item[-] Fortführung des Modells möglich (Prognose)
		\end{itemize}
		Ein Modell hat aber nicht nur Vorteile. Ein wesentlicher \underline{Nachteil des Modells} ist, dass der Informationsgehalt eingeschränkt ist, da Ausschläge und Abweichungen einfach weggelassen werden.
		\newpage
		\section{Der Begriff \glqq{}Funktion\grqq{}}
		\begin{framed}
			\noindent
			\textbf{Definition:} Eine Funktion entspricht einer Menge von Wertepaaren, welche aus zwei Mengen gebildet wird. Jedem Element der Ausgangsmenge (Definitionsmenge oder auch \(\mathbb{D}\)) wird ein Element der Zielmenge (Wertemenge oder auch \(\mathbb{W}\)) zugeordnet.
		\end{framed}
		\noindent
		Nimmt man zum Beispiel das Element \(x=4\) aus \(\mathbb{D}\), so gibt einem die Funktion den entsprechenden Lagerstand zu diesem Zeitpunkt. So ergibt sich das Wertepaar \((4|62)\).\\
		\par\bigskip\noindent
		Das Element der Ausgangsmenge bezeichnet man als \underline{\textbf{x-Koordinate}} und das Element der Wertemenge nennt sich \underline{\textbf{y-Koordinate}}.\\
		\par\bigskip\noindent
		Im Prinzip ist also eine Funktion nichts anderes als eine Menge von Wertepaaren, die aus einer bestimmten Vorschrift (der Funktion) hervorgehen. Daher kann man mit Hilfe von Funktionen Entwicklungen und Abhängigkeiten beschreiben oder modellieren.
		\subsection{Darstellungsformen von Funktionen}
		Eine Funktion kann auf verschiedene Arten dargestellt werden.
		\subsubsection*{Funktionsterm und Funktionsgleichung}
		Der Funktionsterm gibt die Anweisung zur Erzeugung der Wertepaare. Diese entstehen dadurch, dass man die Variable durch eine Zahl (erste Komponente) besetzt und den Wert des Terms (zweite Komponente) berechnet.\\
		Als Funktionsgleichung hingegen bezeichnet man den Ausdruck, dem Term \(y\) den Funktionsterm durch das \glqq{}\(=\)\grqq{}-Zeichen zuweist.\\
		\par\bigskip\noindent
		\begin{align*}
			\underbrace{y = \overbrace{200 -18x}^{\text{Funktionsterm}}}_{\text{Funktionsgleichung}}
		\end{align*}
		\subsubsection*{Wertetabelle}
		Hat man hingegen Wertepaare gebildet, kann man diese in einer Wertetabelle festhalten. Diese spiegeln eindeutig die Funktion wider.\\
		Nachfolgendes Beispiel zeigt die Wertetabelle für die Funktion \(y = -27x+170\).\\
		\par\bigskip\noindent
		\begin{tabularx}{0.5\textwidth}{|l|X|X|X|X|X|X|}
			\hline
			x & 0 & 1 & 2 & 3 & 4 & 5\\
			\hline
			y & 170 & 143 & 116 & 89 & 62 & 35\\
			\hline
		\end{tabularx}
		\subsubsection*{Graph}
		Hat man Wertepaare gegeben, kann man diese als Punkte in ein Koordinatensystem übertragen.\\
		\underline{Anmerkung zur Skalierung:} Man erkennt, dass der Bestand Werte zwischen 170 und 35 annimmt. Daher sollten die Einheit für die y-Achse 17 haben \((170:10 = 17)\)\\
		\includegraphics[width=0.45\textwidth]{../99_Bilder/bierkoord.jpg}\\
		\par\bigskip\noindent
		Die Menge der Punkte, die durch das Verbinden der einzelnen Wertepaar-Punkte entsteht nennt man auch \underline{\textbf{Graph der Funktion}}. Häufig ist es aber nur möglich einen bestimmten Abschnitt des \underline{Funktionsgraphen} zu zeichnen.\\
		\par\bigskip\noindent
		Genaugenommen ist es uns nicht möglich einen Punkt oder den Graph einer Funktion zu zeichnen. Das oben dargestellt ist lediglich der intuitiven Anwendung unserer Erfahrung geschuldet.\\
		Das Verbinden der Punkte mit dem Lineal ist nur bei \textbf{linearen Funktionen} möglich.\\
		\textit{\color{red}{Im Allgemeinen gilt: Das Verwenden des Lineals zum Verbinden der Punkte ist streng verboten!}}\\
		\subsubsection*{\underline{Ihre Aufgabe}}
		Zeichnen Sie einen sauberen Graphen für die Funktionsgleichung \(\mathbf{y = 200-18x}\).\\
		Nutzen Sie dafür eine Wertetabelle.
		\newpage
		\section{Funktionstypen}
		Es gibt verschiedene Arten von Funktionstypen. Diese werden in den nachfolgenden Abschnitten kurz erläutert.
		\subsection{Funktionstypen und ihre Prototypen}
		Die nachfolgende Tabelle gibt einen Überblick über diese. Zusätzlich werden die Prototypen der dazugehörigen Gleichungen angegeben, so dass Sie in Zukunft anhand des Funktionsterms den Funktionstypen erkennen können.\\
		\par\bigskip\noindent
		\begin{tabularx}{0.5\textwidth}{|X|X|}
			\hline
			\textbf{Funktionstyp} & \textbf{Prototyp der Gleichung}\\
			\hline
			Lineare Funktionen & \(y = m\cdot{}x + b\)\\
			\hline
			Quadratische Funktionen & \(y = a_2\cdot{}x^2 + a_1\cdot{}x + a_0\)\\
			\hline
			Ganzrationale Funktionen & \(y = a_n\cdot{}x^n + \ldots{} + a_2\cdot{}x^2 + a_1\cdot{}x + a_0\)\\
			\hline
			Exponential Funktionen & \(y = a\cdot{} b^x\)\\
			\hline
			Potenzfunktion & \(y = x^a\)\\
			& a ist eine rationale Zahl\\
			\hline
			Wurzelfunktion & \(y = \sqrt{x}\)\\
			\hline
			Gebrochen-rationale Funktionen & \(y = \frac{a_n\cdot{}x^n + \ldots{} + a_1\cdot{}x + a_0}{b_m\cdot{}x^m + \ldots{} + b_1\cdot{}x + b_0}\)\\
			\hline
		\end{tabularx}
		\subsection{Verkettete Funktionen}
		Es kann auch passieren, dass verschiedene Funktionstypen miteinander verkettet werden.\\
		So kann zum Beispiel eine \underline{Lineare Funktion} mit einer \underline{Potenzfunktion} verkettet werden: \(y = (x +5)^4\)
	\end{worksheet}
	\begin{worksheet}{}{Mathematik}{Abschnitt: Arbeiten mit Funktionen}
		\setcounter{section}{5}
		\setcounter{page}{14}
		\section{Spezialitäten und Tricks}
		Im Idealfall bereitet eine Funktionsgleichung sogut wie keine Probleme beim Zeichnen des Graphen. Die bekannte Wertetabelle liefert ausreichend viele Punkte, um einen ordentlichen Ausschnitt des Graphen zu zeichnen, aus dem die im Kapitel \glqq{}Grundlagen Funktionen\grqq{} angesprochenen Charakteristika des Graphen deutlich hervorgehen.\\
		\underline{Aber} es gibt auch Funktionstypen mit Besonderheiten. Nachfolgend werden einzelne Besonderheiten aufgeführt und wie sie damit umgehen können.
		\subsection{Besonderheit 1 - Außergewöhnliche Wertetabelle}
		Es kann passieren, dass die aufgestellte Wertetabelle es nicht ermöglicht einen wirklich plausiblen Graphen zu skizzieren.\\
		Bei der Funktion \(y = x^3-3x^2 + 2x\) sieht die Wertetabelle wie folgt aus:
		\begin{tabularx}{0.5\textwidth}{|l|X|X|X|X|X|}
			x & \(-1\) & \(0\) & \(1\) & \(2\) & \(3\)\\
			\hline
			y & \(-6\) & \(0\) & \(0\) & \(0\) & \(6\)
		\end{tabularx}
		Es ist unschwer erkennbar, dass das zugehörige Koordinatensystem so aussieht:\\
		\par\bigskip\noindent
		\includegraphics[width=0.35\textwidth]{../99_Bilder/b1Koord.jpg}\\
		\par\bigskip\noindent
		Hier kann es Helfen, wenn man die Wertetabelle verfeinert und so mehr Punkte zum eintragen erhält.\\
		\par\bigskip\noindent
		\includegraphics[width=0.35\textwidth]{../99_Bilder/b1Koord1.jpg}\\
		\subsection{Besonderheit 2 - Unklare Skalierung}
		Zwar haben wir eine Wertetabelle mit ausreichend vielen Wertepaaren, aber daraus ist nur schwer ersichtlich, wie die Skalierung der y-Achse zu wählen ist, um die Charakteristika zu verdeutlichen.\\
		\textbf{Beispiel: Bevölkerungswachstun Rheinland-Pfalz}\\
		\begin{tabularx}{0.49\textwidth}{|l|X|X|X|X|X|X|}
			\hline
			x & \(0\) & \(1\) & \(2\) & \(3\) & \(4\) & \(5\)\\
			\hline
			y & \(3555\) & \(3602\) & \(3550\) & \(3578\) & \(3629\) & \(3698\)\\
			\hline
		\end{tabularx}
		\par\bigskip\noindent
		\begin{tabularx}{0.5\textwidth}{l|l|l|l|}
			\cline{2-4}
			& \(6\) & \(7\) & \(8\)\\
			\cline{2-4}
			& \(3773\) & \(3848\) & \(3913\)\\
			\cline{2-4}
		\end{tabularx}
		\par\bigskip\noindent
		Natürlich müssen auf der x-Achse die Werte 0 bis 8 abgetragen werden. Die Skalierung kann hier wie üblich gewählt werden.\\
		Wie aber sieht es auf der y-Achse aus? Hier sollen wir auf 10 cm Werte bis \(4000\) abtragen. Eine Möglichkeit wäre es, je \textbf{1 cm} immer \(400\) abzutragen.\\
		\par\bigskip\noindent
		\includegraphics[width=0.45\textwidth]{../99_Bilder/b2Koord.jpg}\\
		\par\bigskip\noindent
		Aus der Wertetabelle ist aber erkennbar, dass unser niedrigster y-Wert \(3550\) ist, wir benötigen also die Werte 0 bis \(3549\) nicht. Damit eröffnet sich eine andere Möglichkeit.\\
		Wir beginnen im Ursprung mit dem y-Wert \(3500\) und erhöhen diesen je \textbf{1 cm} um 50.\\
		\par\bigskip\noindent
		\includegraphics[width=0.45\textwidth]{../99_Bilder/b2Koord1.jpg}\\
		\par\bigskip\noindent
		\textit{\textbf{Vorsicht:} Wurde die Einheit für eine Achse festgelegt, so gilt diese für die gesamte Achse und \underline{darf nicht} geändert werden.}
	\end{worksheet}
\end{document}