\documentclass[11pt,twocolumn,oneside,openany,headings=optiontotoc,11pt,numbers=noenddot]{article}

\usepackage[a4paper]{geometry}
\usepackage[utf8]{inputenc}
\usepackage[T1]{fontenc}
\usepackage{lmodern}
\usepackage[ngerman]{babel}
\usepackage{ngerman}

\usepackage[onehalfspacing]{setspace}

\usepackage{fancyhdr}
\usepackage{fancybox}

\usepackage{rotating}
\usepackage{varwidth}


\usepackage{pdflscape}
\usepackage{graphicx}
\usepackage{graphbox}
\graphicspath{
	{Pics/PDFs/}
	{Pics/JPGs/}
	{Pics/PNGs/}
}
\usepackage{caption}
\usepackage{tabularx}
\usepackage{dashrule}
\usepackage{hhline}
\usepackage{multirow}
\usepackage{enumerate}
\usepackage[hidelinks]{hyperref}
\usepackage{listings}

\usepackage[table]{xcolor}
\usepackage{array}
\usepackage{enumitem,amssymb,amsmath}
\usepackage{interval}
\usepackage{stmaryrd}
\usepackage{polynom}
\usepackage{diagbox}
\usepackage{dashrule}
\usepackage{framed}
\usepackage{mdframed}
\usepackage{karnaugh-map}

\usepackage{blindtext}

\usepackage{eso-pic}

\usepackage{amssymb}
\usepackage{eurosym}
\pagestyle{headings}
\renewcommand{\headrulewidth}{0.2pt}
\renewcommand{\footrulewidth}{0.2pt}
\newcommand*{\underdownarrow}[2]{\ensuremath{\underset{\overset{\Big\downarrow}{#2}}{#1}}}
\setlength{\fboxsep}{5pt}

% Codestyle defined
\definecolor{codegreen}{rgb}{0,0.6,0}
\definecolor{codegray}{rgb}{0.5,0.5,0.5}
\definecolor{codepurple}{rgb}{0.58,0,0.82}
\definecolor{backcolour}{rgb}{0.95,0.95,0.92}
\definecolor{deepgreen}{rgb}{0,0.5,0}
\definecolor{darkblue}{rgb}{0,0,0.65}
\definecolor{mauve}{rgb}{0.40, 0.19,0.28}
\colorlet{exceptioncolour}{yellow!50!red}
\colorlet{commandcolour}{blue!60!black}
\colorlet{numpycolour}{blue!60!green}
\colorlet{specmethodcolour}{violet}

%Neue Spaltendefinition
\newcolumntype{L}[1]{>{\raggedright\let\newline\\\arraybackslash\hspace{0pt}}m{#1}}
\newcolumntype{M}[1]{>{\centering\arraybackslash}X}
\newcommand{\cmnt}[1]{\ignorespaces}
%Textausrichtung ändern
\newcommand\tabrotate[1]{\rotatebox{90}{\raggedright#1\hspace{\tabcolsep}}}

%Intervall-Konfig
\intervalconfig {
	soft open fences
}

%Bash
\lstdefinestyle{BashInputStyle}{
	language=bash,
	basicstyle=\small\sffamily,
	backgroundcolor=\color{backcolour},
	columns=fullflexible,
	backgroundcolor=\color{backcolour},
	breaklines=true,
}
%Java
\lstdefinestyle{JavaInputStyle}{
	language=Java,
	backgroundcolor=\color{backcolour},
	aboveskip=1mm,
	belowskip=1mm,
	showstringspaces=false,
	columns=flexible,
	basicstyle={\footnotesize\ttfamily},
	numberstyle={\tiny},
	numbers=none,
	keywordstyle=\color{purple},,
	commentstyle=\color{deepgreen},
	stringstyle=\color{blue},
	emph={out},
	emphstyle=\color{darkblue},
	emph={[2]rand},
	emphstyle=[2]\color{specmethodcolour},
	breaklines=true,
	breakatwhitespace=true,
	tabsize=2,
}
%Python
\lstdefinestyle{PythonInputStyle}{
	language=Python,
	alsoletter={1234567890},
	aboveskip=1ex,
	basicstyle=\footnotesize,
	breaklines=true,
	breakatwhitespace= true,
	backgroundcolor=\color{backcolour},
	commentstyle=\color{red},
	otherkeywords={\ , \}, \{, \&,\|},
	emph={and,break,class,continue,def,yield,del,elif,else,%
		except,exec,finally,for,from,global,if,import,in,%
		lambda,not,or,pass,print,raise,return,try,while,assert},
	emphstyle=\color{exceptioncolour},
	emph={[2]True,False,None,min},
	emphstyle=[2]\color{specmethodcolour},
	emph={[3]object,type,isinstance,copy,deepcopy,zip,enumerate,reversed,list,len,dict,tuple,xrange,append,execfile,real,imag,reduce,str,repr},
	emphstyle=[3]\color{commandcolour},
	emph={[4]ode, fsolve, sqrt, exp, sin, cos, arccos, pi,  array, norm, solve, dot, arange, , isscalar, max, sum, flatten, shape, reshape, find, any, all, abs, plot, linspace, legend, quad, polyval,polyfit, hstack, concatenate,vstack,column_stack,empty,zeros,ones,rand,vander,grid,pcolor,eig,eigs,eigvals,svd,qr,tan,det,logspace,roll,mean,cumsum,cumprod,diff,vectorize,lstsq,cla,eye,xlabel,ylabel,squeeze},
	emphstyle=[4]\color{numpycolour},
	emph={[5]__init__,__add__,__mul__,__div__,__sub__,__call__,__getitem__,__setitem__,__eq__,__ne__,__nonzero__,__rmul__,__radd__,__repr__,__str__,__get__,__truediv__,__pow__,__name__,__future__,__all__},
	emphstyle=[5]\color{specmethodcolour},
	emph={[6]assert,range,yield},
	emphstyle=[6]\color{specmethodcolour}\bfseries,
	emph={[7]Exception,NameError,IndexError,SyntaxError,TypeError,ValueError,OverflowError,ZeroDivisionError,KeyboardInterrupt},
	emphstyle=[7]\color{specmethodcolour}\bfseries,
	emph={[8]taster,send,sendMail,capture,check,noMsg,go,move,switch,humTem,ventilate,buzz},
	emphstyle=[8]\color{blue},
	keywordstyle=\color{blue}\bfseries,
	rulecolor=\color{black!40},
	showstringspaces=false,
	stringstyle=\color{deepgreen}
}

\lstset{literate=%
	{Ö}{{\"O}}1
	{Ä}{{\"A}}1
	{Ü}{{\"U}}1
	{ß}{{\ss}}1
	{ü}{{\"u}}1
	{ä}{{\"a}}1
	{ö}{{\"o}}1
}

% Neue Klassenarbeits-Umgebung
\newenvironment{worksheet}[3]
% Begin-Bereich
{
	\newpage
	\sffamily
	\setcounter{page}{1}
	\ClearShipoutPicture
	\AddToShipoutPicture{
		\put(55,761){{
				\mbox{\parbox{385\unitlength}{\tiny \color{codegray}BBS I Mainz, #1 \newline #2
						\newline #3
					}
				}
			}
		}
		\put(455,761){{
				\mbox{\hspace{0.3cm}\includegraphics[width=0.2\textwidth]{../../logo.jpg}}
			}
		}
	}
}
% End-Bereich
{
	\clearpage
	\ClearShipoutPicture
}

\setlength{\columnsep}{3em}
\setlength{\columnseprule}{0.5pt}

\geometry{left=1.50cm,right=1.50cm,top=2.50cm,bottom=1.00cm,includeheadfoot}
\pagenumbering{gobble}
\pagestyle{empty}

\begin{document}
	\begin{worksheet}{Mathematik}{Lernabschnitt: Quadratische Funktionen}{Faktorisieren}
		\setcounter{section}{1}
		\section{Quadratische Funktionen faktorisieren}
		Haben wir eine quadratische Funktion gegeben, kann es für das weitere Vorgehen hilfreich sein, wenn man diese faktorisiert.\\
		Wir überführen also
		\begin{center}
			\colorbox{red!10}{\(f(x) = ax^2 +bx + c \)}\\
			zu\\
			\colorbox{green!10}{\(f(x) = d\cdot(x - x_{N1})(x-x_{N2})\)}
		\end{center}
		Um dies zu tun, können wir verschiedene Vorgehen verwenden:
		\begin{itemize}
			\item Differenz von Quadraten
			\item Vollständiges Quadrat
			\item Gruppieren
			\item Teilen und Schieben
			\item Nullstellen bestimmen
		\end{itemize}
		\begin{framed}
			\noindent
			\color{red}{\underline{Vorsicht!}}\normalcolor\\
			Die Faktorisierung einer quadratischen Funktion ist nur möglich, wenn die 
		\end{framed}
		\subsection{Differenz von Quadraten (3. Binomische Formel)}
		Hat unsere quadratische Funktion die Form \[f(x) = x^2 - c; \pm\sqrt{c}\in \mathbb{Z}\]
		Dann handelt es sich um die Differenz von Quadraten auch bekannt als die ausmultiplizierte dritte binomische Formel.\\
		\par\noindent
		\(\Rightarrow\ f(x) = (x+\sqrt{c})\cdot(x-\sqrt{c})\)
		\subsection{Vollständiges Quadrat (1. und 2. Binomische Formel)}
		Hat die quadratische Funktion hingegen die Form \[f(x) = ax^2 \pm bx + c\] und  für die Koeffizienten gilt \(b = 2ad\) und \(c = d^2\) entspricht das \[f(x) = ax^2 \pm 2adx + d^2\]
		Klar erkennbar ist die erste bzw. zweite binomische Formel
		\begin{align*}
			f(x) & = (ax+d)^2\\
			&\text{oder}\\
			f(x) & = (ax-d)^2
		\end{align*}
		\subsection{Gruppieren}
		Wir sind konfrontiert mit einer quadratischen Gleichung der Form \(f(x) = ax^2 + bx + c\). Um gut gruppieren zu können betrachten wir die Koeffizienten \(a,b \text{ und } c\).\\
		Es müssen die Fragen geklärt werden, welche zwei Zahlen \(z_1\) und \(z_2\) ergeben
		\begin{itemize}
			\item[-] bei Multiplikation \(*\Rightarrow a\cdot{}c\)
			\item[-] bei Addition \(+\Rightarrow b\)
		\end{itemize}
		Dieser Schritt ist mit Abstand der schwierigste an diesem Vorgehen.
		\begin{framed}
			\noindent
			Dabei kann es passieren, dass es \textbf{keine solchen zwei Zahlen} gibt. In diesem Fall müssen wir die \textbf{Nullstellen bestimmen}.
			\color{red}
		\end{framed}
		\normalcolor
		\noindent
		Haben wir diese zwei Zahlen gefunden, schreiben wir unsere Funktion neu.
		\[f(x) = ax^2 + z_1x + z_2x + c\]
		Hierbei sollten wir die Reihenfolge von \(z_1\) und \(z_2\) so wählen, dass sie mit \(a\) bzw. mit \(c\) einen gemeinsamen Teiler \(t_1, t_2\) haben.\\
		Diesen gemeinsamen Teiler klammern wir in beiden Fällen aus und erhalten:
		\begin{align*}
			f(x) & = t_1x(\underbrace{\frac{a}{t_1}}_{k_1}x + \underbrace{\frac{z_1}{t_1}}_{k_2}) + t_2(\underbrace{\frac{z_2}{t_2}}_{k_1}x + \underbrace{\frac{c}{t_2}}_{k_2})\\
			& = t_1x(k_1x + k_2) + t_2(k_1x+k_2)
		\end{align*}
		Die beiden Klammerausdrücke sind gleich und ermöglichen uns ein erneutes ausklammern.
		\[f(x) = (t_1x+t_2)(k_1x+k_2)\]
		Schauen wir uns folgendes \underline{\textbf{Beispiel}} an:\\
		\par\noindent
		\begin{tabularx}{0.5\textwidth}{X|l}
			\cline{2-2}
			& \(*:\ 5\cdot(-80)=-400\)\\
			\(f(x) = 5x^2 + 30x -80\) & \(+:\ 30\)\\
			& \(\Rightarrow z_1 = 40; z_2 = -10\)\\
			\cline{2-2}
			\multicolumn{2}{l}{}\\
			\multicolumn{2}{l}{\(f(x) = 5x^2 + \underbrace{40}_{z_1}x \underbrace{-10}_{z_2}x -80\)}\\
			\multicolumn{2}{l}{\(f(x) = 5x(x+8) -10(x+8)\)}\\
			\multicolumn{2}{l}{\(f(x) = (5x-10)(x+8)\)}
		\end{tabularx}\\
		\par\noindent
		\rule{0.45\textwidth}{0.1pt}
		Wir schauen uns noch ein \underline{\textbf{Beispiel}} an:\\
		\par\noindent
		\begin{tabularx}{0.5\textwidth}{X|l}
			\cline{2-2}
			& \(*:\ 1\cdot(-40)=-40\)\\
			\(f(x) = x^2 + 6x - 40\) & \(+:\ 6\)\\
			& \(\Rightarrow z_1 = 10; z_2 = -4\)\\
			\cline{2-2}
			\multicolumn{2}{l}{}\\
			\multicolumn{2}{l}{\(f(x) = x^2 + \underbrace{-4}_{z_2}x \underbrace{10}_{z_1}x - 40\)}\\
			\multicolumn{2}{l}{\(f(x) = x(x-4) + 10(x-4)\)}\\
			\multicolumn{2}{l}{\(f(x) = (x+10)(x-4)\)}
		\end{tabularx}
		\subsection{Teilen und Schieben}
		Auch diesmal haben wir wieder eine quadratische Funktion \(f(x) = ax^2 + bx + c\) gegeben.\\
		Die ersten Schritte entsprechen denen, die wir beim Gruppieren anwenden.\\
		Wir betrachten erneut die Koeffizienten \(a,b \text{ und } c\).\\
		Es müssen die Fragen geklärt werden, welche zwei Zahlen \(z_1\) und \(z_2\) ergeben
		\begin{itemize}
			\item[-] bei Multiplikation \(*\Rightarrow a\cdot{}c\)
			\item[-] bei Addition \(+\Rightarrow b\)
		\end{itemize}
		Dieser Schritt ist mit Abstand der schwierigste an diesem Vorgehen.
		\begin{framed}
			\noindent
			Dabei kann es passieren, dass es \textbf{keine solchen zwei Zahlen} gibt. In diesem Fall müssen wir die \textbf{Nullstellen bestimmen}.
			\color{red}
		\end{framed}
		\normalcolor
		\noindent
		Unter Verwendung der beiden Zahlen schreiben wir nun \[f(x) = (\underline{\ \ }x + \frac{z_1}{a})(\underline{\ \ }x + \frac{z_2}{a})\]
		Lässt sich \(z_1\) bzw. \(z_2\) ganzzahlig durch \(a\) teilen, so erhält \(x\) den Koeffizienten \(1\).\\
		Ist dies nicht der Fall, erhält \(x\) den Koeffizienten \(a\) und die Zahl \(z_1\) oder \(z_2\) bleibt erhalten.\\
		\par\noindent
		\begin{itemize}
			\item[] 1. Möglichkeit: \(z_1\) lässt sich nicht ganzzahlig durch \(a\) teilen, \(z_2\) aber schon\\
			\(f(x) = (ax + z_1)(x+\underbrace{d}_{\frac{z_2}{a}})\)
			\item[] 2. Möglichkeit: \(z_1\) lässt sich ganzzahlig durch \(a\) teilen, \(z_2\) aber nicht\\
			\(f(x) = (x + \underbrace{d}_{\frac{z_1}{a}})(ax+z_2)\)
			\item[] 3. Möglichkeit: \(z_1\) und \(z_2\) lassen sich ganzzahlig durch \(a\) teilen, dann wird \(a\) als Leitkoeffizient eingeführt.\\
			\(f(x) = a(x+\underbrace{d}_{\frac{z_1}{a}})(x+\underbrace{e}_{\frac{z_2}{a}})\)
		\end{itemize}
		Wir betrachten das gleiche \underline{\textbf{Beispiel}} wie eben:\\
		\par\noindent
		\begin{tabularx}{0.5\textwidth}{X|l}
			\cline{2-2}
			& \(*:\ 5\cdot(-80)=-400\)\\
			\(f(x) = 5x^2 + 30x -80\) & \(+:\ 30\)\\
			& \(\Rightarrow z_1 = 40; z_2 = -10\)\\
			\cline{2-2}
			\multicolumn{2}{l}{}\\
			\multicolumn{2}{l}{\(f(x) = (x+\frac{40}{5})(x+\frac{-10}{5})\)}\\
			\multicolumn{2}{l}{\(f(x) = 5(x+8)(x-2)\)}
		\end{tabularx}\\
		\par\noindent
		\rule{0.45\textwidth}{0.1pt}
		Wir schauen uns noch ein \underline{\textbf{Beispiel}} an:\\
		\par\noindent
		\begin{tabularx}{0.5\textwidth}{X|l}
			\cline{2-2}
			& \(*:\ 1\cdot(-40)=-40\)\\
			\(f(x) = x^2 + 6x - 40\) & \(+:\ 6\)\\
			& \(\Rightarrow z_1 = 10; z_2 = -4\)\\
			\cline{2-2}
			\multicolumn{2}{l}{}\\
			\multicolumn{2}{l}{\(f(x) = (x+\frac{10}{1})(x+\frac{-4}{1})\)}\\
			\multicolumn{2}{l}{\(f(x) = (x+10)(x-4)\)}\\
		\end{tabularx}
		\subsection{Nullstellen bestimmen}
		Lässt sich die gegebene quadratische Funktion \(f(x) = ax^2 + bx + c\) nicht mit einer der obigen Verfahren faktorisieren, müssen wir die Nullstellen bestimmen.\\
		Hierfür verwenden wir die \textbf{pq-Formel}.
		\begin{framed}
			\noindent
			\underline{Gegeben:} \(f(x) = x^2 + px + q\)\\
			Dann erfüllen folgende zwei Zahlen \[x_{1,2}=-\frac{p}{2} \pm \sqrt{\left(\frac{p}{2}\right)^2 - q}\] die Gleichung \(f(x) = 0\).\\
			\par\noindent
			Zu beachten ist, dass der \textbf{Leitkoeffizient} (Zahl vor \(x^2\)) \textbf{1 ist}.
		\end{framed}
		Ist unsere quadratische Funktion nicht von der Form \(f(x) = x^2 + px + q\), so müssen wir diese durch Teilen in die entsprechende Form überführen.\\
		\par\noindent
		Wir betrachten das gleiche \underline{\textbf{Beispiel}} wie eben.\\
		\par\noindent
		\begin{tabularx}{0.5\textwidth}{Xl}
			\(f(x) = 5x^2 + 30x - 80\) & |\(:5\)\\
			\(f(x) = x^2 + \underbrace{6}_{p}x \underbrace{- 16}_{q}\) & |pq-Formel\\
			\\
			\(x_{1,2} = -\frac{6}{2} \pm \sqrt{\left(\frac{6}{2}\right)^2 +16}\)\\
			\(x_1 = -\frac{6}{2} + \sqrt{\left(\frac{6}{2}\right)^2 +16}\) & \(x_2 = -\frac{6}{2} - \sqrt{\left(\frac{6}{2}\right)^2 +16}\)\\
			\(x_1 = - 3 + \sqrt{9 + 16}\) & \(x_1 = - 3 - \sqrt{9 + 16}\)\\
			\\
			\colorbox{green!10}{\(x_1 = -3 + 5 = 2\)} & \colorbox{green!10}{\(x_2 = -3 - 5 = -8\)}
		\end{tabularx}\\
		\par\noindent
		Mit diesen Nullstellen können wir faktorisieren (\(f(x) = (x-x_1)(x-x_2)\)) und erhalten:
		\[f(x) = (x-2)(x+8)\]
		\rule{0.45\textwidth}{0.1pt}
		Wir betrachten zudem das nachfolgende \underline{\textbf{Beispiel}}.\\
		\par\noindent
		\begin{tabularx}{0.5\textwidth}{Xl}
			\(f(x) = x^2 + \underbrace{6}_{p}x \underbrace{- 40}_{q}\) &  |pq-Formel\\
			\\
			\(x_{1,2} = -\frac{6}{2} \pm \sqrt{\left(\frac{6}{2}\right)^2 +40}\)\\
			\(x_1 = -\frac{6}{2} + \sqrt{\left(\frac{6}{2}\right)^2 +40}\) & \(x_2 = -\frac{6}{2} - \sqrt{\left(\frac{6}{2}\right)^2 +40}\)\\
			\(x_1 = - 3 + \sqrt{9 + 40}\) & \(x_1 = - 3 - \sqrt{9 + 40}\)\\
			\\
			\colorbox{green!10}{\(x_1 = -3 + 7 = 4\)} & \colorbox{green!10}{\(x_2 = -3 - 7 = -10\)}
		\end{tabularx}\\
		\par\noindent
		Mit diesen Nullstellen können wir faktorisieren (\(f(x) = (x-x_1)(x-x_2)\)) und erhalten:
		\[f(x) = (x-4)(x+10)\]
	\end{worksheet}
\end{document}