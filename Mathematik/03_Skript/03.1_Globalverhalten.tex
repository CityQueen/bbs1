\documentclass[11pt,twocolumn,oneside,openany,headings=optiontotoc,11pt,numbers=noenddot]{article}

\usepackage[a4paper]{geometry}
\usepackage[utf8]{inputenc}
\usepackage[T1]{fontenc}
\usepackage{lmodern}
\usepackage[ngerman]{babel}
\usepackage{ngerman}

\usepackage[onehalfspacing]{setspace}

\usepackage{fancyhdr}
\usepackage{fancybox}

\usepackage{rotating}
\usepackage{varwidth}


\usepackage{pdflscape}
\usepackage{graphicx}
\usepackage{graphbox}
\graphicspath{
	{Pics/PDFs/}
	{Pics/JPGs/}
	{Pics/PNGs/}
}
\usepackage{caption}
\usepackage{tabularx}
\usepackage{dashrule}
\usepackage{hhline}
\usepackage{multirow}
\usepackage{enumerate}
\usepackage[hidelinks]{hyperref}
\usepackage{listings}

\usepackage[table]{xcolor}
\usepackage{array}
\usepackage{enumitem,amssymb,amsmath}
\usepackage{interval}
\usepackage{stmaryrd}
\usepackage{polynom}
\usepackage{diagbox}
\usepackage{dashrule}
\usepackage{framed}
\usepackage{mdframed}
\usepackage{karnaugh-map}

\usepackage{blindtext}

\usepackage{eso-pic}

\usepackage{amssymb}
\usepackage{eurosym}
\pagestyle{headings}
\renewcommand{\headrulewidth}{0.2pt}
\renewcommand{\footrulewidth}{0.2pt}
\newcommand*{\underdownarrow}[2]{\ensuremath{\underset{\overset{\Big\downarrow}{#2}}{#1}}}
\setlength{\fboxsep}{5pt}

% Codestyle defined
\definecolor{codegreen}{rgb}{0,0.6,0}
\definecolor{codegray}{rgb}{0.5,0.5,0.5}
\definecolor{codepurple}{rgb}{0.58,0,0.82}
\definecolor{backcolour}{rgb}{0.95,0.95,0.92}
\definecolor{deepgreen}{rgb}{0,0.5,0}
\definecolor{darkblue}{rgb}{0,0,0.65}
\definecolor{mauve}{rgb}{0.40, 0.19,0.28}
\colorlet{exceptioncolour}{yellow!50!red}
\colorlet{commandcolour}{blue!60!black}
\colorlet{numpycolour}{blue!60!green}
\colorlet{specmethodcolour}{violet}

%Neue Spaltendefinition
\newcolumntype{L}[1]{>{\raggedright\let\newline\\\arraybackslash\hspace{0pt}}m{#1}}
\newcolumntype{M}[1]{>{\centering\arraybackslash}X}
\newcommand{\cmnt}[1]{\ignorespaces}
%Textausrichtung ändern
\newcommand\tabrotate[1]{\rotatebox{90}{\raggedright#1\hspace{\tabcolsep}}}

%Intervall-Konfig
\intervalconfig {
	soft open fences
}

%Bash
\lstdefinestyle{BashInputStyle}{
	language=bash,
	basicstyle=\small\sffamily,
	backgroundcolor=\color{backcolour},
	columns=fullflexible,
	backgroundcolor=\color{backcolour},
	breaklines=true,
}
%Java
\lstdefinestyle{JavaInputStyle}{
	language=Java,
	backgroundcolor=\color{backcolour},
	aboveskip=1mm,
	belowskip=1mm,
	showstringspaces=false,
	columns=flexible,
	basicstyle={\footnotesize\ttfamily},
	numberstyle={\tiny},
	numbers=none,
	keywordstyle=\color{purple},,
	commentstyle=\color{deepgreen},
	stringstyle=\color{blue},
	emph={out},
	emphstyle=\color{darkblue},
	emph={[2]rand},
	emphstyle=[2]\color{specmethodcolour},
	breaklines=true,
	breakatwhitespace=true,
	tabsize=2,
}
%Python
\lstdefinestyle{PythonInputStyle}{
	language=Python,
	alsoletter={1234567890},
	aboveskip=1ex,
	basicstyle=\footnotesize,
	breaklines=true,
	breakatwhitespace= true,
	backgroundcolor=\color{backcolour},
	commentstyle=\color{red},
	otherkeywords={\ , \}, \{, \&,\|},
	emph={and,break,class,continue,def,yield,del,elif,else,%
		except,exec,finally,for,from,global,if,import,in,%
		lambda,not,or,pass,print,raise,return,try,while,assert},
	emphstyle=\color{exceptioncolour},
	emph={[2]True,False,None,min},
	emphstyle=[2]\color{specmethodcolour},
	emph={[3]object,type,isinstance,copy,deepcopy,zip,enumerate,reversed,list,len,dict,tuple,xrange,append,execfile,real,imag,reduce,str,repr},
	emphstyle=[3]\color{commandcolour},
	emph={[4]ode, fsolve, sqrt, exp, sin, cos, arccos, pi,  array, norm, solve, dot, arange, , isscalar, max, sum, flatten, shape, reshape, find, any, all, abs, plot, linspace, legend, quad, polyval,polyfit, hstack, concatenate,vstack,column_stack,empty,zeros,ones,rand,vander,grid,pcolor,eig,eigs,eigvals,svd,qr,tan,det,logspace,roll,mean,cumsum,cumprod,diff,vectorize,lstsq,cla,eye,xlabel,ylabel,squeeze},
	emphstyle=[4]\color{numpycolour},
	emph={[5]__init__,__add__,__mul__,__div__,__sub__,__call__,__getitem__,__setitem__,__eq__,__ne__,__nonzero__,__rmul__,__radd__,__repr__,__str__,__get__,__truediv__,__pow__,__name__,__future__,__all__},
	emphstyle=[5]\color{specmethodcolour},
	emph={[6]assert,range,yield},
	emphstyle=[6]\color{specmethodcolour}\bfseries,
	emph={[7]Exception,NameError,IndexError,SyntaxError,TypeError,ValueError,OverflowError,ZeroDivisionError,KeyboardInterrupt},
	emphstyle=[7]\color{specmethodcolour}\bfseries,
	emph={[8]taster,send,sendMail,capture,check,noMsg,go,move,switch,humTem,ventilate,buzz},
	emphstyle=[8]\color{blue},
	keywordstyle=\color{blue}\bfseries,
	rulecolor=\color{black!40},
	showstringspaces=false,
	stringstyle=\color{deepgreen}
}

\lstset{literate=%
	{Ö}{{\"O}}1
	{Ä}{{\"A}}1
	{Ü}{{\"U}}1
	{ß}{{\ss}}1
	{ü}{{\"u}}1
	{ä}{{\"a}}1
	{ö}{{\"o}}1
}

% Neue Klassenarbeits-Umgebung
\newenvironment{worksheet}[3]
% Begin-Bereich
{
	\newpage
	\sffamily
	\setcounter{page}{1}
	\ClearShipoutPicture
	\AddToShipoutPicture{
		\put(55,761){{
				\mbox{\parbox{385\unitlength}{\tiny \color{codegray}BBS I Mainz, #1 \newline #2
						\newline #3
					}
				}
			}
		}
		\put(455,761){{
				\mbox{\hspace{0.3cm}\includegraphics[width=0.2\textwidth]{../../logo.jpg}}
			}
		}
	}
}
% End-Bereich
{
	\clearpage
	\ClearShipoutPicture
}

\setlength{\columnsep}{3em}
\setlength{\columnseprule}{0.5pt}

\geometry{left=1.50cm,right=1.50cm,top=3.00cm,bottom=1.00cm,includeheadfoot}
\pagenumbering{gobble}
\pagestyle{empty}

\begin{document}
	\begin{worksheet}{Höhere Berufsfachschule IT-Systeme}{Grundstufe - Mathematik}{Lernabschnitt 3: Ganzrationale Funktionen - Globalverhalten}
		\setcounter{section}{6}
		\setcounter{subsection}{2}
		\subsection{Globalverhalten von ganzrationalen Funktionen}
		Betrachtet man den Graphen einer ganzrationalen Funktion (GRF) kann man insgesamt vier verschiedene Verhalten erkennen.\\
		Zunächst einmal \underline{kommt} der Graph einer GRF aus dem \textbf{negativen} oder \textbf{positiven} Unendlichen und \underline{hauen} entsprechend ins \textbf{negative} bzw. \textbf{positive} Unendliche ab.\\
		\par\noindent
		\includegraphics[width=0.48\textwidth]{../99_Bilder/03-1_gVerh.png}\\
		\par\noindent
		Möchten wir nun das Verhalten der Funktionswerte für große x-Beträge ausdrücken, nutzen wir folgende Symbolik (an obigem Beispiel):\\
		\par\noindent
		\begin{tabularx}{0.48\textwidth}{X|X}
			\hline
			& \\
			\(f(x) \xrightarrow{x\rightarrow-\infty}\infty\) & \(f(x) \xrightarrow{x\rightarrow\infty}-\infty\)\\
			\\
			die Funktionswerte kommen aus dem positiven Unendlichen & die Funktionswerte hauen ins negative Unendliche ab\\
			\hline
		\end{tabularx}\\
		\par\noindent
		Leider haben wir den Graphen einer Funktion nicht immer gegeben.\\
		Das Verhalten muss sich also auch an der Funktion ablesen lassen. Über dieses Verhalten für große x-Beträge gibt der \underline{charakteristische Summand} \(a_n\cdot{}x^n\) Auskunft.\\
		\par\noindent
		Die Regelmäßigkeit beim Verhalten der Funktionswerte kann in folgender Tabelle festgehalten werden:\\
		\begin{tabular}{|l|L{0.2\textwidth}|L{0.15\textwidth}|}
			\hline
			\diagbox{\(a_{n}\)}{n} & & \\
			\hline
			& \(f(x) \xrightarrow{x \rightarrow -\infty}\) & \(f(x) \xrightarrow{x \rightarrow -\infty}\)\\
			& & \\
			& \(f(x) \xrightarrow{x \rightarrow \infty} \) & \(f(x) \xrightarrow{x \rightarrow \infty} \)\\
			\hline
			& \(f(x) \xrightarrow{x \rightarrow -\infty}\) & \(f(x) \xrightarrow{x \rightarrow -\infty}\)\\
			& & \\
			& \(f(x) \xrightarrow{x \rightarrow \infty} \) & \(f(x) \xrightarrow{x \rightarrow \infty} \)\\
			\hline
		\end{tabular}\\
		\par\noindent
		\textbf{\underline{Übungen}}: Geben Sie das Verhalten der folgenden Funktionen für große x-Beträge an.
		\begin{align*}
			d(x) & = -21x^{3} - 10x + 1 \\
			e(x) & = -10x^{7} + 8x^{5} - 6x^{3} + 1\\
			f(x) & = 0.01x^{4} - 200x^{2} - 1000x\\
			g(x) & = -5x^{3} + 500x^{2} - 30\\
			h(x) & = -70x^{6} + 10x^{3} - 2x\\
			k(x) & = 25x^{4} + 20x^{3} - 14x + 500\\
			l(x) & = 0.5x^{2} - 12x + 200\\
			m(x) & = x^{3} + x^{2} - 4x - 1\\
			n(x) & = -x^{3} - x^{2} + 4x - 1\\
			o(x) & = 0.2x^{4} + 2x^{3} + 5x^{2} + x - 2\\
			p(x) & = -0.2x^{4} - 2x^{3} - 5x^{2} - x + 2\\
			q(x) & = 2x^{2} - 1\\
			r(x) & = -2x^{2} + 1\\
			s(x) & = -3x^{3} - 2x^{2} + 3x - 1\\
			t(x) & = -x^{5} - x^{3} + x\\
			u(x) & = x^{5} + x^{3} - x\\
			v(x) & = 100x^{10} - 50x^{6} + 10x^{2}\\
			w(x) & = 12x^{5} - 2x
		\end{align*}
	\end{worksheet}
\end{document}