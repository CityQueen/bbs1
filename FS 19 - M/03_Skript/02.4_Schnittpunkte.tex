\documentclass[11pt,twocolumn,oneside,openany,headings=optiontotoc,11pt,numbers=noenddot]{article}

\usepackage[a4paper]{geometry}
\usepackage[utf8]{inputenc}
\usepackage[T1]{fontenc}
\usepackage{lmodern}
\usepackage[ngerman]{babel}
\usepackage{ngerman}

\usepackage[onehalfspacing]{setspace}

\usepackage{fancyhdr}
\usepackage{fancybox}

\usepackage{rotating}
\usepackage{varwidth}


\usepackage{pdflscape}
\usepackage{graphicx}
\usepackage{graphbox}
\graphicspath{
	{Pics/PDFs/}
	{Pics/JPGs/}
	{Pics/PNGs/}
}
\usepackage{caption}
\usepackage{tabularx}
\usepackage{dashrule}
\usepackage{hhline}
\usepackage{multirow}
\usepackage{enumerate}
\usepackage[hidelinks]{hyperref}
\usepackage{listings}

\usepackage[table]{xcolor}
\usepackage{array}
\usepackage{enumitem,amssymb,amsmath}
\usepackage{interval}
\usepackage{stmaryrd}
\usepackage{polynom}
\usepackage{diagbox}
\usepackage{dashrule}
\usepackage{framed}
\usepackage{mdframed}
\usepackage{karnaugh-map}

\usepackage{blindtext}

\usepackage{eso-pic}

\usepackage{amssymb}
\usepackage{eurosym}
\pagestyle{headings}
\renewcommand{\headrulewidth}{0.2pt}
\renewcommand{\footrulewidth}{0.2pt}
\newcommand*{\underdownarrow}[2]{\ensuremath{\underset{\overset{\Big\downarrow}{#2}}{#1}}}
\setlength{\fboxsep}{5pt}

% Codestyle defined
\definecolor{codegreen}{rgb}{0,0.6,0}
\definecolor{codegray}{rgb}{0.5,0.5,0.5}
\definecolor{codepurple}{rgb}{0.58,0,0.82}
\definecolor{backcolour}{rgb}{0.95,0.95,0.92}
\definecolor{deepgreen}{rgb}{0,0.5,0}
\definecolor{darkblue}{rgb}{0,0,0.65}
\definecolor{mauve}{rgb}{0.40, 0.19,0.28}
\colorlet{exceptioncolour}{yellow!50!red}
\colorlet{commandcolour}{blue!60!black}
\colorlet{numpycolour}{blue!60!green}
\colorlet{specmethodcolour}{violet}

%Neue Spaltendefinition
\newcolumntype{L}[1]{>{\raggedright\let\newline\\\arraybackslash\hspace{0pt}}m{#1}}
\newcolumntype{M}[1]{>{\centering\arraybackslash}X}
\newcommand{\cmnt}[1]{\ignorespaces}
%Textausrichtung ändern
\newcommand\tabrotate[1]{\rotatebox{90}{\raggedright#1\hspace{\tabcolsep}}}

%Intervall-Konfig
\intervalconfig {
	soft open fences
}

%Bash
\lstdefinestyle{BashInputStyle}{
	language=bash,
	basicstyle=\small\sffamily,
	backgroundcolor=\color{backcolour},
	columns=fullflexible,
	backgroundcolor=\color{backcolour},
	breaklines=true,
}
%Java
\lstdefinestyle{JavaInputStyle}{
	language=Java,
	backgroundcolor=\color{backcolour},
	aboveskip=1mm,
	belowskip=1mm,
	showstringspaces=false,
	columns=flexible,
	basicstyle={\footnotesize\ttfamily},
	numberstyle={\tiny},
	numbers=none,
	keywordstyle=\color{purple},,
	commentstyle=\color{deepgreen},
	stringstyle=\color{blue},
	emph={out},
	emphstyle=\color{darkblue},
	emph={[2]rand},
	emphstyle=[2]\color{specmethodcolour},
	breaklines=true,
	breakatwhitespace=true,
	tabsize=2,
}
%Python
\lstdefinestyle{PythonInputStyle}{
	language=Python,
	alsoletter={1234567890},
	aboveskip=1ex,
	basicstyle=\footnotesize,
	breaklines=true,
	breakatwhitespace= true,
	backgroundcolor=\color{backcolour},
	commentstyle=\color{red},
	otherkeywords={\ , \}, \{, \&,\|},
	emph={and,break,class,continue,def,yield,del,elif,else,%
		except,exec,finally,for,from,global,if,import,in,%
		lambda,not,or,pass,print,raise,return,try,while,assert},
	emphstyle=\color{exceptioncolour},
	emph={[2]True,False,None,min},
	emphstyle=[2]\color{specmethodcolour},
	emph={[3]object,type,isinstance,copy,deepcopy,zip,enumerate,reversed,list,len,dict,tuple,xrange,append,execfile,real,imag,reduce,str,repr},
	emphstyle=[3]\color{commandcolour},
	emph={[4]ode, fsolve, sqrt, exp, sin, cos, arccos, pi,  array, norm, solve, dot, arange, , isscalar, max, sum, flatten, shape, reshape, find, any, all, abs, plot, linspace, legend, quad, polyval,polyfit, hstack, concatenate,vstack,column_stack,empty,zeros,ones,rand,vander,grid,pcolor,eig,eigs,eigvals,svd,qr,tan,det,logspace,roll,mean,cumsum,cumprod,diff,vectorize,lstsq,cla,eye,xlabel,ylabel,squeeze},
	emphstyle=[4]\color{numpycolour},
	emph={[5]__init__,__add__,__mul__,__div__,__sub__,__call__,__getitem__,__setitem__,__eq__,__ne__,__nonzero__,__rmul__,__radd__,__repr__,__str__,__get__,__truediv__,__pow__,__name__,__future__,__all__},
	emphstyle=[5]\color{specmethodcolour},
	emph={[6]assert,range,yield},
	emphstyle=[6]\color{specmethodcolour}\bfseries,
	emph={[7]Exception,NameError,IndexError,SyntaxError,TypeError,ValueError,OverflowError,ZeroDivisionError,KeyboardInterrupt},
	emphstyle=[7]\color{specmethodcolour}\bfseries,
	emph={[8]taster,send,sendMail,capture,check,noMsg,go,move,switch,humTem,ventilate,buzz},
	emphstyle=[8]\color{blue},
	keywordstyle=\color{blue}\bfseries,
	rulecolor=\color{black!40},
	showstringspaces=false,
	stringstyle=\color{deepgreen}
}

\lstset{literate=%
	{Ö}{{\"O}}1
	{Ä}{{\"A}}1
	{Ü}{{\"U}}1
	{ß}{{\ss}}1
	{ü}{{\"u}}1
	{ä}{{\"a}}1
	{ö}{{\"o}}1
}

% Neue Klassenarbeits-Umgebung
\newenvironment{worksheet}[3]
% Begin-Bereich
{
	\newpage
	\sffamily
	\setcounter{page}{1}
	\ClearShipoutPicture
	\AddToShipoutPicture{
		\put(55,761){{
				\mbox{\parbox{385\unitlength}{\tiny \color{codegray}BBS I Mainz, #1 \newline #2
						\newline #3
					}
				}
			}
		}
		\put(455,761){{
				\mbox{\hspace{0.3cm}\includegraphics[width=0.2\textwidth]{../../logo.jpg}}
			}
		}
	}
}
% End-Bereich
{
	\clearpage
	\ClearShipoutPicture
}

\setlength{\columnsep}{3em}
\setlength{\columnseprule}{0.5pt}

\geometry{left=1.50cm,right=1.50cm,top=3.00cm,bottom=1.00cm,includeheadfoot}
\pagenumbering{gobble}
\pagestyle{empty}

\begin{document}
	\begin{worksheet}{Höhere Berufsfachschule IT-Systeme}{Grundstufe - Mathematik}{Lernabschnitt 2: Schnittpunkte bestimmen}
		\setcounter{section}{4}
		\section{Schnittpunkte}
		Betrachtet man den Verlauf von mehreren Funktionsgraphen in einem Koordinatensystem\\
		\par\noindent
		\includegraphics[width=0.48\textwidth]{../99_Bilder/024_NST.png}\\
		können wir Schnittpunkte markieren. In obigem Bild sind die Koordinaten einfach abzulesen. Manchmal ist dies aber nicht so einfach möglich und manchmal haben wir keinen Graphen, sondern lediglich die Funktionsgleichung.\\
		Wir kommen also manchmal um eine genaue Bestimmung/Berechnung der Schnittstellen bzw. Schnittpunkte nicht drum herum.\\
		\par\noindent
		\textbf{Und wie?}\\
		Zunächst müssen wir die Schnittstelle zweier Funktionen bestimmen.\\
		Wie schon bei \textit{Schnittstellen von linearen Funktionen}
		\begin{itemize}
			\item setzen wir die zwei Funktionen gleich
			\item bringen alle Terme auf eine Seite
			\item bestimmen dann \(x\)
		\end{itemize}
		Alle Terme auf einer Seite heißt, dass wir im Prinzip \(0 = \text{Terme}\) lösen.\\
		\par\noindent
		Das Problem der Schnittstellenbestimmung reduziert sich also auf das Finden von Nullstellen.\\
		\par\noindent
		Haben wir die Schnittstelle \(x_{ST}\), so bestimmen wir den Funktionswert von \(x_{ST}\). Dies liefert und die y-Koordinate und wir können den Schnittpunkt (\(SP(x_{ST}|f(x_{ST}))\)) angeben.
		\par\noindent
		\underline{\textbf{Beispiel:}}\\
		Wir haben die Funktionen \(f(x) = x^2 - 4\) und \(g(x) = x+2\).\\
		\begin{align*}
			f(x) & = g(x)\\
			x^2 - 4 & = x + 2 & | -2\\
			x^2 - 6 & = x & | -x\\
			x^2 -x -6 = 0
		\end{align*}
		Wir sehen, dass es sich um keine binomische Formel handelt, also benötigen wir die \underline{pq-Formel}.
		\begin{align*}
			p & = -1; & q = - 6\\
			x_{1/2} & = -\frac{-1}{2} \pm \sqrt{\left(\frac{-1}{2}\right)^2 - (-6)}\\
			x_{1/2} & = \frac{1}{2} \pm \sqrt{\left(\frac{1}{4}\right) + 6}\\
			x_1 & = \frac{1}{2} + \sqrt{\left(\frac{1}{4}\right) + 6} & \Rightarrow x_1 = 3\\
			x_2 & = \frac{1}{2} \pm \sqrt{\left(\frac{1}{4}\right) + 6} & \Rightarrow x_2 = -2
		\end{align*}
		Unsere Schnittstellen sind also \colorbox{green!10}{\(x_1 = 3\)} und \colorbox{green!10}{\(x_2 = -2\)}. Wir bestimmen den zugehörige Funktionswert.\\
		\begin{align*}
			f(x_1) & = f(3)  & = 3^2 - 4  & = 5\\
			& \Rightarrow & SP_1(3|5) & \\
			f(x_2) & = f(-2) & = (-2)^2 - 4 & = 0\\
			& \Rightarrow & SP_2(-2|0) &
		\end{align*}
	\end{worksheet}
\end{document}