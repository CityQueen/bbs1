\documentclass[11pt,twocolumn,oneside,openany,headings=optiontotoc,11pt,numbers=noenddot]{article}

\usepackage[a4paper]{geometry}
\usepackage[utf8]{inputenc}
\usepackage[T1]{fontenc}
\usepackage{lmodern}
\usepackage[ngerman]{babel}
\usepackage{ngerman}

\usepackage[onehalfspacing]{setspace}

\usepackage{fancyhdr}
\usepackage{fancybox}

\usepackage{rotating}
\usepackage{varwidth}


\usepackage{pdflscape}
\usepackage{graphicx}
\usepackage{graphbox}
\graphicspath{
	{Pics/PDFs/}
	{Pics/JPGs/}
	{Pics/PNGs/}
}
\usepackage{caption}
\usepackage{tabularx}
\usepackage{dashrule}
\usepackage{hhline}
\usepackage{multirow}
\usepackage{enumerate}
\usepackage[hidelinks]{hyperref}
\usepackage{listings}

\usepackage[table]{xcolor}
\usepackage{array}
\usepackage{enumitem,amssymb,amsmath}
\usepackage{interval}
\usepackage{stmaryrd}
\usepackage{polynom}
\usepackage{diagbox}
\usepackage{dashrule}
\usepackage{framed}
\usepackage{mdframed}
\usepackage{karnaugh-map}

\usepackage{blindtext}

\usepackage{eso-pic}

\usepackage{amssymb}
\usepackage{eurosym}
\pagestyle{headings}
\renewcommand{\headrulewidth}{0.2pt}
\renewcommand{\footrulewidth}{0.2pt}
\newcommand*{\underdownarrow}[2]{\ensuremath{\underset{\overset{\Big\downarrow}{#2}}{#1}}}
\setlength{\fboxsep}{5pt}

% Codestyle defined
\definecolor{codegreen}{rgb}{0,0.6,0}
\definecolor{codegray}{rgb}{0.5,0.5,0.5}
\definecolor{codepurple}{rgb}{0.58,0,0.82}
\definecolor{backcolour}{rgb}{0.95,0.95,0.92}
\definecolor{deepgreen}{rgb}{0,0.5,0}
\definecolor{darkblue}{rgb}{0,0,0.65}
\definecolor{mauve}{rgb}{0.40, 0.19,0.28}
\colorlet{exceptioncolour}{yellow!50!red}
\colorlet{commandcolour}{blue!60!black}
\colorlet{numpycolour}{blue!60!green}
\colorlet{specmethodcolour}{violet}

%Neue Spaltendefinition
\newcolumntype{L}[1]{>{\raggedright\let\newline\\\arraybackslash\hspace{0pt}}m{#1}}
\newcolumntype{M}[1]{>{\centering\arraybackslash}X}
\newcommand{\cmnt}[1]{\ignorespaces}
%Textausrichtung ändern
\newcommand\tabrotate[1]{\rotatebox{90}{\raggedright#1\hspace{\tabcolsep}}}

%Intervall-Konfig
\intervalconfig {
	soft open fences
}

%Bash
\lstdefinestyle{BashInputStyle}{
	language=bash,
	basicstyle=\small\sffamily,
	backgroundcolor=\color{backcolour},
	columns=fullflexible,
	backgroundcolor=\color{backcolour},
	breaklines=true,
}
%Java
\lstdefinestyle{JavaInputStyle}{
	language=Java,
	backgroundcolor=\color{backcolour},
	aboveskip=1mm,
	belowskip=1mm,
	showstringspaces=false,
	columns=flexible,
	basicstyle={\footnotesize\ttfamily},
	numberstyle={\tiny},
	numbers=none,
	keywordstyle=\color{purple},,
	commentstyle=\color{deepgreen},
	stringstyle=\color{blue},
	emph={out},
	emphstyle=\color{darkblue},
	emph={[2]rand},
	emphstyle=[2]\color{specmethodcolour},
	breaklines=true,
	breakatwhitespace=true,
	tabsize=2,
}
%Python
\lstdefinestyle{PythonInputStyle}{
	language=Python,
	alsoletter={1234567890},
	aboveskip=1ex,
	basicstyle=\footnotesize,
	breaklines=true,
	breakatwhitespace= true,
	backgroundcolor=\color{backcolour},
	commentstyle=\color{red},
	otherkeywords={\ , \}, \{, \&,\|},
	emph={and,break,class,continue,def,yield,del,elif,else,%
		except,exec,finally,for,from,global,if,import,in,%
		lambda,not,or,pass,print,raise,return,try,while,assert},
	emphstyle=\color{exceptioncolour},
	emph={[2]True,False,None,min},
	emphstyle=[2]\color{specmethodcolour},
	emph={[3]object,type,isinstance,copy,deepcopy,zip,enumerate,reversed,list,len,dict,tuple,xrange,append,execfile,real,imag,reduce,str,repr},
	emphstyle=[3]\color{commandcolour},
	emph={[4]ode, fsolve, sqrt, exp, sin, cos, arccos, pi,  array, norm, solve, dot, arange, , isscalar, max, sum, flatten, shape, reshape, find, any, all, abs, plot, linspace, legend, quad, polyval,polyfit, hstack, concatenate,vstack,column_stack,empty,zeros,ones,rand,vander,grid,pcolor,eig,eigs,eigvals,svd,qr,tan,det,logspace,roll,mean,cumsum,cumprod,diff,vectorize,lstsq,cla,eye,xlabel,ylabel,squeeze},
	emphstyle=[4]\color{numpycolour},
	emph={[5]__init__,__add__,__mul__,__div__,__sub__,__call__,__getitem__,__setitem__,__eq__,__ne__,__nonzero__,__rmul__,__radd__,__repr__,__str__,__get__,__truediv__,__pow__,__name__,__future__,__all__},
	emphstyle=[5]\color{specmethodcolour},
	emph={[6]assert,range,yield},
	emphstyle=[6]\color{specmethodcolour}\bfseries,
	emph={[7]Exception,NameError,IndexError,SyntaxError,TypeError,ValueError,OverflowError,ZeroDivisionError,KeyboardInterrupt},
	emphstyle=[7]\color{specmethodcolour}\bfseries,
	emph={[8]taster,send,sendMail,capture,check,noMsg,go,move,switch,humTem,ventilate,buzz},
	emphstyle=[8]\color{blue},
	keywordstyle=\color{blue}\bfseries,
	rulecolor=\color{black!40},
	showstringspaces=false,
	stringstyle=\color{deepgreen}
}

\lstset{literate=%
	{Ö}{{\"O}}1
	{Ä}{{\"A}}1
	{Ü}{{\"U}}1
	{ß}{{\ss}}1
	{ü}{{\"u}}1
	{ä}{{\"a}}1
	{ö}{{\"o}}1
}

% Neue Klassenarbeits-Umgebung
\newenvironment{worksheet}[3]
% Begin-Bereich
{
	\newpage
	\sffamily
	\setcounter{page}{1}
	\ClearShipoutPicture
	\AddToShipoutPicture{
		\put(55,761){{
				\mbox{\parbox{385\unitlength}{\tiny \color{codegray}BBS I Mainz, #1 \newline #2
						\newline #3
					}
				}
			}
		}
		\put(455,761){{
				\mbox{\hspace{0.3cm}\includegraphics[width=0.2\textwidth]{../../logo.jpg}}
			}
		}
	}
}
% End-Bereich
{
	\clearpage
	\ClearShipoutPicture
}

\setlength{\columnsep}{3em}
\setlength{\columnseprule}{0.5pt}

\geometry{left=2.50cm,right=2.50cm,top=3.00cm,bottom=1.00cm,includeheadfoot}
\pagenumbering{arabic}
\pagestyle{plain}

\begin{document}
	\begin{worksheet}{Höhere Berufsfachschule IT-Systeme}{Grundstufe - Mathematik}{Grundlagen}
		\begin{framed}
			Datum:
		\end{framed}
		\section{Allgemeine Begrifflichkeiten und Grundlagen}
		\subsection{Zahlen und Zahlenmengen}
		Wir kennen die unterschiedlichsten Zahlen. Zu Beginn lernen wir Zahlen wie  \(0,1,2,3,\ldots\). Diese Zahlen bilden die \textbf{Menge der natürlichen Zahlen} \(\mathbb{N}\). Manchmal ist es nötig, dass die \(0\) aus dieser Menge ausgeschlossen wird. Dann schreibt man \(\mathbb{N}\backslash\{0\}\).\\
		\par\noindent
		Häufig wird man als nächstes mit den negativen Zahlen konfrontiert. Also \(-1,-2,-3,\ldots\). Nehmen wir also die \textit{negativen Zahlen} zu den bereits bekannten \textit{natürlichen Zahlen} hinzu, erhalten wir die \textbf{Menge der ganzen Zahlen} \(\mathbb{G}\). Im Allgemeinen wird diese Menge auch mit dem Buchstaben \(\mathbb{Z}\) beschrieben.\\
		\par\noindent
		Jene Zahlen, die man als Bruch darstellen kann, werden ebenfalls zu einer Menge zusammengefasst. Diese \textbf{Menge der rationalen Zahlen} \(\mathbb{Q}\).\\
		Zu ihr zählen auch alle endlichen oder periodischen Dezimalzahlen.\\
		\par\noindent
		Wenn es Zahlen gibt, die als Bruch darstellbar sind, dann existieren logischerweise auch Zahlen, die \underline{nicht} als Bruch darstellbar sind. Dazu zählen \(\sqrt{2}, -\sqrt{3}, \pi \). Diese Zahlen bezeichnet man als \textbf{irrationale Zahlen}. Wir merken uns, die Zahlen der Zahlengerade, die \underline{nicht rational} sind, bezeichnet man als \underline{irrational}.\\
		\par\noindent
		Nehmen wir die \textit{Menge der rationalen Zahlen} und die \textit{Menge der irrationalen Zahlen} zusammen, erhält man praktisch alle Zahlen auf der Zahlengerade. Diese bezeichnet man als \textbf{Menge der reellen Zahlen} \(\mathbb{R}\).\\
		\par\noindent
		Wie bereits erwähnt können wir uns merken, dass je eine Menge ein Teil der größeren Menge ist. Also gilt
		\[\mathbb{N} \subset \mathbb{G} \subset \mathbb{Q} \subset \mathbb{R}\]
		In bestimmten Fällen ist man gezwungen nur eine bestimmte Teilmenge von \(\mathbb{R}\) anzugeben. Diese Teilmenge bezeichnet man auch als \textbf{Intervall}.\\
		Bei Intervallen wird unterschieden zwischen dem \textbf{abgeschlossenen Intervall} \(\left[a;b\right]\), welches \underline{sowohl a wie auch b} enthält.\\
		Dem \textbf{halboffenen Intervall} \(\left]a;b\right]\). Dieses Intervall enthält \underline{nicht a, aber b}. Das Intervall \(\left[a;b\right[\) hingegen enthält \underline{a, aber nicht b}.\\
		Zu guter letzt noch das \textbf{offene Intervall} \(\left]a;b\right[\). In diesem Intervall sind \underline{weder a noch b} enthalten.
		\subsection{Terme}
		\subsubsection*{Was sind Terme?} Unter einem Term versteht man ein \mbox{\textbf{mathematisches Gebilde}}. Dieses besteht aus Zahlen und Rechenzeichen.\\
		Eine Zahl kann hier aber auch durch einen Buchstaben angegeben werden.\\
		\par
		\begin{tabularx}{0.5\textwidth}{ll}
			\noindent
			\textbf{Beispiele:} & \(3\)\\
			& \(1,70+2,30\)\\
			& \(2,30\cdot x\)\\
			& \(5x + 20\)\\
			& \(5^2\)\\
			& \(\sqrt{5}\)\\
		\end{tabularx}
		\subsection{Termstrukturen}
		\paragraph{Warum sind sie wichtig?}
		Durch die Verwendung von Rechenzeichen innerhalb von Termen erhalten diese eine gewisse Struktur. In diversen Standardsituationen kann es sehr hilfreich sein, wenn man diese Strukturen erkennt.
		\begin{framed}
			\noindent
			\textbf{Beispiele:}\\
			\underline{Situation 1:} Berechnen von Funktionswerten:
			\[f(-3) = -2\cdot (-3)^2 + 4\]
			\textit{Welche Reihenfolge der Rechenoperationen muss beachtet werden? Tastenfolge im Taschenrechner?}\\
			\par\noindent
			\underline{Situation 2:} Lösen einer Gleichung mithilfe von Äquivalenzumformungen (z.B. zur Nullstellenbestimmung)
			\[-2x^2 +4 = 0\]
			\textit{Rechnet man zuerst }|\textit{\(:(-2)\), }|\textit{\(-4\) oder }|\textit{\glqq{}Wurzel\grqq{}?}
		\end{framed}
		\subsubsection*{Woraus bestehen diese?}
		Die Terme, mit denen Sie im Laufe der Zeit konfrontiert werden bestehen aus Summen, Produkten oder Potenzen.\\
		\par\noindent
		\textbf{\underline{Summen}}\\
		Eine \underline{Summe} besteht aus mindestens \textbf{zwei Summanden}, die durch ein \underline{\(+\)-Zeichen} verbunden sind.
		\begin{framed}\noindent
			\underline{Kurz:} Summand + Summand = Summe
		\end{framed}
		\noindent
		Führt man die Operation (Addition) aus, so erhält man den \underline{Summenwert}.\\
		\par\noindent
		\underline{Hinweis:} Sind zwei zahlen durch ein \underline{\glqq{}\(-\)\grqq{}-Zeichen} verbunden, spricht man von der \underline{Differenz}. Dieses Gebilde kann aber auch als Summe bezeichnet werden.\\
		\(7-4\) bedeutet nämlich eigentlich nichts anderes als \(7 + (-4)\). Es ist also eine Summe aus den Summanden \(7\) und \(-4\).\\
		\par\noindent
		\textbf{\underline{Produkte}}\\
		Ein \underline{Produkt} besteht aus mindestens \textbf{zwei Faktoren}. Diese werden durch ein \underline{\glqq{}\(\cdot\)\grqq{}-Zeichen} miteinander verbunden.
		\begin{framed}
			\noindent
			\underline{Kurz:} Faktor $\cdot$ Faktor = Produkt
		\end{framed}
		\noindent
		Führt man die Operation (Multiplikation) aus, erhält man den sogenannten \underline{Produktwert}.\\
		\par\noindent
		\underline{Hinweis:} Der Mal-Punkt wird häufig weggelassen. Also \(5\cdot x\) wird auch als \(5x\) geschrieben.\\
		\par\noindent
		\textbf{\underline{Potenzen}}\\
		Eine \underline{Potenz} besteht immer aus einer \textbf{Basis} und einem \textbf{Exponenten}. Dabei gibt der Exponent an, wie häufig die Basis mit sich selbst multipliziert wird.\\
		Führt man diese Rechnung aus, ergibt das den \underline{Potenzwert}.
		\begin{framed}
			\noindent
			\underline{Kurz:} \(Basis^{\text{Exponent}}\) = Potenz
		\end{framed}
		\noindent
		\underline{\textbf{Wurzel}}\\
		Die \underline{Wurzel} einer Zahl \(a\) bezeichnet die Zahl, die mit sich selbst multipliziert, den Wert \(a\) ergibt. In der Regel schreibt man \(\sqrt{a}\).\\
		Die Zahl unterhalb der Wurzel nennt man auch \textbf{Radikand}.\\
		\par\noindent
		\underline{\textbf{Absolutbetrag}}\\
		Der \underline{Betrag} einer Zahl gibt ihren \grqq{}Abstand\grqq{} zur Null an. Daher ist dieser auch \textbf{nie}negativ.
		\begin{align*}
			|4| = 4\\
			|-4| = 4
		\end{align*}
		Man liest dann beispielsweise \glqq{}Der Betrag von \(-4\) gleich \(4\)\grqq{}.
		\subsubsection{Wichtige Verknüpfungsregel}
		Es ist häufig der Fall, dass Potenzen, Produkte und Summern miteinander verknüpft werden. Ist dies der Fall, zerren unterschiedliche Rechenzeichen an einer Zahl herum.\\
		Bei der Berechnung des Werts eines Terms gilt die folgende Hierarchie:
		\begin{framed}
			\centering
			\color{red}Po\normalcolor{}tenzrechnung\\
			vor\\
			\color{red}Pu\normalcolor{}nktrechnung\\
			vor\\
			\color{red}Stri\normalcolor{}chrechnung!\\
			\normalcolor
			\par\noindent
			\raggedright
			Sind auch \color{red}Kla\normalcolor{}mmern beteiligt, so haben diese die größte Macht und binden am stärksten.
		\end{framed}
		\noindent
		\centering
		\fcolorbox{red!15}{red!5}{\textbf{\color{red}KlaPoPuStri}\normalcolor{} bewahrt vor Fehlern!}\\
		\raggedright
		\par\noindent
		\textbf{Beispiel 1:} \(4\cdot 2 + 5\)\\
		Wir analysieren die Termstruktur:
		\begin{tabularx}{0.5\textwidth}{c}
			\(4\ \boxed{\cdot}\ 2\ +\ 5\)\\
			\multicolumn{1}{l}{Zuerst \color{red}Pu\normalcolor{}nktrechnung:}\\
			\(=\ 8\ \boxed{+}\ 5\)\\
			\multicolumn{1}{l}{Dann \color{red}Stri\normalcolor{}chrechnung:}\\
			= 13
		\end{tabularx}
		\par\noindent
		\textbf{Beispiel 2:} \(4\cdot (2 + 5)\)\\
		Wir analysieren die Termstruktur:
		\begin{tabularx}{0.5\textwidth}{c}
			\(4\ \cdot\ \boxed{(}\ 2\ +\ 5\ \boxed{)}\)\\
			\multicolumn{1}{l}{Zuerst \color{red}Kla\normalcolor{}mmerausdruck auflösen:}\\
			\(=\ 4\ \boxed{\cdot}\ 7\)\\
			\multicolumn{1}{l}{Dann \color{red}Pu\normalcolor{}nktrechnung:}\\
			= 28
		\end{tabularx}
		\par\noindent
		\textbf{Beispiel 3:} \(-2\cdot{}(2+4)^2 + 7\)\\
		Wir analysieren die Termstruktur:
		\begin{tabularx}{0.5\textwidth}{c}
			\(-2\ \cdot{}\ \boxed{(}\ 2\ +\ 4\ \boxed{)}^2 +\ 7\)\\
			\multicolumn{1}{l}{Zuerst \color{red}Kla\normalcolor{}mmerausdruck auflösen:}\\
			\(=\ -2\ \cdot{}\ \boxed{6^2} +\ 7\)\\
			\multicolumn{1}{l}{Danach \color{red}Po\normalcolor{}tenz bestimmen:}\\
			\(=\ -2\ \boxed{\cdot{}}\ 36\ +\ 7\)\\
			\multicolumn{1}{l}{Anschließend \color{red}Kla\normalcolor{}mmerausdruck auflösen:}\\
			\(=\ -72\ \boxed{+}\ 7\)\\
			\multicolumn{1}{l}{Zuletzt \color{red}Stri\normalcolor{}chrechnung:}\\
			\(=\ -655\)\\
		\end{tabularx}
		\newpage
		\subsection{Rechenregeln}
		Ihnen ist sicher schon in den Sinn gekommen, dass Terme gegebenenfalls vereinfacht bzw. umgeformt werden können, um das Rechnen damit zu vereinfachen.
		\subsubsection*{Vorzeichenregel (VZ)}
		Für die Subtraktion und Addition negativer Zahlen gilt:
		\(-(-a) = +a\)\\
		\(+(-a) = -a\)\\
		\underline{Beispiel:} \(2 + (-3) = -1\)\\
		\par\noindent
		Für die Multiplikation von negativen Zahlen gilt:
		\((-)\cdot(-) = +\)\\
		\((+)\cdot(-) = +\)\\
		\((-)\cdot(+) = -\)\\
		\underline{Beispiel:} \((-2)\cdot(-4) = 8\)
		\subsubsection*{Klammerregeln - Ausmultiplizieren (AM)}
		Wird eine Summe als Klammerausdruck mit einer Zahl multipliziert, so muss jeder Summand aus der Klammer mit dieser Zahl multipliziert werden.\\
		\underline{Beispiel:} \(4\cdot(3+x) = 4\cdot{}3 + 4\cdot{}x = 12 + 4x\)
		\subsubsection*{Faktorisieren bzw. Ausklammern (FAK)}
		Bei einer Summe kann es durchaus passieren, dass alle Summanden einer Summe denselben Faktor enthalten. Diesen Faktor kann man dann ausklammern. Er wird also vor geschrieben. In die Klammer werden die Summe mit den verbleibenden Faktoren geschrieben.\\
		\par\noindent
		\underline{Beispiel:} Zwei Summanden: \(3x - 6\)\\
		Zunächst erkennen wir den gemeinsamen Faktor \((3)\).\\
		\(\mbox{3}x - \mbox{3}\cdot{}2\)\\
		Der Faktor wird zuerst aufgeschrieben, die übriggebliebenen Faktoren werden in der Klammer beibehalten.\\
		\(= \mbox{3}\cdot{}(x - 2)\)\\
		\par\noindent
		Es ergibt sich ein Produkt aus zwei Faktoren. Daher nennt man diese Operation auch \textbf{Faktorisieren}.\\
		\par\noindent
		Vielleicht haben Sie es bereits erkannt. \underline{Ausklammern ist die Umkehrung des Ausmul-} \underline{tiplizierens.}
		\begin{align*}
			\xrightarrow{\ Ausmultiplizieren}\\
			3(x-2) = 3x - 6\\
			\xleftarrow[\ \ \ Faktorisieren\ \ \ ]{}
		\end{align*}
		\subsubsection*{Minus vor der Klammer (MK)}
		Steht vor einem Klammerausdruck ein Minus (-), ist das nichts anderes als die Multiplikation der Klammer mit dem Wert \((-1)\).\\
		\underline{Beispiel:} \(-(2x + 2) = (-1)\cdot{}(2x+2) = -2x - 2\)\\
		Innerhalb einer Rechnung entspricht das dann:
		\begin{align*}
			& \ \ \ \ (2+a) - (3 - a)\\
			& = 2 + a + (-1)\cdot{}(3-a)\\
			& = 2 + a + (-1)\cdot{}3 + (-1)\cdot{}(-1)\\
			& = 2 + a - 3 + a\\
			& = -1 + 2a
		\end{align*}
		\subsubsection*{Binomische Formeln (BF)}
		Wird eine Summe/Differenz mit zwei Summanden bzw. Minuend und Subtrahend mit sich selbst multipliziert, kann man die binomische Formel anwenden:\\
		\begin{align*}
			\text{1. Binomische Formel}\\
			\mathbf{(a+b)^2} = (a + b)\cdot{}(a + b) & = \mathbf{a^2 + 2ab + b^2}\\
			\text{2. Binomische Formel}\\
			\mathbf{(a-b)^2} = (a - b)\cdot{}(a - b) & = \mathbf{a^2 - 2ab + b^2}\\
			\text{3. Binomische Formel}\\
			\mathbf{(a + b)\cdot{}(a - b)} & = \mathbf{a^2 - b^2}
		\end{align*}
		\underline{Beispiel:}
		\begin{align*}
			(x + 2)^2 & = x^2 + 2\cdot{}2\cdot{}x + 2^2\\
			& = x^2 + 4x + 4
		\end{align*}
		\subsubsection*{Zusammenfassen (ZUS)}
		Innerhalb einer Summe kann man Summanden zu einem Term zusammenfassen, sofern diese \glqq{}gleichartig\grqq{} sind.\\
		\underline{Beispiel:}\\
		\fcolorbox{red}{white}{\(x^2\)} \fcolorbox{blue}{white}{\(-\ 4x\)} \fcolorbox{green}{white}{\(+\ 3\)} \fcolorbox{blue}{white}{\(+\ 2x\)} \fcolorbox{green}{white}{\(-\ 4\)} \fcolorbox{red}{white}{\(+\ 2x^2\)}\\
		\par\noindent
		lässt sich wie folgt vereinfachen:\\
		\fcolorbox{white}{red!15}{\(3x^2\)} \fcolorbox{white}{blue!15}{\(-2x\)} \fcolorbox{white}{green!15}{\(-1\)}
		\subsubsection*{Potenzgesetze (PG)}
		Werden \underline{Potenzen} multipliziert oder dividiert, können auch diese \underline{zusammengefasst} werden. Dies ist aber nur dann möglich, wenn die \underline{Basis} oder der \underline{Exponent} gleich sind.\\
		Es gelten die folgenden Regeln:
		\begin{align*}
			a^m\cdot{}a^n & = a ^{m+n}\\
			\frac{a^m}{a^n} & = a^{m-n}\\
			a^m\cdot{}b^m & = (a\cdot{}b)^m\\
			\frac{a^m}{b^m} & = \left(\frac{a}{b}\right)^m\\
			(a^m)^n & = a^{m\cdot{}n}
		\end{align*}
		\newpage
		\underline{\textbf{Auch beim vereinfachen gilt} die Reihenfol-} \underline{ge von \color{red}KlaPoPuStri\normalcolor}. Also zuerst innerhalb der Klammer vereinfachen, dann die Potenz, im Anschluss die Punktrechnung und abschließend die Strichrechnung!\\
		\par\bigskip\noindent
		\underline{\textbf{Beispiel:}} Versuchen sie den folgenden Term mit Hilfe der genannten Rechenregeln zu vereinfachen: \(2\cdot{}(x+2)^2 + 4x + 2\)
		\subsubsection*{Wurzelgesetze (WG)}
		Bei der Multiplikation oder Division von \underline{Wurzelausdrücken} kann \underline{zusammengefasst} und \underline{vereinfacht} werden. Dabei gelten, wie bei den Potenzen, verschiedene Regeln, abhängig davon, ob der gleiche \underline{Wurzelexponent} oder der gleiche \underline{Radikant} gegeben ist.
		\begin{align*}
			\sqrt[n]{a}\cdot\sqrt[n]{b} & = \sqrt[n]{a\cdot{}b}\\
			\sqrt[n]{a}:\sqrt[n]{b} & = \sqrt[n]{a:b}\\
			(\sqrt[n]{a})^m & = \sqrt[n]{a^m}\\
			\sqrt[m]{\sqrt[n]{a}} & = \sqrt[m\cdot{}n]{a}\\
			\sqrt[n\cdot{}k]{a^{m\cdot{}k}} & = \sqrt[n]{a^m}\\
			\sqrt[n]{a} & = a^{\frac{1}{n}}\ a\geq 0, n\in\mathbb{N}\backslash\{0\}
		\end{align*}
		\subsection{Brüche und Bruchterme}
		Als Bruch bezeichnen wir die Darstellung \(\frac{a}{b}\), wobei man \(a\) die Bezeichnung Zähler und \(b\) den Namen Nenner trägt.\\
		Vertauschen wir \(a\) und \(b\), kehren also die Positionen um, so erhalten wir den \textbf{Kehrwert} \(\frac{b}{a}\).\\
		\subsubsection*{Addition und Subtraktion von Brüchen}
		Bei der Addition bzw. Subtraktion von Brüchen muss man darauf achten, dass die beteiligten Brüche \textbf{gleichnamig} sind. Also den \underline{gleichen Nenner} haben.\\
		Ist das schon so, addieren wir einfach die Zähler und behalten den Nenner bei. \[\frac{1}{3} + \frac{4}{3} = \frac{1+4}{3}\]
		\par\noindent
		Sind die Nenner hingegen \textit{ungleichnamig}, müssen wir diese erst \textit{gleichnamig} machen, bevor wir weiterrechnen dürfen.\\
		Hierfür wählen wir eine Zahl aus, die sowohl durch den einen wie auch durch den anderen Nenner darstellbar ist. Nun \textbf{erweitern} wir \textit{Zähler} und \textit{Nenner} so, dass die ausgewählte Zahl im Nenner steht.\\
		Beim \textit{erweitern} werden \underline{Zähler und Nenner} mit der gleichen Zahl multipliziert.\\
		\par\noindent
		Schauen wir uns ein Beispiel an:
		\[\frac{1}{2} + \frac{4}{3} = \underbrace{\frac{3}{6}}_{\frac{1}{2}\cdot{}\frac{3}{3}} + \underbrace{\frac{8}{6}}_{\frac{4}{3}\cdot{}\frac{2}{2}} = \frac{11}{6}\]
		Um unnötig große Nenner zu erhalten, wählt man in der Regel die kleinstmögliche Zahl. Diese wird auch als \textbf{kleinstes gemeinsames Vielfaches} (kgV) bezeichnet. Der kleinste aller gemeinsamen Nenner nennt man \textbf{Hauptnenner}.
		\subsubsection*{Multiplikation von Brüchen}
		Die Multiplikation zweier Brüche hingegen ist so simpel, dass es fast keiner Erklärung bedarf. Dabei werden jeweils die beiden Zähler bzw. die Nenner multipliziert.
		\[\frac{3}{5}\cdot{}\frac{2}{7} = \frac{3\cdot{}2}{5\cdot{}7} = \frac{6}{35}\]
		Multiplizieren wir aber eine ganze Zahl mit einem Bruch, so wird diese ganze Zahl lediglich mit dem Zähler multipliziert.
		\[\frac{2}{7}\cdot{}3 = \frac{2\cdot{}3}{7}\]
		\color{red!40}{\textit{Vorsicht}:}\normalcolor{} Ist der Zähler eine Summe bzw. Differenz, muss bei der Multiplikation mit einem anderen Bruch oder einer Zahl Klammern gesetzt werden.
		\begin{align*}
			\frac{3a + 5}{9} + \frac{6}{x+y} & = \frac{(3a+5)\cdot{}6}{9\cdot{}(x+y)}\\
			= \frac{(3a+5)\cdot{}\cancelto{2}{6}}{\cancelto{3}{9}\cdot{}(x+y)} & = \frac{6a+10}{3x+3y}
		\end{align*}
		\subsubsection*{Division von Brüchen}
		Zuletzt können wir auch durch einen Bruch dividieren.\\
		Um dies zu tun, multiplizieren wir den ersten Faktor einfach mit dem \underline{Kehrwert} des Bruchs.
		\begin{align*}
			\frac{3}{5}:\frac{2}{7} & = \frac{3}{5}\cdot{}\frac{7}{2} = \frac{21}{10}\\
			\\
			5:\frac{4}{3} & = 5\cdot{}\frac{3}{4} = \frac{15}{4}
		\end{align*}
		Diese Rechenregel verwenden wir auch bei Doppelbrüchen.
		\[\frac{\frac{2}{3}}{\frac{5}{6}} = \frac{2}{3}\cdot{}\frac{6}{5} = \frac{12}{15} = \frac{4}{5}\]
		Wie in der letzten Rechnung erkennbar, sollten Ergebnisse immer in gekürzter Form angegeben werden.\\
		\par\noindent
		\color{red!40}{\textit{Vorsicht}:}\normalcolor{} Gemischten Brüche dürfen nicht verwechselt werden mit der Multiplikation einer Zahl und einem Bruch!
		\begin{align*}
			2\frac{3}{4} = 2 + \frac{3}{4} & = \frac{8}{4} +\frac{3}{4} = \frac{11}{4}\\
			& \neq\\
			\frac{6}{4} & = 2\cdot{}\frac{3}{4}
		\end{align*}
		%ToDo: Zu jedem Bereich Aufgaben
		\subsection{Ihre Aufgaben}
		\subsubsection*{(Zahlen und Zahlenmengen)}
		(1) Zu welchen Zahlenmengen gehören die folgenden Zahlen? Geben sie \underline{alle} Möglichkeiten an.\\
		\begin{tabularx}{0.48\textwidth}{XX}
			(a) -5 & (e) 0,2\\
			(b) $\pi$ & (f) $-\frac{6}{3}$\\
			(c) 1,23456789 & (g) $\sqrt{5}$\\
			(d) 69 & (h) $\frac{3}{7}$
		\end{tabularx}\\
		\par\bigskip\noindent
		(2) Schreiben Sie als Intervall\\
		\begin{tabularx}{0.48\textwidth}{X}
			(a) \(I = \{x|-7\leq x\leq 2\}\)\\
			(b) \(I = \{x|-6< x\leq 2\}\)\\
			(c) \(I = \{x|-3\leq x< 2\}\)\\
			(d) \(I = \{x|-8< x< 2\}\)
		\end{tabularx}
		\subsubsection*{(Terme)}
		(1) Handelt es sich bei folgenden Ausdrücken um Terme?\\
		\begin{tabularx}{0.5\textwidth}{XX}
			(a) \(2a+b\) & (d) \(16\left(8x-22\right.\)\\
			(b) \(8 + \cdot{} 2\) & (e) \(0 < x| <5\)\\
			(c) \(1,5x\) & (f) \(1<2\)
		\end{tabularx}\\
		\par\bigskip\noindent
		(2) Berechnen Sie \underline{\textbf{ohne}} Taschenrechner!\\
		\begin{tabularx}{0.5\textwidth}{X}
			(a) \(\left[-12 + |-6| -2\cdot{}(16-3)\right]:2 + 1\)\\
			(b) \(5\cdot{}10 -20\cdot\left[25+(30-5)-50\right]-5\cdot{}3\)\\
			(c) \(2-3\cdot{}(72+21+8)\cdot{}x\)
		\end{tabularx}
		\subsubsection*{(Rechenregeln)}
		(1) Lösen Sie die Klammern auf. Fassen Sie so weit wie möglich zusammen.\\
		\begin{tabularx}{0.5\textwidth}{X}
			(a) \(5-(-3x+6)-7x\)\\
			(b) \(-(3x+4a)-(-3x+40a)\)\\
			(c) \(-(2x-1)\cdot{}19 + 32x\)\\
			(d) \(2(3(-2(x-5))+15)\)
		\end{tabularx}\\
		\par\bigskip\noindent
		(2) Multiplizieren Sie möglichst geschickt aus.\\
		\begin{tabularx}{0.5\textwidth}{X}
			(a) \((x+y)(3x-3y)\)\\
			(b) \((12a-8b)(15a+10b)\)
		\end{tabularx}\\
		\newpage
		(3) Faktorisieren Sie.\\
		\begin{tabularx}{0.5\textwidth}{XX}
			(a) \(8x-2y\) & (c) \(36a^2 - 60ab +25b^2\)\\
			(b) \(10a + 15b -10\) & (d) \(\frac{1}{16}a^2 + \frac{1}{4}ab + \frac{1}{4}b^2\)
		\end{tabularx}
		\subsubsection*{(Brüche)}
		(1) Bestimmen Sie das kleinste gemeinsame Vielfache.\\
		\begin{tabularx}{0.5\textwidth}{XX}
			(a) 2, 3, 9 & (c) 2, 6, 36\\
			(b) 7, 9, 15 & (d) 4, 18, 28
		\end{tabularx}
		\par\bigskip\noindent
		(2) Addieren bzw. subtrahieren Sie und kürzen Sie das Ergebnis
		\begin{tabularx}{0.5\textwidth}{XX}
			(a) \(\frac{5}{6} + \frac{7}{3}\) & (c) \(\frac{1}{4} + \frac{5}{8 + \frac{11}{24}}\)\\
			(b) \(-\frac{1}{2}+\frac{5}{2}-\frac{3}{8}\) & (d) \(-\frac{7}{12} + \frac{2}{3} - \frac{1}{6}\)
		\end{tabularx}
		\par\bigskip\noindent
		(3) Multiplizieren sie die Brüche und kürzen, wenn möglich.
		\begin{tabularx}{0.5\textwidth}{XXX}
			(a) \(\frac{1}{2}\cdot\frac{2}{3}\) & (d) \(\frac{9}{4}\cdot\frac{6}{5}\) & (g) \(\frac{29}{7}\cdot\frac{42}{5}\)\\
			(b) \(\frac{5}{7}\cdot\frac{3}{6}\) & (e) \(\frac{5}{3}\cdot\frac{6}{5}\) & (h) \(\frac{9}{8}\cdot{}5\frac{1}{3}\)\\
			(c) \(\frac{7}{8}\cdot\frac{4}{5}\) & (f) \(3\cdot\frac{25}{9}\cdot\frac{3}{5}\)
		\end{tabularx}\\
		\par\bigskip\noindent
		(4) Führen Sie die Division aus und kürzen Sie, wenn möglich.
		\begin{tabularx}{0.5\textwidth}{XXX}
			(a) \(\frac{1}{3}:\frac{1}{4}\) & (d) \(\frac{8}{3}:\frac{2}{9}\) & (g) \(\frac{36}{11}:\frac{9}{7}\)\\
			(b) \(\frac{5}{7}:\frac{3}{7}\) & (e) \(\frac{17}{3}:\frac{2}{9}\) & (h) \(\frac{\frac{3}{7}:\frac{1}{2}}{\frac{4}{5}}\)\\
			(c) \(\frac{7}{9}:\frac{2}{5}\) & (f) \(\frac{9}{7}:\frac{9}{4}\)
		\end{tabularx}\\
		\subsubsection*{(Potenzen)}
		(1) Berechnen Sie folgende Potenzen.
		\begin{tabularx}{0.5\textwidth}{XXX}
			(a) \(3^4\) & (d) \((-3)^{-4}\) & (g) \(\frac{7^2}{8}\)\\
			(b) \(-3^{\frac{1}{4}}\) & (e) \((-3)^4\) & (h) \(-\frac{3}{4^2}\)\\
			(c) \(3^{\frac{1}{4}}\) & (f) \((-3^2)^3\)
		\end{tabularx}
		\par\bigskip\noindent
		(2) Wandeln Sie jeweils die Wurzeln in Potenzen um.
		\begin{tabularx}{0.5\textwidth}{XX}
			(a) \(\sqrt[3]{4}\) & (e) \(\sqrt[6]{5^3}\)\\
			(b) \(\sqrt[5]{3}\) & (f) \(\sqrt[8]{a^3}; a\geq 0\)\\
			(c) \(\sqrt{7}\) & (g) \(\sqrt[3]{a^5}; a\geq 0\)\\
			(d) \(\sqrt[3]{2^2}\) & (h) \(\sqrt[3]{a^2\cdot{}b^4}\)
		\end{tabularx}\\
		\subsubsection*{(Wurzel)}
		(1) Berechnen Sie die folgenden Wurzelterme.
		\begin{tabularx}{0.5\textwidth}{XX}
			(a) \(\sqrt{2}\cdot\sqrt{2}\) & (e) \(\sqrt[3]{9}\cdot\sqrt[3]{3}\)\\
			(b) \(\sqrt{6}\cdot\sqrt{54}\) & (f) \(5\sqrt{2,45}\cdot{}6\cdot\sqrt{5}\)\\
			(c) \(3\sqrt{5}\cdot{}2\sqrt{0,2}\) & (g) \(2\sqrt{9a^2b}\cdot\sqrt{4a^2b}\)\\
			(d) \(\sqrt[3]{8}\cdot\sqrt[4]{16}\) & (h) \(\sqrt{\sqrt{81a}}\)
		\end{tabularx}\\
		\newpage
		\section{Gleichungen und Ungleichungen}
		\subsection{Allgemeines zu Gleichungen und Ungleichungen}
		Sie wissen nun was ein Term ist. Es ist möglich zwei Terme mittels dem \glqq{}\(=\)\grqq{}-Zeichen miteinander zu verbinden. Ist dies der Fall, so liegt eine \underline{\textbf{Gleichung}} vor.\\
		\par\bigskip\noindent
		\underline{Beispiel:} \(18 + 0,5\cdot{}x = x\)\\
		\glqq{}\textit{Bei welcher Fahrtzeit macht es keinen Unterschied, ob man die Silber-Karte (18 \euro{} Grundgebühr und 0,50 \euro{} pro angefangene halbe Stunde) hat oder nicht (1 \euro{} pro angefangene Stunde)?}\grqq{}\\
		\par\bigskip\noindent
		Es besteht aber auch die Möglichkeit, zwei Terme mit einem \glqq{}\(<\)\grqq{} bzw. \glqq{}\(\leq\)\grqq{} (\glqq{}kleiner\grqq{} bzw. \glqq{}kleiner-gleich\grqq{}) oder \glqq{}\(>\)\grqq{} bzw. \glqq{}\(\geq\)\grqq{} (\glqq{}größer\grqq{} bzw. \glqq{}größer-gleich\grqq{}) zu verbinden.\\
		In diesem Fall handelt es sich um eine \underline{\textbf{Ungleichung}}.\\
		\par\bigskip\noindent
		\underline{Beispiel:} \(18 + 0,5x > 50\)\\
		\glqq{}\textit{Ab welcher Fahrtzeit übersteigen die Kosten einen Wert von 50 \euro{} pro Jahr?}\grqq{}\\
		\subsection{Die Lösungsmenge einer Gleichung}
		Gleichungen mit einer Variablen kann man lösen. Als Lösung einer Gleichung bezeichnet man die Zahlen als Besetzung der Variable, für die der Wert des rechten Terms gleich dem Wert des linken Terms ist.\\
		Alle Lösungen zusammen nennt man auch \underline{Lösungsmenge \(\mathbb{L}\)}.\\
		\par\bigskip\noindent
		\underline{Beispiel:} Die Lösungsmenge der Gleichung \(18 + 0,5x = x\) beträgt \(\mathbb{L} = {36}\) oder auch \(x=36\).
		\subsection{Die Lösungsmenge einer Gleichung bestimmen und angeben}
		Im Prinzip kann man sich eine Gleichung wie eine altmodische Waage vorstellen.\\
		\includegraphics[width=0.1\textwidth]{../99_Bilder/waage.jpg} Dabei müssen die Gegenstände in den einzelnen Schalen nicht dieselbe Gestalt haben. Es ist aber unabdingbar, dass diese das gleiche Gewicht haben.\\
		\par\bigskip\noindent
		Ziel bei einer Waage ist es immer ein Gleichgewicht zu halten. Das bedeutet, führt man eine Operation auf der einen Seite durch, so muss diese auch auf der anderen Seite durchgeführt werden, damit das Gleichgewicht nicht gestört wird.\\
		Die Operationen kann man solange durchführen, bis man die Lösungsmenge erhält.\\
		\begin{framed}
			\noindent
			Im nachfolgenden Schauen wir uns ein Beispiel an.\\
			\centering
			\includegraphics[width=0.4\textwidth]{../99_Bilder/L.jpg}\\
			\raggedright
			Wir entfernen nun zunächst einen Kreis und einen Kasten.\\
			\centering
			\includegraphics[width=0.4\textwidth]{../99_Bilder/L1.jpg}\\
			\raggedright
			Es sind auf beiden Seiten eine gerade Anzahl an \glqq{}Gegenständen\grqq{}. Also halbieren wir diese.\\
			\centering
			\includegraphics[width=0.4\textwidth]{../99_Bilder/L2.jpg}\\
			\raggedright
			Wir sehen also, ein Kasten entspricht zwei Kreisen.\\
		\end{framed}
		\par\bigskip\noindent
		Eine Gleichung immer mit der Waagendarstellung zu lösen kann sehr aufwendig sein. Man kann das Ganze auch symbolisch darstellen. Der Lösungsweg sieht dann so aus:
		\begin{tabularx}{0.5\textwidth}{rcll}
			\(3x + 1\) & \(=\) & \(x+5\) & |\(-1\)\\
			\(3x\) & \(=\) & \(x+4\) & |\(-x\)\\
			\(2x\) & \(=\) & \(4\) & |\(:2\)\\
			\(x\) & \(=\) & \(2\)
		\end{tabularx}
		Die Operationen, die man auf beiden Seiten der Gleichung durchführt, nennt man \underline{Äquivalenzumformung}. Die Umformungen sind so gewählt, dass die Termbestandteile mit \(x\) auf eine Seite und die Termbestandteile mit Zahlen auf die andere Seite gebracht werden.\\
		Man erhält also eine Gleichung der Form \(x = \text{Zahl}\).\\
		\par\bigskip\noindent
		Dabei ist \underline{zu beachten}, dass zunächst alle schwächsten Bindungen (\color{blue}Stri\normalcolor{}chrechnung) umgekehrt werden, danach die zweitschwächsten (\color{blue}Pu\normalcolor{}nktrechnung), dann die drittschwächsten (\color{blue}Po\normalcolor{}tenz) und zuletzt werden \color{blue}Kla\normalcolor{}mmern aufgelöst.\\
		\par\bigskip\noindent
		\centering
		\fcolorbox{blue}{blue!15}{\color{blue}StriPuPoKla\normalcolor{} löst die Gleichung!}\\
		\raggedright
		\par\bigskip\noindent
		Welche Operation welche Operation umkehrt ist nachfolgend aufgeführt:
		\begin{tabularx}{0.45\textwidth}{|c|c|X|}
			\cline{1-2}
			Operation & Umkehrung & \multicolumn{1}{l}{}\\
			\cline{1-2}
			\(+a\) & \(-a\) & \multicolumn{1}{l}{} \\
			\cline{1-2}
			\(-a\) & \(+a\) & \multicolumn{1}{l}{} \\
			\hline
			\multicolumn{1}{|l|}{} & & \\
			\(\cdot{}a\) & \(:a\) oder \(\cdot{}\frac{1}{a}\) &
			\multirow{7}{0.1\textheight}{Diese Operationen sind mit \textbf{allen Summanden auf beiden Seiten} durchzuführen!}\\
			\multicolumn{1}{|l|}{} & & \\
			\cline{1-2}
			\multicolumn{1}{|l|}{} & & \\
			\(:a\) oder \(\frac{\ldots}{a}\) & \(\cdot{}a\) & \\
			\multicolumn{1}{|l|}{} & & \\
			\cline{1-2}
			\multicolumn{1}{|l|}{} & & \\
			\(\ldots^2\) & \glqq{}\(\sqrt{\ldots}\)\grqq{} & \\
			\multicolumn{1}{|l|}{} & & \\
			\hline
		\end{tabularx}
		\par\bigskip\noindent
		\underline{Anmerkung zu \textbf{Äquivalenzumformung bei}} \underline{\textbf{Ungleichungen}:}\\
		Multipliziert oder dividiert man bei Ungleichungen mit einer negativen Zahl, dreht sich das Ungleichheitszeichen um.\\
		\underline{Beispiel:}
		\begin{tabularx}{0.5\textwidth}{cccl}
			\(-2x\) & \(<\) & \(4\) & |\(:(-2)\)\\
			\(x\) & \(>\) & \(-2\)
		\end{tabularx}
	\end{worksheet}
\end{document}