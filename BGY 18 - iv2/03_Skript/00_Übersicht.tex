\documentclass[11pt,oneside,openany,headings=optiontotoc,11pt,numbers=noenddot]{article}

\usepackage[a4paper]{geometry}
\usepackage[utf8]{inputenc}
\usepackage[T1]{fontenc}
\usepackage{lmodern}
\usepackage[ngerman]{babel}
\usepackage{ngerman}

\usepackage[onehalfspacing]{setspace}

\usepackage{fancyhdr}
\usepackage{fancybox}

\usepackage{rotating}
\usepackage{varwidth}


\usepackage{pdflscape}
\usepackage{graphicx}
\usepackage{graphbox}
\graphicspath{
	{Pics/PDFs/}
	{Pics/JPGs/}
	{Pics/PNGs/}
}
\usepackage{caption}
\usepackage{tabularx}
\usepackage{dashrule}
\usepackage{hhline}
\usepackage{multirow}
\usepackage{enumerate}
\usepackage[hidelinks]{hyperref}
\usepackage{listings}

\usepackage[table]{xcolor}
\usepackage{array}
\usepackage{enumitem,amssymb,amsmath}
\usepackage{interval}
\usepackage{stmaryrd}
\usepackage{polynom}
\usepackage{diagbox}
\usepackage{dashrule}
\usepackage{framed}
\usepackage{mdframed}
\usepackage{karnaugh-map}

\usepackage{blindtext}

\usepackage{eso-pic}

\usepackage{amssymb}
\usepackage{eurosym}
\pagestyle{headings}
\renewcommand{\headrulewidth}{0.2pt}
\renewcommand{\footrulewidth}{0.2pt}
\newcommand*{\underdownarrow}[2]{\ensuremath{\underset{\overset{\Big\downarrow}{#2}}{#1}}}
\setlength{\fboxsep}{5pt}

% Codestyle defined
\definecolor{codegreen}{rgb}{0,0.6,0}
\definecolor{codegray}{rgb}{0.5,0.5,0.5}
\definecolor{codepurple}{rgb}{0.58,0,0.82}
\definecolor{backcolour}{rgb}{0.95,0.95,0.92}
\definecolor{deepgreen}{rgb}{0,0.5,0}
\definecolor{darkblue}{rgb}{0,0,0.65}
\definecolor{mauve}{rgb}{0.40, 0.19,0.28}
\colorlet{exceptioncolour}{yellow!50!red}
\colorlet{commandcolour}{blue!60!black}
\colorlet{numpycolour}{blue!60!green}
\colorlet{specmethodcolour}{violet}

%Neue Spaltendefinition
\newcolumntype{L}[1]{>{\raggedright\let\newline\\\arraybackslash\hspace{0pt}}m{#1}}
\newcolumntype{M}[1]{>{\centering\arraybackslash}X}
\newcommand{\cmnt}[1]{\ignorespaces}
%Textausrichtung ändern
\newcommand\tabrotate[1]{\rotatebox{90}{\raggedright#1\hspace{\tabcolsep}}}

%Intervall-Konfig
\intervalconfig {
	soft open fences
}

%Bash
\lstdefinestyle{BashInputStyle}{
	language=bash,
	basicstyle=\small\sffamily,
	backgroundcolor=\color{backcolour},
	columns=fullflexible,
	backgroundcolor=\color{backcolour},
	breaklines=true,
}
%Java
\lstdefinestyle{JavaInputStyle}{
	language=Java,
	backgroundcolor=\color{backcolour},
	aboveskip=1mm,
	belowskip=1mm,
	showstringspaces=false,
	columns=flexible,
	basicstyle={\footnotesize\ttfamily},
	numberstyle={\tiny},
	numbers=none,
	keywordstyle=\color{purple},,
	commentstyle=\color{deepgreen},
	stringstyle=\color{blue},
	emph={out},
	emphstyle=\color{darkblue},
	emph={[2]rand},
	emphstyle=[2]\color{specmethodcolour},
	breaklines=true,
	breakatwhitespace=true,
	tabsize=2,
}
%Python
\lstdefinestyle{PythonInputStyle}{
	language=Python,
	alsoletter={1234567890},
	aboveskip=1ex,
	basicstyle=\footnotesize,
	breaklines=true,
	breakatwhitespace= true,
	backgroundcolor=\color{backcolour},
	commentstyle=\color{red},
	otherkeywords={\ , \}, \{, \&,\|},
	emph={and,break,class,continue,def,yield,del,elif,else,%
		except,exec,finally,for,from,global,if,import,in,%
		lambda,not,or,pass,print,raise,return,try,while,assert},
	emphstyle=\color{exceptioncolour},
	emph={[2]True,False,None,min},
	emphstyle=[2]\color{specmethodcolour},
	emph={[3]object,type,isinstance,copy,deepcopy,zip,enumerate,reversed,list,len,dict,tuple,xrange,append,execfile,real,imag,reduce,str,repr},
	emphstyle=[3]\color{commandcolour},
	emph={[4]ode, fsolve, sqrt, exp, sin, cos, arccos, pi,  array, norm, solve, dot, arange, , isscalar, max, sum, flatten, shape, reshape, find, any, all, abs, plot, linspace, legend, quad, polyval,polyfit, hstack, concatenate,vstack,column_stack,empty,zeros,ones,rand,vander,grid,pcolor,eig,eigs,eigvals,svd,qr,tan,det,logspace,roll,mean,cumsum,cumprod,diff,vectorize,lstsq,cla,eye,xlabel,ylabel,squeeze},
	emphstyle=[4]\color{numpycolour},
	emph={[5]__init__,__add__,__mul__,__div__,__sub__,__call__,__getitem__,__setitem__,__eq__,__ne__,__nonzero__,__rmul__,__radd__,__repr__,__str__,__get__,__truediv__,__pow__,__name__,__future__,__all__},
	emphstyle=[5]\color{specmethodcolour},
	emph={[6]assert,range,yield},
	emphstyle=[6]\color{specmethodcolour}\bfseries,
	emph={[7]Exception,NameError,IndexError,SyntaxError,TypeError,ValueError,OverflowError,ZeroDivisionError,KeyboardInterrupt},
	emphstyle=[7]\color{specmethodcolour}\bfseries,
	emph={[8]taster,send,sendMail,capture,check,noMsg,go,move,switch,humTem,ventilate,buzz},
	emphstyle=[8]\color{blue},
	keywordstyle=\color{blue}\bfseries,
	rulecolor=\color{black!40},
	showstringspaces=false,
	stringstyle=\color{deepgreen}
}

\lstset{literate=%
	{Ö}{{\"O}}1
	{Ä}{{\"A}}1
	{Ü}{{\"U}}1
	{ß}{{\ss}}1
	{ü}{{\"u}}1
	{ä}{{\"a}}1
	{ö}{{\"o}}1
}

% Neue Klassenarbeits-Umgebung
\newenvironment{worksheet}[3]
% Begin-Bereich
{
	\newpage
	\sffamily
	\setcounter{page}{1}
	\ClearShipoutPicture
	\AddToShipoutPicture{
		\put(55,761){{
				\mbox{\parbox{385\unitlength}{\tiny \color{codegray}BBS I Mainz, #1 \newline #2
						\newline #3
					}
				}
			}
		}
		\put(455,761){{
				\mbox{\hspace{0.3cm}\includegraphics[width=0.2\textwidth]{../../logo.jpg}}
			}
		}
	}
}
% End-Bereich
{
	\clearpage
	\ClearShipoutPicture
}

\setlength{\columnsep}{3em}
\setlength{\columnseprule}{0.5pt}

\geometry{left=1.00cm,right=1.00cm,top=3.00cm,bottom=1.00cm,includeheadfoot}
\pagestyle{plain}
\pagenumbering{arabic}

\begin{document}
	\begin{worksheet}{Berufliches Gymnasium}{Informationsverarbeitung - Grundkurs}{Lernabschnitt 0: Übersicht}
		\setlength{\columnseprule}{0pt}
		\setcounter{section}{0}
		\section{Übersicht}
		Zur Informationsverarbeitung gehören verschiedene Bereiche, die im Laufe der Oberstufe behandelt werden. Damit Sie sich jetzt schon eine Idee bekommen, was in den einzelnen Abschnitten auf Sie zukommt, finden Sie nachfolgend eine grobe Übersicht über die einzelnen Bereiche.
		\subsection{Algorithmen und Datenstrukturen planen und realisieren}
		Dieses Themengebiet beschäftigt sich mit technischen, naturwissenschaftlichen, mathematischen oder auch betriebswirtschaftlichen \textbf{Problemstellungen}, welche zunächst \textbf{analysiert} - \texttt{was soll passieren}, \textbf{strukturiert} - \texttt{wann soll was passieren} und anschließend \textbf{grafisch dargestellt} - \texttt{wir überführen die Struktur in ein Schaubild\footnote{z.B.: Struktogramm, Ablaufplan}} werden.\\
		\par\noindent
		Bei der Analyse und Strukturierung greifen wir auf \textbf{diverse Kontrollstrukturen}\footnote{z.B. Schleifen oder Wiederholungsanweisungen.} zurück. Um die Funktionalität unseres Schaubildes zu testen, nutzen wir eine \textbf{geeignete Programmiersprache} - Java oder Python - um den Algorithmus zu kodieren und anschließend mit geeigneten Verfahren zu \textbf{testen}.
		\subsection{Ein privates Daten-Verarbeitungs-System für die Kommunikation in Netzen konzipieren}
		\LARGE{TO-DO}\normalsize\\
		Grundlegende Verfahren der Kommunikation in Netzen abbilden.
		Technische Voraussetzungen zur Kommunikation in Netzen analysieren und darstellen.
		Ein sicheres privates Netzwerk zur Kommunikation im Internet planen und einrichten.
		Probleme in der Kommunikation in Netzwerken identifizieren und beschreiben.
		Inhaltliche Orientierung
		Sender - Empfänger - Botschaft
		Komponenten
		Dienste, Protokolle
		Provider
		\subsection{Digitale Daten schützen und sichern}
		\LARGE{TO-DO}\normalsize\\
		Datenschutz, Datensicherheit und Datensicherung definieren und deren Überschneidungen erläutern.
		Veröffentlichung sensibler Daten im Internet kritisch beurteilen.
		Rechte und Pflichten der Bürger sowie der datenspeichern den Einrichtungen darstellen. Organisatorische
		Maßnahmen zu deren Umsetzung beschreiben. Aktuelle Beispiele aus Gesellschaft,
		Politik und Wirtschaft anhand der geltenden rechtlichen Vorgaben diskutieren.
		Geeignete Maßnahmen zum Schutz vor Diebstahl, Verfälschung, Verlust und Zerstörung von
		Daten gegenüberstellen.
		Aktuelle Bedrohungen durch Schadprogramme und Programmfehler analysieren und entsprechende
		Maßnahmen zu deren Abwehr ergreifen.
		Datensicherung mit Hard- und Software situativ beurteilen und auswählen. Einfache Strategien
		der Datensicherung erläutern und umsetzen.
		\subsection{Ein Projekt durchführen}
		\LARGE{TO-DO}\normalsize\\
		Ein Projekt unter Bezugnahme auf das Grundfach Informationsverarbeitung wissenschaftsorientiert
		durchführen. Die Ergebnisse dokumentieren, präsentieren und evaluieren.
		\subsection*{Zusätzliche Angebote}
		\LARGE{TO-DO}\normalsize\\
	\end{worksheet}
\end{document}