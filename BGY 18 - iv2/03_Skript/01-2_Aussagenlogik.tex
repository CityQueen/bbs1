\documentclass[11pt,oneside,openany,headings=optiontotoc,11pt,numbers=noenddot]{article}

\usepackage[a4paper]{geometry}
\usepackage[utf8]{inputenc}
\usepackage[T1]{fontenc}
\usepackage{lmodern}
\usepackage[ngerman]{babel}
\usepackage{ngerman}

\usepackage[onehalfspacing]{setspace}

\usepackage{fancyhdr}
\usepackage{fancybox}

\usepackage{rotating}
\usepackage{varwidth}


\usepackage{pdflscape}
\usepackage{graphicx}
\usepackage{graphbox}
\graphicspath{
	{Pics/PDFs/}
	{Pics/JPGs/}
	{Pics/PNGs/}
}
\usepackage{caption}
\usepackage{tabularx}
\usepackage{dashrule}
\usepackage{hhline}
\usepackage{multirow}
\usepackage{enumerate}
\usepackage[hidelinks]{hyperref}
\usepackage{listings}

\usepackage[table]{xcolor}
\usepackage{array}
\usepackage{enumitem,amssymb,amsmath}
\usepackage{interval}
\usepackage{stmaryrd}
\usepackage{polynom}
\usepackage{diagbox}
\usepackage{dashrule}
\usepackage{framed}
\usepackage{mdframed}
\usepackage{karnaugh-map}

\usepackage{blindtext}

\usepackage{eso-pic}

\usepackage{amssymb}
\usepackage{eurosym}
\pagestyle{headings}
\renewcommand{\headrulewidth}{0.2pt}
\renewcommand{\footrulewidth}{0.2pt}
\newcommand*{\underdownarrow}[2]{\ensuremath{\underset{\overset{\Big\downarrow}{#2}}{#1}}}
\setlength{\fboxsep}{5pt}

% Codestyle defined
\definecolor{codegreen}{rgb}{0,0.6,0}
\definecolor{codegray}{rgb}{0.5,0.5,0.5}
\definecolor{codepurple}{rgb}{0.58,0,0.82}
\definecolor{backcolour}{rgb}{0.95,0.95,0.92}
\definecolor{deepgreen}{rgb}{0,0.5,0}
\definecolor{darkblue}{rgb}{0,0,0.65}
\definecolor{mauve}{rgb}{0.40, 0.19,0.28}
\colorlet{exceptioncolour}{yellow!50!red}
\colorlet{commandcolour}{blue!60!black}
\colorlet{numpycolour}{blue!60!green}
\colorlet{specmethodcolour}{violet}

%Neue Spaltendefinition
\newcolumntype{L}[1]{>{\raggedright\let\newline\\\arraybackslash\hspace{0pt}}m{#1}}
\newcolumntype{M}[1]{>{\centering\arraybackslash}X}
\newcommand{\cmnt}[1]{\ignorespaces}
%Textausrichtung ändern
\newcommand\tabrotate[1]{\rotatebox{90}{\raggedright#1\hspace{\tabcolsep}}}

%Intervall-Konfig
\intervalconfig {
	soft open fences
}

%Bash
\lstdefinestyle{BashInputStyle}{
	language=bash,
	basicstyle=\small\sffamily,
	backgroundcolor=\color{backcolour},
	columns=fullflexible,
	backgroundcolor=\color{backcolour},
	breaklines=true,
}
%Java
\lstdefinestyle{JavaInputStyle}{
	language=Java,
	backgroundcolor=\color{backcolour},
	aboveskip=1mm,
	belowskip=1mm,
	showstringspaces=false,
	columns=flexible,
	basicstyle={\footnotesize\ttfamily},
	numberstyle={\tiny},
	numbers=none,
	keywordstyle=\color{purple},,
	commentstyle=\color{deepgreen},
	stringstyle=\color{blue},
	emph={out},
	emphstyle=\color{darkblue},
	emph={[2]rand},
	emphstyle=[2]\color{specmethodcolour},
	breaklines=true,
	breakatwhitespace=true,
	tabsize=2,
}
%Python
\lstdefinestyle{PythonInputStyle}{
	language=Python,
	alsoletter={1234567890},
	aboveskip=1ex,
	basicstyle=\footnotesize,
	breaklines=true,
	breakatwhitespace= true,
	backgroundcolor=\color{backcolour},
	commentstyle=\color{red},
	otherkeywords={\ , \}, \{, \&,\|},
	emph={and,break,class,continue,def,yield,del,elif,else,%
		except,exec,finally,for,from,global,if,import,in,%
		lambda,not,or,pass,print,raise,return,try,while,assert},
	emphstyle=\color{exceptioncolour},
	emph={[2]True,False,None,min},
	emphstyle=[2]\color{specmethodcolour},
	emph={[3]object,type,isinstance,copy,deepcopy,zip,enumerate,reversed,list,len,dict,tuple,xrange,append,execfile,real,imag,reduce,str,repr},
	emphstyle=[3]\color{commandcolour},
	emph={[4]ode, fsolve, sqrt, exp, sin, cos, arccos, pi,  array, norm, solve, dot, arange, , isscalar, max, sum, flatten, shape, reshape, find, any, all, abs, plot, linspace, legend, quad, polyval,polyfit, hstack, concatenate,vstack,column_stack,empty,zeros,ones,rand,vander,grid,pcolor,eig,eigs,eigvals,svd,qr,tan,det,logspace,roll,mean,cumsum,cumprod,diff,vectorize,lstsq,cla,eye,xlabel,ylabel,squeeze},
	emphstyle=[4]\color{numpycolour},
	emph={[5]__init__,__add__,__mul__,__div__,__sub__,__call__,__getitem__,__setitem__,__eq__,__ne__,__nonzero__,__rmul__,__radd__,__repr__,__str__,__get__,__truediv__,__pow__,__name__,__future__,__all__},
	emphstyle=[5]\color{specmethodcolour},
	emph={[6]assert,range,yield},
	emphstyle=[6]\color{specmethodcolour}\bfseries,
	emph={[7]Exception,NameError,IndexError,SyntaxError,TypeError,ValueError,OverflowError,ZeroDivisionError,KeyboardInterrupt},
	emphstyle=[7]\color{specmethodcolour}\bfseries,
	emph={[8]taster,send,sendMail,capture,check,noMsg,go,move,switch,humTem,ventilate,buzz},
	emphstyle=[8]\color{blue},
	keywordstyle=\color{blue}\bfseries,
	rulecolor=\color{black!40},
	showstringspaces=false,
	stringstyle=\color{deepgreen}
}

\lstset{literate=%
	{Ö}{{\"O}}1
	{Ä}{{\"A}}1
	{Ü}{{\"U}}1
	{ß}{{\ss}}1
	{ü}{{\"u}}1
	{ä}{{\"a}}1
	{ö}{{\"o}}1
}

% Neue Klassenarbeits-Umgebung
\newenvironment{worksheet}[3]
% Begin-Bereich
{
	\newpage
	\sffamily
	\setcounter{page}{1}
	\ClearShipoutPicture
	\AddToShipoutPicture{
		\put(55,761){{
				\mbox{\parbox{385\unitlength}{\tiny \color{codegray}BBS I Mainz, #1 \newline #2
						\newline #3
					}
				}
			}
		}
		\put(455,761){{
				\mbox{\hspace{0.3cm}\includegraphics[width=0.2\textwidth]{../../logo.jpg}}
			}
		}
	}
}
% End-Bereich
{
	\clearpage
	\ClearShipoutPicture
}

\setlength{\columnsep}{3em}
\setlength{\columnseprule}{0.5pt}

\geometry{left=1.50cm,right=1.50cm,top=3.00cm,bottom=1.00cm,includeheadfoot}
\pagestyle{plain}
\pagenumbering{arabic}

\begin{document}
	\begin{worksheet}{Berufliches Gymnasium}{Klassenstufe 12 - Informationsverarbeitung - Grundkurs}{Lernabschnitt 1: Strukturiert programmieren - Aussagenlogik}
		\noindent
		\setcounter{section}{1}
		\setcounter{page}{3}
		\noindent
		Für die Programmierung ist es, abgesehen von der Kenntnis einer Programmiersprache, unabdingbar, dass man sich mit der Aussagenlogik beschäftigt. Also damit, welche Funktion jede einzelne Verknüpfung liefert und wie die dazugehörigen \grqq.{}Wahrheitstabellen\grqq{} aussehen.
		\section{Aussagenlogik}
		\subsection{Grundbegriffe}
		Eine \textbf{Aussage} ist eine Äußerung, die wahr oder falsch ist, je nachdem, ob der durch sie beschriebene Sachverhalt vorliegt oder nicht.\\
		\par\noindent
		In der nachfolgenden Tabelle finden Sie einen Überblick über die klassischen Aussagefunktionen:\\
		\par\noindent
		\begin{tabularx}{\textwidth}{|X|X|X|X|}
			\hline
			\rowcolor{gray!10} \textbf{Verknüpfung} & \textbf{Formulierung} & \textbf{Zeichen} & \textbf{\glqq{}Übersetzung\grqq{}}\\
			\hline
			Negation & Nicht & \(\neg\) & Gegenteil\\
			\hline
			Konjunktion & Und & \(\wedge\) & Verbindung\\
			\hline
			Disjunktion & Oder & \(\vee\) & \\
			\hline
			Implikation & Wenn ... dann & \(\Rightarrow\) & \\
			\hline
			Äquivalenz & genau dann, wenn ... & \(\Leftrightarrow\) & \\
			\hline
		\end{tabularx}
		\subsection{Die Negation \grqq{}\(\neg\)\grqq{}}
		Die Negation \grqq{}$\neg$\grqq{} ordnet jeder Aussage A die Aussage \grqq{}nicht A\grqq{} ($\neg$A) zu.
		\begin{framed}
			\noindent
			Das bedeutet: \textit{Die Aussage $\neg$A hat den zur Aussage A entgegengesetzten Wahrheitswert.}
		\end{framed}
		\par\noindent
		\begin{tabularx}{\textwidth}{l|l|l|}
			\cline{2-3}
			\underline{Wahrheitstabelle:} & \textbf{A} & \textbf{$\mathbf{\neg}$ A}\\
			\cline{2-3}
			& W & F\\
			\cline{2-3}
			& F & W\\
			\cline{2-3}
		\end{tabularx}
		\subsection{Die Konjunktion \grqq{}$\wedge$\grqq{}}
		Die Verknüpfung zweier Aussagen A und B durch das Wort \grqq{}und\grqq{} heißt \textbf{Konjunktion}.
		\begin{framed}
			\noindent
			Dabei ist die Verknüpfung so definiert, dass mit A und B zwei Aussagen sind. Die Konjunktion von A und B ist wahr genau dann, wenn die beiden Aussagen A und B wahr sind.\\
			\textbf{Schreibweise:} \(A\wedge{}B\)
		\end{framed}
		\par\noindent
		\begin{tabularx}{\textwidth}{l|c|c|c|}
			\cline{2-4}
			\underline{Wahrheitstabelle für Konjunktion} & \textbf{A} & \textbf{B} & \(\mathbf{A\wedge{}B}\)\\
			\cline{2-4}
			& F & F & F\\
			\cline{2-4}
			& F & W & F\\
			\cline{2-4}
			& W & F & F\\
			\cline{2-4}
			& W & W & W\\
			\cline{2-4}
		\end{tabularx}
		\subsection{Die Disjunktion \grqq{}$\vee$\grqq{}}
		Die Verknüpfung zweier Aussagen A und B durch das Wort \grqq{}oder\grqq{} heißt \textbf{Disjunktion}.
		\begin{framed}
			\noindent
			Die Verknüpfung ist dann wie folgt definiert, sind A und B zwei Aussagen, dann ist die Disjunktion von A und B wahr genau dann, wenn mindestens eine der beiden Aussagen A oder B wahr ist.\\
			\textbf{Schreibweise:} A$\vee$B
		\end{framed}
		\par\noindent
		\begin{tabularx}{\textwidth}{l|c|c|c|}
			\cline{2-4}
			\underline{Wahrheitstabelle für Disjunktion} & \textbf{A} & \textbf{B} & \(\mathbf{A\vee{}B}\)\\
			\cline{2-4}
			& F & F & F\\
			\cline{2-4}
			& F & W & W\\
			\cline{2-4}
			& W & F & W\\
			\cline{2-4}
			& W & W & W\\
			\cline{2-4}
		\end{tabularx}
		\subsection{Die Implikation \grqq{}$\Rightarrow$\grqq{}}
		Die Verknüpfung zweier Aussagen A und B durch \grqq{}wenn - dann\grqq{} (bzw. \grqq{}wenn - so\grqq{}) heißt \textbf{Implikation} (wenn A, dann B).
		\begin{framed}
			\noindent
			Es gilt, wenn A und B zwei Aussagen sind, dann ist die Implikation von A und B genau dann falsch, wenn A wahr und B falsch ist. In allen anderen Fällen ist sie wahr.\\
			\textbf{Schreibweise:} A $\Rightarrow$ B
		\end{framed}
		\par\noindent
		\begin{tabularx}{\textwidth}{l|c|c|c|}
			\cline{2-4}
			\underline{Wahrheitstabelle für Implikation} & \textbf{A} & \textbf{B} & \(\mathbf{A \Rightarrow{}B}\)\\
			\cline{2-4}
			& F & F & F\\
			\cline{2-4}
			& F & W & F\\
			\cline{2-4}
			& W & F & F\\
			\cline{2-4}
			& W & W & W\\
			\cline{2-4}
		\end{tabularx}
		\subsection{Die Äquivalenz \grqq{}$\Leftrightarrow$\grqq{}}
		Die Verknüpfung zweier Aussagen A und B durch \grqq{}genau dann, wenn\grqq{} oder \grqq{}dann und nur dann, wenn\grqq{} heißt \textbf{Äquivalenz}.
		\begin{framed}
			\noindent
			Sind A und B zwei Aussagen. Dann ist die Äquivalenz von A und B wahr genau dann, wenn A und B denselben Wahrheitswert haben. In allen anderen Fällen ist sie falsch.\\
			\textbf{Schreibweise:} A$\Leftrightarrow$B
		\end{framed}
		\par\noindent
		\begin{tabularx}{\textwidth}{l|c|c|c|c|}
			\cline{2-5}
			\underline{Wahrheitstabelle für Implikation} & \textbf{A} & \textbf{B} & \(\mathbf{A \Rightarrow{}B}\) & \(\mathbf{B \Rightarrow{}A}\)\\
			\cline{2-5}
			& F & F & W & W\\
			\cline{2-5}
			& F & W & F & F\\
			\cline{2-5}
			& W & F & F & F\\
			\cline{2-5}
			& W & W & W & W\\
			\cline{2-5}
		\end{tabularx}
		\subsection{Aussagenlogische Verbindungen}
		Bei den Aussagenverbindungen sollten wie beim Rechnen mit Zahlen Klammern gesetzt werden, um zu entscheiden, in welcher Reihenfolge die Aussage verknüpft werden soll.\\
		\par\noindent
		Werden keine Klammern gesetzt, so gilt folgende Reihenfolge:
		\begin{itemize}[label=-]
			\item Negation (NICHT)
			\item Konjunktion (UND)
			\item Disjunktion (ODER)
			\item Implikation
			\item Äquivalenz
		\end{itemize}
	\end{worksheet}
\end{document}