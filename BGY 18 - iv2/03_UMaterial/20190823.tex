\documentclass[oneside,openany,headings=optiontotoc,11pt,numbers=noenddot]{scrreprt}

\usepackage[a4paper]{geometry}
\usepackage[utf8]{inputenc}
\usepackage[T1]{fontenc}
\usepackage{lmodern}
\usepackage[ngerman]{babel}
\usepackage{ngerman}

\usepackage[onehalfspacing]{setspace}

\usepackage{fancyhdr}
\usepackage{fancybox}

\usepackage{rotating}
\usepackage{varwidth}


\usepackage{pdflscape}
\usepackage{graphicx}
\usepackage{graphbox}
\graphicspath{
	{Pics/PDFs/}
	{Pics/JPGs/}
	{Pics/PNGs/}
}
\usepackage{caption}
\usepackage{tabularx}
\usepackage{dashrule}
\usepackage{hhline}
\usepackage{multirow}
\usepackage{enumerate}
\usepackage[hidelinks]{hyperref}
\usepackage{listings}

\usepackage[table]{xcolor}
\usepackage{array}
\usepackage{enumitem,amssymb,amsmath}
\usepackage{interval}
\usepackage{stmaryrd}
\usepackage{polynom}
\usepackage{diagbox}
\usepackage{dashrule}
\usepackage{framed}
\usepackage{mdframed}
\usepackage{karnaugh-map}

\usepackage{blindtext}

\usepackage{eso-pic}

\usepackage{amssymb}
\usepackage{eurosym}
\pagestyle{headings}
\renewcommand{\headrulewidth}{0.2pt}
\renewcommand{\footrulewidth}{0.2pt}
\newcommand*{\underdownarrow}[2]{\ensuremath{\underset{\overset{\Big\downarrow}{#2}}{#1}}}
\setlength{\fboxsep}{5pt}

% Codestyle defined
\definecolor{codegreen}{rgb}{0,0.6,0}
\definecolor{codegray}{rgb}{0.5,0.5,0.5}
\definecolor{codepurple}{rgb}{0.58,0,0.82}
\definecolor{backcolour}{rgb}{0.95,0.95,0.92}
\definecolor{deepgreen}{rgb}{0,0.5,0}
\definecolor{darkblue}{rgb}{0,0,0.65}
\definecolor{mauve}{rgb}{0.40, 0.19,0.28}
\colorlet{exceptioncolour}{yellow!50!red}
\colorlet{commandcolour}{blue!60!black}
\colorlet{numpycolour}{blue!60!green}
\colorlet{specmethodcolour}{violet}

%Neue Spaltendefinition
\newcolumntype{L}[1]{>{\raggedright\let\newline\\\arraybackslash\hspace{0pt}}m{#1}}
\newcolumntype{M}[1]{>{\centering\arraybackslash}X}
\newcommand{\cmnt}[1]{\ignorespaces}
%Textausrichtung ändern
\newcommand\tabrotate[1]{\rotatebox{90}{\raggedright#1\hspace{\tabcolsep}}}

%Intervall-Konfig
\intervalconfig {
	soft open fences
}

%Bash
\lstdefinestyle{BashInputStyle}{
	language=bash,
	basicstyle=\small\sffamily,
	backgroundcolor=\color{backcolour},
	columns=fullflexible,
	backgroundcolor=\color{backcolour},
	breaklines=true,
}
%Java
\lstdefinestyle{JavaInputStyle}{
	language=Java,
	backgroundcolor=\color{backcolour},
	aboveskip=1mm,
	belowskip=1mm,
	showstringspaces=false,
	columns=flexible,
	basicstyle={\footnotesize\ttfamily},
	numberstyle={\tiny},
	numbers=none,
	keywordstyle=\color{purple},,
	commentstyle=\color{deepgreen},
	stringstyle=\color{blue},
	emph={out},
	emphstyle=\color{darkblue},
	emph={[2]rand},
	emphstyle=[2]\color{specmethodcolour},
	breaklines=true,
	breakatwhitespace=true,
	tabsize=2,
}
%Python
\lstdefinestyle{PythonInputStyle}{
	language=Python,
	alsoletter={1234567890},
	aboveskip=1ex,
	basicstyle=\footnotesize,
	breaklines=true,
	breakatwhitespace= true,
	backgroundcolor=\color{backcolour},
	commentstyle=\color{red},
	otherkeywords={\ , \}, \{, \&,\|},
	emph={and,break,class,continue,def,yield,del,elif,else,%
		except,exec,finally,for,from,global,if,import,in,%
		lambda,not,or,pass,print,raise,return,try,while,assert},
	emphstyle=\color{exceptioncolour},
	emph={[2]True,False,None,min},
	emphstyle=[2]\color{specmethodcolour},
	emph={[3]object,type,isinstance,copy,deepcopy,zip,enumerate,reversed,list,len,dict,tuple,xrange,append,execfile,real,imag,reduce,str,repr},
	emphstyle=[3]\color{commandcolour},
	emph={[4]ode, fsolve, sqrt, exp, sin, cos, arccos, pi,  array, norm, solve, dot, arange, , isscalar, max, sum, flatten, shape, reshape, find, any, all, abs, plot, linspace, legend, quad, polyval,polyfit, hstack, concatenate,vstack,column_stack,empty,zeros,ones,rand,vander,grid,pcolor,eig,eigs,eigvals,svd,qr,tan,det,logspace,roll,mean,cumsum,cumprod,diff,vectorize,lstsq,cla,eye,xlabel,ylabel,squeeze},
	emphstyle=[4]\color{numpycolour},
	emph={[5]__init__,__add__,__mul__,__div__,__sub__,__call__,__getitem__,__setitem__,__eq__,__ne__,__nonzero__,__rmul__,__radd__,__repr__,__str__,__get__,__truediv__,__pow__,__name__,__future__,__all__},
	emphstyle=[5]\color{specmethodcolour},
	emph={[6]assert,range,yield},
	emphstyle=[6]\color{specmethodcolour}\bfseries,
	emph={[7]Exception,NameError,IndexError,SyntaxError,TypeError,ValueError,OverflowError,ZeroDivisionError,KeyboardInterrupt},
	emphstyle=[7]\color{specmethodcolour}\bfseries,
	emph={[8]taster,send,sendMail,capture,check,noMsg,go,move,switch,humTem,ventilate,buzz},
	emphstyle=[8]\color{blue},
	keywordstyle=\color{blue}\bfseries,
	rulecolor=\color{black!40},
	showstringspaces=false,
	stringstyle=\color{deepgreen}
}

\lstset{literate=%
	{Ö}{{\"O}}1
	{Ä}{{\"A}}1
	{Ü}{{\"U}}1
	{ß}{{\ss}}1
	{ü}{{\"u}}1
	{ä}{{\"a}}1
	{ö}{{\"o}}1
}

% Neue Klassenarbeits-Umgebung
\newenvironment{worksheet}[3]
% Begin-Bereich
{
	\newpage
	\sffamily
	\setcounter{page}{1}
	\ClearShipoutPicture
	\AddToShipoutPicture{
		\put(55,761){{
				\mbox{\parbox{385\unitlength}{\tiny \color{codegray}BBS I Mainz, #1 \newline #2
						\newline #3
					}
				}
			}
		}
		\put(455,761){{
				\mbox{\hspace{0.3cm}\includegraphics[width=0.2\textwidth]{../../logo.jpg}}
			}
		}
	}
}
% End-Bereich
{
	\clearpage
	\ClearShipoutPicture
}

\geometry{left=1.50cm,right=1.50cm,top=3.00cm,bottom=1.00cm,includeheadfoot}

\begin{document}
	\begin{worksheet}{Berufliches Gymnasium}{Klassenstufe 12 - Informationsverarbeitung - Grundkurs}{Lernabschnitt 1: Grundlagen - Mathematische Operatoren}
				
		\noindent
		\sffamily
		\begin{framed}
			\noindent
			\textbf{Aufgabe 1:} Was wird ausgegeben?\\
			\par\noindent
			\begin{tabularx}{\textwidth}{|l|X|}
				\hline
				{\lstinline[style=JavaInputStyle]|System.out.print("Pi ist " + 3.14);|} & \\
				\hline
				{\lstinline[style=JavaInputStyle]|System.out.print(3*5 - 12.5);|} & \\
				\hline
				{\lstinline[style=JavaInputStyle]|System.out.print("Hallo" + " " + "Welt");|} & \\
				\hline
				{\lstinline[style=JavaInputStyle]|int a = 12;|} & \\
				{\lstinline[style=JavaInputStyle]|System.out.print("Der Wert von a ist " + a + "km.");|} & \\
				\hline
			\end{tabularx}\\
			\par\noindent
			\textbf{Aufgabe 2:} Entscheiden Sie unter Berücksichtigung der folgenden Programmzeilen, welche Aussagen wahr sind.\\
			\par\noindent
			\begin{minipage}{0.25\textwidth}
				{\lstinline[style=JavaInputStyle]|int a,b;|}\\
				{\lstinline[style=JavaInputStyle]|b = 5;|}\\
			\end{minipage}
			\hfill
			\begin{minipage}{0.7\textwidth}
				\begin{tabularx}{0.9\textwidth}{cl}
					\rowcolor{gray!10} \fbox{} & a) Variable a ist nicht deklariert.\\
					\rowcolor{gray!20} \fbox{} & b) Variable b ist nicht deklariert.\\
					\rowcolor{gray!10} \fbox{} & c) Variable a ist deklariert, aber nicht initialisiert.\\
					\rowcolor{gray!20} \fbox{} & d) Variable b ist deklariert, aber nicht initialisiert.\\
					\rowcolor{gray!10} \fbox{} & e) Variable b ist initialisiert, aber nicht deklariert.\\
				\end{tabularx}
			\end{minipage}\\
			\par\noindent
			\textbf{Aufgabe 3} Wie werden die Ausdrücke ausgewertet? Von welchen Datentypen sind die Ergebnisse?\\
			\texttt{Klammern haben die gleiche Bedeutung wie in der Mathematik.}\\
			\par\noindent
			\begin{tabularx}{\textwidth}{|l|X|X|}
				\hline
				& \textbf{Ergebnis} & \textbf{Datentyp}\\
				\hline
				\hline
				{\lstinline[style=JavaInputStyle]|23.4 + 7|} & {\lstinline[style=JavaInputStyle]|30.4|} & {\lstinline[style=JavaInputStyle]|double|}\\
				\hline
				{\lstinline[style=JavaInputStyle]|30 - 5|} & & \\
				\hline
				{\lstinline[style=JavaInputStyle]|(10/3) + 0.5|} & & \\
				\hline
				{\lstinline[style=JavaInputStyle]|'a' == 'b'|} & & \\
				\hline
				{\lstinline[style=JavaInputStyle]|\"ab\" \!= \"cd\"|} & & \\
				\hline
				{\lstinline[style=JavaInputStyle]|6.6 / 3.3|} & & \\
				\hline
				{\lstinline[style=JavaInputStyle]|10 / 4 == 2|} & & \\
				\hline
				{\lstinline[style=JavaInputStyle]|1 / 3 * 1234567891234|} & & \\
				\hline
			\end{tabularx}\\
			\par\noindent
			\textbf{Aufgabe 4} Welche der folgenden Aussagen zu Operatoren sind richtig?\\
			Kreuzen Sie die richtige(n) Aussage(n) an.\\
			\par\noindent
			\begin{tabularx}{0.9\textwidth}{cl}
				\fbox{} & Der Operator + kann als unärer Operator eingesetzt werden.\\
				\fbox{} & Mehrere Operatoren in einem Ausdruck werden in Java immer von links nach rechts ausgewertet.\\
				\fbox{} & Das Ergebnis des Ausdrucks {\lstinline[style=JavaInputStyle]|5 / 2|} ist vom Datentyp {\lstinline[style=JavaInputStyle]|int|}.\\
				\fbox{} & Das Ergebnis des Ausdrucks {\lstinline[style=JavaInputStyle]|3.0 + 4|} ist eine ganze Zahl vom Datentyp {\lstinline[style=JavaInputStyle]|int|}.\\
				\fbox{} & {\lstinline[style=JavaInputStyle]|\& \&|} ergibt wahr, wenn beide Operatoren wahr sind.\\
				\fbox{} & Das Ergebnis des Ausdrucks {\lstinline[style=JavaInputStyle]|5 / 2|} ist eine Gleitkommazahl.\\
				\fbox{} & Der Operator {\lstinline[style=JavaInputStyle]|==|} hat als Ergebnis einen Wahrheitswert.\\
			\end{tabularx}
			\newpage
			\noindent
			\textbf{Aufgabe 1} [Flaeche.java]\\
			\textit{Die Fläche F eines Kreises berechnet sich bei gegebenem Radius \(r\) mit \(F = r^2\cdot\pi\).}\\
			Entwickeln Sie ein Java-Programm, das zu einem eingegebenen Radius \(r\) die Kreisfläche \(F\) berechnet und diese auf dem Bildschirm ausgibt!\\
			\(r\) soll im Programm als Konstante deklariert werden.\\
			\par\noindent
			\textbf{Aufgabe 2} [Rechteck.java]\\
			Schreiben Sie ein Programm, das zunächst nach der Eingabe von Länge und Breite eines Rechtecks fragt und anschließend den Flächeninhalt und den Umfang ausgibt.\\
			\par\noindent
			\textbf{Aufgabe 3} [Produkt.java]\\
			Entwerfen Sie ein Java-Programm \lstinline[style=JavaInputStyle]|Produkt.java|, das vier Zahlvariablen \lstinline[style=JavaInputStyle]|zahl1|, \lstinline[style=JavaInputStyle]|zahl2|, \lstinline[style=JavaInputStyle]|zahl3| und \lstinline[style=JavaInputStyle]|zahl4| deklariert, mit den Werten \(45,\ 25,\ 39\) und \(5\) initialisiert.\\
			Das Produkt der vier Zahlen soll gebildet und anschließend ausgegeben werden.\\
			\par\noindent
			\textbf{Aufgabe 4} [Schwimmbad.java]\\
			Schreiben Sie ein Java-Programm, das die Menge an Wasser in einem rechteckigen Schwimmbecken (mit konstanter Wassertiefe) passt.\\
			Der Anwender soll Tiefe, Breite und Länge eingeben können. Die Ausgabe der Wassermengen erfolgt in Litern.\\
			\par\noindent
			\textbf{Aufgabe 5} [Meilen.java]\\
			Schreiben Sie ein Java-Programm, das Meilen in Kilometer umrechnet.\\
			Der Benutzer gibt nach Aufforderung eine Strecke in Meilen ein. Das Programm berechnet daraus den entsprechenden Wert in Kilometern und gibt diesen aus.\\
			Nutzen Sie die Formel: \(km = mi/0.62137\).\\
			\par\noindent
			\textbf{Aufgabe 6} [Spritverbrauch.java]\\
			Schreiben Sie ein Programm, das anhand der folgenden Eingaben die Spritkosten berechnet.\\
			Nutzereingabe:
			\begin{itemize}[label=-]
				\itemsep0em
				\item aktueller Spritpreis
				\item durchschnittlicher Spritverbrauch pro 100 km
				\item Anzahl der zu fahrenden Kilometer
			\end{itemize}
			\textbf{Aufgabe 7} [Bremsweg.java]\\
			Der minimale Bremsweg eines Autos bei einer Vollbremsung berechnet sich nach der Näherungsformel: \(Bremsweg = \frac{[Geschwindigkeit in km/h]^2}{10}\).\\
			Schreiben Sie ein Java-Programm, das Geschwindigkeit vom Benutzer einliest und mit Hilfe der obigen Formel den Bremsweg berechnet und anschließend in einem sinnvollen Satz auf dem Bildschirm ausgibt.
		\end{framed}
	\end{worksheet}
\end{document}