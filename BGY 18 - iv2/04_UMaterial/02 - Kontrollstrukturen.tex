\documentclass[oneside,openany,headings=optiontotoc,11pt,numbers=noenddot]{scrreprt}

\usepackage[a4paper]{geometry}
\usepackage[utf8]{inputenc}
\usepackage[T1]{fontenc}
\usepackage{lmodern}
\usepackage[ngerman]{babel}
\usepackage{ngerman}

\usepackage[onehalfspacing]{setspace}

\usepackage{fancyhdr}
\usepackage{fancybox}

\usepackage{rotating}
\usepackage{varwidth}


\usepackage{pdflscape}
\usepackage{graphicx}
\usepackage{graphbox}
\graphicspath{
	{Pics/PDFs/}
	{Pics/JPGs/}
	{Pics/PNGs/}
}
\usepackage{caption}
\usepackage{tabularx}
\usepackage{dashrule}
\usepackage{hhline}
\usepackage{multirow}
\usepackage{enumerate}
\usepackage[hidelinks]{hyperref}
\usepackage{listings}

\usepackage[table]{xcolor}
\usepackage{array}
\usepackage{enumitem,amssymb,amsmath}
\usepackage{interval}
\usepackage{stmaryrd}
\usepackage{polynom}
\usepackage{diagbox}
\usepackage{dashrule}
\usepackage{framed}
\usepackage{mdframed}
\usepackage{karnaugh-map}

\usepackage{blindtext}

\usepackage{eso-pic}

\usepackage{amssymb}
\usepackage{eurosym}
\pagestyle{headings}
\renewcommand{\headrulewidth}{0.2pt}
\renewcommand{\footrulewidth}{0.2pt}
\newcommand*{\underdownarrow}[2]{\ensuremath{\underset{\overset{\Big\downarrow}{#2}}{#1}}}
\setlength{\fboxsep}{5pt}

% Codestyle defined
\definecolor{codegreen}{rgb}{0,0.6,0}
\definecolor{codegray}{rgb}{0.5,0.5,0.5}
\definecolor{codepurple}{rgb}{0.58,0,0.82}
\definecolor{backcolour}{rgb}{0.95,0.95,0.92}
\definecolor{deepgreen}{rgb}{0,0.5,0}
\definecolor{darkblue}{rgb}{0,0,0.65}
\definecolor{mauve}{rgb}{0.40, 0.19,0.28}
\colorlet{exceptioncolour}{yellow!50!red}
\colorlet{commandcolour}{blue!60!black}
\colorlet{numpycolour}{blue!60!green}
\colorlet{specmethodcolour}{violet}

%Neue Spaltendefinition
\newcolumntype{L}[1]{>{\raggedright\let\newline\\\arraybackslash\hspace{0pt}}m{#1}}
\newcolumntype{M}[1]{>{\centering\arraybackslash}X}
\newcommand{\cmnt}[1]{\ignorespaces}
%Textausrichtung ändern
\newcommand\tabrotate[1]{\rotatebox{90}{\raggedright#1\hspace{\tabcolsep}}}

%Intervall-Konfig
\intervalconfig {
	soft open fences
}

%Bash
\lstdefinestyle{BashInputStyle}{
	language=bash,
	basicstyle=\small\sffamily,
	backgroundcolor=\color{backcolour},
	columns=fullflexible,
	backgroundcolor=\color{backcolour},
	breaklines=true,
}
%Java
\lstdefinestyle{JavaInputStyle}{
	language=Java,
	backgroundcolor=\color{backcolour},
	aboveskip=1mm,
	belowskip=1mm,
	showstringspaces=false,
	columns=flexible,
	basicstyle={\footnotesize\ttfamily},
	numberstyle={\tiny},
	numbers=none,
	keywordstyle=\color{purple},,
	commentstyle=\color{deepgreen},
	stringstyle=\color{blue},
	emph={out},
	emphstyle=\color{darkblue},
	emph={[2]rand},
	emphstyle=[2]\color{specmethodcolour},
	breaklines=true,
	breakatwhitespace=true,
	tabsize=2,
}
%Python
\lstdefinestyle{PythonInputStyle}{
	language=Python,
	alsoletter={1234567890},
	aboveskip=1ex,
	basicstyle=\footnotesize,
	breaklines=true,
	breakatwhitespace= true,
	backgroundcolor=\color{backcolour},
	commentstyle=\color{red},
	otherkeywords={\ , \}, \{, \&,\|},
	emph={and,break,class,continue,def,yield,del,elif,else,%
		except,exec,finally,for,from,global,if,import,in,%
		lambda,not,or,pass,print,raise,return,try,while,assert},
	emphstyle=\color{exceptioncolour},
	emph={[2]True,False,None,min},
	emphstyle=[2]\color{specmethodcolour},
	emph={[3]object,type,isinstance,copy,deepcopy,zip,enumerate,reversed,list,len,dict,tuple,xrange,append,execfile,real,imag,reduce,str,repr},
	emphstyle=[3]\color{commandcolour},
	emph={[4]ode, fsolve, sqrt, exp, sin, cos, arccos, pi,  array, norm, solve, dot, arange, , isscalar, max, sum, flatten, shape, reshape, find, any, all, abs, plot, linspace, legend, quad, polyval,polyfit, hstack, concatenate,vstack,column_stack,empty,zeros,ones,rand,vander,grid,pcolor,eig,eigs,eigvals,svd,qr,tan,det,logspace,roll,mean,cumsum,cumprod,diff,vectorize,lstsq,cla,eye,xlabel,ylabel,squeeze},
	emphstyle=[4]\color{numpycolour},
	emph={[5]__init__,__add__,__mul__,__div__,__sub__,__call__,__getitem__,__setitem__,__eq__,__ne__,__nonzero__,__rmul__,__radd__,__repr__,__str__,__get__,__truediv__,__pow__,__name__,__future__,__all__},
	emphstyle=[5]\color{specmethodcolour},
	emph={[6]assert,range,yield},
	emphstyle=[6]\color{specmethodcolour}\bfseries,
	emph={[7]Exception,NameError,IndexError,SyntaxError,TypeError,ValueError,OverflowError,ZeroDivisionError,KeyboardInterrupt},
	emphstyle=[7]\color{specmethodcolour}\bfseries,
	emph={[8]taster,send,sendMail,capture,check,noMsg,go,move,switch,humTem,ventilate,buzz},
	emphstyle=[8]\color{blue},
	keywordstyle=\color{blue}\bfseries,
	rulecolor=\color{black!40},
	showstringspaces=false,
	stringstyle=\color{deepgreen}
}

\lstset{literate=%
	{Ö}{{\"O}}1
	{Ä}{{\"A}}1
	{Ü}{{\"U}}1
	{ß}{{\ss}}1
	{ü}{{\"u}}1
	{ä}{{\"a}}1
	{ö}{{\"o}}1
}

% Neue Klassenarbeits-Umgebung
\newenvironment{worksheet}[3]
% Begin-Bereich
{
	\newpage
	\sffamily
	\setcounter{page}{1}
	\ClearShipoutPicture
	\AddToShipoutPicture{
		\put(55,761){{
				\mbox{\parbox{385\unitlength}{\tiny \color{codegray}BBS I Mainz, #1 \newline #2
						\newline #3
					}
				}
			}
		}
		\put(455,761){{
				\mbox{\hspace{0.3cm}\includegraphics[width=0.2\textwidth]{../../logo.jpg}}
			}
		}
	}
}
% End-Bereich
{
	\clearpage
	\ClearShipoutPicture
}

\geometry{left=1.50cm,right=1.50cm,top=3.00cm,bottom=1.00cm,includeheadfoot}

\begin{document}
	\begin{worksheet}{Informationsverarbeitung}{Lernabschnitt: Strukturiert Programmieren}{Kontrollstrukturen}
		\noindent
		\sffamily
		\begin{framed}
			\noindent
			\textbf{Aufgabe 6:} \underline{Kleinste von vier Zahlen}\\
			Erstellen Sie ein Programm, das von vier eingegebenen Zahlen die Kleinste ermittelt und ausgibt!\\
			\par\noindent
			\textbf{Aufgabe 7:} \underline{Division zweier Zahlen}\\
			Erstellen Sie ein Programm, das folgende Aufgaben erfüllt:\\
			Es sind zwei Zahlen einzugeben. Das Ergebnis der Division der größeren Zahl durch die kleinere Zahl ist auszugeben.\\
			Beachten Sie, dass die Division durch Null nicht erlaubt ist.\\
			\par\noindent
			\textbf{Aufgabe 8:} \underline{Satz des Pythagoras}\\
			Der Satz des Pythagoras definiert die Beziehung der Seitenlängen in einem rechtwinkligen Dreieck zu \(a^2 + b^2 = c^2\). Die Formel für die Berechnung der Hypotenuse bei gegebenen Katheten ist demnach \lstinline[style=JavaInputStyle]|c = Math.sqrt(a*a + b*b)|. Es ist eine bemerkenswerte Tatsache, dass die Hypotenusenlänge auch berechnet werden kann, ohne explizit die Wurzel zu ziehen.\\
			Das folgende Verfahren wurde 1983 von C. Moler und D. Morrison beschrieben. Zunächst berechnen Sie:\\
			\lstinline[style=JavaInputStyle]|c = maximum(a,b);|\\
			\lstinline[style=JavaInputStyle]|q = minimum(a,b);|\\
			Dann führen Sie drei Iterationen der folgenden Anweisungen aus:\\
			\lstinline[style=JavaInputStyle]|r = (q/c)^2;|\\
			\lstinline[style=JavaInputStyle]|s = r / (4.0 + r);|\\
			\lstinline[style=JavaInputStyle]|c = c + (2.0 * s*c);|\\
			\lstinline[style=JavaInputStyle]|q = s*q;|\\
			Bereits nach nur drei Iterationen findet sich eine sehr gute Annäherung der Hypotenuse c.\\
			Schreiben Sie ein Programm, das dieses Verfahren implementiert und überprüfen Sie die Korrektheit.\\
			(\underline{Hinweis:} Für die Bestimmung des Maximums der Zahlen a und b können Sie die Anweisung \lstinline[style=JavaInputStyle]|Math.max(a,b)| und für das Minimum die Anweisung \lstinline[style=JavaInputStyle]|Math.min(a,b)| nutzen.)\\
			\par\noindent
			\textbf{Aufgabe 9:} \underline{Summe von Zahlen}\\
			Schreiben Sie ein Programm zur Berechnung der Summe der Zahlen von 1 bis n. Ermitteln Sie das Resultat durch Summierung innerhalb einer Schleife.\\
			\par\noindent
			\textbf{Aufgabe 10:} \underline{Mittelwert}\\
			Schreiben Sie ein Programm, das den Mittelwert von beliebig vielen einzugebenden Zahlen berechnet.\\
			\newpage
			\noindent
			\textbf{Aufgabe 11:} \underline{Tabelle Quadrat- und Kubikzahlen}\\
			Schreiben Sie ein Programm, das folgende Tabelle von Quadrat- und Kubikzahlen ausgibt:\\
			\begin{tabular}{ccc}
				n & \(n^2\) & \(n^3\)\\
				\hline
				1 & 1 & 1\\
				2 & 4 & 8\\
				3 & 9 & 27
			\end{tabular}\\
			\par\noindent
			\textbf{Aufgabe 12:} \underline{Quadratwurzel - do-while-Schleife}\\
			Schreiben Sie ein Programm, welches die die Quadratwurzel einer natürlichen Zahl annähert. Benutzen Sie dazu die Folge von Heron:\\
			\(X_{n+1} = \frac{1}{2}*(X_n+\frac{Zahl}{X_n})\)\\
			Das Programm soll abbrechen, wenn der Unterschied zum vorherigen Näherungswert kleiner als \(0,000 000 000 000 001\) wird. Der Wert \(X_0\) ist die Zahl selbst.\\
			\par\noindent
			\textbf{Aufgabe 13:} \underline{switch-case}\\
			Schreiben Sie ein Programm, um nach Eingabe einer Schulnote die Bewertung in Textform auszugeben. Wird eine Note größer als 6 eingegeben, soll eine Meldung ausgegeben werden.\\
			Die Bewertung lauten:\\
			\begin{tabular}{lll}
				1: Sehr Gut & 2: Gut & 3: Befriedigend\\
				4: Ausreichend & 5: Mangelhaft & 6: Ungenügend
			\end{tabular}\\
			\par\noindent
			\textbf{Aufgabe 14:} \underline{Fallunterscheidungen}\\
			Schreiben Sie ein Programm, das zwei Zahlen einliest und nach Wahl eine Berechnung folgender Art durchführt:\\
			\begin{tabular}{ll}
				1: Addition & 2: Subtraktion\\
				3: Multiplikation & 4: Division
			\end{tabular}\\
			\par\noindent
			\textbf{Aufgabe 15:} \underline{Teiler einer Zahl}\\
			Geben Sie zu einer einzugebenden Zahl alle Teiler dieser Zahl aus. Realisieren Sie die Lösung sowohl mit einer \lstinline[style=JavaInputStyle]|for|- als auch mit einer \lstinline[style=JavaInputStyle]|while|-Schleife.
		\end{framed}
	\end{worksheet}
\end{document}