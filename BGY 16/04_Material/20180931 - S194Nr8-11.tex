\documentclass[oneside,openany,headings=optiontotoc,11pt,numbers=noenddot]{scrreprt}

\usepackage[a4paper]{geometry}
\usepackage[utf8]{inputenc}
\usepackage[T1]{fontenc}
\usepackage{lmodern}
\usepackage[ngerman]{babel}
\usepackage{ngerman}

\usepackage[onehalfspacing]{setspace}

\usepackage{fancyhdr}
\usepackage{fancybox}

\usepackage{rotating}
\usepackage{varwidth}


\usepackage{pdflscape}
\usepackage{graphicx}
\usepackage{graphbox}
\graphicspath{
	{Pics/PDFs/}
	{Pics/JPGs/}
	{Pics/PNGs/}
}
\usepackage{caption}
\usepackage{tabularx}
\usepackage{dashrule}
\usepackage{hhline}
\usepackage{multirow}
\usepackage{enumerate}
\usepackage[hidelinks]{hyperref}
\usepackage{listings}

\usepackage[table]{xcolor}
\usepackage{array}
\usepackage{enumitem,amssymb,amsmath}
\usepackage{interval}
\usepackage{stmaryrd}
\usepackage{polynom}
\usepackage{diagbox}
\usepackage{dashrule}
\usepackage{framed}
\usepackage{mdframed}
\usepackage{karnaugh-map}

\usepackage{blindtext}

\usepackage{eso-pic}

\usepackage{amssymb}
\usepackage{eurosym}
\pagestyle{headings}
\renewcommand{\headrulewidth}{0.2pt}
\renewcommand{\footrulewidth}{0.2pt}
\newcommand*{\underdownarrow}[2]{\ensuremath{\underset{\overset{\Big\downarrow}{#2}}{#1}}}
\setlength{\fboxsep}{5pt}

% Codestyle defined
\definecolor{codegreen}{rgb}{0,0.6,0}
\definecolor{codegray}{rgb}{0.5,0.5,0.5}
\definecolor{codepurple}{rgb}{0.58,0,0.82}
\definecolor{backcolour}{rgb}{0.95,0.95,0.92}
\definecolor{deepgreen}{rgb}{0,0.5,0}
\definecolor{darkblue}{rgb}{0,0,0.65}
\definecolor{mauve}{rgb}{0.40, 0.19,0.28}
\colorlet{exceptioncolour}{yellow!50!red}
\colorlet{commandcolour}{blue!60!black}
\colorlet{numpycolour}{blue!60!green}
\colorlet{specmethodcolour}{violet}

%Neue Spaltendefinition
\newcolumntype{L}[1]{>{\raggedright\let\newline\\\arraybackslash\hspace{0pt}}m{#1}}
\newcolumntype{M}[1]{>{\centering\arraybackslash}X}
\newcommand{\cmnt}[1]{\ignorespaces}
%Textausrichtung ändern
\newcommand\tabrotate[1]{\rotatebox{90}{\raggedright#1\hspace{\tabcolsep}}}

%Intervall-Konfig
\intervalconfig {
	soft open fences
}

%Bash
\lstdefinestyle{BashInputStyle}{
	language=bash,
	basicstyle=\small\sffamily,
	backgroundcolor=\color{backcolour},
	columns=fullflexible,
	backgroundcolor=\color{backcolour},
	breaklines=true,
}
%Java
\lstdefinestyle{JavaInputStyle}{
	language=Java,
	backgroundcolor=\color{backcolour},
	aboveskip=1mm,
	belowskip=1mm,
	showstringspaces=false,
	columns=flexible,
	basicstyle={\footnotesize\ttfamily},
	numberstyle={\tiny},
	numbers=none,
	keywordstyle=\color{purple},,
	commentstyle=\color{deepgreen},
	stringstyle=\color{blue},
	emph={out},
	emphstyle=\color{darkblue},
	emph={[2]rand},
	emphstyle=[2]\color{specmethodcolour},
	breaklines=true,
	breakatwhitespace=true,
	tabsize=2,
}
%Python
\lstdefinestyle{PythonInputStyle}{
	language=Python,
	alsoletter={1234567890},
	aboveskip=1ex,
	basicstyle=\footnotesize,
	breaklines=true,
	breakatwhitespace= true,
	backgroundcolor=\color{backcolour},
	commentstyle=\color{red},
	otherkeywords={\ , \}, \{, \&,\|},
	emph={and,break,class,continue,def,yield,del,elif,else,%
		except,exec,finally,for,from,global,if,import,in,%
		lambda,not,or,pass,print,raise,return,try,while,assert},
	emphstyle=\color{exceptioncolour},
	emph={[2]True,False,None,min},
	emphstyle=[2]\color{specmethodcolour},
	emph={[3]object,type,isinstance,copy,deepcopy,zip,enumerate,reversed,list,len,dict,tuple,xrange,append,execfile,real,imag,reduce,str,repr},
	emphstyle=[3]\color{commandcolour},
	emph={[4]ode, fsolve, sqrt, exp, sin, cos, arccos, pi,  array, norm, solve, dot, arange, , isscalar, max, sum, flatten, shape, reshape, find, any, all, abs, plot, linspace, legend, quad, polyval,polyfit, hstack, concatenate,vstack,column_stack,empty,zeros,ones,rand,vander,grid,pcolor,eig,eigs,eigvals,svd,qr,tan,det,logspace,roll,mean,cumsum,cumprod,diff,vectorize,lstsq,cla,eye,xlabel,ylabel,squeeze},
	emphstyle=[4]\color{numpycolour},
	emph={[5]__init__,__add__,__mul__,__div__,__sub__,__call__,__getitem__,__setitem__,__eq__,__ne__,__nonzero__,__rmul__,__radd__,__repr__,__str__,__get__,__truediv__,__pow__,__name__,__future__,__all__},
	emphstyle=[5]\color{specmethodcolour},
	emph={[6]assert,range,yield},
	emphstyle=[6]\color{specmethodcolour}\bfseries,
	emph={[7]Exception,NameError,IndexError,SyntaxError,TypeError,ValueError,OverflowError,ZeroDivisionError,KeyboardInterrupt},
	emphstyle=[7]\color{specmethodcolour}\bfseries,
	emph={[8]taster,send,sendMail,capture,check,noMsg,go,move,switch,humTem,ventilate,buzz},
	emphstyle=[8]\color{blue},
	keywordstyle=\color{blue}\bfseries,
	rulecolor=\color{black!40},
	showstringspaces=false,
	stringstyle=\color{deepgreen}
}

\lstset{literate=%
	{Ö}{{\"O}}1
	{Ä}{{\"A}}1
	{Ü}{{\"U}}1
	{ß}{{\ss}}1
	{ü}{{\"u}}1
	{ä}{{\"a}}1
	{ö}{{\"o}}1
}

% Neue Klassenarbeits-Umgebung
\newenvironment{worksheet}[3]
% Begin-Bereich
{
	\newpage
	\sffamily
	\setcounter{page}{1}
	\ClearShipoutPicture
	\AddToShipoutPicture{
		\put(55,761){{
				\mbox{\parbox{385\unitlength}{\tiny \color{codegray}BBS I Mainz, #1 \newline #2
						\newline #3
					}
				}
			}
		}
		\put(455,761){{
				\mbox{\hspace{0.3cm}\includegraphics[width=0.2\textwidth]{../../logo.jpg}}
			}
		}
	}
}
% End-Bereich
{
	\clearpage
	\ClearShipoutPicture
}

\geometry{left=2.50cm,right=2.50cm,top=3.00cm,bottom=1.00cm,includeheadfoot}

\begin{document}
	\begin{worksheet}{BGY 16}{Klassenstufe 13 - Mathematik}{Lernabschnitt 1: Kettenregel und Produktregel}
				
		\noindent
		\sffamily
		\begin{framed}
			\noindent
			Wir erinnern uns an die \textbf{Potenzregeln}, welche in beide Richtungen angewendet werden können.\\
			\par\noindent
			\(x\in\mathbb{R}, n,m\in\mathbb{N}\)\\
			\begin{align*}
				& x^n\cdot{}x^m = x^{n+m}\\
				& \frac{x^n}{x^m} = x^{n-m}\\
				& \left(x^n\right)^m = x^{n\cdot{}m}\\
				& \frac{1}{x^n} = x^{-n}\\
				& \sqrt[m]{x^n} = x^{\frac{n}{m}}
			\end{align*}
		\end{framed}
		\begin{framed}
			\noindent
			\small{\color{codegray}\underline{S. 194 \textbf{Aufgabe 8}:} Schreiben Sie den Funktionsterm der Funktion \(f\) in der Form \(f(x) = c\cdot{}a^x\).}\\
			\par
			\begin{tabularx}{\textwidth}{l|l|l|l}
				(a) \(f(x)=3^{2x+3}\) & (b) \(f(x) = 16^{2x+0,5}\) & (c) \(f(x) = \frac{1}{2^{1+x}}\) & (d) \(f(x) = \frac{1}{2^{x-1}}\)\\
				& & & \\
				\(= 3^{2x}\cdot{}3^3\) & \(=16^{2x}\cdot{}16^{\frac{1}{2}}\) & \(= 2^{-(1+x)} = 2^{-1-x}\) & \(= 2^{-(x-1) = 2^{-x+1}}\)\\
				& & & \\
				\(= (3^2)^x\cdot{}27\) & \(= (16^2)^x\cdot{}\sqrt{16}\) & \(=\frac{1}{2}\cdot{}(2^{-1})^x\) & \(=2\cdot{}(2^{-1})^x\)\\
				& & & \\
				\(\Rightarrow f(x) = 27\cdot{}9^x\) & \(\Rightarrow f(x) = 4\cdot{}256^x\) & \(\Rightarrow f(x) = \frac{1}{2}\cdot{}\frac{1}{2}^x\) & \(\Rightarrow f(x) = 2\cdot{}\frac{1}{2}^x\)\\
				& & & \\
				\hline
				& & & \\
				(e) \(f(x) = \left(\frac{1}{2}\right)^{x-2}\) & (f) \(f(x) = 3^{\frac{1}{3}x-3}\) & (g) \(f(x)= \left(\frac{1}{4}\right)^{\frac{1}{4}x-\frac{1}{4}}\) & (h) \(f(x) = \frac{48}{4^{-0,5x+2}}\)\\
				& & & \\
				\(= \frac{\frac{1}{2}^x}{\frac{1}{2}^2} = \frac{1}{2}^x\cdot{}2^2\) & \(= \frac{3^{\frac{1}{3}x}}{3^3}= \frac{1}{3^3}\cdot{}\left(3^\frac{1}{3}\right)^x\) & \(= \frac{\frac{1}{4}^{\frac{1}{4}x}}{\frac{1}{4}^{\frac{1}{4}}} = \left(\frac{1}{4}^{\frac{1}{4}}\right)^x\cdot{}4^{\frac{1}{4}}\) & \(= 48\cdot4^{-(-0,5x+2)}\)\\
				& & & \\
				\(\Rightarrow f(x) = 4\cdot{}\frac{1}{2}^x\) & \(\Rightarrow f(x)= \frac{1}{27}\cdot{}\sqrt[3]{3}^x\) & \(\Rightarrow f(x)= \sqrt[4]{4}\cdot{}\sqrt[4]{\frac{1}{4}}^x\) & \(= 48\cdot{}4^{0,5x-2}\)\\
				& & & \\
				& & & \(= 48\cdot{}4^{0,5x}\cdot4^{-2}\)\\
				& & & \\
				& & & \(= 48\cdot{}\frac{1}{16}\cdot{}(\sqrt{4})^x\)\\
				& & & \\
				& & & \(\Rightarrow f(x) = 3\cdot{}2^x\)
			\end{tabularx}
		\end{framed}
		\newpage
		\begin{framed}
			\noindent
			Der \textbf{Logarithmus} \(\log_b(a)\), gelesen als \textit{Logarithmus von \textbf{a} zur Basis \textbf{b}} gibt uns den Exponenten \(e\) von b, so dass \( b^e = a\).\\
			\par\bigskip\noindent
			Zudem erinnern wir uns, dass bei Gleichungen eine Operation (in unserem Fall \(log_b\)) immer auf beide Seiten der Gleichung angewendet werden muss.\\
			\[b^x = a^y\ \ |log_b\]
			\[x = log_b(a^y)\]
			Es ist also sinnvoll, die \textbf{gleiche Basis} auf beiden Seiten der Gleichung zu haben.
		\end{framed}
		\begin{framed}
			\noindent
			\small{\color{codegray}\underline{S. 194 \textbf{Aufgabe 11}:} Lösen Sie die Gleichungen.}\\
			\par
			\begin{tabularx}{\textwidth}{Xl|Xl|Xl}
				(a) \(5^x = 125\) & |\(log_5\) & (b) \(5^x = \frac{1}{25}\) & |\(log_5\) & (c) \(5^x = 625\) & |\(log_5\)\\
				& & & & & \\
				\(x = 3\) & & \(x = -2\) & & \(x = 4\)\\
				& & & & & \\
				\hline
				& & & & & \\
				(d) \(3^{x-1} = 9\) & |\(log_3\) & (e) \(3^{x+2} = 3^{2x}\) & |\(log_3\) & (f) \(0,5^x=2\) & \\
				& & & & & \\
				\(x-1 = 2\) & |\(+1\) & \(x+2 = 2x\) & |\(-x\) & \(0,5^x = 0,5^{-4}\) & |\(log_{0,5}\)\\
				& & & & & \\
				\(x = 3\) & & \(x = 2\) & & \(x = -4\)\\
				& & & & & \\
				\hline
				& & & & & \\
				(g) \(2^{3x-4} = 8\) & & (h) \(3\cdot\left(\frac{1}{3}\right)^{3x+2} = \frac{1}{27}\) & |\(:3\) & (i) \(\frac{1}{16}\cdot{}4^{\frac{1}{2}x-2} = 2^{3x}\)\\
				& & & & & \\
				\(2^{3x-4} = 2^3\) & |\(log_2\) & \(\left(\frac{1}{3}\right)^{3x+2} = \left(\frac{1}{3}\right)^4\) & |\(log_\frac{1}{3}\) & \(\frac{1}{16}\cdot{}\frac{4^{\frac{1}{2}x}}{4^2} = 2^{3x}\)\\
				& & & & & \\
				\(3x-4 = 3\) & |\(+4\) & \(3x+2 = 4\) & |\(-2\) & \(\frac{1}{2^4}\cdot{}2^x\cdot{}\frac{1}{2^4} = 2^{3x}\)\\
				& & & & & \\
				\(3x = 7\) & |\(:3\) & \(3x = 2\) & |\(:3\) & \(2^{x-8} = 2^{3x}\) & \(log_2\)\\
				& & & & & \\
				\(x = \frac{7}{3}\) & & \(x = \frac{2}{3}\) & & \(x-8 = 3x\) & |\(-x\)\\
				& & & & & \\
				& & & & \(-8 = 2x\) & |\(:2\)\\
				& & & & & \\
				& & & & \(x = -4\)\\
			\end{tabularx}
			\begin{tabularx}{\textwidth}{Xl|Xl|Xl}
				(j) \(\left(\frac{2}{3}\right)^{x-1} = \left(\frac{8}{27}\right)^{x+2}\) & & \(0,5^x = 2^{x+1}\) & & \(2\cdot{}0,25^x = 4x\) & \\
				& & & & & \\
				\(\left(\frac{2}{3}\right)^{x-1} = \left[\left(\frac{2}{3}\right)^3\right]^{x+2}\) & & \(2^{-x} = 2^{x+1}\) & |\(log_2\) & \(2^{1-2x} = \left(2^2\right)^x\)\\
				& & & & & \\
				\(\left(\frac{2}{3}\right)^{x-1} = \left(\frac{2}{3}\right)^{3(x+2)}\) & |\(log_\frac{2}{3}\) & \(-x = x+1\) & |\(-x\) & \(2^{1-2x} = 2^{2x}\) & |\(log_2\)\\
				& & & & & \\
				\(x-1 = 3(x+2)\) & & \(-2x = 1\) & |\(:(-2)\) & \(1-2x = 2x\) & |\(+2x\)\\
				& & & & & \\
				\(x-1 = 3x+6\) & |\(+1;-3x\) & \(x = -\frac{1}{2}\) & & \(1 = 4x\) & |\(:4\)\\
				& & & & & \\
				\(-2x = 7\) & |\(:(-2)\) & & & \(x = \frac{1}{4}\)\\
				& & & & & \\
				\(x = -\frac{7}{2}\) & & & & 
			\end{tabularx}
		\end{framed}
	\end{worksheet}
\end{document}