\documentclass[oneside,openany,headings=optiontotoc,11pt,numbers=noenddot]{scrreprt}

\usepackage[a4paper]{geometry}
\usepackage[utf8]{inputenc}
\usepackage[T1]{fontenc}
\usepackage{lmodern}
\usepackage[ngerman]{babel}
\usepackage{ngerman}

\usepackage[onehalfspacing]{setspace}

\usepackage{fancyhdr}
\usepackage{fancybox}

\usepackage{rotating}
\usepackage{varwidth}


\usepackage{pdflscape}
\usepackage{graphicx}
\usepackage{graphbox}
\graphicspath{
	{Pics/PDFs/}
	{Pics/JPGs/}
	{Pics/PNGs/}
}
\usepackage{caption}
\usepackage{tabularx}
\usepackage{dashrule}
\usepackage{hhline}
\usepackage{multirow}
\usepackage{enumerate}
\usepackage[hidelinks]{hyperref}
\usepackage{listings}

\usepackage[table]{xcolor}
\usepackage{array}
\usepackage{enumitem,amssymb,amsmath}
\usepackage{interval}
\usepackage{stmaryrd}
\usepackage{polynom}
\usepackage{diagbox}
\usepackage{dashrule}
\usepackage{framed}
\usepackage{mdframed}
\usepackage{karnaugh-map}

\usepackage{blindtext}

\usepackage{eso-pic}

\usepackage{amssymb}
\usepackage{eurosym}
\pagestyle{headings}
\renewcommand{\headrulewidth}{0.2pt}
\renewcommand{\footrulewidth}{0.2pt}
\newcommand*{\underdownarrow}[2]{\ensuremath{\underset{\overset{\Big\downarrow}{#2}}{#1}}}
\setlength{\fboxsep}{5pt}

% Codestyle defined
\definecolor{codegreen}{rgb}{0,0.6,0}
\definecolor{codegray}{rgb}{0.5,0.5,0.5}
\definecolor{codepurple}{rgb}{0.58,0,0.82}
\definecolor{backcolour}{rgb}{0.95,0.95,0.92}
\definecolor{deepgreen}{rgb}{0,0.5,0}
\definecolor{darkblue}{rgb}{0,0,0.65}
\definecolor{mauve}{rgb}{0.40, 0.19,0.28}
\colorlet{exceptioncolour}{yellow!50!red}
\colorlet{commandcolour}{blue!60!black}
\colorlet{numpycolour}{blue!60!green}
\colorlet{specmethodcolour}{violet}

%Neue Spaltendefinition
\newcolumntype{L}[1]{>{\raggedright\let\newline\\\arraybackslash\hspace{0pt}}m{#1}}
\newcolumntype{M}[1]{>{\centering\arraybackslash}X}
\newcommand{\cmnt}[1]{\ignorespaces}
%Textausrichtung ändern
\newcommand\tabrotate[1]{\rotatebox{90}{\raggedright#1\hspace{\tabcolsep}}}

%Intervall-Konfig
\intervalconfig {
	soft open fences
}

%Bash
\lstdefinestyle{BashInputStyle}{
	language=bash,
	basicstyle=\small\sffamily,
	backgroundcolor=\color{backcolour},
	columns=fullflexible,
	backgroundcolor=\color{backcolour},
	breaklines=true,
}
%Java
\lstdefinestyle{JavaInputStyle}{
	language=Java,
	backgroundcolor=\color{backcolour},
	aboveskip=1mm,
	belowskip=1mm,
	showstringspaces=false,
	columns=flexible,
	basicstyle={\footnotesize\ttfamily},
	numberstyle={\tiny},
	numbers=none,
	keywordstyle=\color{purple},,
	commentstyle=\color{deepgreen},
	stringstyle=\color{blue},
	emph={out},
	emphstyle=\color{darkblue},
	emph={[2]rand},
	emphstyle=[2]\color{specmethodcolour},
	breaklines=true,
	breakatwhitespace=true,
	tabsize=2,
}
%Python
\lstdefinestyle{PythonInputStyle}{
	language=Python,
	alsoletter={1234567890},
	aboveskip=1ex,
	basicstyle=\footnotesize,
	breaklines=true,
	breakatwhitespace= true,
	backgroundcolor=\color{backcolour},
	commentstyle=\color{red},
	otherkeywords={\ , \}, \{, \&,\|},
	emph={and,break,class,continue,def,yield,del,elif,else,%
		except,exec,finally,for,from,global,if,import,in,%
		lambda,not,or,pass,print,raise,return,try,while,assert},
	emphstyle=\color{exceptioncolour},
	emph={[2]True,False,None,min},
	emphstyle=[2]\color{specmethodcolour},
	emph={[3]object,type,isinstance,copy,deepcopy,zip,enumerate,reversed,list,len,dict,tuple,xrange,append,execfile,real,imag,reduce,str,repr},
	emphstyle=[3]\color{commandcolour},
	emph={[4]ode, fsolve, sqrt, exp, sin, cos, arccos, pi,  array, norm, solve, dot, arange, , isscalar, max, sum, flatten, shape, reshape, find, any, all, abs, plot, linspace, legend, quad, polyval,polyfit, hstack, concatenate,vstack,column_stack,empty,zeros,ones,rand,vander,grid,pcolor,eig,eigs,eigvals,svd,qr,tan,det,logspace,roll,mean,cumsum,cumprod,diff,vectorize,lstsq,cla,eye,xlabel,ylabel,squeeze},
	emphstyle=[4]\color{numpycolour},
	emph={[5]__init__,__add__,__mul__,__div__,__sub__,__call__,__getitem__,__setitem__,__eq__,__ne__,__nonzero__,__rmul__,__radd__,__repr__,__str__,__get__,__truediv__,__pow__,__name__,__future__,__all__},
	emphstyle=[5]\color{specmethodcolour},
	emph={[6]assert,range,yield},
	emphstyle=[6]\color{specmethodcolour}\bfseries,
	emph={[7]Exception,NameError,IndexError,SyntaxError,TypeError,ValueError,OverflowError,ZeroDivisionError,KeyboardInterrupt},
	emphstyle=[7]\color{specmethodcolour}\bfseries,
	emph={[8]taster,send,sendMail,capture,check,noMsg,go,move,switch,humTem,ventilate,buzz},
	emphstyle=[8]\color{blue},
	keywordstyle=\color{blue}\bfseries,
	rulecolor=\color{black!40},
	showstringspaces=false,
	stringstyle=\color{deepgreen}
}

\lstset{literate=%
	{Ö}{{\"O}}1
	{Ä}{{\"A}}1
	{Ü}{{\"U}}1
	{ß}{{\ss}}1
	{ü}{{\"u}}1
	{ä}{{\"a}}1
	{ö}{{\"o}}1
}

% Neue Klassenarbeits-Umgebung
\newenvironment{worksheet}[3]
% Begin-Bereich
{
	\newpage
	\sffamily
	\setcounter{page}{1}
	\ClearShipoutPicture
	\AddToShipoutPicture{
		\put(55,761){{
				\mbox{\parbox{385\unitlength}{\tiny \color{codegray}BBS I Mainz, #1 \newline #2
						\newline #3
					}
				}
			}
		}
		\put(455,761){{
				\mbox{\hspace{0.3cm}\includegraphics[width=0.2\textwidth]{../../logo.jpg}}
			}
		}
	}
}
% End-Bereich
{
	\clearpage
	\ClearShipoutPicture
}

\geometry{left=2.50cm,right=2.50cm,top=3.00cm,bottom=1.00cm,includeheadfoot}

\begin{document}
	\begin{worksheet}{BGY 16}{Klassenstufe 13 - Mathematik}{Lernabschnitt 1: Ketten-, Produkt und \textbf{Quotientenregel}}
				
		\begin{framed}
			\noindent
			Wir erinnern uns an die drei Ableitungsregeln, die hier notwendig sind.
			\paragraph{Produktregel} Gegeben ist eine Funktion \(f(x) = u(x)\cdot{}v(x)\), mit \(u(x)\) und \(v(x)\) differenzierbar in \(x\). Dann gilt für die Ableitung von \(f(x)\) folgendes:
			\[f'(x) = u'(x)\cdot{}v(x) + u(x)\cdot{}v'(x)\]
			\paragraph{Kettenregel} Gegeben ist eine verkettete Funktion der Form \(f(x) = u(v(x))\), mit \(u(x)\) und \(v(x)\) differenzierbar in \(x\). Dann gilt für die Ableitung von \(f(x)\) folgendes:
			\[f'(x) = u'(v(x))\cdot{}v'(x)\]
			\paragraph{Quotientenregel} Gegeben ist eine die Funktion der Form \(f(x) = \frac{u(x)}{v(x)}\), mit \(u(x)\) und \(v(x)\) differenzierbar in \(x\) und \(v(x) \neq 0\). Dann gilt für die Ableitung von \(f(x)\) folgendes:
			\[f'(x) = \frac{u'(x)\cdot{}v(x)-u(x)\cdot{}v'(x)}{[v(x)]^2}\]
		\end{framed}
		\begin{framed}
			\noindent
			\textbf{\underline{Aufgabe 1}} Leiten Sie mit Hilfe der Kettenregel ab und vereinfachen Sie das Ergebnis (falls möglich).\\
			\par
			\begin{tabularx}{\textwidth}{lX}
				(a) & \(f(x)=(2+3x)^3\)\\
				& \(f'(x) = 3(2+3x)^2\cdot{}3 =\) \colorbox{green!10}{\(9(2+3x)^2\)}\\
				\\
				(b) & \(f(x) = (2x-3)^5\)\\
				& \(f'(x) = 5(2x-3)^4\cdot{}2 =\) \colorbox{green!10}{\(10(2x-3)^4\)}\\
				\\
				(c) & \(f(x) = \sqrt{3x-4} = (3x-4)^{\frac{1}{2}}\)\\
				& \(f'(x) = \frac{1}{2}(3x-4)^{-\frac{1}{2}}\cdot{}3 = \) \colorbox{green!10}{\(\frac{3}{2\sqrt{3x-4}}\)}\\
				\\
				(d) & \(f(x) = (x+4x^3)^{-3}\)\\
				& \(f'(x) =  (-3)(x+4x^3)^{-4}\cdot{}(1+12x^2) =\) \colorbox{green!10}{\(\frac{-3-36x^2}{(x+4x^3)^4}\)}\\
				\\
			\end{tabularx}
			\begin{tabularx}{\textwidth}{lX}
				(e) & \(f(x) = (x-x^4)^{-2}\)\\
				& \(f'(x) = (-2)(x-x^4)^{-3}\cdot(1-4x^3) =\) \colorbox{green!10}{\(\frac{8x^3-2}{(x-x^4)^3}\)}\\
				\\
				(f) & \(f(x) = \sqrt{x^3+1} = (x^3+1)^{\frac{1}{2}}\)\\
				& \(f'(x) = \frac{1}{2}(x^3+1)^{-\frac{1}{2}}\cdot{}3x^2 =\) \colorbox{green!10}{\(\frac{3x^2}{2\sqrt{x^3+1}}\)}\\
				\\
				(g) & \(f(x) = (x^2+x)^{\frac{3}{2}}\)\\
				& \(f'(x) = \frac{3}{2}(x^2+x)^{\frac{1}{2}}\cdot(2x+1) =\) \colorbox{green!10}{\(\frac{(6x+3)\sqrt{x^2+x}}{2}\)}\\
				\\
				(h) & \(f(x) = (1-\sqrt{x})^4\)\\
				& \(f'(x) = 4(1-\sqrt{x})^3\cdot{}(-\frac{1}{2}x^{-\frac{1}{2}}) =\) \colorbox{green!10}{\(\frac{-2(1-\sqrt{x})^3}{\sqrt{x}}\)}
			\end{tabularx}\\
			\par\noindent
			\rule{0.9\textwidth}{0.1pt}\\
			\textbf{\underline{Aufgabe 2}} Bestimmen Sie mit Hilfe der Produktregel die Ableitung von:\\
			\par\noindent
			\begin{tabularx}{\textwidth}{lX}
				(a) & \(f(x) = x(2+3x)\)\\
				& \(f'(x) = 1\cdot(2+3x) + x\cdot{}3x =\) \colorbox{green!10}{\(3x^2 + 3x +2\)}\\
				\\
				(b) & \(f(x) = \sqrt{1-x}(x^2+3x) = (1-x)^{\frac{1}{2}}(x^2+3x)\)\\
				& \(f'(x) = \frac{1}{2}(1-x)^{-\frac{1}{2}}\cdot(-1)\cdot(x^2+3x) + (1-x)^{\frac{1}{2}}\cdot{}(2x+3) =\)\\
				& \(\frac{x^2+3x+2(1-x)(2x+3)}{2\sqrt{1-x}} =\) \colorbox{green!10}{\(\frac{-3x^2+x+6}{2\sqrt{1-x}}\)}\\
				\\
				(c) & \(f(x) = (2x-3)(x^2+x)\)\\
				& \colorbox{green!10}{\(f'(x) = 2\cdot(x^2+x) + (2x-3)\cdot(2x+1)\)}\\
				\\
				(d) & \(f(x) = (x+1)(x^2+3x^3)\)\\
				& \colorbox{green!10}{\(1\cdot(x^2+3x^3) + (x+1)\cdot(2x+9x^2)\)}
			\end{tabularx}\\
			\newpage
			\textbf{\underline{Aufgabe 3}} Nutzen Sie zur Ableitung die Quotientenregel.\\
			\par
			\begin{tabularx}{\textwidth}{XX}
				(a) \(f(x) = \frac{x}{x+1}\) & (b) \(f(t) = \frac{2t+1}{4t^2-5}\) \\
				\\
				\(f'(x) = \frac{(x+1) - x}{(x+1)^2} =\) \colorbox{green!10}{\(\frac{1}{(x+1)^2}\)} & \(f'(x) = \frac{2(4t^2-5) - (2t+1)(8t)}{(4t^2-5)^2} =\) \colorbox{green!10}{\(\frac{-8t^2+8t-10}{(4t^2-5)^2}\)}\\
				\\
				(c) \(f(a) = \frac{2a + a^3}{3a-4}\) & (d) \(f(x) = \frac{\sqrt{x+1}}{x^2+1}\)\\
				\(f'(x) = \frac{(2+3a^2)(3a-4) - (2a+a^3)\cdot{}3}{(3a-4)^2} =\) & \(f'(x) = \frac{\frac{1}{2}(x+1)^{-\frac{1}{2}}(x^2+1) - (\sqrt{x+1})\cdot{}2x}{(x^2+1)^2} =\)\\
				\colorbox{green!10}{\(\frac{8a^3-12a^2 + 4a -8}{(3a-4)^2}\)} & \colorbox{green!10}{\(\frac{(x^2+1) - 2x(x+1)}{(\sqrt{x+1})(x^2+1)^2}\)}\\
				\\
				(e) \(f(x) = \frac{3x^2-1}{15-x^2}\) & (f) \(f(x) = \frac{\sqrt{x}}{x-1}\)\\
				\(f'(x) = \frac{6x(12-x^2) - (3x^2-1)(-2x)}{(15-x^2)^2} =\) & \(f'(x) = \frac{\frac{1}{2}x^{-\frac{1}{2}}(x-1) - \sqrt{x}\cdot{}1}{(x-1)^2} = \frac{(x-1)-x}{\sqrt{x}(x-1)^2} =\)\\
				\colorbox{green!10}{\(\frac{6x^3-6x^2+70x}{(15-x^2)^2}\)} & \colorbox{green!10}{\(\frac{-1}{\sqrt{x}(x-1)^2}\)}\\
				\\
				(g) \(f(x) = \frac{x^2+x+1}{x^2-1}\) & (h) \(f(x) = \frac{1-x^2}{x+2}\)\\
				\(f'(x) = \frac{(2x+1)(x^2-1) - (x^2+x+1)(2x)}{(x^2-1)^2}\) & \(f'(x) = \frac{-2x(x+2) - (1-x^2)\cdot{}1}{(x+2)^2} =\)\\
				\colorbox{green!10}{\(\frac{-x^2-4x-1}{(x^2-1)^2}\)} & \colorbox{green!10}{\(\frac{-x^2-4x-1}{(x+2)^2}\)}\\
				\\
				(k) \(f(x) = \frac{2x-3}{4x+1}\) & (l) \(f(x) = \frac{\sqrt{x}+1}{1-\sqrt{x}}\)\\
				\(f'(x) = \frac{2\cdot(4x+1) - (2x-3)\cdot{}4}{(4x+1)^2} =\) & \(f'(x) = \frac{\frac{1}{2}x^{-\frac{1}{2}}\cdot(1-\sqrt{x}) - (\sqrt{x}+1)\cdot(-\frac{1}{2}x^{-\frac{1}{2}})}{(1-\sqrt{x})^2} =\)\\
				\colorbox{green!10}{\(\frac{14}{(4x+1)^2}\)} & \colorbox{green!10}{\(\frac{1}{2\sqrt{x}(1-\sqrt{x})^2}\)}
			\end{tabularx}\\
		\end{framed}
	\end{worksheet}
\end{document}