\documentclass[oneside,openany,headings=optiontotoc,11pt,numbers=noenddot]{scrreprt}

\usepackage[a4paper]{geometry}
\usepackage[utf8]{inputenc}
\usepackage[T1]{fontenc}
\usepackage{lmodern}
\usepackage[ngerman]{babel}
\usepackage{ngerman}

\usepackage[onehalfspacing]{setspace}

\usepackage{fancyhdr}
\usepackage{fancybox}

\usepackage{rotating}
\usepackage{varwidth}


\usepackage{pdflscape}
\usepackage{graphicx}
\usepackage{graphbox}
\graphicspath{
	{Pics/PDFs/}
	{Pics/JPGs/}
	{Pics/PNGs/}
}
\usepackage{caption}
\usepackage{tabularx}
\usepackage{dashrule}
\usepackage{hhline}
\usepackage{multirow}
\usepackage{enumerate}
\usepackage[hidelinks]{hyperref}
\usepackage{listings}

\usepackage[table]{xcolor}
\usepackage{array}
\usepackage{enumitem,amssymb,amsmath}
\usepackage{interval}
\usepackage{stmaryrd}
\usepackage{polynom}
\usepackage{diagbox}
\usepackage{dashrule}
\usepackage{framed}
\usepackage{mdframed}
\usepackage{karnaugh-map}

\usepackage{blindtext}

\usepackage{eso-pic}

\usepackage{amssymb}
\usepackage{eurosym}
\pagestyle{headings}
\renewcommand{\headrulewidth}{0.2pt}
\renewcommand{\footrulewidth}{0.2pt}
\newcommand*{\underdownarrow}[2]{\ensuremath{\underset{\overset{\Big\downarrow}{#2}}{#1}}}
\setlength{\fboxsep}{5pt}

% Codestyle defined
\definecolor{codegreen}{rgb}{0,0.6,0}
\definecolor{codegray}{rgb}{0.5,0.5,0.5}
\definecolor{codepurple}{rgb}{0.58,0,0.82}
\definecolor{backcolour}{rgb}{0.95,0.95,0.92}
\definecolor{deepgreen}{rgb}{0,0.5,0}
\definecolor{darkblue}{rgb}{0,0,0.65}
\definecolor{mauve}{rgb}{0.40, 0.19,0.28}
\colorlet{exceptioncolour}{yellow!50!red}
\colorlet{commandcolour}{blue!60!black}
\colorlet{numpycolour}{blue!60!green}
\colorlet{specmethodcolour}{violet}

%Neue Spaltendefinition
\newcolumntype{L}[1]{>{\raggedright\let\newline\\\arraybackslash\hspace{0pt}}m{#1}}
\newcolumntype{M}[1]{>{\centering\arraybackslash}X}
\newcommand{\cmnt}[1]{\ignorespaces}
%Textausrichtung ändern
\newcommand\tabrotate[1]{\rotatebox{90}{\raggedright#1\hspace{\tabcolsep}}}

%Intervall-Konfig
\intervalconfig {
	soft open fences
}

%Bash
\lstdefinestyle{BashInputStyle}{
	language=bash,
	basicstyle=\small\sffamily,
	backgroundcolor=\color{backcolour},
	columns=fullflexible,
	backgroundcolor=\color{backcolour},
	breaklines=true,
}
%Java
\lstdefinestyle{JavaInputStyle}{
	language=Java,
	backgroundcolor=\color{backcolour},
	aboveskip=1mm,
	belowskip=1mm,
	showstringspaces=false,
	columns=flexible,
	basicstyle={\footnotesize\ttfamily},
	numberstyle={\tiny},
	numbers=none,
	keywordstyle=\color{purple},,
	commentstyle=\color{deepgreen},
	stringstyle=\color{blue},
	emph={out},
	emphstyle=\color{darkblue},
	emph={[2]rand},
	emphstyle=[2]\color{specmethodcolour},
	breaklines=true,
	breakatwhitespace=true,
	tabsize=2,
}
%Python
\lstdefinestyle{PythonInputStyle}{
	language=Python,
	alsoletter={1234567890},
	aboveskip=1ex,
	basicstyle=\footnotesize,
	breaklines=true,
	breakatwhitespace= true,
	backgroundcolor=\color{backcolour},
	commentstyle=\color{red},
	otherkeywords={\ , \}, \{, \&,\|},
	emph={and,break,class,continue,def,yield,del,elif,else,%
		except,exec,finally,for,from,global,if,import,in,%
		lambda,not,or,pass,print,raise,return,try,while,assert},
	emphstyle=\color{exceptioncolour},
	emph={[2]True,False,None,min},
	emphstyle=[2]\color{specmethodcolour},
	emph={[3]object,type,isinstance,copy,deepcopy,zip,enumerate,reversed,list,len,dict,tuple,xrange,append,execfile,real,imag,reduce,str,repr},
	emphstyle=[3]\color{commandcolour},
	emph={[4]ode, fsolve, sqrt, exp, sin, cos, arccos, pi,  array, norm, solve, dot, arange, , isscalar, max, sum, flatten, shape, reshape, find, any, all, abs, plot, linspace, legend, quad, polyval,polyfit, hstack, concatenate,vstack,column_stack,empty,zeros,ones,rand,vander,grid,pcolor,eig,eigs,eigvals,svd,qr,tan,det,logspace,roll,mean,cumsum,cumprod,diff,vectorize,lstsq,cla,eye,xlabel,ylabel,squeeze},
	emphstyle=[4]\color{numpycolour},
	emph={[5]__init__,__add__,__mul__,__div__,__sub__,__call__,__getitem__,__setitem__,__eq__,__ne__,__nonzero__,__rmul__,__radd__,__repr__,__str__,__get__,__truediv__,__pow__,__name__,__future__,__all__},
	emphstyle=[5]\color{specmethodcolour},
	emph={[6]assert,range,yield},
	emphstyle=[6]\color{specmethodcolour}\bfseries,
	emph={[7]Exception,NameError,IndexError,SyntaxError,TypeError,ValueError,OverflowError,ZeroDivisionError,KeyboardInterrupt},
	emphstyle=[7]\color{specmethodcolour}\bfseries,
	emph={[8]taster,send,sendMail,capture,check,noMsg,go,move,switch,humTem,ventilate,buzz},
	emphstyle=[8]\color{blue},
	keywordstyle=\color{blue}\bfseries,
	rulecolor=\color{black!40},
	showstringspaces=false,
	stringstyle=\color{deepgreen}
}

\lstset{literate=%
	{Ö}{{\"O}}1
	{Ä}{{\"A}}1
	{Ü}{{\"U}}1
	{ß}{{\ss}}1
	{ü}{{\"u}}1
	{ä}{{\"a}}1
	{ö}{{\"o}}1
}

% Neue Klassenarbeits-Umgebung
\newenvironment{worksheet}[3]
% Begin-Bereich
{
	\newpage
	\sffamily
	\setcounter{page}{1}
	\ClearShipoutPicture
	\AddToShipoutPicture{
		\put(55,761){{
				\mbox{\parbox{385\unitlength}{\tiny \color{codegray}BBS I Mainz, #1 \newline #2
						\newline #3
					}
				}
			}
		}
		\put(455,761){{
				\mbox{\hspace{0.3cm}\includegraphics[width=0.2\textwidth]{../../logo.jpg}}
			}
		}
	}
}
% End-Bereich
{
	\clearpage
	\ClearShipoutPicture
}

\geometry{left=2.50cm,right=2.50cm,top=3.00cm,bottom=1.00cm,includeheadfoot}

\begin{document}
	\begin{worksheet}{BGY 16}{Mathematik - Lernbereich 3, Algebraisierung}{Gegenseitige Lage von Geraden I - Übung}
				
		\begin{framed}
			\noindent
			Gegeben ist die Ebene \(E: \vec{x} = \left(\begin{array}{c}2\\1\\-3\end{array}\right) + r\left(\begin{array}{c}5\\2\\1\end{array}\right) +s\left(\begin{array}{c}-2\\-1\\3\end{array}\right)\)\\
			Bestimmen Sie jeweils den Punkt P für die nachfolgenden Parameter.\\
			\hdashrule[0.5ex][x]{\textwidth}{0.1mm}{8mm 2pt}\\
			\begin{tabularx}{\textwidth}{lX}
				(a) & \(r=2; s= -1\)\\
				& \(\vec{x} = \left(\begin{array}{c}2\\1\\-3\end{array}\right) + 2*\left(\begin{array}{c}5\\2\\1\end{array}\right) -\left(\begin{array}{c}-2\\-1\\3\end{array}\right) = \left(\begin{array}{c}14\\6\\-4\end{array}\right)\)\\
				\multicolumn{2}{l}{Der zugehörige Punkt ist \(P(14|6|-4)\).}\\
				(b)& \(r=-2; s=2\)\\
				& \(\vec{x} = \left(\begin{array}{c}2\\1\\-3\end{array}\right) - 2*\left(\begin{array}{c}5\\2\\1\end{array}\right) +2\left(\begin{array}{c}-2\\-1\\3\end{array}\right) = \left(\begin{array}{c}-12\\-5\\1\end{array}\right)\)\\
				\multicolumn{2}{l}{Der zugehörige Punkt ist \(P(-12|-5|1)\).}\\
			\end{tabularx}
			\par\noindent
			\hdashrule[0.5ex][x]{\textwidth}{0.2mm}{8mm 2pt}\\
			\par\noindent
			Geben Sie zwei verschiedene Parametergleichungen der Ebene E an, die durch die drei Punkte festgelegt ist.\\
			\begin{tabularx}{\textwidth}{lX}
				\hline
				& Die Parameterform der Ebene hat folgende Form: \(E: \vec{x} = \vec{p} + r\vec{u} + s\vec{v}\)\\
				\hline
			\end{tabularx}	
			(a) \(A(2|1|0); B(5|5|-1); C(-4|-7|2)\)\\
			\begin{tabularx}{\textwidth}{X|X}
				\(\vec{p} = \vec{0A} = \left(\begin{matrix}2\\1\\0\end{matrix}\right), \vec{u_1} = \vec{AB} = \left(\begin{matrix}3\\4\\-1\end{matrix}\right)\) & \(\vec{q} = \vec{0B} =  \left(\begin{matrix}5\\5\\-1\end{matrix}\right), \vec{u_2} = \vec{BA} = \left(\begin{matrix}-3\\-4\\1\end{matrix}\right)\)\\
				\(\vec{v_1} = \vec{AC} = \left(\begin{matrix}-6\\-8\\2\end{matrix}\right)\) & \(\vec{v_2} = \vec{BC} = \left(\begin{matrix}-9\\-12\\3\end{matrix}\right)\)\\
			\end{tabularx}\\
			So ergeben sich dann die beiden Parametergleichungen\\
			\par\noindent
			\begin{tabularx}{\textwidth}{X|X}
				\colorbox{green!10}{\(E_1: \vec{x} = \left(\begin{matrix}2\\1\\0\end{matrix}\right) + r*\left(\begin{matrix}3\\4\\-1\end{matrix}\right) +s*\left(\begin{matrix}-6\\-8\\2\end{matrix}\right)\)} & \colorbox{green!10}{\(E_2: \vec{x} =  \left(\begin{matrix}5\\5\\-1\end{matrix}\right) + r*\left(\begin{matrix}-3\\-4\\1\end{matrix}\right) + s*\left(\begin{matrix}-9\\-12\\3\end{matrix}\right)\)}\\
			\end{tabularx}
			\newpage\noindent
			(b) \(A(8|1|-3); B(-11|4|3); C(6|-1|0)\)\\
			\begin{tabularx}{\textwidth}{X|X}
				\(\vec{p} = \vec{0A} = \left(\begin{matrix}8\\1\\-3\end{matrix}\right), \vec{u_1} = \vec{AB} = \left(\begin{matrix}-19\\3\\6\end{matrix}\right)\) & \(\vec{q} = \vec{0B} =  \left(\begin{matrix}-11\\4\\3\end{matrix}\right), \vec{u_2} = \vec{BA} = \left(\begin{matrix}19\\-3\\-6\end{matrix}\right)\)\\
				\(\vec{v_1} = \vec{AC} = \left(\begin{matrix}-2\\-2\\3\end{matrix}\right)\) & \(\vec{v_2} = \vec{BC} = \left(\begin{matrix}17\\-5\\-3\end{matrix}\right)\)\\
			\end{tabularx}\\
			So ergeben sich dann die beiden Parametergleichungen\\
			\par\noindent
			\begin{tabularx}{\textwidth}{X|X}
				\colorbox{green!10}{\(E_1: \vec{x} = \left(\begin{matrix}8\\1\\-3\end{matrix}\right) + r*\left(\begin{matrix}-19\\3\\6\end{matrix}\right) +s*\left(\begin{matrix}-2\\-2\\3\end{matrix}\right)\)} & \colorbox{green!10}{\(E_2: \vec{x} =  \left(\begin{matrix}-11\\4\\3\end{matrix}\right) + r*\left(\begin{matrix}19\\-3\\-6\end{matrix}\right) + s*\left(\begin{matrix}17\\-5\\-3\end{matrix}\right)\)}\\
			\end{tabularx}
			\par\bigskip\noindent
			(c) \(A(2|4|7); B(13|25|8); C(6|7|-13)\)\\
			\begin{tabularx}{\textwidth}{X|X}
				\(\vec{p} = \vec{0A} = \left(\begin{matrix}2\\4\\7\end{matrix}\right), \vec{u_1} = \vec{AB} = \left(\begin{matrix}11\\21\\1\end{matrix}\right), \) & \(\vec{q} = \vec{0B} =  \left(\begin{matrix}13\\25\\8\end{matrix}\right), \vec{u_2} = \vec{BA} = \left(\begin{matrix}-11\\-21\\-1\end{matrix}\right), \)\\
				\(\vec{v_1} = \vec{AC} = \left(\begin{matrix}4\\3\\-20\end{matrix}\right)\) & \(\vec{v_2} = \vec{BC} = \left(\begin{matrix}-7\\-18\\-21\end{matrix}\right)\)\\
			\end{tabularx}\\
			So ergeben sich dann die beiden Parametergleichungen\\
			\par\noindent
			\begin{tabularx}{\textwidth}{X|X}
				\colorbox{green!10}{\(E_1: \vec{x} = \left(\begin{matrix}2\\4\\7\end{matrix}\right) + r*\left(\begin{matrix}11\\21\\1\end{matrix}\right) +s*\left(\begin{matrix}4\\3\\-20\end{matrix}\right)\)} & \colorbox{green!10}{\(E_2: \vec{x} =  \left(\begin{matrix}13\\25\\8\end{matrix}\right) + r*\left(\begin{matrix}-11\\-21\\-1\end{matrix}\right) + s*\left(\begin{matrix}-7\\-18\\-21\end{matrix}\right)\)}\\
			\end{tabularx}
			\par\bigskip\noindent
		\end{framed}
	\end{worksheet}
\end{document}