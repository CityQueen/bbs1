\documentclass[oneside,openany,headings=optiontotoc,11pt,numbers=noenddot]{scrreprt}

\usepackage[a4paper]{geometry}
\usepackage[utf8]{inputenc}
\usepackage[T1]{fontenc}
\usepackage{lmodern}
\usepackage[ngerman]{babel}
\usepackage{ngerman}

\usepackage[onehalfspacing]{setspace}

\usepackage{fancyhdr}
\usepackage{fancybox}

\usepackage{rotating}
\usepackage{varwidth}


\usepackage{pdflscape}
\usepackage{graphicx}
\usepackage{graphbox}
\graphicspath{
	{Pics/PDFs/}
	{Pics/JPGs/}
	{Pics/PNGs/}
}
\usepackage{caption}
\usepackage{tabularx}
\usepackage{dashrule}
\usepackage{hhline}
\usepackage{multirow}
\usepackage{enumerate}
\usepackage[hidelinks]{hyperref}
\usepackage{listings}

\usepackage[table]{xcolor}
\usepackage{array}
\usepackage{enumitem,amssymb,amsmath}
\usepackage{interval}
\usepackage{stmaryrd}
\usepackage{polynom}
\usepackage{diagbox}
\usepackage{dashrule}
\usepackage{framed}
\usepackage{mdframed}
\usepackage{karnaugh-map}

\usepackage{blindtext}

\usepackage{eso-pic}

\usepackage{amssymb}
\usepackage{eurosym}
\pagestyle{headings}
\renewcommand{\headrulewidth}{0.2pt}
\renewcommand{\footrulewidth}{0.2pt}
\newcommand*{\underdownarrow}[2]{\ensuremath{\underset{\overset{\Big\downarrow}{#2}}{#1}}}
\setlength{\fboxsep}{5pt}

% Codestyle defined
\definecolor{codegreen}{rgb}{0,0.6,0}
\definecolor{codegray}{rgb}{0.5,0.5,0.5}
\definecolor{codepurple}{rgb}{0.58,0,0.82}
\definecolor{backcolour}{rgb}{0.95,0.95,0.92}
\definecolor{deepgreen}{rgb}{0,0.5,0}
\definecolor{darkblue}{rgb}{0,0,0.65}
\definecolor{mauve}{rgb}{0.40, 0.19,0.28}
\colorlet{exceptioncolour}{yellow!50!red}
\colorlet{commandcolour}{blue!60!black}
\colorlet{numpycolour}{blue!60!green}
\colorlet{specmethodcolour}{violet}

%Neue Spaltendefinition
\newcolumntype{L}[1]{>{\raggedright\let\newline\\\arraybackslash\hspace{0pt}}m{#1}}
\newcolumntype{M}[1]{>{\centering\arraybackslash}X}
\newcommand{\cmnt}[1]{\ignorespaces}
%Textausrichtung ändern
\newcommand\tabrotate[1]{\rotatebox{90}{\raggedright#1\hspace{\tabcolsep}}}

%Intervall-Konfig
\intervalconfig {
	soft open fences
}

%Bash
\lstdefinestyle{BashInputStyle}{
	language=bash,
	basicstyle=\small\sffamily,
	backgroundcolor=\color{backcolour},
	columns=fullflexible,
	backgroundcolor=\color{backcolour},
	breaklines=true,
}
%Java
\lstdefinestyle{JavaInputStyle}{
	language=Java,
	backgroundcolor=\color{backcolour},
	aboveskip=1mm,
	belowskip=1mm,
	showstringspaces=false,
	columns=flexible,
	basicstyle={\footnotesize\ttfamily},
	numberstyle={\tiny},
	numbers=none,
	keywordstyle=\color{purple},,
	commentstyle=\color{deepgreen},
	stringstyle=\color{blue},
	emph={out},
	emphstyle=\color{darkblue},
	emph={[2]rand},
	emphstyle=[2]\color{specmethodcolour},
	breaklines=true,
	breakatwhitespace=true,
	tabsize=2,
}
%Python
\lstdefinestyle{PythonInputStyle}{
	language=Python,
	alsoletter={1234567890},
	aboveskip=1ex,
	basicstyle=\footnotesize,
	breaklines=true,
	breakatwhitespace= true,
	backgroundcolor=\color{backcolour},
	commentstyle=\color{red},
	otherkeywords={\ , \}, \{, \&,\|},
	emph={and,break,class,continue,def,yield,del,elif,else,%
		except,exec,finally,for,from,global,if,import,in,%
		lambda,not,or,pass,print,raise,return,try,while,assert},
	emphstyle=\color{exceptioncolour},
	emph={[2]True,False,None,min},
	emphstyle=[2]\color{specmethodcolour},
	emph={[3]object,type,isinstance,copy,deepcopy,zip,enumerate,reversed,list,len,dict,tuple,xrange,append,execfile,real,imag,reduce,str,repr},
	emphstyle=[3]\color{commandcolour},
	emph={[4]ode, fsolve, sqrt, exp, sin, cos, arccos, pi,  array, norm, solve, dot, arange, , isscalar, max, sum, flatten, shape, reshape, find, any, all, abs, plot, linspace, legend, quad, polyval,polyfit, hstack, concatenate,vstack,column_stack,empty,zeros,ones,rand,vander,grid,pcolor,eig,eigs,eigvals,svd,qr,tan,det,logspace,roll,mean,cumsum,cumprod,diff,vectorize,lstsq,cla,eye,xlabel,ylabel,squeeze},
	emphstyle=[4]\color{numpycolour},
	emph={[5]__init__,__add__,__mul__,__div__,__sub__,__call__,__getitem__,__setitem__,__eq__,__ne__,__nonzero__,__rmul__,__radd__,__repr__,__str__,__get__,__truediv__,__pow__,__name__,__future__,__all__},
	emphstyle=[5]\color{specmethodcolour},
	emph={[6]assert,range,yield},
	emphstyle=[6]\color{specmethodcolour}\bfseries,
	emph={[7]Exception,NameError,IndexError,SyntaxError,TypeError,ValueError,OverflowError,ZeroDivisionError,KeyboardInterrupt},
	emphstyle=[7]\color{specmethodcolour}\bfseries,
	emph={[8]taster,send,sendMail,capture,check,noMsg,go,move,switch,humTem,ventilate,buzz},
	emphstyle=[8]\color{blue},
	keywordstyle=\color{blue}\bfseries,
	rulecolor=\color{black!40},
	showstringspaces=false,
	stringstyle=\color{deepgreen}
}

\lstset{literate=%
	{Ö}{{\"O}}1
	{Ä}{{\"A}}1
	{Ü}{{\"U}}1
	{ß}{{\ss}}1
	{ü}{{\"u}}1
	{ä}{{\"a}}1
	{ö}{{\"o}}1
}

% Neue Klassenarbeits-Umgebung
\newenvironment{worksheet}[3]
% Begin-Bereich
{
	\newpage
	\sffamily
	\setcounter{page}{1}
	\ClearShipoutPicture
	\AddToShipoutPicture{
		\put(55,761){{
				\mbox{\parbox{385\unitlength}{\tiny \color{codegray}BBS I Mainz, #1 \newline #2
						\newline #3
					}
				}
			}
		}
		\put(455,761){{
				\mbox{\hspace{0.3cm}\includegraphics[width=0.2\textwidth]{../../logo.jpg}}
			}
		}
	}
}
% End-Bereich
{
	\clearpage
	\ClearShipoutPicture
}

\geometry{left=2.50cm,right=2.50cm,top=3.00cm,bottom=1.00cm,includeheadfoot}

\begin{document}
	\begin{worksheet}{BGY 16}{Mathematik - Lernbereich 3, Algebraisierung}{Gegenseitige Lage von Geraden - Hausaufgabe Lösung}
		
		\begin{framed}
			\noindent
			\tiny{\color{codegray}Allgemeines Vorgehen zur Bestimmung der gegenseitigen Lage von Geraden}\\
			\normalsize
			Sollen zwei Geraden \(g: \vec{x} = \vec{p} + r\vec{u}\) und \(h: \vec{x} = \vec{q}+ t\vec{v}\) auf ihre gegenseitige Lage untersucht werden, betrachten wir zunächst die Richtungsvektoren \(\vec{u}\) und \(\vec{v}\). Sind \(\vec{u}\) und \(\vec{v}\)...\\
			\par\noindent
			\begin{tabularx}{\textwidth}{X|X}
				\color{blue}linear abhängig\normalcolor, & \color{blue}linear unabhängig\normalcolor,\\
				\hline
				dann sind \(g\) und \(h\)
				\begin{itemize}
					\item[+] \color{codegreen}parallel\normalcolor{}, wenn \[\vec{p} + r\vec{u} = \vec{q} + t\vec{v}\] \underline{\textbf{keine}} Lösung besitzt.
					\item[+] \color{red}gleich\normalcolor{}, wenn \[\vec{p} + r\vec{u} = \vec{q} + t\vec{v}\] \underline{\textbf{unendlich viele}} Lösungen besitzt.
				\end{itemize} & dann haben \(g\) und \(h\)
			\begin{itemize}
				\item[+] \color{codegreen}keinen\normalcolor{} Schnittpunkt (sie sind \textit{windschief}), wenn \[\vec{p} + r\vec{u} = \vec{q} + t\vec{v}\] \underline{\textbf{keine}} Lösung besitzt.
				\item[+] \color{red}einen\normalcolor{} Schnittpunkt S, wenn \[\vec{p} + r\vec{u} = \vec{q} + t\vec{v}\] \underline{\textbf{eine}} Lösung besitzt.
				
		\end{itemize}\\
			\end{tabularx}
		\end{framed}
		
		\begin{framed}
			\noindent
			\underline{\textit{S. 64 Nr. 3 c \& d}} Untersuchen Sie die gegenseitige Lage der Geraden g und h. Berechnen Sie gegebenenfalls die Koordinaten des Schnittpunktes S.\\
			\par\noindent
			\begin{tabularx}{\textwidth}{XX}
				\textbf{(c)} \(g: \vec{x} = \left(\begin{array}{c}7\\3\end{array}\right) + r\left(\begin{array}{c}1\\0\end{array}\right)\) & \(h: \vec{x} = \left(\begin{array}{c}2\\5\end{array}\right) + t\left(\begin{array}{c}1\\1\end{array}\right)\)
			\end{tabularx}
			Die Richtungsvektoren \(\left(\begin{array}{c}1\\1\end{array}\right)\) und \(\left(\begin{array}{c}1\\1\end{array}\right)\) sind \color{blue}linear unabhängig\normalcolor. Die Geraden können also \color{red}einen\normalcolor{} oder \color{codegreen}keinen\normalcolor{} Schnittpunkt S haben. Um dies zu bestimmen, setzen wir \(g=h\) und lösen das Gleichungssystem.\\
			\par\noindent
			\begin{tabularx}{\textwidth}{X|X}
				\(\left(\begin{array}{c}7\\3\end{array}\right) + r\left(\begin{array}{c}1\\0\end{array}\right) = \left(\begin{array}{c}2\\5\end{array}\right) + t\left(\begin{array}{c}1\\1\end{array}\right)\) & \begin{tabular}{lll}
					\( 7 + r = 2 + t\)& & \(\Rightarrow r = -7\)\\
					\(3 = 5 + t\) & \(\Rightarrow t = -2\)
				\end{tabular}
			\end{tabularx}\\
			\par\noindent
			Den Schnittpunkt der beiden Geraden bestimmen wir nun, indem wir die ermittelten Werte für \(r\) und \(t\) in die dazugehörigen Geradengleichungen einsetzen.\\
			\par\noindent
			\begin{tabularx}{\textwidth}{XX}
				\(\left(\begin{array}{c}7\\3\end{array}\right) - 7\left(\begin{array}{c}1\\0\end{array}\right) = \left(\begin{array}{c}0\\3\end{array}\right)\) & \(\left(\begin{array}{c}2\\5\end{array}\right) - 2\left(\begin{array}{c}1\\1\end{array}\right) = \left(\begin{array}{c}0\\3\end{array}\right)\)
			\end{tabularx}
			\par\noindent			
			Die Geraden schneiden sich also im Punkt S(0|3).\\
			\noindent
			\hdashrule[0.5ex][x]{\textwidth}{0.1mm}{8mm 2pt}\\
			\par\noindent
			\begin{tabularx}{\textwidth}{XX}
				\textbf{(d)} \(g: \vec{x} = \left(\begin{array}{c}1\\3\end{array}\right) + r\left(\begin{array}{c}3\\6\end{array}\right)\) & \(h: \vec{x} = \left(\begin{array}{c}2\\5\end{array}\right) + t\left(\begin{array}{c}-5\\-10\end{array}\right)\)
			\end{tabularx}
			Die Richtungsvektoren \(\vec{u} = \left(\begin{array}{c}3\\6\end{array}\right)\) und \(\vec{v} = \left(\begin{array}{c}-5\\-10\end{array}\right)\) sind \color{blue}linear abhängig \normalcolor{} (\(\vec{u} = -\frac{3}{5}\vec{v}\)). Die Geraden können also entweder \color{codegreen}parallel\normalcolor{} oder \color{red}gleich\normalcolor{} sein.\\
			Um dies zu bestimmen, setzen wir \(g=h\) und lösen das Gleichungssystem nach \(r\) und \(t\).\\
			\par\noindent
			\par\noindent
			\(\left(\begin{array}{c}1\\3\end{array}\right) + r\left(\begin{array}{c}3\\6\end{array}\right) = \left(\begin{array}{c}2\\5\end{array}\right) + t\left(\begin{array}{c}-5\\-10\end{array}\right)\)
			\begin{center}
				\begin{tabular}{cl}
					&\(1 + 3r = 2 - 5t\)\\
					&\(3 + 6r = 5 - 10t \Rightarrow r = \frac{2 - 10t}{6}\)\\
					\\
					\hline\\
					\par
					&\(1 + 3(\frac{2 - 10t}{6}) = 2 -5t\) | \(-1; +5t\)\\
					\(\Leftrightarrow\) & \(\frac{2 - 10t}{2} + 5t = 1\)\\
					\(\Leftrightarrow\) & \(1 - 5t + 5t = 1\)\\
					\(\Leftrightarrow\) & \(1 = 1\)\\
					\(\Rightarrow\) & Wir erhalten eine wahre Aussage\\
					& Die Geraden g und h sind also \textbf{\underline{gleich}}.
				\end{tabular}				
			\end{center}
			Da \(g=h\) müssen wir keinen Schnittpunkt mehr berechnen.		
		\end{framed}
		
		\begin{framed}
			\noindent
			\underline{\textit{S. 64 Nr. 8}} Die Gerade f geht durch den Punkt A(3|8|0) und hat den Richtungsvektor \(\left(\begin{array}{c}2\\5\\0\end{array}\right)\). Die Gerade h geht durch den Punkt B(-2|3|1) und hat den Stützvektor \(\left(\begin{array}{c}3\\1\\0\end{array}\right)\).\\
			Überprüfen Sie, ob sich die Geraden g und h schneiden. Berechnen Sie gegebenenfalls die Koordinaten des Schnittpunktes.\\
			\par\noindent
			Zunächst müssen wir die Geradengleichungen für \(g\) und \(h\) bestimmen.\\
			Für \(g\) können wir wie bekannt vorgehen. So ergibt sich die folgende Geradengleichung:
			\[g: \vec{x} = \left(\begin{array}{c}2\\8\\0\end{array}\right) + r\left(\begin{array}{c}2\\5\\0\end{array}\right)\]
			Um die Geradengleichung von \(h\) aufzustellen müssen wir folgendes beachten: Wir kennen den \color{codegreen}\underline{Stützvektor \(\vec{s}\)}\normalcolor{} sowie \color{codegreen}\underline{einen weiteren Punkt}\normalcolor, nicht aber den \color{red}\underline{Richtungsvektor}\normalcolor. Diesen müssen wir noch bestimmen. \tiny{\color{codegray}Hierfür berechnen wir \(\vec{B} - \vec{s} = \left(\begin{array}{c}-2\\3\\1\end{array}\right) - \left(\begin{array}{c}3\\1\\0\end{array}\right) = \left(\begin{array}{c}-5\\2\\1\end{array}\right)\).}
			\normalsize\\
			Wir erhalten damit \(h: \vec{x} = \left(\begin{array}{c}3\\1\\0\end{array}\right) + t\left(\begin{array}{c}-5\\2\\1\end{array}\right)\)\\
			\par\noindent
			Die Richtungsvektoren von sind \color{blue}linear unabhängig\normalcolor. Die Geraden können also \textbf{\underline{einen}} oder \textbf{\underline{keinen}} Schnittpunkt S haben. Um dies zu bestimmen, setzen wir \(g=h\) und lösen das Gleichungssystem.\\
			\begin{center}
				\begin{tabular}{lll}
					I & \(2 + 2r = 2 -5t\)\\
					II & \(8 +5r = 1 + 2t\) \\
					III & \(0 = t\) & \(\Rightarrow t = 0\)\\
					\hline 
					I' & \(2 + 2r = 2\) & \(\Rightarrow r = 0\)\\
					II' & \(8+5r = 1\) & \(\Rightarrow r = -\frac{7}{5}\)\\
					III & \( 0 = t\)
				\end{tabular}\\
			\end{center}
			Wir erhalten für \(r\) zwei unterschiedliche Werte. Somit hat das Gleichungssystem \textbf{\underline{keine}} Lösung \( \Rightarrow \) \(g\) und \(h\) \color{codegreen}\underline{schneiden sich nicht}\normalcolor{}. Sie sind also \textit{windschief}.
		\end{framed}
	\end{worksheet}
\end{document}