\documentclass[11pt,twocolumn,oneside,openany,headings=optiontotoc,11pt,numbers=noenddot]{article}

\usepackage[a4paper]{geometry}
\usepackage[utf8]{inputenc}
\usepackage[T1]{fontenc}
\usepackage{lmodern}
\usepackage[ngerman]{babel}
\usepackage{ngerman}

\usepackage[onehalfspacing]{setspace}

\usepackage{fancyhdr}
\usepackage{fancybox}

\usepackage{rotating}
\usepackage{varwidth}


\usepackage{pdflscape}
\usepackage{graphicx}
\usepackage{graphbox}
\graphicspath{
	{Pics/PDFs/}
	{Pics/JPGs/}
	{Pics/PNGs/}
}
\usepackage{caption}
\usepackage{tabularx}
\usepackage{dashrule}
\usepackage{hhline}
\usepackage{multirow}
\usepackage{enumerate}
\usepackage[hidelinks]{hyperref}
\usepackage{listings}

\usepackage[table]{xcolor}
\usepackage{array}
\usepackage{enumitem,amssymb,amsmath}
\usepackage{interval}
\usepackage{stmaryrd}
\usepackage{polynom}
\usepackage{diagbox}
\usepackage{dashrule}
\usepackage{framed}
\usepackage{mdframed}
\usepackage{karnaugh-map}

\usepackage{blindtext}

\usepackage{eso-pic}

\usepackage{amssymb}
\usepackage{eurosym}
\pagestyle{headings}
\renewcommand{\headrulewidth}{0.2pt}
\renewcommand{\footrulewidth}{0.2pt}
\newcommand*{\underdownarrow}[2]{\ensuremath{\underset{\overset{\Big\downarrow}{#2}}{#1}}}
\setlength{\fboxsep}{5pt}

% Codestyle defined
\definecolor{codegreen}{rgb}{0,0.6,0}
\definecolor{codegray}{rgb}{0.5,0.5,0.5}
\definecolor{codepurple}{rgb}{0.58,0,0.82}
\definecolor{backcolour}{rgb}{0.95,0.95,0.92}
\definecolor{deepgreen}{rgb}{0,0.5,0}
\definecolor{darkblue}{rgb}{0,0,0.65}
\definecolor{mauve}{rgb}{0.40, 0.19,0.28}
\colorlet{exceptioncolour}{yellow!50!red}
\colorlet{commandcolour}{blue!60!black}
\colorlet{numpycolour}{blue!60!green}
\colorlet{specmethodcolour}{violet}

%Neue Spaltendefinition
\newcolumntype{L}[1]{>{\raggedright\let\newline\\\arraybackslash\hspace{0pt}}m{#1}}
\newcolumntype{M}[1]{>{\centering\arraybackslash}X}
\newcommand{\cmnt}[1]{\ignorespaces}
%Textausrichtung ändern
\newcommand\tabrotate[1]{\rotatebox{90}{\raggedright#1\hspace{\tabcolsep}}}

%Intervall-Konfig
\intervalconfig {
	soft open fences
}

%Bash
\lstdefinestyle{BashInputStyle}{
	language=bash,
	basicstyle=\small\sffamily,
	backgroundcolor=\color{backcolour},
	columns=fullflexible,
	backgroundcolor=\color{backcolour},
	breaklines=true,
}
%Java
\lstdefinestyle{JavaInputStyle}{
	language=Java,
	backgroundcolor=\color{backcolour},
	aboveskip=1mm,
	belowskip=1mm,
	showstringspaces=false,
	columns=flexible,
	basicstyle={\footnotesize\ttfamily},
	numberstyle={\tiny},
	numbers=none,
	keywordstyle=\color{purple},,
	commentstyle=\color{deepgreen},
	stringstyle=\color{blue},
	emph={out},
	emphstyle=\color{darkblue},
	emph={[2]rand},
	emphstyle=[2]\color{specmethodcolour},
	breaklines=true,
	breakatwhitespace=true,
	tabsize=2,
}
%Python
\lstdefinestyle{PythonInputStyle}{
	language=Python,
	alsoletter={1234567890},
	aboveskip=1ex,
	basicstyle=\footnotesize,
	breaklines=true,
	breakatwhitespace= true,
	backgroundcolor=\color{backcolour},
	commentstyle=\color{red},
	otherkeywords={\ , \}, \{, \&,\|},
	emph={and,break,class,continue,def,yield,del,elif,else,%
		except,exec,finally,for,from,global,if,import,in,%
		lambda,not,or,pass,print,raise,return,try,while,assert},
	emphstyle=\color{exceptioncolour},
	emph={[2]True,False,None,min},
	emphstyle=[2]\color{specmethodcolour},
	emph={[3]object,type,isinstance,copy,deepcopy,zip,enumerate,reversed,list,len,dict,tuple,xrange,append,execfile,real,imag,reduce,str,repr},
	emphstyle=[3]\color{commandcolour},
	emph={[4]ode, fsolve, sqrt, exp, sin, cos, arccos, pi,  array, norm, solve, dot, arange, , isscalar, max, sum, flatten, shape, reshape, find, any, all, abs, plot, linspace, legend, quad, polyval,polyfit, hstack, concatenate,vstack,column_stack,empty,zeros,ones,rand,vander,grid,pcolor,eig,eigs,eigvals,svd,qr,tan,det,logspace,roll,mean,cumsum,cumprod,diff,vectorize,lstsq,cla,eye,xlabel,ylabel,squeeze},
	emphstyle=[4]\color{numpycolour},
	emph={[5]__init__,__add__,__mul__,__div__,__sub__,__call__,__getitem__,__setitem__,__eq__,__ne__,__nonzero__,__rmul__,__radd__,__repr__,__str__,__get__,__truediv__,__pow__,__name__,__future__,__all__},
	emphstyle=[5]\color{specmethodcolour},
	emph={[6]assert,range,yield},
	emphstyle=[6]\color{specmethodcolour}\bfseries,
	emph={[7]Exception,NameError,IndexError,SyntaxError,TypeError,ValueError,OverflowError,ZeroDivisionError,KeyboardInterrupt},
	emphstyle=[7]\color{specmethodcolour}\bfseries,
	emph={[8]taster,send,sendMail,capture,check,noMsg,go,move,switch,humTem,ventilate,buzz},
	emphstyle=[8]\color{blue},
	keywordstyle=\color{blue}\bfseries,
	rulecolor=\color{black!40},
	showstringspaces=false,
	stringstyle=\color{deepgreen}
}

\lstset{literate=%
	{Ö}{{\"O}}1
	{Ä}{{\"A}}1
	{Ü}{{\"U}}1
	{ß}{{\ss}}1
	{ü}{{\"u}}1
	{ä}{{\"a}}1
	{ö}{{\"o}}1
}

% Neue Klassenarbeits-Umgebung
\newenvironment{worksheet}[3]
% Begin-Bereich
{
	\newpage
	\sffamily
	\setcounter{page}{1}
	\ClearShipoutPicture
	\AddToShipoutPicture{
		\put(55,761){{
				\mbox{\parbox{385\unitlength}{\tiny \color{codegray}BBS I Mainz, #1 \newline #2
						\newline #3
					}
				}
			}
		}
		\put(455,761){{
				\mbox{\hspace{0.3cm}\includegraphics[width=0.2\textwidth]{../../logo.jpg}}
			}
		}
	}
}
% End-Bereich
{
	\clearpage
	\ClearShipoutPicture
}

\setlength{\columnsep}{3em}
\setlength{\columnseprule}{0.5pt}

\geometry{left=1.50cm,right=2.00cm,top=3.00cm,bottom=1.00cm,includeheadfoot}
\pagenumbering{arabic}
\pagestyle{plain}

\begin{document}
	\begin{worksheet}{BS FI 1. Lehrjahr}{LF 4 - Einfache IT-Systeme}{Digitaltechnik - Logische Verknüpfungen}
		\setcounter{section}{1}
		\setcounter{page}{5}
		\section{Logische Verknüpfungen}
		Betrachtet man die binären Zahlen \(0\) und \(1\) als Zustände, so kann man ihnen entsprechende Wahrheitswerte \(0 = false\) und \(1 = true\) zuweisen.\\
		Verknüpft man zwei Zustände, so ist der Ausgang (\textit{Output}) abhängig von den eingehenden (\textit{Input}) Zuständen. Dabei können auch wieder nur die entsprechenden Zustände \(0 = false\) und \(1 = true\) angenommen werden.\\
		\par\noindent
		Es stellt sich nun die Frage, \textbf{welche Verknüpfungen gibt es?}\\
		Hier ist zu unterscheiden zwischen \textbf{Grundverknüpfungen} und \textbf{Erweiterte Verknüpfungen}\footnote{Die Bezeichnung ist selbst gewählt.}.\\
		Nachfolgend betrachten wir zunächst die \underline{drei} Grundverknüpfungen. Dazu zählt zum einen die Funktionalität, das gebräuchlichste Schaltsymbol sowie die dazugehörige Wahrheitstabelle.
		\subsection{Grundverknüpfungen}
		Wir bezeichnen diese Verknüpfungen als \textit{Grundverknüpfungen}, da sie selbst nur unter Verwendung der anderen Grundverknüpfung dargestellt werden kann und sie zudem in Kombination alle in \textit{\nameref{ssec:erwVerk}} dargestellten Verknüpfungen erzeugen können.
		\subsubsection*{NOT}
		Die \textbf{NOT}-Verknüpfung wird auch als \textit{Negation} bezeichnet. Sie negiert also ihren Eingangszustand.\\
		Geht ein \textbf{0} rein, so kommt eine \textbf{1} raus.
		\par\noindent
		\begin{tabularx}{0.48\textwidth}{ll|l}
			\multirow{3}{*}{\includegraphics[width=0.1\textwidth,align=t]{../99_Bilder/NOT.jpg}} & \(x_1\) & \(f(x_1)\)\\
			\cline{2-3}
			 & 0 & 1\\
			& 1 & 0\\
		\end{tabularx}
		\subsubsection*{AND}
		Für die \textbf{AND}-Verknüpfung kann auch der Begriff \textit{Konjunktion} verwendet werden.\\
		Sie verknüpft die \underline{zwei} Eingangszustände als logisches UND. Dabei ist der Output nur dann \textbf{1}, wenn \underline{beide} Eingangszustände \textbf{1} sind.\\
		\par\noindent
		\begin{tabularx}{0.48\textwidth}{lll|l}
			& \(x_1\) & \(x_2\)& \(f(x_1;x_2)\)\\
			\cline{2-4}
			\multirow{3}{*}{\includegraphics[width=0.1\textwidth,align=t]{../99_Bilder/AND.jpg}} & 0 & 0 & 0\\
			& 0 & 1 & 0\\
			& 1 & 0 & 0\\
			& 1 & 1 & 1\\
		\end{tabularx}
		\subsubsection*{OR}
		Die \textbf{OR}-Verknüpfung wird auch als \textit{Disjunktion} bezeichnet.\\
		Sie verknüpft die \underline{zwei} Eingangszustände als logisches ODER. Dabei ist der Output in dem Moment \textbf{1}, wenn \underline{mindestens einer} der beiden Eingangszustände \textbf{1} ist.\\
		\par\noindent
		\begin{tabularx}{0.48\textwidth}{lll|l}
			& \(x_1\) & \(x_2\)& \(f(x_1;x_2)\)\\
			\cline{2-4}
			\multirow{3}{*}{\includegraphics[width=0.1\textwidth,align=t]{../99_Bilder/OR.jpg}} & 0 & 0 & 0\\
			& 0 & 1 & 1\\
			& 1 & 0 & 1\\
			& 1 & 1 & 1\\
		\end{tabularx}
		\subsection{Erweiterte Verknüpfungen}
		\label{ssec:erwVerk}
		\subsubsection*{NAND}
		Die Verknüpfung entspricht dem AND. \textbf{\underline{Vorsicht:}} Das Ausgangssignal wird negiert. Es ist also \textbf{immer 1}, \underline{außer wenn beide} Eingangszustände \textbf{1} sind.\\
		\par\noindent
		\begin{tabularx}{0.48\textwidth}{lll|l}
			& \(x_1\) & \(x_2\)& \(f(x_1;x_2)\)\\
			\cline{2-4}
			\multirow{3}{*}{\includegraphics[width=0.1\textwidth,align=t]{../99_Bilder/NAND.jpg}} & 0 & 0 & 1\\
			& 0 & 1 & 1\\
			& 1 & 0 & 1\\
			& 1 & 1 & 0\\
		\end{tabularx}
		\subsubsection*{NOR}
		Die Verknüpfung entspricht dem OR. \textbf{\underline{Aber}} auch hier wird das Ausgangssignal negiert. Entsprechend ist der Output  \textbf{nur 1}, \underline{wenn beide} Eingangszustände \textbf{0} sind.\\
		\par\noindent
		\begin{tabularx}{0.48\textwidth}{lll|l}
			& \(x_1\) & \(x_2\)& \(f(x_1;x_2)\)\\
			\cline{2-4}
			\multirow{3}{*}{\includegraphics[width=0.1\textwidth,align=t]{../99_Bilder/NOR.jpg}} & 0 & 0 & 1\\
			& 0 & 1 & 0\\
			& 1 & 0 & 0\\
			& 1 & 1 & 0\\
		\end{tabularx}
		\subsubsection*{XOR}
		Die logische Verknüpfung OR wird hier verschärft. Das Ausgangssignal ist \textbf{nur genau dann 1}, \underline{wenn eine} der beiden Eingangssignale \textbf{1} ist.\\
		\par\noindent
		\begin{tabularx}{0.48\textwidth}{lll|l}
			& \(x_1\) & \(x_2\)& \(f(x_1;x_2)\)\\
			\cline{2-4}
			\multirow{3}{*}{\includegraphics[width=0.1\textwidth,align=t]{../99_Bilder/XOR.jpg}} & 0 & 0 & 0\\
			& 0 & 1 & 1\\
			& 1 & 0 & 1\\
			& 1 & 1 & 0\\
		\end{tabularx}
		\subsubsection*{<-> (XNOR)}
		Diese Verknüpfung entspricht dem invertierten XOR. Das bedeutet, das Ausgangssignal ist \textbf{genau dann 1}, wenn \underline{beide} Eingangszustände \textbf{0} oder \underline{beide} \textbf{1} sind.\\
		\par\noindent
		\begin{tabularx}{0.48\textwidth}{lll|l}
			& \(x_1\) & \(x_2\)& \(f(x_1;x_2)\)\\
			\cline{2-4}
			\multirow{3}{*}{\includegraphics[width=0.1\textwidth,align=t]{../99_Bilder/equi.jpg}} & 0 & 0 & 1\\
			& 0 & 1 & 0\\
			& 1 & 0 & 0\\
			& 1 & 1 & 1\\
		\end{tabularx}\\
		\par\noindent
		\underline{\textbf{Ihre Aufgabe}}:\\
		\par\noindent
		Überlegen Sie für die einzelnen Erweiterten Verknüpfungen, wie sie diese \underline{nur} unter Verwendung der \textbf{Grundverknüpfungen} realisieren können.
	\end{worksheet}
\end{document}