\documentclass[oneside,openany,headings=optiontotoc,11pt,numbers=noenddot]{article}

\usepackage[a4paper]{geometry}
\usepackage[utf8]{inputenc}
\usepackage[T1]{fontenc}
\usepackage{lmodern}
\usepackage[ngerman]{babel}
\usepackage{ngerman}

\usepackage[onehalfspacing]{setspace}

\usepackage{fancyhdr}
\usepackage{fancybox}

\usepackage{rotating}
\usepackage{varwidth}


\usepackage{pdflscape}
\usepackage{graphicx}
\usepackage{graphbox}
\graphicspath{
	{Pics/PDFs/}
	{Pics/JPGs/}
	{Pics/PNGs/}
}
\usepackage{caption}
\usepackage{tabularx}
\usepackage{dashrule}
\usepackage{hhline}
\usepackage{multirow}
\usepackage{enumerate}
\usepackage[hidelinks]{hyperref}
\usepackage{listings}

\usepackage[table]{xcolor}
\usepackage{array}
\usepackage{enumitem,amssymb,amsmath}
\usepackage{interval}
\usepackage{stmaryrd}
\usepackage{polynom}
\usepackage{diagbox}
\usepackage{dashrule}
\usepackage{framed}
\usepackage{mdframed}
\usepackage{karnaugh-map}

\usepackage{blindtext}

\usepackage{eso-pic}

\usepackage{amssymb}
\usepackage{eurosym}
\pagestyle{headings}
\renewcommand{\headrulewidth}{0.2pt}
\renewcommand{\footrulewidth}{0.2pt}
\newcommand*{\underdownarrow}[2]{\ensuremath{\underset{\overset{\Big\downarrow}{#2}}{#1}}}
\setlength{\fboxsep}{5pt}

% Codestyle defined
\definecolor{codegreen}{rgb}{0,0.6,0}
\definecolor{codegray}{rgb}{0.5,0.5,0.5}
\definecolor{codepurple}{rgb}{0.58,0,0.82}
\definecolor{backcolour}{rgb}{0.95,0.95,0.92}
\definecolor{deepgreen}{rgb}{0,0.5,0}
\definecolor{darkblue}{rgb}{0,0,0.65}
\definecolor{mauve}{rgb}{0.40, 0.19,0.28}
\colorlet{exceptioncolour}{yellow!50!red}
\colorlet{commandcolour}{blue!60!black}
\colorlet{numpycolour}{blue!60!green}
\colorlet{specmethodcolour}{violet}

%Neue Spaltendefinition
\newcolumntype{L}[1]{>{\raggedright\let\newline\\\arraybackslash\hspace{0pt}}m{#1}}
\newcolumntype{M}[1]{>{\centering\arraybackslash}X}
\newcommand{\cmnt}[1]{\ignorespaces}
%Textausrichtung ändern
\newcommand\tabrotate[1]{\rotatebox{90}{\raggedright#1\hspace{\tabcolsep}}}

%Intervall-Konfig
\intervalconfig {
	soft open fences
}

%Bash
\lstdefinestyle{BashInputStyle}{
	language=bash,
	basicstyle=\small\sffamily,
	backgroundcolor=\color{backcolour},
	columns=fullflexible,
	backgroundcolor=\color{backcolour},
	breaklines=true,
}
%Java
\lstdefinestyle{JavaInputStyle}{
	language=Java,
	backgroundcolor=\color{backcolour},
	aboveskip=1mm,
	belowskip=1mm,
	showstringspaces=false,
	columns=flexible,
	basicstyle={\footnotesize\ttfamily},
	numberstyle={\tiny},
	numbers=none,
	keywordstyle=\color{purple},,
	commentstyle=\color{deepgreen},
	stringstyle=\color{blue},
	emph={out},
	emphstyle=\color{darkblue},
	emph={[2]rand},
	emphstyle=[2]\color{specmethodcolour},
	breaklines=true,
	breakatwhitespace=true,
	tabsize=2,
}
%Python
\lstdefinestyle{PythonInputStyle}{
	language=Python,
	alsoletter={1234567890},
	aboveskip=1ex,
	basicstyle=\footnotesize,
	breaklines=true,
	breakatwhitespace= true,
	backgroundcolor=\color{backcolour},
	commentstyle=\color{red},
	otherkeywords={\ , \}, \{, \&,\|},
	emph={and,break,class,continue,def,yield,del,elif,else,%
		except,exec,finally,for,from,global,if,import,in,%
		lambda,not,or,pass,print,raise,return,try,while,assert},
	emphstyle=\color{exceptioncolour},
	emph={[2]True,False,None,min},
	emphstyle=[2]\color{specmethodcolour},
	emph={[3]object,type,isinstance,copy,deepcopy,zip,enumerate,reversed,list,len,dict,tuple,xrange,append,execfile,real,imag,reduce,str,repr},
	emphstyle=[3]\color{commandcolour},
	emph={[4]ode, fsolve, sqrt, exp, sin, cos, arccos, pi,  array, norm, solve, dot, arange, , isscalar, max, sum, flatten, shape, reshape, find, any, all, abs, plot, linspace, legend, quad, polyval,polyfit, hstack, concatenate,vstack,column_stack,empty,zeros,ones,rand,vander,grid,pcolor,eig,eigs,eigvals,svd,qr,tan,det,logspace,roll,mean,cumsum,cumprod,diff,vectorize,lstsq,cla,eye,xlabel,ylabel,squeeze},
	emphstyle=[4]\color{numpycolour},
	emph={[5]__init__,__add__,__mul__,__div__,__sub__,__call__,__getitem__,__setitem__,__eq__,__ne__,__nonzero__,__rmul__,__radd__,__repr__,__str__,__get__,__truediv__,__pow__,__name__,__future__,__all__},
	emphstyle=[5]\color{specmethodcolour},
	emph={[6]assert,range,yield},
	emphstyle=[6]\color{specmethodcolour}\bfseries,
	emph={[7]Exception,NameError,IndexError,SyntaxError,TypeError,ValueError,OverflowError,ZeroDivisionError,KeyboardInterrupt},
	emphstyle=[7]\color{specmethodcolour}\bfseries,
	emph={[8]taster,send,sendMail,capture,check,noMsg,go,move,switch,humTem,ventilate,buzz},
	emphstyle=[8]\color{blue},
	keywordstyle=\color{blue}\bfseries,
	rulecolor=\color{black!40},
	showstringspaces=false,
	stringstyle=\color{deepgreen}
}

\lstset{literate=%
	{Ö}{{\"O}}1
	{Ä}{{\"A}}1
	{Ü}{{\"U}}1
	{ß}{{\ss}}1
	{ü}{{\"u}}1
	{ä}{{\"a}}1
	{ö}{{\"o}}1
}

% Neue Klassenarbeits-Umgebung
\newenvironment{worksheet}[3]
% Begin-Bereich
{
	\newpage
	\sffamily
	\setcounter{page}{1}
	\ClearShipoutPicture
	\AddToShipoutPicture{
		\put(55,761){{
				\mbox{\parbox{385\unitlength}{\tiny \color{codegray}BBS I Mainz, #1 \newline #2
						\newline #3
					}
				}
			}
		}
		\put(455,761){{
				\mbox{\hspace{0.3cm}\includegraphics[width=0.2\textwidth]{../../logo.jpg}}
			}
		}
	}
}
% End-Bereich
{
	\clearpage
	\ClearShipoutPicture
}

\geometry{left=2.50cm,right=2.50cm,top=3.00cm,bottom=1.00cm,includeheadfoot}
\pagestyle{plain}
\pagenumbering{arabic}

\begin{document}				
	\begin{worksheet}{Erfahrungsbericht Auslandsaufenthalt}{St. Joseph\grq{}s Academy, Baton Rouge, Louisiana USA}{22. September 2018 bis 12. Oktober 2018}
		\begin{center}
			\section*{\underline{My stay at St. Joseph\grq{}s Academy in Baton Rouge}}
		\end{center}
		\setcounter{section}{1}
		\subsection*{\textit{A brief introduction}}
		Ich bin Carolyn Wesp, 28 Jahre und gehöre der \textbf{H17} an. Ich unterrichte die Fächer \textbf{Informatik} und \textbf{Mathematik} an der \textbf{Berufsbildenden Schule 1 Gewerbe und Technik} in Mainz.
		\subsection*{\textit{How did I find SJA}}
		Als wir im Frühjahr 2018 während einer Veranstaltung auf die Möglichkeit eines Auslandsaufenthaltes während des Referendariats hingewiesen wurde, war mir sofort klar, das möchte ich machen.\\
		Um diese Möglichkeit auf jeden Fall wahrnehmen zu können, habe ich mich im Bekanntenkreis meiner Mutter, die am Flughafen arbeitet und somit Kontakte in der ganzen Welt verteilt hat, umgehört, ob dort jemand Kontakt zu einer Schule hat. So gelangte ich an die \textbf{St. Joseph\grq{}s Academy}.
		\subsection{About SJA}
		\begin{wrapfigure}{L}{0.3\textwidth}
			\centering
			\includegraphics[width=0.25\textwidth]{SJA.jpg}
			\caption{\label{fig:sja}Schulemblem der St. Joseph\grq{}s Academy}
		\end{wrapfigure}
		St. Josph\grq{}s Academy, eine katholische Privatschule, ist die größte Mädchenschule des Landes und obwohl sie erst ab 1977 eine reine Oberschule wurde, ist sie die älteste High School in Baton Rouge, LA.\marginpar{\textit{\tiny{Louisiana}}} Die Schule wurde 1868 von den Schwestern des St. Joseph gegründet. Sie haben es sich zur Aufgabe gemacht, innerhalb der Schulgemeinschaft ein Umfeld für Exzellenz zu schaffen, in welchem die Schülerinnen zu verantwortungsbewussten Mitgliedern der Gesellschaft zu erziehen und ihnen Möglichkeiten zu eröffnen, in ihrem Glauben zu wachsen, sich schulisch wie auch persönlich weiterzuentwickeln.\\
		Dieses Leitmotiv, \textit{Sactity, Joy, Action}, verfolgt die Schule nun bereits seit 150 Jahren.\\
		
		\noindent
		1941 siedelte die Schule vom Zentrum von Baton Rouge in eine Wohngegend um. In direkter Nachbarschaft befindet sich die \textbf{Catholic High School}, eine katholische Jungen-Privatschule.\\
		Bestimmte Fächer, \small{\textit{Chor, Theater}},\normalsize werden in Kooperation mit Catholic High unterrichtet, so dass die Schülerinnen hierfür in das nahegelegene Schulgebäude wechseln. Für andere Fächer, \small{\textit{Kunst}},\normalsize  kommen die Schüler der Catholic High in die Räume der SJA.\\
		
		\noindent
		1998 trat einer der Führungskräfte von Hewlett-Packard an die Schule heran und spendete eine große Geldsumme, um jede Schülerin mit einem eigenen persönlichen Laptop auszustatten, welchen sie während ihrer gesamten Schulzeit nutzen kann.\\
		An diesem System wurde festgehalten, so dass die Schülerinnen wie auch die Lehrerkräfte seit 1998 alle mit einem persönlichen Laptop ausgestattet sind und der gesamte Schulcampus über eine fächendeckende drahtlose Netzwerkabdeckung verfügt.\\
		Die Nutzung und Akzeptanz dieser technische Ausstattung lässt sich in allen Fächern erkennen.\\
		
		\noindent
		Die Schule hat einen besonderen Schwerpunkt in \textbf{STEM} (\textit{science, technology, engeneering and mathematics}) gewählt, da Frauen in der Geschichte dieser Bereiche im Allgemeinen unterrepräsentiert sind. Daher hat Andrea Clesi McMakin 1974 das STEM Lab ins Leben gerufen, um einen fächerübergreifenden, praktischen Bereich zu schaffen, in dem die Schülerinnen mit Laser-Cuttern, Mikrocomputern, 3D Druckern und andere technischen Gerätschaften die Möglichkeit haben ihrer Kreativität freien lauf zu lassen und dabei Unterrichtsinhalte \grq{}spielerisch\grq{} zu erlernen und zu verstehen.\\
		
		\noindent
		Zustäzlich dazu wurde das \textit{Innovation and Design Lab}\marginpar{\textit{\tiny{I\&D}}} gegründet. Dieses ermöglicht den Schülerinnen innnovativ und probjektbasiert lernen und kann sie unterstützen, ihre Kompetenzen in den Bereichen kritisches Denken, Zusammenarbeit sowie Problemlösung zu entwickeln und zu fördern.\\
		Hierfür erhalten Sie die Möglichkeit, Projekt zu entwerfen, mit denen sie reale Probleme lösen und ihre Gruppen positiv verändern können. Das I\&D Lab ist also ein Klassenraum in dem über den kurikularen Rand hinaus gelernt werden und sich entwickelt werden kann.
		
		\subsection{The Schedule}
		Der Unterricht beginnt an jeden Tag um 7:30 und endet um 14:47. Wie es zu dieser ungewöhnlichen Zeit kommt, lässt sich an folgender Tabelle erkennen.\\
		\begin{center}
			\tiny
			\begin{tabularx}{0.5\textwidth}{X|X}
				Class: & 50 Minuten\\
				\hline
				\hline
				First Bell & 7:25\\
				Prayer \& Pledge & 7:28\\
				\hline
				1\(^{st}\) Period & 7:30 - 8:20\\
				2\(^{nd}\) Period & 8:25 - 9:15\\
				\hline
				Announcements & 9:20 - 9:22\\
				\hline
				3\(^{rd}\) Period & 9:22 - 10:12\\
				4\(^{th}\) Period & 10:17 - 11:07\\
				5\(^{th}\) Period & 11:12 - 12:02\\
				6\(^{th}\) Period & 12:07 - 12:57\\
				7\(^{th}\) Period & 13:02 - 13:52\\
				8\(^{th}\) Period & 13:57 - 14:47
			\end{tabularx}
		\end{center}
		\normalsize
		\small{\textbf{Erläuterungen}:\\
		\indent
		\textit{Prayer \& Pledge} - Jeder Schultag bzw. jede Schulstunde beginnt mit einem Gebet. Dabei Bekreuzigen sich die Schülerinnen und die Lehrkräfte vor und nach dem Gebet. Das Morgengebet wird durch die Lautsprecheransage \glqq{}\textit{Please stand for morning prayer!}\grqq{} angekündigt. Die Schülerinnen wie auch die Lehrkräfte verharren also an ihrem aktuellen Ort, bekreuzigen sich und beten gemeinsam im Sprechgesang.\\
		Lediglich vor der ersten Schulstunde sprechen die Schülerinnen und Lehrkräfte den Treueschwur. Dafür drehen sie sich den amerikanischen Flagge zu, die in jedem Klassensaal aufgehängt ist.
		\begin{center}
			\begin{minipage}{0.7\textwidth}
				\begin{center}
					\tiny
					
					\textit{\glqq{}I pledge allegiance to the flag of the United States of America, and to the republic for which it stands, one Nation under God, indivisible, with liberty and justice for all.\grqq{}}
				\end{center}
				\normalsize
			\end{minipage}
		\end{center}	
		Während dieses Treueschwurs ist ein wahrer Sprechgesang in den Gängen der Schule zu hören.\\
		\indent
		\textit{Announcements} Bei den Announcements werden aktuelle Aktionen (z.B. Unterstützung von Bedürftigen), anstehende Termine oder Ergebnisse von Sportwettkämpfen der Schulteams verkündet. Ebenfalls wird den Schülerinnen gratuliert, die an diesem Tag bzw. am vergangenen Wochenende Geburtstag hatten gratuliert.}\\
		\normalsize
		\par\noindent
		Dies sind die Unterrichtszeiten für einen regulären Schultag. Während meines Aufenthalts wurde allerdings an einem Tag eine Messe abgehalten, es fand eine Pep-Rally und die Schule hatte eine Autorin für einen Vortrag im Rahmen der \textit{Speaker-Series} eingeladen.\\
		An diesen Tagen waren die Unterrichtsstunden verkürzt, so dass kein Unterricht entfiel.\\
		\par\noindent
		\tiny
		\begin{tabularx}{\textwidth}{X|X||X|X}
			\multicolumn{2}{c}{Schedule 4} & \multicolumn{2}{c}{Schedule 3}\\
			(Messe bzw. Vortrag) & & (Pep-Rally)\\
			Class: 40 Minuten & Assembly: 78 Minuten & Class: 42 Minuten & Assemby: 58 Minuten\\
			\hline
			\hline
			First Bell & 7:25 & First Bell & 7:25\\
			Prayer \& Pledge & 7:28 & Prayer \& Pledge & 7:28\\
			1\(^{st}\) Period & 7:30 - 8:10 & 1\(^{st}\) Period & 7:30 - 8:10\\
			2\(^{nd}\) Period & 8:15 - 8:55 & 2\(^{nd}\) Period & 8:15 - 8:55\\
			\cline{3-4}
			3\(^{rd}\) Period & 9:00 - 9:40 & Announcements & 9:45 - 11:02\\
			\hline
			\textbf{Assembly/} & \textbf{9:45 - 11:02} & 3\(^{rd}\) Period & 9:00 - 9:40\\
			\textbf{Announcements} & & \\
			\cline{1-2}
			4\(^{th}\) Period & 11:07 - 11:47 & 4\(^{th}\) Period & 11:07 - 11:47\\
			5\(^{th}\) Period & 11:52 - 12:32 & 5\(^{th}\) Period & 11:52 - 12:32\\
			6\(^{th}\) Period & 12:37 - 13:17 & 6\(^{th}\) Period & 12:37 - 13:17\\
			7\(^{th}\) Period & 13:22 - 14:02 & 7\(^{th}\) Period & 13:22 - 14:02\\
			8\(^{th}\) Period & 14:07 - 14:47 & 8\(^{th}\) Period & 14:07 - 14:47\\
			\cline{3-4}
			& & \textbf{Assembly} & \textbf{13:49 - 14:47}\\
			\cline{3-4}
		\end{tabularx}
		\normalsize
		\subsection{General Information}
		Die Schülerinnen erhalten zu Schuljahresbeginn einen Stundenplan, der jeden Tag der gleiche ist. Das bedeutet, sie haben jeden Tag höchstens sieben verschiedene Fächer. Diese verteilen sich auf die Stunden 1 bis 8. Abhängig von den zugeteilten Kursen haben die Schülerinnen in der vierten, fünften oder sechsten Stunde eine Mittagspause.\marginpar{\textit{\tiny{Diese Pausenzeiten gelten auch für die Lehrkräfte.}}}\\
		Um die Schülerinnen in den Gängen aber auch in den Klassensälen zum einen mit ihrem Namen ansprechen und sie zum anderen einer bestimmten Klassenstufe zuordnen zu können, sind die Schülerinnen dazu verpflichtet ihren Schülerausweis sichtbar am Körper zu tragen. Dabei gilt die folgende farbliche Zuordnung\\
		\begin{tabularx}{\textwidth}{llll}
			Grün - Freshmen - 9 & Blau - Sophomore - 10 & Pink - Junior - 11 & Gelb - Senior - 12
		\end{tabularx}
		Die Lehrkräfte hingegen haben jeder ein Namensschild, welches sie auch verpflichtend den ganzen Tag tragen. Wie auch die Schülerinnen haben alle Lehrkräfte einen \grqq{}Mitarbeiterausweis\grqq{}, dieser ordnet die Tragenden durch den roten Streifen automatisch der Angestelltenschaft zu.
		\subsection{Mathematics department}
		Das \textit{mathematics department} besteht aus 10 Lehrkräften. Diese haben alle ihren Bachelor of Science in Mathematik gemacht. Einige haben im Anschluss ihren Master mit dem Schwerpunkt Education in irgendeiner Form gemacht. Ein Studium im Bereich Education ist bei allen vorhanden.\\
		\par\noindent
		In der ersten Schulwoche, während die neuen Schülerinnen eine erste Orientierung für das anstehende Schuljahr erhalten, haben die Lehrkräfte der einzelnen Abteilungen Zeit, sich zusammenzusetzen und den Unterricht für das erste Halbjahr zu planen.
		\subsection{STEM Lab}
		...
		\subsection{Helpdesk}
		...
		\section{What happened}
		\subsection{Week 1}
		...
		\subsection{Week 2}
		...
		\subsection{Week 3}
		...
		\subsection{Memorable Impressions}
		LSU Football Game am 01.10.2018\\
		Ich habe mir ein LSU T-Shirt gekauft - \textbf{GO Tigers!}
	\end{worksheet}
\end{document}