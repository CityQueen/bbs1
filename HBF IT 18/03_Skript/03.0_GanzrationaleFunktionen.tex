\documentclass[11pt,twocolumn,oneside,openany,headings=optiontotoc,11pt,numbers=noenddot]{article}

\usepackage[a4paper]{geometry}
\usepackage[utf8]{inputenc}
\usepackage[T1]{fontenc}
\usepackage{lmodern}
\usepackage[ngerman]{babel}
\usepackage{ngerman}

\usepackage[onehalfspacing]{setspace}

\usepackage{fancyhdr}
\usepackage{fancybox}

\usepackage{rotating}
\usepackage{varwidth}


\usepackage{pdflscape}
\usepackage{graphicx}
\usepackage{graphbox}
\graphicspath{
	{Pics/PDFs/}
	{Pics/JPGs/}
	{Pics/PNGs/}
}
\usepackage{caption}
\usepackage{tabularx}
\usepackage{dashrule}
\usepackage{hhline}
\usepackage{multirow}
\usepackage{enumerate}
\usepackage[hidelinks]{hyperref}
\usepackage{listings}

\usepackage[table]{xcolor}
\usepackage{array}
\usepackage{enumitem,amssymb,amsmath}
\usepackage{interval}
\usepackage{stmaryrd}
\usepackage{polynom}
\usepackage{diagbox}
\usepackage{dashrule}
\usepackage{framed}
\usepackage{mdframed}
\usepackage{karnaugh-map}

\usepackage{blindtext}

\usepackage{eso-pic}

\usepackage{amssymb}
\usepackage{eurosym}
\pagestyle{headings}
\renewcommand{\headrulewidth}{0.2pt}
\renewcommand{\footrulewidth}{0.2pt}
\newcommand*{\underdownarrow}[2]{\ensuremath{\underset{\overset{\Big\downarrow}{#2}}{#1}}}
\setlength{\fboxsep}{5pt}

% Codestyle defined
\definecolor{codegreen}{rgb}{0,0.6,0}
\definecolor{codegray}{rgb}{0.5,0.5,0.5}
\definecolor{codepurple}{rgb}{0.58,0,0.82}
\definecolor{backcolour}{rgb}{0.95,0.95,0.92}
\definecolor{deepgreen}{rgb}{0,0.5,0}
\definecolor{darkblue}{rgb}{0,0,0.65}
\definecolor{mauve}{rgb}{0.40, 0.19,0.28}
\colorlet{exceptioncolour}{yellow!50!red}
\colorlet{commandcolour}{blue!60!black}
\colorlet{numpycolour}{blue!60!green}
\colorlet{specmethodcolour}{violet}

%Neue Spaltendefinition
\newcolumntype{L}[1]{>{\raggedright\let\newline\\\arraybackslash\hspace{0pt}}m{#1}}
\newcolumntype{M}[1]{>{\centering\arraybackslash}X}
\newcommand{\cmnt}[1]{\ignorespaces}
%Textausrichtung ändern
\newcommand\tabrotate[1]{\rotatebox{90}{\raggedright#1\hspace{\tabcolsep}}}

%Intervall-Konfig
\intervalconfig {
	soft open fences
}

%Bash
\lstdefinestyle{BashInputStyle}{
	language=bash,
	basicstyle=\small\sffamily,
	backgroundcolor=\color{backcolour},
	columns=fullflexible,
	backgroundcolor=\color{backcolour},
	breaklines=true,
}
%Java
\lstdefinestyle{JavaInputStyle}{
	language=Java,
	backgroundcolor=\color{backcolour},
	aboveskip=1mm,
	belowskip=1mm,
	showstringspaces=false,
	columns=flexible,
	basicstyle={\footnotesize\ttfamily},
	numberstyle={\tiny},
	numbers=none,
	keywordstyle=\color{purple},,
	commentstyle=\color{deepgreen},
	stringstyle=\color{blue},
	emph={out},
	emphstyle=\color{darkblue},
	emph={[2]rand},
	emphstyle=[2]\color{specmethodcolour},
	breaklines=true,
	breakatwhitespace=true,
	tabsize=2,
}
%Python
\lstdefinestyle{PythonInputStyle}{
	language=Python,
	alsoletter={1234567890},
	aboveskip=1ex,
	basicstyle=\footnotesize,
	breaklines=true,
	breakatwhitespace= true,
	backgroundcolor=\color{backcolour},
	commentstyle=\color{red},
	otherkeywords={\ , \}, \{, \&,\|},
	emph={and,break,class,continue,def,yield,del,elif,else,%
		except,exec,finally,for,from,global,if,import,in,%
		lambda,not,or,pass,print,raise,return,try,while,assert},
	emphstyle=\color{exceptioncolour},
	emph={[2]True,False,None,min},
	emphstyle=[2]\color{specmethodcolour},
	emph={[3]object,type,isinstance,copy,deepcopy,zip,enumerate,reversed,list,len,dict,tuple,xrange,append,execfile,real,imag,reduce,str,repr},
	emphstyle=[3]\color{commandcolour},
	emph={[4]ode, fsolve, sqrt, exp, sin, cos, arccos, pi,  array, norm, solve, dot, arange, , isscalar, max, sum, flatten, shape, reshape, find, any, all, abs, plot, linspace, legend, quad, polyval,polyfit, hstack, concatenate,vstack,column_stack,empty,zeros,ones,rand,vander,grid,pcolor,eig,eigs,eigvals,svd,qr,tan,det,logspace,roll,mean,cumsum,cumprod,diff,vectorize,lstsq,cla,eye,xlabel,ylabel,squeeze},
	emphstyle=[4]\color{numpycolour},
	emph={[5]__init__,__add__,__mul__,__div__,__sub__,__call__,__getitem__,__setitem__,__eq__,__ne__,__nonzero__,__rmul__,__radd__,__repr__,__str__,__get__,__truediv__,__pow__,__name__,__future__,__all__},
	emphstyle=[5]\color{specmethodcolour},
	emph={[6]assert,range,yield},
	emphstyle=[6]\color{specmethodcolour}\bfseries,
	emph={[7]Exception,NameError,IndexError,SyntaxError,TypeError,ValueError,OverflowError,ZeroDivisionError,KeyboardInterrupt},
	emphstyle=[7]\color{specmethodcolour}\bfseries,
	emph={[8]taster,send,sendMail,capture,check,noMsg,go,move,switch,humTem,ventilate,buzz},
	emphstyle=[8]\color{blue},
	keywordstyle=\color{blue}\bfseries,
	rulecolor=\color{black!40},
	showstringspaces=false,
	stringstyle=\color{deepgreen}
}

\lstset{literate=%
	{Ö}{{\"O}}1
	{Ä}{{\"A}}1
	{Ü}{{\"U}}1
	{ß}{{\ss}}1
	{ü}{{\"u}}1
	{ä}{{\"a}}1
	{ö}{{\"o}}1
}

% Neue Klassenarbeits-Umgebung
\newenvironment{worksheet}[3]
% Begin-Bereich
{
	\newpage
	\sffamily
	\setcounter{page}{1}
	\ClearShipoutPicture
	\AddToShipoutPicture{
		\put(55,761){{
				\mbox{\parbox{385\unitlength}{\tiny \color{codegray}BBS I Mainz, #1 \newline #2
						\newline #3
					}
				}
			}
		}
		\put(455,761){{
				\mbox{\hspace{0.3cm}\includegraphics[width=0.2\textwidth]{../../logo.jpg}}
			}
		}
	}
}
% End-Bereich
{
	\clearpage
	\ClearShipoutPicture
}

\setlength{\columnsep}{3em}
\setlength{\columnseprule}{0.5pt}

\geometry{left=1.50cm,right=1.50cm,top=3.00cm,bottom=1.00cm,includeheadfoot}
\pagenumbering{gobble}
\pagestyle{empty}

\begin{document}
	\begin{worksheet}{Höhere Berufsfachschule IT-Systeme}{Grundstufe - Mathematik}{Lernabschnitt 3: Ganzrationale Funktionen}
		\setcounter{section}{5}
		\section{Ganzrationale Funktionen}
		Spricht man von \textbf{ganzrationalen Funktionen}, meint man immer eine Funktion der Form \[f(x) = a_nx^n + a_{n-1}x^{n-1} + \ldots + a_1x + a_0\color{codegray}{x^0}\] Dabei besteht jeder Summand aus \(a_n\cdot{}x^n\) und \(n\) ist eine natürliche Zahle, also \(n \in \mathbb{N}\).\\
		\par\noindent
		\rule{0.48\textwidth}{0.1pt}\\
		\underline{\color{red}{Vorsicht!}}\\
		In einer ganzrationalen Funktion \underline{müssen nicht} alle Exponenten bis Null (0) vorkommen.\\
		\rule{0.48\textwidth}{0.1pt}
		\subsection{Begrifflichkeiten}
		\subsubsection*{\underline{Grad}}
		Als \textbf{Grad} \(\mathbf{n}\) einer Funktion bezeichnet man den größten vorkommenden Exponenten.\\
		\par\noindent
		\textbf{\underline{Beispiel:}} \(f(x) = 3x^4 + 5x^2 -3\)\\
		Diese Funktion hat den Grad \(n = 4\) - wegen \fbox{\(3x\)\colorbox{green!10}{\(^4\)}}.
		\subsubsection*{\underline{Koeffizient}}
		Mit dem Begriff \textbf{Koeffizient} \(a_n\) bezeichnet man immer den \textbf{Faktor vor} dem \(x^n\).\\
		\par\noindent
		\underline{\textbf{Beispiel:}} \(f(x) = -4x^3 + 3x^{\mathit{1}} - 1\cdot{}\mathit{x^0}\) hat die Koeffizienten: \(a_3 = -4; a_1 = 3; a_0 = -1\)\\
		\par\noindent
		\(f(x) = \)\colorbox{green!10}{\(2\)}\(x^4 + \)\colorbox{green!10}{\(1\)}\(\cdot{}x^3 \)\colorbox{green!10}{\(- 4\)}\(x^{\mathit{1}}\) hat die Koeffizienten \(a_4 = 2; a_3 = 1; a_1 = -4\).
		\subsubsection*{\underline{Charakteristischer Summand}}
		Bei einer ganzrationalen Funktion bezeichnet man den Summanden, der den größten Exponenten hat, als \textbf{charakteristischen Summanden}.\\
		\par\noindent
		\underline{\textbf{Beispiel:}} Wir betrachten die Funktionsgleichung \(f(x) = -\frac{1}{3}x^4 + 3x^2 - 6x\).\\
		Der charakteristische Summand ist \colorbox{green!10}{\(-\frac{1}{3}x^4\)}. Setzt sich also zusammen aus dem \(x\)-Term und seinem Koeffizienten.
		\subsection{Prototypen}
		Eine ganzrationale Funktion kann in unterschiedlicher Form auftreten. Diese bezeichnen wir als Prototypen.
		\subsubsection*{\underline{Polynomform}}
		Der oben bereits erwähnte Prototyp \[f(x) = a_nx^n + a_{n-1}x^{n-1} + \ldots + a_1x^1 + a_0\] wird Polynom genannt und dementsprechend heißt diese Darstellung \textbf{Polynomform}.
		\subsubsection*{\underline{Faktorform}}
		Wie auch bei den quadratischen Funktionen kann eine ganzrationale Funktion als Produkt der \grq{}Nullstellen-Polynome\grq{} dargestellt werden.
		\[f(x) = a(x-N_1)(x-N_2)\cdot{}\ldots{}\cdot{}(x-N_{n_1})(x-N_n)\]
		Dabei geben uns die einzelnen \grq{}Nullstellen-Polynome\grq{} Auskunft über die Nullstellen.\\
		\par\noindent
		\underline{\textbf{Beispiel:}} \(f(x) = (x-\underbrace{2}_{N_1})\cdot(x-\underbrace{1}_{N_2})\cdot(x-\underbrace{(-3)}_{N_3})\)
		\subsubsection*{Polynomform \(\Leftrightarrow\) Faktorform}
		\textbf{FF \(\Rightarrow\) PF}\\
		Haben wir eine ganzrationale Funktion in \textbf{Faktorform} \(f(x) = a(x-N_1)\cdot\ldots\cdot(x-N_{n-1})\cdot(x-N_{n})\) gegeben und möchten diese \textbf{in die Polynomform} überführen, so multiplizieren wir den Faktor aus und erhalten so die gewünschte Form.\\
		\par\noindent
		\underline{\textbf{Beispiel:}} \(f(x) = 0,5(x-3)(x+2)(x-1)\)
		\begin{align*}
			f_{FF}(x) & = 0,5(x-3)\underbrace{(x+2)(x-1)}_{ausmultiplizieren}\\
			& = 0,5(x-3)(x^2 - x +2x -2)\\
			& = 0,5\underbrace{(x-3)(x^2 + x -2)}_{ausmultiplizieren}\\
			& = 0,5\cdot(x^3 +x^2 -2x -3x^2 -3x +6)\\
			& = \underbrace{0,5(x^3 -2x^2 -5x +6)}_{ausmultiplizieren}\\
			& = 0,5x^3 - x^2 -2,5x +3\\
			\\
			& \Rightarrow f_{PF}(x) = 0,5x^3 -x^2 -2,5x + 3
		\end{align*}
		\textbf{PF \(\Rightarrow\) FF}\\
		\par\noindent
		Haben wir eine ganzrationale Funktion in \textbf{Polynomform} \(f(x) = a_nx^n + a_{n-1}x^{n-1} + \ldots a_1x^1 + a_0\) gegeben und wollen diese \textbf{in die Faktorform} überführen, klammern wir zunächst den Koeffizienten des \underline{charakteristischen Summanden} aus und bestimmen im Anschluss die Nullstellen des Klammerausdrucks.\\
		Im Anschluss setzen wir die berechneten Nullstellen in das \underline{Gerüst der Faktorform} ein.\\
		\par\noindent
	\end{worksheet}
\end{document}