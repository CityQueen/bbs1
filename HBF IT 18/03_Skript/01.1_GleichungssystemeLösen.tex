\documentclass[11pt,twocolumn,oneside,openany,headings=optiontotoc,11pt,numbers=noenddot]{article}

\usepackage[a4paper]{geometry}
\usepackage[utf8]{inputenc}
\usepackage[T1]{fontenc}
\usepackage{lmodern}
\usepackage[ngerman]{babel}
\usepackage{ngerman}

\usepackage[onehalfspacing]{setspace}

\usepackage{fancyhdr}
\usepackage{fancybox}

\usepackage{rotating}
\usepackage{varwidth}


\usepackage{pdflscape}
\usepackage{graphicx}
\usepackage{graphbox}
\graphicspath{
	{Pics/PDFs/}
	{Pics/JPGs/}
	{Pics/PNGs/}
}
\usepackage{caption}
\usepackage{tabularx}
\usepackage{dashrule}
\usepackage{hhline}
\usepackage{multirow}
\usepackage{enumerate}
\usepackage[hidelinks]{hyperref}
\usepackage{listings}

\usepackage[table]{xcolor}
\usepackage{array}
\usepackage{enumitem,amssymb,amsmath}
\usepackage{interval}
\usepackage{stmaryrd}
\usepackage{polynom}
\usepackage{diagbox}
\usepackage{dashrule}
\usepackage{framed}
\usepackage{mdframed}
\usepackage{karnaugh-map}

\usepackage{blindtext}

\usepackage{eso-pic}

\usepackage{amssymb}
\usepackage{eurosym}
\pagestyle{headings}
\renewcommand{\headrulewidth}{0.2pt}
\renewcommand{\footrulewidth}{0.2pt}
\newcommand*{\underdownarrow}[2]{\ensuremath{\underset{\overset{\Big\downarrow}{#2}}{#1}}}
\setlength{\fboxsep}{5pt}

% Codestyle defined
\definecolor{codegreen}{rgb}{0,0.6,0}
\definecolor{codegray}{rgb}{0.5,0.5,0.5}
\definecolor{codepurple}{rgb}{0.58,0,0.82}
\definecolor{backcolour}{rgb}{0.95,0.95,0.92}
\definecolor{deepgreen}{rgb}{0,0.5,0}
\definecolor{darkblue}{rgb}{0,0,0.65}
\definecolor{mauve}{rgb}{0.40, 0.19,0.28}
\colorlet{exceptioncolour}{yellow!50!red}
\colorlet{commandcolour}{blue!60!black}
\colorlet{numpycolour}{blue!60!green}
\colorlet{specmethodcolour}{violet}

%Neue Spaltendefinition
\newcolumntype{L}[1]{>{\raggedright\let\newline\\\arraybackslash\hspace{0pt}}m{#1}}
\newcolumntype{M}[1]{>{\centering\arraybackslash}X}
\newcommand{\cmnt}[1]{\ignorespaces}
%Textausrichtung ändern
\newcommand\tabrotate[1]{\rotatebox{90}{\raggedright#1\hspace{\tabcolsep}}}

%Intervall-Konfig
\intervalconfig {
	soft open fences
}

%Bash
\lstdefinestyle{BashInputStyle}{
	language=bash,
	basicstyle=\small\sffamily,
	backgroundcolor=\color{backcolour},
	columns=fullflexible,
	backgroundcolor=\color{backcolour},
	breaklines=true,
}
%Java
\lstdefinestyle{JavaInputStyle}{
	language=Java,
	backgroundcolor=\color{backcolour},
	aboveskip=1mm,
	belowskip=1mm,
	showstringspaces=false,
	columns=flexible,
	basicstyle={\footnotesize\ttfamily},
	numberstyle={\tiny},
	numbers=none,
	keywordstyle=\color{purple},,
	commentstyle=\color{deepgreen},
	stringstyle=\color{blue},
	emph={out},
	emphstyle=\color{darkblue},
	emph={[2]rand},
	emphstyle=[2]\color{specmethodcolour},
	breaklines=true,
	breakatwhitespace=true,
	tabsize=2,
}
%Python
\lstdefinestyle{PythonInputStyle}{
	language=Python,
	alsoletter={1234567890},
	aboveskip=1ex,
	basicstyle=\footnotesize,
	breaklines=true,
	breakatwhitespace= true,
	backgroundcolor=\color{backcolour},
	commentstyle=\color{red},
	otherkeywords={\ , \}, \{, \&,\|},
	emph={and,break,class,continue,def,yield,del,elif,else,%
		except,exec,finally,for,from,global,if,import,in,%
		lambda,not,or,pass,print,raise,return,try,while,assert},
	emphstyle=\color{exceptioncolour},
	emph={[2]True,False,None,min},
	emphstyle=[2]\color{specmethodcolour},
	emph={[3]object,type,isinstance,copy,deepcopy,zip,enumerate,reversed,list,len,dict,tuple,xrange,append,execfile,real,imag,reduce,str,repr},
	emphstyle=[3]\color{commandcolour},
	emph={[4]ode, fsolve, sqrt, exp, sin, cos, arccos, pi,  array, norm, solve, dot, arange, , isscalar, max, sum, flatten, shape, reshape, find, any, all, abs, plot, linspace, legend, quad, polyval,polyfit, hstack, concatenate,vstack,column_stack,empty,zeros,ones,rand,vander,grid,pcolor,eig,eigs,eigvals,svd,qr,tan,det,logspace,roll,mean,cumsum,cumprod,diff,vectorize,lstsq,cla,eye,xlabel,ylabel,squeeze},
	emphstyle=[4]\color{numpycolour},
	emph={[5]__init__,__add__,__mul__,__div__,__sub__,__call__,__getitem__,__setitem__,__eq__,__ne__,__nonzero__,__rmul__,__radd__,__repr__,__str__,__get__,__truediv__,__pow__,__name__,__future__,__all__},
	emphstyle=[5]\color{specmethodcolour},
	emph={[6]assert,range,yield},
	emphstyle=[6]\color{specmethodcolour}\bfseries,
	emph={[7]Exception,NameError,IndexError,SyntaxError,TypeError,ValueError,OverflowError,ZeroDivisionError,KeyboardInterrupt},
	emphstyle=[7]\color{specmethodcolour}\bfseries,
	emph={[8]taster,send,sendMail,capture,check,noMsg,go,move,switch,humTem,ventilate,buzz},
	emphstyle=[8]\color{blue},
	keywordstyle=\color{blue}\bfseries,
	rulecolor=\color{black!40},
	showstringspaces=false,
	stringstyle=\color{deepgreen}
}

\lstset{literate=%
	{Ö}{{\"O}}1
	{Ä}{{\"A}}1
	{Ü}{{\"U}}1
	{ß}{{\ss}}1
	{ü}{{\"u}}1
	{ä}{{\"a}}1
	{ö}{{\"o}}1
}

% Neue Klassenarbeits-Umgebung
\newenvironment{worksheet}[3]
% Begin-Bereich
{
	\newpage
	\sffamily
	\setcounter{page}{1}
	\ClearShipoutPicture
	\AddToShipoutPicture{
		\put(55,761){{
				\mbox{\parbox{385\unitlength}{\tiny \color{codegray}BBS I Mainz, #1 \newline #2
						\newline #3
					}
				}
			}
		}
		\put(455,761){{
				\mbox{\hspace{0.3cm}\includegraphics[width=0.2\textwidth]{../../logo.jpg}}
			}
		}
	}
}
% End-Bereich
{
	\clearpage
	\ClearShipoutPicture
}

\setlength{\columnsep}{3em}
\setlength{\columnseprule}{0.5pt}

\geometry{left=2.50cm,right=2.50cm,top=3.00cm,bottom=1.00cm,includeheadfoot}
\pagenumbering{arabic}
\pagestyle{plain}

\begin{document}
	\begin{worksheet}{Höhere Berufsfachschule IT-Systeme}{Grundstufe - 
		Mathematik}{Lernabschnitt 1: Lineare Funktionen}
		\section*{Lösen von Gleichungssystemen}
		\label{sec:lgs}
		\setcounter{section}{1}
		\subsection{Was ist ein Gleichungssystem?}
		Zu einem \textbf{Gleichungssystem} gehören immer mindestens zwei Gleichungen. Dabei ist zu beachten, dass diese \underline{zusammengehören}.\\
		Diese \textit{Zusammengehörigkeit} resultiert aus der Situation, aus welcher die Gleichungen entstanden sind.\\
		\par\noindent
		\textbf{Beispiel:} Die zwei Handytarife\\
		\(f_k(x) = 45x + 239\)\\
		\(f_v(x) = 52x + 109\)\\
		aus der vergangenen Stunde.\\
		\par\noindent
		Aktuell können wir die Tarife unabhängig voneinander betrachten. Also für \(f_v(x)\) x-Stellen bestimmen, die für \(f_k(x)\) keine Bedeutung haben. Möchte man aber die Tarife vergleichen, stellt man sie in einem Gleichungssystem dar. Dafür wird jeder Zeile (also jeder Gleichung) eine Nummer zugewiesen. So erhalten wir:\\
		\((I)\ \ f_k(x) = 45x + 239\)\\
		\((II)\ f_v(x) = 52x + 109\)\\
		\par\noindent
		\underline{\textit{Vorsicht:}} Die Gleichungen sind nun voneinander abhängig. Das bedeutet, jede x-Stelle, die die Gleichung \(f_k(x)\) erfüllt, muss auch die Gleichung \(f_v(x)\) erfüllen.\\
		Wir können also x nicht mehr mit beliebigen Zahlen belegen, sondern es existiert höchstens \underline{\textbf{eine}} Lösung.
		\subsection{Was sagt uns die Lösung eines Gleichungssystems?}
		Die Lösung eines Gleichungssystems sind also die Zahlen, mit denen man die Unbekannten (früher Variablen - aber jetzt nicht mehr!) belegen muss, um für beide Zeilen eine wahre Aussage zu erhalten.\\
		Haben wir zwei Gleichungen mit zwei Unbekannten (\(x\) und das dazugehörige \(f(x)\)), so entspricht die Lösung dem grafischen Schnittpunkt der beiden Geraden.\\
		\section*{Lösungsverfahren für Gleichungssysteme}
		Um die eben angesprochene Belegung zu bestimmen, gibt es drei Vorgehensweisen.
		\subsection{Das Gleichungsverfahren}
		Haben wir zwei Gleichungen der Form\\
		\((I)\ \ f(x) = a\cdot{}x\)\\
		\((II)\ f(x) = b\cdot{}x\)\\
		So kann man, wie wir es zuletzt getan haben, die beiden Zeilen gleichsetzen: \((I) = (II)\) und löst die resultierende Gleichung nach \(x\) auf.
		\subsection{Das Einsetzungsverfahren}
		Entsprechen die Gleichungen der Form\\
		\((I)\ \ f(x) = a\cdot{}x\)\\
		\((II)\ g(x) = b\cdot{}x + c\cdot{}f(x)\)\\
		so können wir den Term \(f(x)\) in Zeile (II) durch die entsprechende Termstruktur aus Zeile (I) ersetzen. Im Anschluss lösen wir wie gewohnt nach \(x\) auf.
		\subsection{Das Additionsverfahren}
		Sehen die Gleichungen so aus\\
		\((I)\ \ f(x) = a\cdot{}x + b\cdot{}y\)\\
		\((II) f(x) = c\cdot{}x + d\cdot{}y\)\\
		Also die Gleichungen weisen die gleichen Unbekannten auf, kann man das Additionsverfahren anwenden.\\
		Wir versuchen dabei in einer Zeile eine der beiden Unbekannten zu eliminieren, so dass man aus der übrig gebliebenen Unbekannten die Belegung für diese ablesen kann.\\
		Dieses Vorgehen kann man auch für \underline{drei Unbekannte} nutzen.\\
		\par\bigskip\noindent
		Zur Verdeutlichung betrachten wir folgendes Beispiel:
		\begin{align*}
			(I)\ & a + c = 2 & |\cdot(-3)\\
			(II)\ & 3a + c = 0 \\
			\\
			(I')\ & 3a + 3c = 6\\
			(II)\ & 3a + c = 0 & |(II)-(I')\\
			\\
			(I')\ & 3a + 3c = 6\\
			(II)\ & -2c = -6 & |(II)-(I')
		\end{align*}
		Aus (II') können wir folgern: \(c = 3\)\\
		Wir setzen also \(c=3\) in (I) ein und bestimmen den Wert für a.
		\begin{align*}
			(I)\ & a + \underbrace{3}_{c} = 2 & \Rightarrow a = -1
		\end{align*}
		Wir können also sagen, für \(a = -1\) und \(c = 3\) sind die Gleichungen (I) und (II) beide erfüllt.
		\newpage
		\noindent
		\textbf{\underline{Ihre Aufgabe:}} Bestimmen Sie unter Verwendung des \underline{Additionsverfahren} die Lösung für folgendes Gleichungssystem:\\
		\par\noindent
		(a)
		\begin{align*}
			(I)\ & 6x + 12y = 30\\
			(II)\ & 3x + 3y = 9\\
		\end{align*}
		\vfill
		(b)
		\begin{align*}
			(I)\ & -x + y + z = 0\\
			(II)\ & x - 3y -2z = 5\\
			(III)\ & 5x + y + 4z = 3
		\end{align*}
		\vfill
	\end{worksheet}
\end{document}