\documentclass[11pt,twocolumn,oneside,openany,headings=optiontotoc,11pt,numbers=noenddot]{article}

\usepackage[a4paper]{geometry}
\usepackage[utf8]{inputenc}
\usepackage[T1]{fontenc}
\usepackage{lmodern}
\usepackage[ngerman]{babel}
\usepackage{ngerman}

\usepackage[onehalfspacing]{setspace}

\usepackage{fancyhdr}
\usepackage{fancybox}

\usepackage{rotating}
\usepackage{varwidth}


\usepackage{pdflscape}
\usepackage{graphicx}
\usepackage{graphbox}
\graphicspath{
	{Pics/PDFs/}
	{Pics/JPGs/}
	{Pics/PNGs/}
}
\usepackage{caption}
\usepackage{tabularx}
\usepackage{dashrule}
\usepackage{hhline}
\usepackage{multirow}
\usepackage{enumerate}
\usepackage[hidelinks]{hyperref}
\usepackage{listings}

\usepackage[table]{xcolor}
\usepackage{array}
\usepackage{enumitem,amssymb,amsmath}
\usepackage{interval}
\usepackage{stmaryrd}
\usepackage{polynom}
\usepackage{diagbox}
\usepackage{dashrule}
\usepackage{framed}
\usepackage{mdframed}
\usepackage{karnaugh-map}

\usepackage{blindtext}

\usepackage{eso-pic}

\usepackage{amssymb}
\usepackage{eurosym}
\pagestyle{headings}
\renewcommand{\headrulewidth}{0.2pt}
\renewcommand{\footrulewidth}{0.2pt}
\newcommand*{\underdownarrow}[2]{\ensuremath{\underset{\overset{\Big\downarrow}{#2}}{#1}}}
\setlength{\fboxsep}{5pt}

% Codestyle defined
\definecolor{codegreen}{rgb}{0,0.6,0}
\definecolor{codegray}{rgb}{0.5,0.5,0.5}
\definecolor{codepurple}{rgb}{0.58,0,0.82}
\definecolor{backcolour}{rgb}{0.95,0.95,0.92}
\definecolor{deepgreen}{rgb}{0,0.5,0}
\definecolor{darkblue}{rgb}{0,0,0.65}
\definecolor{mauve}{rgb}{0.40, 0.19,0.28}
\colorlet{exceptioncolour}{yellow!50!red}
\colorlet{commandcolour}{blue!60!black}
\colorlet{numpycolour}{blue!60!green}
\colorlet{specmethodcolour}{violet}

%Neue Spaltendefinition
\newcolumntype{L}[1]{>{\raggedright\let\newline\\\arraybackslash\hspace{0pt}}m{#1}}
\newcolumntype{M}[1]{>{\centering\arraybackslash}X}
\newcommand{\cmnt}[1]{\ignorespaces}
%Textausrichtung ändern
\newcommand\tabrotate[1]{\rotatebox{90}{\raggedright#1\hspace{\tabcolsep}}}

%Intervall-Konfig
\intervalconfig {
	soft open fences
}

%Bash
\lstdefinestyle{BashInputStyle}{
	language=bash,
	basicstyle=\small\sffamily,
	backgroundcolor=\color{backcolour},
	columns=fullflexible,
	backgroundcolor=\color{backcolour},
	breaklines=true,
}
%Java
\lstdefinestyle{JavaInputStyle}{
	language=Java,
	backgroundcolor=\color{backcolour},
	aboveskip=1mm,
	belowskip=1mm,
	showstringspaces=false,
	columns=flexible,
	basicstyle={\footnotesize\ttfamily},
	numberstyle={\tiny},
	numbers=none,
	keywordstyle=\color{purple},,
	commentstyle=\color{deepgreen},
	stringstyle=\color{blue},
	emph={out},
	emphstyle=\color{darkblue},
	emph={[2]rand},
	emphstyle=[2]\color{specmethodcolour},
	breaklines=true,
	breakatwhitespace=true,
	tabsize=2,
}
%Python
\lstdefinestyle{PythonInputStyle}{
	language=Python,
	alsoletter={1234567890},
	aboveskip=1ex,
	basicstyle=\footnotesize,
	breaklines=true,
	breakatwhitespace= true,
	backgroundcolor=\color{backcolour},
	commentstyle=\color{red},
	otherkeywords={\ , \}, \{, \&,\|},
	emph={and,break,class,continue,def,yield,del,elif,else,%
		except,exec,finally,for,from,global,if,import,in,%
		lambda,not,or,pass,print,raise,return,try,while,assert},
	emphstyle=\color{exceptioncolour},
	emph={[2]True,False,None,min},
	emphstyle=[2]\color{specmethodcolour},
	emph={[3]object,type,isinstance,copy,deepcopy,zip,enumerate,reversed,list,len,dict,tuple,xrange,append,execfile,real,imag,reduce,str,repr},
	emphstyle=[3]\color{commandcolour},
	emph={[4]ode, fsolve, sqrt, exp, sin, cos, arccos, pi,  array, norm, solve, dot, arange, , isscalar, max, sum, flatten, shape, reshape, find, any, all, abs, plot, linspace, legend, quad, polyval,polyfit, hstack, concatenate,vstack,column_stack,empty,zeros,ones,rand,vander,grid,pcolor,eig,eigs,eigvals,svd,qr,tan,det,logspace,roll,mean,cumsum,cumprod,diff,vectorize,lstsq,cla,eye,xlabel,ylabel,squeeze},
	emphstyle=[4]\color{numpycolour},
	emph={[5]__init__,__add__,__mul__,__div__,__sub__,__call__,__getitem__,__setitem__,__eq__,__ne__,__nonzero__,__rmul__,__radd__,__repr__,__str__,__get__,__truediv__,__pow__,__name__,__future__,__all__},
	emphstyle=[5]\color{specmethodcolour},
	emph={[6]assert,range,yield},
	emphstyle=[6]\color{specmethodcolour}\bfseries,
	emph={[7]Exception,NameError,IndexError,SyntaxError,TypeError,ValueError,OverflowError,ZeroDivisionError,KeyboardInterrupt},
	emphstyle=[7]\color{specmethodcolour}\bfseries,
	emph={[8]taster,send,sendMail,capture,check,noMsg,go,move,switch,humTem,ventilate,buzz},
	emphstyle=[8]\color{blue},
	keywordstyle=\color{blue}\bfseries,
	rulecolor=\color{black!40},
	showstringspaces=false,
	stringstyle=\color{deepgreen}
}

\lstset{literate=%
	{Ö}{{\"O}}1
	{Ä}{{\"A}}1
	{Ü}{{\"U}}1
	{ß}{{\ss}}1
	{ü}{{\"u}}1
	{ä}{{\"a}}1
	{ö}{{\"o}}1
}

% Neue Klassenarbeits-Umgebung
\newenvironment{worksheet}[3]
% Begin-Bereich
{
	\newpage
	\sffamily
	\setcounter{page}{1}
	\ClearShipoutPicture
	\AddToShipoutPicture{
		\put(55,761){{
				\mbox{\parbox{385\unitlength}{\tiny \color{codegray}BBS I Mainz, #1 \newline #2
						\newline #3
					}
				}
			}
		}
		\put(455,761){{
				\mbox{\hspace{0.3cm}\includegraphics[width=0.2\textwidth]{../../logo.jpg}}
			}
		}
	}
}
% End-Bereich
{
	\clearpage
	\ClearShipoutPicture
}

\setlength{\columnsep}{3em}
\setlength{\columnseprule}{0.5pt}

\geometry{left=2.50cm,right=2.50cm,top=3.00cm,bottom=1.00cm,includeheadfoot}
\pagenumbering{gobble}
\pagestyle{empty}

\begin{document}
	\begin{worksheet}{Höhere Berufsfachschule IT-Systeme}{Grundstufe - Mathematik}{Grundlagen Funktionen}
		\section{Allgemeine Begrifflichkeiten und Grundlagen}
		\subsection{Terme}
		\subsubsection*{Was sind Terme?} Unter einem Term versteht man ein \mbox{\textbf{mathematisches Gebilde}}. Dieses besteht aus Zahlen und Rechenzeichen.\\
		Eine Zahl kann hier aber auch durch einen Buchstaben angegeben werden.\\
		\par
		\begin{tabularx}{0.5\textwidth}{ll}
			\noindent
			\textbf{Beispiele:} & \(3\)\\
			& \(1,70+2,30\)\\
			& \(2,30\cdot x\)\\
			& \(5x + 20\)\\
			& \(5^2\)\\
			& \(\sqrt{5}\)\\
		\end{tabularx}
		\subsection{Termstrukturen}
		\paragraph{Warum sind die wichtig?}
		Durch die Verwendung von Rechenzeichen innerhalb von Termen erhalten diese eine gewisse Struktur. In diversen Standardsituationen kann es sehr hilfreich sein, wenn man diese Strukturen erkennt.
		\begin{framed}
			\noindent
			\textbf{Beispiele:}\\
			\underline{Situation 1:} Berechnen von Funktionswerten:
			\[f(-3) = -2\cdot (-3)^2 + 4\]
			\textit{Welche Reihenfolge der Rechenoperationen muss beachtet werden? Tastenfolge im Taschenrechner?}\\
			\par
			\underline{Situation 2:} Lösen einer Gleichung mithilfe von Äquivalenzumformungen (z.B. zur Nullstellenbestimmung)
			\[-2x^2 +4 = 0\]
			\textit{Rechnet man zuerst }|\textit{\(:(-2)\), }|\textit{\(-4\) oder }|\textit{\glqq{}Wurzel\grqq{}?}
		\end{framed}
		\subsubsection*{Woraus bestehen diese?}
		Die Terme, mit denen Sie im Laufe der Zeit konfrontiert werden bestehen aus Summen, Produkten oder Potenzen.\\
		\par\noindent
		\textbf{\underline{Summen}}\\
		Eine \underline{Summe} besteht aus mindestens \textbf{zwei Summanden}, die durch ein \underline{\(+\)-Zeichen} verbunden sind.
		\begin{framed}\noindent
			\underline{Kurz:} Summand + Summand = Summe
		\end{framed}
		\noindent
		Führt man die Operation (Addition) aus, so erhält man den \underline{Summenwert}.\\
		\par\noindent
		\underline{Hinweis:} Sind zwei zahlen durch ein \underline{\glqq{}\(-\)\grqq{}-Zeichen} verbunden, spricht man von der \underline{Differenz}. Dieses Gebilde kann aber auch als Summe bezeichnet werden.\\
		\(7-4\) bedeutet nämlich eigentlich nichts anderes als \(7 + (-4)\). Es ist also eine Summe aus den Summanden \(7\) und \(-4\).\\
		\par\noindent
		\textbf{\underline{Produkte}}\\
		Ein \underline{Produkt} besteht aus mindestens \textbf{zwei Faktoren}. Diese werden durch ein \underline{\glqq{}\(\cdot\)\grqq{}-Zeichen} miteinander verbunden.
		\begin{framed}
			\noindent
			\underline{Kurz:} Faktor $\cdot$ Faktor = Produkt
		\end{framed}
		\noindent
		Führt man die Operation (Multiplikation) aus, erhält man den sogenannten \underline{Produktwert}.\\
		\par\noindent
		\underline{Hinweis:} Der Mal-Punkt wird häufig weggelassen. Also \(5\cdot x\) wird auch als \(5x\) geschrieben.\\
		\par\noindent
		\textbf{\underline{Potenzen}}\\
		Eine \underline{Potenz} besteht immer aus einer \textbf{Basis} und einem \textbf{Exponenten}. Dabei gibt der Exponent an, wie häufig die Basis mit sich selbst multipliziert wird.\\
		Führt man diese Rechnung aus, ergibt das den \underline{Potenzwert}.
		\begin{framed}
			\noindent
			\underline{Kurz:} \(Basis^{\text{Exponent}}\) = Potenz
		\end{framed}
		\noindent
		\underline{\textbf{Wurzel}}\\
		Die \underline{Wurzel} einer Zahl \(a\) bezeichnet die Zahl, die mit sich selbst multipliziert, den Wert \(a\) ergibt. In der Regel schreibt man \(\sqrt{a}\).\\
		Die Zahl unterhalb der Wurzel nennt man auch \textbf{Radikand}.
		\subsubsection{Wichtige Verknüpfungsregel}
		Es ist häufig der Fall, dass Potenzen, Produkte und Summern miteinander verknüpft werden. Ist dies der Fall, zerren unterschiedliche Rechenzeichen an einer Zahl herum.\\
		Bei der Berechnung des Werts eines Terms gilt die folgende Hierarchie:
		\begin{framed}
			\centering
			\color{red}Po\normalcolor{}tenzrechnung\\
			vor\\
			\color{red}Pu\normalcolor{}nktrechnung\\
			vor\\
			\color{red}Stri\normalcolor{}chrechnung!\\
			\normalcolor
			\par\noindent
			\raggedright
			Sind auch \color{red}Kla\normalcolor{}mmern beteiligt, so haben diese die größte Macht und binden am stärksten.
		\end{framed}
		\noindent
		\centering
		\fcolorbox{red!15}{red!5}{\textbf{\color{red}KlaPoPuStri}\normalcolor{} bewahrt vor Fehlern!}\\
		\raggedright
		\par\noindent
		\textbf{Beispiel 1:} \(4\cdot 2 + 5\)\\
		Wir analysieren die Termstruktur:
		\begin{tabularx}{0.5\textwidth}{c}
			\(4\ \boxed{\cdot}\ 2\ +\ 5\)\\
			\multicolumn{1}{l}{Zuerst \color{red}Pu\normalcolor{}nktrechnung:}\\
			\(=\ 8\ \boxed{+}\ 5\)\\
			\multicolumn{1}{l}{Dann \color{red}Stri\normalcolor{}chrechnung:}\\
			= 13
		\end{tabularx}
		\par\noindent
		\textbf{Beispiel 2:} \(4\cdot (2 + 5)\)\\
		Wir analysieren die Termstruktur:
		\begin{tabularx}{0.5\textwidth}{c}
			\(4\ \cdot\ \boxed{(}\ 2\ +\ 5\ \boxed{)}\)\\
			\multicolumn{1}{l}{Zuerst \color{red}Kla\normalcolor{}mmerausdruck auflösen:}\\
			\(=\ 4\ \boxed{\cdot}\ 7\)\\
			\multicolumn{1}{l}{Dann \color{red}Pu\normalcolor{}nktrechnung:}\\
			= 28
		\end{tabularx}
		\par\noindent
		\textbf{Beispiel 3:} \(-2\cdot{}(2+4)^2 + 7\)\\
		Wir analysieren die Termstruktur:
		\begin{tabularx}{0.5\textwidth}{c}
			\(-2\ \cdot{}\ \boxed{(}\ 2\ +\ 4\ \boxed{)}^2 +\ 7\)\\
			\multicolumn{1}{l}{Zuerst \color{red}Kla\normalcolor{}mmerausdruck auflösen:}\\
			\(=\ -2\ \cdot{}\ \boxed{6^2} +\ 7\)\\
			\multicolumn{1}{l}{Danach \color{red}Po\normalcolor{}tenz bestimmen:}\\
			\(=\ -2\ \boxed{\cdot{}}\ 36\ +\ 7\)\\
			\multicolumn{1}{l}{Anschließend \color{red}Kla\normalcolor{}mmerausdruck auflösen:}\\
			\(=\ -72\ \boxed{+}\ 7\)\\
			\multicolumn{1}{l}{Zuletzt \color{red}Stri\normalcolor{}chrechnung:}\\
			\(=\ -655\)\\
		\end{tabularx}
		\newpage
		\subsection{Rechenregeln}
		\subsubsection*{Vorzeichenregel (VZ)}
		\subsubsection*{Klammerregeln - Ausmultiplizieren (AM)}
		\subsubsection*{Faktorisieren und Ausklammern (FAK)}
		\subsubsection*{Minus vor der Klammer (MK)}
		\subsubsection*{Binomische Formeln (BF)}
		\subsubsection*{Zusammenfassen (ZUS)}
		\subsubsection*{Potenzgesetze (PG)}
		\section{Begrifflichkeiten Funktionen}
		\subsection{Wertepaare}
		\subsubsection{x-Koordinate}
		\subsubsection*{Nullstelle(n)}
		\subsection*{Extremstelle}
		\subsection*{Wendestelle}
		\subsubsection{y-Koordinate}
		\subsubsection*{y-Achsenabschnittswert}
	\end{worksheet}
\end{document}