\documentclass[11pt,twocolumn,oneside,openany,headings=optiontotoc,11pt,numbers=noenddot]{article}

\usepackage[a4paper]{geometry}
\usepackage[utf8]{inputenc}
\usepackage[T1]{fontenc}
\usepackage{lmodern}
\usepackage[ngerman]{babel}
\usepackage{ngerman}

\usepackage[onehalfspacing]{setspace}

\usepackage{fancyhdr}
\usepackage{fancybox}

\usepackage{rotating}
\usepackage{varwidth}


\usepackage{pdflscape}
\usepackage{graphicx}
\usepackage{graphbox}
\graphicspath{
	{Pics/PDFs/}
	{Pics/JPGs/}
	{Pics/PNGs/}
}
\usepackage{caption}
\usepackage{tabularx}
\usepackage{dashrule}
\usepackage{hhline}
\usepackage{multirow}
\usepackage{enumerate}
\usepackage[hidelinks]{hyperref}
\usepackage{listings}

\usepackage[table]{xcolor}
\usepackage{array}
\usepackage{enumitem,amssymb,amsmath}
\usepackage{interval}
\usepackage{stmaryrd}
\usepackage{polynom}
\usepackage{diagbox}
\usepackage{dashrule}
\usepackage{framed}
\usepackage{mdframed}
\usepackage{karnaugh-map}

\usepackage{blindtext}

\usepackage{eso-pic}

\usepackage{amssymb}
\usepackage{eurosym}
\pagestyle{headings}
\renewcommand{\headrulewidth}{0.2pt}
\renewcommand{\footrulewidth}{0.2pt}
\newcommand*{\underdownarrow}[2]{\ensuremath{\underset{\overset{\Big\downarrow}{#2}}{#1}}}
\setlength{\fboxsep}{5pt}

% Codestyle defined
\definecolor{codegreen}{rgb}{0,0.6,0}
\definecolor{codegray}{rgb}{0.5,0.5,0.5}
\definecolor{codepurple}{rgb}{0.58,0,0.82}
\definecolor{backcolour}{rgb}{0.95,0.95,0.92}
\definecolor{deepgreen}{rgb}{0,0.5,0}
\definecolor{darkblue}{rgb}{0,0,0.65}
\definecolor{mauve}{rgb}{0.40, 0.19,0.28}
\colorlet{exceptioncolour}{yellow!50!red}
\colorlet{commandcolour}{blue!60!black}
\colorlet{numpycolour}{blue!60!green}
\colorlet{specmethodcolour}{violet}

%Neue Spaltendefinition
\newcolumntype{L}[1]{>{\raggedright\let\newline\\\arraybackslash\hspace{0pt}}m{#1}}
\newcolumntype{M}[1]{>{\centering\arraybackslash}X}
\newcommand{\cmnt}[1]{\ignorespaces}
%Textausrichtung ändern
\newcommand\tabrotate[1]{\rotatebox{90}{\raggedright#1\hspace{\tabcolsep}}}

%Intervall-Konfig
\intervalconfig {
	soft open fences
}

%Bash
\lstdefinestyle{BashInputStyle}{
	language=bash,
	basicstyle=\small\sffamily,
	backgroundcolor=\color{backcolour},
	columns=fullflexible,
	backgroundcolor=\color{backcolour},
	breaklines=true,
}
%Java
\lstdefinestyle{JavaInputStyle}{
	language=Java,
	backgroundcolor=\color{backcolour},
	aboveskip=1mm,
	belowskip=1mm,
	showstringspaces=false,
	columns=flexible,
	basicstyle={\footnotesize\ttfamily},
	numberstyle={\tiny},
	numbers=none,
	keywordstyle=\color{purple},,
	commentstyle=\color{deepgreen},
	stringstyle=\color{blue},
	emph={out},
	emphstyle=\color{darkblue},
	emph={[2]rand},
	emphstyle=[2]\color{specmethodcolour},
	breaklines=true,
	breakatwhitespace=true,
	tabsize=2,
}
%Python
\lstdefinestyle{PythonInputStyle}{
	language=Python,
	alsoletter={1234567890},
	aboveskip=1ex,
	basicstyle=\footnotesize,
	breaklines=true,
	breakatwhitespace= true,
	backgroundcolor=\color{backcolour},
	commentstyle=\color{red},
	otherkeywords={\ , \}, \{, \&,\|},
	emph={and,break,class,continue,def,yield,del,elif,else,%
		except,exec,finally,for,from,global,if,import,in,%
		lambda,not,or,pass,print,raise,return,try,while,assert},
	emphstyle=\color{exceptioncolour},
	emph={[2]True,False,None,min},
	emphstyle=[2]\color{specmethodcolour},
	emph={[3]object,type,isinstance,copy,deepcopy,zip,enumerate,reversed,list,len,dict,tuple,xrange,append,execfile,real,imag,reduce,str,repr},
	emphstyle=[3]\color{commandcolour},
	emph={[4]ode, fsolve, sqrt, exp, sin, cos, arccos, pi,  array, norm, solve, dot, arange, , isscalar, max, sum, flatten, shape, reshape, find, any, all, abs, plot, linspace, legend, quad, polyval,polyfit, hstack, concatenate,vstack,column_stack,empty,zeros,ones,rand,vander,grid,pcolor,eig,eigs,eigvals,svd,qr,tan,det,logspace,roll,mean,cumsum,cumprod,diff,vectorize,lstsq,cla,eye,xlabel,ylabel,squeeze},
	emphstyle=[4]\color{numpycolour},
	emph={[5]__init__,__add__,__mul__,__div__,__sub__,__call__,__getitem__,__setitem__,__eq__,__ne__,__nonzero__,__rmul__,__radd__,__repr__,__str__,__get__,__truediv__,__pow__,__name__,__future__,__all__},
	emphstyle=[5]\color{specmethodcolour},
	emph={[6]assert,range,yield},
	emphstyle=[6]\color{specmethodcolour}\bfseries,
	emph={[7]Exception,NameError,IndexError,SyntaxError,TypeError,ValueError,OverflowError,ZeroDivisionError,KeyboardInterrupt},
	emphstyle=[7]\color{specmethodcolour}\bfseries,
	emph={[8]taster,send,sendMail,capture,check,noMsg,go,move,switch,humTem,ventilate,buzz},
	emphstyle=[8]\color{blue},
	keywordstyle=\color{blue}\bfseries,
	rulecolor=\color{black!40},
	showstringspaces=false,
	stringstyle=\color{deepgreen}
}

\lstset{literate=%
	{Ö}{{\"O}}1
	{Ä}{{\"A}}1
	{Ü}{{\"U}}1
	{ß}{{\ss}}1
	{ü}{{\"u}}1
	{ä}{{\"a}}1
	{ö}{{\"o}}1
}

% Neue Klassenarbeits-Umgebung
\newenvironment{worksheet}[3]
% Begin-Bereich
{
	\newpage
	\sffamily
	\setcounter{page}{1}
	\ClearShipoutPicture
	\AddToShipoutPicture{
		\put(55,761){{
				\mbox{\parbox{385\unitlength}{\tiny \color{codegray}BBS I Mainz, #1 \newline #2
						\newline #3
					}
				}
			}
		}
		\put(455,761){{
				\mbox{\hspace{0.3cm}\includegraphics[width=0.2\textwidth]{../../logo.jpg}}
			}
		}
	}
}
% End-Bereich
{
	\clearpage
	\ClearShipoutPicture
}

\setlength{\columnsep}{3em}
\setlength{\columnseprule}{0.5pt}

\geometry{left=1.50cm,right=1.50cm,top=3.00cm,bottom=1.00cm,includeheadfoot}
\pagenumbering{gobble}
\pagestyle{empty}

\begin{document}
	\begin{worksheet}{Höhere Berufsfachschule IT-Systeme}{Grundstufe - Mathematik}{Lernabschnitt 2: Nullstellen einer quadratischen Funktion}
		\setcounter{section}{3}
		\section{Nullstellen bestimmen}
		Betrachtet  man \textbf{quadratischen Funktionen} \(\mathbf{f(x)}\), so begegnen uns verschiedene Typen von quadratischen Funktionen.\\
		Wollen wir die Nullstellen bestimmen, so müssen wir die Funktion gleich Null setzen (\(= 0\)) und anschießend den Wert für \(\mathbf{x}\) bestimmen, für den die Gleichung erfüllt ist.\\
		\par\noindent
		Nachfolgenden sind verschiedene Verfahren aufgeführt, die zur Lösung dieser Fragestellung hilfreich sein können.
		\subsection{\(\mathbf{f(x) = x^2 - c}\) / \(\mathbf{f(x) = -x^2 + c}\)}
		Hat unsere Funktion die Form \underline{\(f(x) = x^2 - c\)}, dann wollen wir folgendes tun:
		\begin{align*}
			f(x) & = x^2 - c & | = 0\\
			x^2 - c & = 0 & | +c\\
			x^2 & = c & | \sqrt{}\\
			x & = \pm\sqrt{c}
		\end{align*}
		Wir haben also durch \textbf{Nullsetzen} und \textbf{nach} \(\mathbf{x}\) \textbf{umformen} bestimmt, dass \colorbox{green!10}{\(x = \pm\sqrt{c}\)} die \underline{Nullstellen} der Funktion \(f(x) = x^2 - c\) sind.\\
		\par\noindent
		Hat unsere Funktion die Form \underline{\(f(x) = -x^2 + c\)}, dann wollen wir folgendes tun:
		\begin{align*}
			f(x) & = - x^2 + c & | = 0\\
			- x^2 + c & = 0 & | + x^2\\
			c & = x^2 & | \sqrt{}\\
			\pm\sqrt{c} & = x
		\end{align*}
		Wir haben also durch \textbf{Nullsetzen} und \textbf{nach} \(\mathbf{x}\) \textbf{umformen} bestimmt, dass \colorbox{green!10}{\(x = \pm\sqrt{c}\)} auch die \underline{Nullstellen} der Funktion \(f(x) = -x^2 + c\) sind.\\
		\par\noindent
		\textbf{\underline{Beispiel:}} Wir betrachten die Funktion \(f(x) = x^2 - 9\)
		\begin{align*}
			0 & = x^2 - 9 & |+9\\
			9 & = x^2 & |\sqrt{}\\
			\pm{}\sqrt{9} = \pm{}3 & = x
		\end{align*}
		\subsection{\(\mathbf{f(x) = ax^2 + bx + c}\)}
		Bei Funktionen der Form \(f(x) = ax^2 + bx +c\) gibt es abhängig von den Parameterwerten von \(a, b\) und \(c\) verschiedene Vorgehensweisen.\\
		\subsubsection*{\(\mathbf{f(x) = x^2 + px + q}\)}
		Ist der Koeffizient von \(x^2\) eine \(\mathbf{1}\), so können wir die pq-Formel anwenden. Diese liefert und die Nullstellen der Funktion.
		\begin{framed}
			\noindent
			\underline{\textbf{pq-Formel}}\\
			Ist eine Funktion der Form \(f(x) = x^2 + px + q\) gegeben, so liefert \[x_{1/2} = -\frac{p}{2}\pm \sqrt{\left(\frac{p}{2}\right)^2 - q}\] die Nullstellen der Funktion.
		\end{framed}
		\noindent
		\underline{\textbf{Beispiel:}} Gegeben ist die Funktion \(f(x) = x^2 -5x + 3\). Wir setzen \(f(x) = 0\) und bestimmen mit Hilfe der pq-Formel die Nullstellen.
		\begin{align*}
			0 & = x^2 \underbrace{- 5}_{p}x \underbrace{+ 3}_{q}\\
			\text{\underline{pq-Formel}}\\
			\Rightarrow x_{1/2} & = -\frac{-5}{2} \pm \sqrt{\left(\frac{-5}{2}\right)^2 - 3}\\
			x_{1/2} & = \frac{5}{2} \pm \sqrt{\left(\frac{25}{4}\right) - 3}\\
		\end{align*}
		\begin{align*}
			x_1 & = \frac{5}{2} + \sqrt{\left(\frac{25}{4}\right) - 3} & = \frac{5 +\sqrt{13}}{2}\\
			x_2 & = \frac{5}{2} - \sqrt{\left(\frac{25}{4}\right) - 3} & = \frac{5 +\sqrt{13}}{2}\\\\
		\end{align*}
		\subsubsection*{\(\mathbf{f(x) = x^2 + bx + c}\)}
		Haben wir eine Funktion \(f(x) = x^2 + bx + c\) gegeben und für die Koeffizienten \(b\) und \(c\) gilt: \(\sqrt{c}\) ist eine ganze Zahl und \(b = \pm{}2\cdot\sqrt{c}\), dann haben wir die \underline{1. oder 2. Binomische Formel} gegeben.\\
		\par\noindent
		Ist \(f(x) = x^2 + bx + c\), prüft man:
		\begin{itemize}
			\item[-] Ist \(\sqrt{c}\) eine Ganze Zahl?
			\begin{itemize}
				\item[-] Wenn ja, dann prüfe wir ob \( b = \pm{}2\cdot\sqrt{c}\)
				\begin{itemize}
					\item[-] Ist beides erfüllt, dann ist\\
					\(x =\) (\textit{\tiny{Setze hier das Vorzeichen von b ein}}) \(\sqrt{c}\)\\
					eine doppelte Nullstelle.\\
					\(f(x) = (x\pm\sqrt{c})^2\)
				\end{itemize}
			\end{itemize}
		\end{itemize}
		\subsubsection*{\(\mathbf{f(x) = ax^2 + bx + c}\)}
		Hat unser \(x^2\) einen Koeffizienten \(a \neq{} 0\), so können wir \underline{\textbf{nicht}} die pq-Formel anwenden.\\
		Um dennoch die Gleichung \(0 = ax^2 +bx + c\) lösen zu können bleiben uns diverse Möglichkeiten.\\
		Zunächst können wir die Funktion faktorisieren, sie also in die bekannte Linearfaktorform \(f(x) = a\cdot{}(x-x_{N_1})\cdot(x-x_{N_2})\) überführen.
		\begin{framed}
			\noindent
			\underline{\textbf{Produkt ist Null (PIN)}}\\
			Ein Produkt ist Null, wenn einer der Faktoren Null ist. In Zeichen: \[a\cdot{}b = 0 \Leftrightarrow a = 0 \text{\ oder\ } b = 0\]
			\small{\(\Leftrightarrow\) - Genau dann wenn}\normalsize
		\end{framed}
		\begin{framed}
			\noindent
			\underline{Vorsicht:} Eine Funktion lässt sich nur Faktorisieren, wenn sie Nullstellen hat.
		\end{framed}
		\noindent
		Zum Faktorisieren von Funktionen wenden Sie eines der Verfahren aus dem Skript \textit{Quadratischen Funktionen faktorisieren}.\\
		\small{Für Beispiele schlagen Sie in eben genanntem Skript nach.}\normalsize\\
		\par\noindent
		\underline{\textbf{Alternativ}} können Sie auch den Koeffizienten \(a\) ausklammern und im Anschluss die pq-Formel anwenden.\\
		\(f(x) = ax^2 + bx + c\) wird durch ausklammern von \(a\) zu \(f(x) = a\cdot{}(x^2 + \frac{b}{a}x~ \frac{c}{a})\). Auf den Klammerausdruck können Sie nun die \underline{pq-Formel} anwenden.
		\subsection{\(\mathbf{f(x) = ax^2 + bx}\)}
		Ist in unserer Funktione Konstante \(c\) vorhanden, können wir zunächst Faktorisieren indem wir das \textbf{kleinste gemeinsame Vielfache} von \(ax^2\) und \(bx\) ausklammern (mindestens \(x\)).\\
		Dann sind wird mit einem Produkt konfrontiert, für welches wir die oben erwähnte Regel anwenden (\textit{Produkt ist Null}).
		\underline{\textbf{Beispiel:}}
		\begin{align*}
			f(x) & = 15x^2 + 5x & |5x\cdot()\\
			f(x) & = 5x\cdot(3x + 1) 6\\
			0 & = 5x\cdot(3x+1) & \Leftrightarrow 5x = 0\ \text{oder}\ 3x+1 = 0\\
			\\
			5x & = 0  & |:5\\
			x_1 & = 0\\
			\\
			3x + 1 & = 0 & | -1\\
			3x & = -1 & |:3\\
			x_2 & = -\frac{1}{3}
		\end{align*}
	\end{worksheet}
\end{document}