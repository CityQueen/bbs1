\documentclass[oneside,openany,headings=optiontotoc,11pt,numbers=noenddot]{scrreprt}

\usepackage[a4paper]{geometry}
\usepackage[utf8]{inputenc}
\usepackage[T1]{fontenc}
\usepackage{lmodern}
\usepackage[ngerman]{babel}
\usepackage{ngerman}

\usepackage[onehalfspacing]{setspace}

\usepackage{fancyhdr}
\usepackage{fancybox}

\usepackage{rotating}
\usepackage{varwidth}


\usepackage{pdflscape}
\usepackage{graphicx}
\usepackage{graphbox}
\graphicspath{
	{Pics/PDFs/}
	{Pics/JPGs/}
	{Pics/PNGs/}
}
\usepackage{caption}
\usepackage{tabularx}
\usepackage{dashrule}
\usepackage{hhline}
\usepackage{multirow}
\usepackage{enumerate}
\usepackage[hidelinks]{hyperref}
\usepackage{listings}

\usepackage[table]{xcolor}
\usepackage{array}
\usepackage{enumitem,amssymb,amsmath}
\usepackage{interval}
\usepackage{stmaryrd}
\usepackage{polynom}
\usepackage{diagbox}
\usepackage{dashrule}
\usepackage{framed}
\usepackage{mdframed}
\usepackage{karnaugh-map}

\usepackage{blindtext}

\usepackage{eso-pic}

\usepackage{amssymb}
\usepackage{eurosym}
\pagestyle{headings}
\renewcommand{\headrulewidth}{0.2pt}
\renewcommand{\footrulewidth}{0.2pt}
\newcommand*{\underdownarrow}[2]{\ensuremath{\underset{\overset{\Big\downarrow}{#2}}{#1}}}
\setlength{\fboxsep}{5pt}

% Codestyle defined
\definecolor{codegreen}{rgb}{0,0.6,0}
\definecolor{codegray}{rgb}{0.5,0.5,0.5}
\definecolor{codepurple}{rgb}{0.58,0,0.82}
\definecolor{backcolour}{rgb}{0.95,0.95,0.92}
\definecolor{deepgreen}{rgb}{0,0.5,0}
\definecolor{darkblue}{rgb}{0,0,0.65}
\definecolor{mauve}{rgb}{0.40, 0.19,0.28}
\colorlet{exceptioncolour}{yellow!50!red}
\colorlet{commandcolour}{blue!60!black}
\colorlet{numpycolour}{blue!60!green}
\colorlet{specmethodcolour}{violet}

%Neue Spaltendefinition
\newcolumntype{L}[1]{>{\raggedright\let\newline\\\arraybackslash\hspace{0pt}}m{#1}}
\newcolumntype{M}[1]{>{\centering\arraybackslash}X}
\newcommand{\cmnt}[1]{\ignorespaces}
%Textausrichtung ändern
\newcommand\tabrotate[1]{\rotatebox{90}{\raggedright#1\hspace{\tabcolsep}}}

%Intervall-Konfig
\intervalconfig {
	soft open fences
}

%Bash
\lstdefinestyle{BashInputStyle}{
	language=bash,
	basicstyle=\small\sffamily,
	backgroundcolor=\color{backcolour},
	columns=fullflexible,
	backgroundcolor=\color{backcolour},
	breaklines=true,
}
%Java
\lstdefinestyle{JavaInputStyle}{
	language=Java,
	backgroundcolor=\color{backcolour},
	aboveskip=1mm,
	belowskip=1mm,
	showstringspaces=false,
	columns=flexible,
	basicstyle={\footnotesize\ttfamily},
	numberstyle={\tiny},
	numbers=none,
	keywordstyle=\color{purple},,
	commentstyle=\color{deepgreen},
	stringstyle=\color{blue},
	emph={out},
	emphstyle=\color{darkblue},
	emph={[2]rand},
	emphstyle=[2]\color{specmethodcolour},
	breaklines=true,
	breakatwhitespace=true,
	tabsize=2,
}
%Python
\lstdefinestyle{PythonInputStyle}{
	language=Python,
	alsoletter={1234567890},
	aboveskip=1ex,
	basicstyle=\footnotesize,
	breaklines=true,
	breakatwhitespace= true,
	backgroundcolor=\color{backcolour},
	commentstyle=\color{red},
	otherkeywords={\ , \}, \{, \&,\|},
	emph={and,break,class,continue,def,yield,del,elif,else,%
		except,exec,finally,for,from,global,if,import,in,%
		lambda,not,or,pass,print,raise,return,try,while,assert},
	emphstyle=\color{exceptioncolour},
	emph={[2]True,False,None,min},
	emphstyle=[2]\color{specmethodcolour},
	emph={[3]object,type,isinstance,copy,deepcopy,zip,enumerate,reversed,list,len,dict,tuple,xrange,append,execfile,real,imag,reduce,str,repr},
	emphstyle=[3]\color{commandcolour},
	emph={[4]ode, fsolve, sqrt, exp, sin, cos, arccos, pi,  array, norm, solve, dot, arange, , isscalar, max, sum, flatten, shape, reshape, find, any, all, abs, plot, linspace, legend, quad, polyval,polyfit, hstack, concatenate,vstack,column_stack,empty,zeros,ones,rand,vander,grid,pcolor,eig,eigs,eigvals,svd,qr,tan,det,logspace,roll,mean,cumsum,cumprod,diff,vectorize,lstsq,cla,eye,xlabel,ylabel,squeeze},
	emphstyle=[4]\color{numpycolour},
	emph={[5]__init__,__add__,__mul__,__div__,__sub__,__call__,__getitem__,__setitem__,__eq__,__ne__,__nonzero__,__rmul__,__radd__,__repr__,__str__,__get__,__truediv__,__pow__,__name__,__future__,__all__},
	emphstyle=[5]\color{specmethodcolour},
	emph={[6]assert,range,yield},
	emphstyle=[6]\color{specmethodcolour}\bfseries,
	emph={[7]Exception,NameError,IndexError,SyntaxError,TypeError,ValueError,OverflowError,ZeroDivisionError,KeyboardInterrupt},
	emphstyle=[7]\color{specmethodcolour}\bfseries,
	emph={[8]taster,send,sendMail,capture,check,noMsg,go,move,switch,humTem,ventilate,buzz},
	emphstyle=[8]\color{blue},
	keywordstyle=\color{blue}\bfseries,
	rulecolor=\color{black!40},
	showstringspaces=false,
	stringstyle=\color{deepgreen}
}

\lstset{literate=%
	{Ö}{{\"O}}1
	{Ä}{{\"A}}1
	{Ü}{{\"U}}1
	{ß}{{\ss}}1
	{ü}{{\"u}}1
	{ä}{{\"a}}1
	{ö}{{\"o}}1
}

% Neue Klassenarbeits-Umgebung
\newenvironment{worksheet}[3]
% Begin-Bereich
{
	\newpage
	\sffamily
	\setcounter{page}{1}
	\ClearShipoutPicture
	\AddToShipoutPicture{
		\put(55,761){{
				\mbox{\parbox{385\unitlength}{\tiny \color{codegray}BBS I Mainz, #1 \newline #2
						\newline #3
					}
				}
			}
		}
		\put(455,761){{
				\mbox{\hspace{0.3cm}\includegraphics[width=0.2\textwidth]{../../logo.jpg}}
			}
		}
	}
}
% End-Bereich
{
	\clearpage
	\ClearShipoutPicture
}

\geometry{left=2.50cm,right=2.50cm,top=3.00cm,bottom=1.00cm,includeheadfoot}

\begin{document}
	\begin{worksheet}{Höhere Berufsfachschule IT-Systeme}{Grundstufe - Mathematik}{Wochenplan Ganzrationale Funktionen}
		\noindent
		\begin{tabularx}{\textwidth}{XXl}
			Wochenplan Nr.: \rule{0.15\textwidth}{1pt} & Erledigt: & Zeitraum: \underline{10.12 - 16.12}
		\end{tabularx}\\
		\par\noindent
		Die Aufgaben gliedern sich nach folgender Schwierigkeitsstufe.\\
		\begin{tabularx}{\textwidth}{XXX}
			(I) Grundlagen & (II) Forstgeschritten & (III) Experte
		\end{tabularx}\\
		\par\noindent
		\textbf{\underline{Pflicht}}: Sie bearbeiten \textit{pro Teil} jeweils \textbf{eine Aufgabe} vom Schwierigkeitsgrad ihrer Wahl.\\
		\underline{\textbf{Wahl}}: Zur Vertiefung und Festigung stehen ihnen die übrigen Aufgaben zur Verfügung.
		\begin{framed}
			\noindent
			\textbf{Teil 1:} \textbf{Markieren} Sie die Nullstellen der Funktion. Schreiben Sie die Funktion gegebenenfalls um.\\
			\par\noindent
			\begin{tabularx}{\textwidth}{M|M}
				(I) \(f(x) = (x-2)(x-\frac{1}{3})(x-4,5)\) & (II) \(f(x) = 7(x-2)(x+3)(x-3)\)\\
				\\
				\hline
				\multicolumn{2}{c}{}\\
				\multicolumn{2}{c}{(III) \(f(x) = \frac{1}{4}x(x+2)(x-4)(x+0,5)\)}
			\end{tabularx}\\
			\par\bigskip\noindent
			\textbf{Erläutern} Sie, welchen Vorteil diese Darstellung im Bezug auf die Nullstellen hat.
		\end{framed}
		\begin{framed}
			\noindent
			\textbf{Teil 2:} \textbf{Erklären} Sie an einer der ganzrationalen Funktionen genau, wie Sie Vorgehen um die Nullstellen der Funktionen zu bestimmen.\\
			\par\noindent
			\begin{tabularx}{\textwidth}{M|M}
				(I) \(f(x) = (x-2)(x^2 - 6x + 9)\) & (II) \(f(x) = (x+3)(x^3-3,5x^2 - 9,5x +30)\)\\
				\\
				\hline
				\multicolumn{2}{c}{}\\
				\multicolumn{2}{c}{(III) \(f(x) = (x-2)(5x^4 - 10x^2 +2)\)}
			\end{tabularx}\\
			\par\noindent
			\textbf{Bestimmen} Sie alle Nullstellen zu einer der Funktionen.
		\end{framed}
		\normalsize
		\begin{framed}
			\noindent
			\textbf{Teil 3:} \textbf{Überführen} Sie eine der Funktionen in die Faktorform.\\
			\par\noindent
			\begin{tabularx}{\textwidth}{M|M}
				(I) \(f(x) = 0,5x^3-6x+8\) & (II) \(f(x) = x^3+0,5x^2-13x-20\)\\
				\\
				\hline
				\multicolumn{2}{c}{}\\
				\multicolumn{2}{c}{(III) \(f(x) = x^4-6x^3+5x^2+24x-36\)}
			\end{tabularx}\\
			\par\noindent
		\end{framed}
		\newpage
		\begin{framed}
			\noindent
			\textbf{Teil 4:} \textbf{Markieren} Sie die Nullstellen der Funktion im Koordinatensystem.\\
			\par\noindent
			\begin{tabularx}{\textwidth}{M|M}
				(I) \(f(x) = 0,25(x-2,5)(x-2)(x-4)\) & (II) \(f(x) = (x - \frac{1}{3})(x + 2)^2(x - \frac{7}{3})\)\\
				\includegraphics[width=0.48\textwidth,align=t]{../99_Bilder/WP/WP11_T3a.png} & \includegraphics[width=0.48\textwidth,align=t]{../99_Bilder/WP/WP11_T3b.png}\\
				& \\
				\hline
				\multicolumn{2}{c}{}\\
				\multicolumn{2}{c}{(III) \(f(x) = 2.5 (x - \frac{1}{5}) (x + 2) (x - 3)^3\)}\\
				\multicolumn{2}{c}{\includegraphics[width=0.48\textwidth,align=t]{../99_Bilder/WP/WP11_T3c.png}}
			\end{tabularx}
		\end{framed}
		\begin{framed}
			\noindent
			\textbf{Teil 5:} \textbf{Geben} Sie zu einer Funktionen aus \textit{\underline{Teil 4}} den \textit{charakteristischen Summanden} und das \textit{Absolutglied} an.\\
			\textbf{Untersuchen} Sie das Verhalten der Funktion für große \(x\)-Beträge.
		\end{framed}
	\end{worksheet}
\end{document}