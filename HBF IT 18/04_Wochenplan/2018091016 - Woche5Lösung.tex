\documentclass[oneside,openany,headings=optiontotoc,11pt,numbers=noenddot]{scrreprt}

\usepackage[a4paper]{geometry}
\usepackage[utf8]{inputenc}
\usepackage[T1]{fontenc}
\usepackage{lmodern}
\usepackage[ngerman]{babel}
\usepackage{ngerman}

\usepackage[onehalfspacing]{setspace}

\usepackage{fancyhdr}
\usepackage{fancybox}

\usepackage{rotating}
\usepackage{varwidth}


\usepackage{pdflscape}
\usepackage{graphicx}
\usepackage{graphbox}
\graphicspath{
	{Pics/PDFs/}
	{Pics/JPGs/}
	{Pics/PNGs/}
}
\usepackage{caption}
\usepackage{tabularx}
\usepackage{dashrule}
\usepackage{hhline}
\usepackage{multirow}
\usepackage{enumerate}
\usepackage[hidelinks]{hyperref}
\usepackage{listings}

\usepackage[table]{xcolor}
\usepackage{array}
\usepackage{enumitem,amssymb,amsmath}
\usepackage{interval}
\usepackage{stmaryrd}
\usepackage{polynom}
\usepackage{diagbox}
\usepackage{dashrule}
\usepackage{framed}
\usepackage{mdframed}
\usepackage{karnaugh-map}

\usepackage{blindtext}

\usepackage{eso-pic}

\usepackage{amssymb}
\usepackage{eurosym}
\pagestyle{headings}
\renewcommand{\headrulewidth}{0.2pt}
\renewcommand{\footrulewidth}{0.2pt}
\newcommand*{\underdownarrow}[2]{\ensuremath{\underset{\overset{\Big\downarrow}{#2}}{#1}}}
\setlength{\fboxsep}{5pt}

% Codestyle defined
\definecolor{codegreen}{rgb}{0,0.6,0}
\definecolor{codegray}{rgb}{0.5,0.5,0.5}
\definecolor{codepurple}{rgb}{0.58,0,0.82}
\definecolor{backcolour}{rgb}{0.95,0.95,0.92}
\definecolor{deepgreen}{rgb}{0,0.5,0}
\definecolor{darkblue}{rgb}{0,0,0.65}
\definecolor{mauve}{rgb}{0.40, 0.19,0.28}
\colorlet{exceptioncolour}{yellow!50!red}
\colorlet{commandcolour}{blue!60!black}
\colorlet{numpycolour}{blue!60!green}
\colorlet{specmethodcolour}{violet}

%Neue Spaltendefinition
\newcolumntype{L}[1]{>{\raggedright\let\newline\\\arraybackslash\hspace{0pt}}m{#1}}
\newcolumntype{M}[1]{>{\centering\arraybackslash}X}
\newcommand{\cmnt}[1]{\ignorespaces}
%Textausrichtung ändern
\newcommand\tabrotate[1]{\rotatebox{90}{\raggedright#1\hspace{\tabcolsep}}}

%Intervall-Konfig
\intervalconfig {
	soft open fences
}

%Bash
\lstdefinestyle{BashInputStyle}{
	language=bash,
	basicstyle=\small\sffamily,
	backgroundcolor=\color{backcolour},
	columns=fullflexible,
	backgroundcolor=\color{backcolour},
	breaklines=true,
}
%Java
\lstdefinestyle{JavaInputStyle}{
	language=Java,
	backgroundcolor=\color{backcolour},
	aboveskip=1mm,
	belowskip=1mm,
	showstringspaces=false,
	columns=flexible,
	basicstyle={\footnotesize\ttfamily},
	numberstyle={\tiny},
	numbers=none,
	keywordstyle=\color{purple},,
	commentstyle=\color{deepgreen},
	stringstyle=\color{blue},
	emph={out},
	emphstyle=\color{darkblue},
	emph={[2]rand},
	emphstyle=[2]\color{specmethodcolour},
	breaklines=true,
	breakatwhitespace=true,
	tabsize=2,
}
%Python
\lstdefinestyle{PythonInputStyle}{
	language=Python,
	alsoletter={1234567890},
	aboveskip=1ex,
	basicstyle=\footnotesize,
	breaklines=true,
	breakatwhitespace= true,
	backgroundcolor=\color{backcolour},
	commentstyle=\color{red},
	otherkeywords={\ , \}, \{, \&,\|},
	emph={and,break,class,continue,def,yield,del,elif,else,%
		except,exec,finally,for,from,global,if,import,in,%
		lambda,not,or,pass,print,raise,return,try,while,assert},
	emphstyle=\color{exceptioncolour},
	emph={[2]True,False,None,min},
	emphstyle=[2]\color{specmethodcolour},
	emph={[3]object,type,isinstance,copy,deepcopy,zip,enumerate,reversed,list,len,dict,tuple,xrange,append,execfile,real,imag,reduce,str,repr},
	emphstyle=[3]\color{commandcolour},
	emph={[4]ode, fsolve, sqrt, exp, sin, cos, arccos, pi,  array, norm, solve, dot, arange, , isscalar, max, sum, flatten, shape, reshape, find, any, all, abs, plot, linspace, legend, quad, polyval,polyfit, hstack, concatenate,vstack,column_stack,empty,zeros,ones,rand,vander,grid,pcolor,eig,eigs,eigvals,svd,qr,tan,det,logspace,roll,mean,cumsum,cumprod,diff,vectorize,lstsq,cla,eye,xlabel,ylabel,squeeze},
	emphstyle=[4]\color{numpycolour},
	emph={[5]__init__,__add__,__mul__,__div__,__sub__,__call__,__getitem__,__setitem__,__eq__,__ne__,__nonzero__,__rmul__,__radd__,__repr__,__str__,__get__,__truediv__,__pow__,__name__,__future__,__all__},
	emphstyle=[5]\color{specmethodcolour},
	emph={[6]assert,range,yield},
	emphstyle=[6]\color{specmethodcolour}\bfseries,
	emph={[7]Exception,NameError,IndexError,SyntaxError,TypeError,ValueError,OverflowError,ZeroDivisionError,KeyboardInterrupt},
	emphstyle=[7]\color{specmethodcolour}\bfseries,
	emph={[8]taster,send,sendMail,capture,check,noMsg,go,move,switch,humTem,ventilate,buzz},
	emphstyle=[8]\color{blue},
	keywordstyle=\color{blue}\bfseries,
	rulecolor=\color{black!40},
	showstringspaces=false,
	stringstyle=\color{deepgreen}
}

\lstset{literate=%
	{Ö}{{\"O}}1
	{Ä}{{\"A}}1
	{Ü}{{\"U}}1
	{ß}{{\ss}}1
	{ü}{{\"u}}1
	{ä}{{\"a}}1
	{ö}{{\"o}}1
}

% Neue Klassenarbeits-Umgebung
\newenvironment{worksheet}[3]
% Begin-Bereich
{
	\newpage
	\sffamily
	\setcounter{page}{1}
	\ClearShipoutPicture
	\AddToShipoutPicture{
		\put(55,761){{
				\mbox{\parbox{385\unitlength}{\tiny \color{codegray}BBS I Mainz, #1 \newline #2
						\newline #3
					}
				}
			}
		}
		\put(455,761){{
				\mbox{\hspace{0.3cm}\includegraphics[width=0.2\textwidth]{../../logo.jpg}}
			}
		}
	}
}
% End-Bereich
{
	\clearpage
	\ClearShipoutPicture
}

\geometry{left=2.50cm,right=2.50cm,top=3.00cm,bottom=1.00cm,includeheadfoot}

\begin{document}
	\begin{worksheet}{Höhere Berufsfachschule IT-Systeme}{Grundstufe - Mathematik}{Wochenplan Lineare Funktionen}
		\noindent
		\begin{tabularx}{\textwidth}{XXl}
			Wochenplan Nr.: \rule{0.15\textwidth}{1pt} & Erledigt: & Zeitraum: \underline{10.09 - 16.09}
		\end{tabularx}
	
		\begin{framed}
			\noindent
			\textbf{Montag:} Geben Sie die Steigung und den y-Achsenabschnitt der linearen Funktion an.\\
			\par\noindent
			\begin{tabularx}{\textwidth}{XX}
				(a) \(f(x) = 3x\) & (b) \(f(x) = \frac{1}{5}x +2\)\\
				\colorbox{green!10}{\(m = 3; y-AAS = 0\)} & \colorbox{green!10}{\(m = \frac{1}{5}; y-AAS = 2\)}\\
				\\
				(c) \(f(x) = -\frac{4}{3}x -\frac{5}{2}\) & (d) \(f(x) = 1,5x+0,5\)\\
				\colorbox{green!10}{\(m=-\frac{4}{3}; y-AAS = -\frac{5}{2}\)} & \colorbox{green!10}{\(m = 1,5; y-AAS = 0,5\)}
			\end{tabularx}
		\end{framed}
		\begin{framed}
			\noindent
			\textbf{Dienstag:} Bestimmen Sie die Funktionsgleichung der wie folgt gegebenen linearen Funktion.
			\indent{} Die Gerade steigt um ein Drittel pro Einheit auf der x-Achse und geht durch den Punkt \(P(-3|-4)\).
			\begin{center}
				\rule{0.9\textwidth}{0.1pt}\\
			\end{center}
			Wir erinnern uns, dass eine lineare Funktion die Form \colorbox{green!10}{\(y = mx + b\)} hat, wobei \underline{m die Steigung} und \underline{b den y-Achsenabschnitt} angibt.\\
			\par\noindent
			Aus dem Text wissen wir: \(m = \frac{1}{3}\). Wir benötigen also lediglich noch den y-Achsenabschnitt. Um diesen zu bestimmen, verwenden wir die \textbf{Punkt-Steigung-Form} \(y-y_1 = m*(x-x_1)\).\\
			\par\noindent
			Der gegebene Punkt \(P(\underbrace{-3}_{x_1}|\underbrace{-4}_{y_1})\) gibt uns die nötigen Informationen. Wir setzen diese nur noch in die Gleichung ein und lösen anschließend nach \(y\) auf.\\
			\par\noindent
			\begin{tabularx}{\textwidth}{ll}
				\(y - (-4) = \frac{1}{3}\cdot(x-(-3))\) & |\(AM\)\\
				\(y + 4 = \frac{1}{3}x + 1\) & |\(-4\)\\
				\colorbox{green!10}{\(y = \frac{1}{3}x -1\)}
			\end{tabularx}
		\end{framed}
		\begin{framed}
			\noindent
			\textbf{Mittwoch:} Bestimmen Sie jeweils die Gleichung zu der Geraden, die durch P geht und die Steigung \(m\) hat.
			\begin{framed}
				\noindent
				Wir kennen sowohl einen Punkt wie auch die Steigung der Geraden. Das bedeutet, um die entsprechende Funktionsgleichung aufzustellen, benötigen wir lediglich die \textbf{Punkt-Steigung-Form}: \(\mathbf{y -y_1 = m\cdot(x-x_1)}\).
			\end{framed}
			\begin{tabularx}{\textwidth}{Xl|Xl}
				(a) \(P(2|5); m=3\) & & (b) \(P(4|-2); m=0\)\\
				\(y - 5 = 3(x-2)\) & |\(AM\) & \(y-(-2) = 4(x-4)\) & |\(AM\)\\
				\(y - 5 = 3x -6\) & |\(+5\) & \(y+2 = 4x -16\) & |\(-2\)\\
				\colorbox{green!10}{\(y= 3x - 1\)} & & \colorbox{green!10}{\(y = 4x - 18\)}\\
				\\
				\hline
				\\
				(c) \(P(-3|1); m=-1\) & & (d) \(P(1,5|0,5); m=-4\)\\
				\(y - 1 = (-1)(x-(-3))\) &  & \( y-0,5 = (-4)(x-1,5)\) & |\(AM\)\\
				\(y -1 = (-1)(x+3)\) & |\(AM\) & \( y-0,5 = -4x +6\) & |\(+0,5\)\\
				\(y - 1 = -x -3\) & |\(+1\) & \colorbox{green!10}{\(y = -4x +6,5\)}\\
				\colorbox{green!10}{\(y = -x -2\)} & & 
			\end{tabularx}
		\end{framed}
		\begin{framed}
			\noindent
			\textbf{Donnerstag:} Käpt\grq{}n Blaubär fährt mit seinem Tankschiff A bei einer Durchschnittsgeschwindigkeit von \(400\ sm\) pro Tag von Hong Kong nach Hamburg.\\
			Hein Blöd steuert Tankschiff B von Hamburg nach Hong Kong mit \(550\ sm\) pro Tag zurück.
			\begin{itemize}
				\item[(b)] Geben Sie die Funktionsgleichung von \(f_A\) bzw. \(f_B\) an, die die Fahrt der Tanker A bzw B beschreiben.\\
				\colorbox{green!10}{\(f_A(x) = 400x; f_B(x) = 550x\)}
				\item[(c)] Stellen Sie zu jeder Funktion eine Wertetabelle auf.
			\end{itemize}
			\begin{tabularx}{\textwidth}{l|l|l|l|l|l|l|l|l}
				x & 0 & 1 & 2 & 3 & 4 & 5 & 6 & 7\\
				\hline
				& & & & & & & &\\
				\(f_A(x)\) & 0 & 400 & 800 & 1200 & 1600 & 2000 & 2400 & 2800\\
				\hline
				& & & & & & & &\\
				\(f_B(x)\) & 0 & 550 & 1100 & 1650 & 2200 & 2750 & 3300 & 3850\\
			\end{tabularx}
			\begin{itemize}
				\item[(d)] Nach wie vielen Tagen können  sich Käpt\grq{}n Blaubär und Hein Blöd auf hoher See zuwinken?\\
				\colorbox{red!10}{Da wir keine Information über die Strecke haben, die zwischen Hong Kong und Ham-}\\
				\colorbox{red!10}{burg liegt, wir also nicht sagen können, wie weit die beiden bereits von ihrem Ursprungs-}\\
				\colorbox{red!10}{ort entfernt sind, lässt sich die Frage mit den gegebenen Informationen nicht lösen.}
			\end{itemize}
		\end{framed}
		\begin{framed}
			\noindent
			\textbf{Freitag:} Zei Motorradfahrer fahren auf derselben Straße von A nach B. Die beiden Orte sind \(270\ km\) voneinander entfernt.\\
			Fahrer M1 fährt um 9 Uhr ab und hält eine Durchschnittsgeschwindigkeit von \(45\ \frac{km}{h}\). 75 Minuten später startet Fahrer M2 und fährt durchschnittlich \(60\ \frac{km}{h}\).
			\begin{itemize}
				\item[(a)] Stellen Sie den Sachverhalt mithilfe zweier Funktionen dar.
				\item[(b)] Ermitteln Sie durch Rechnung die Ankunftszeiten der beiden Fahrer.
				\item[(c)] Zu welchem Zeitpunkt treffen sich die beiden Fahrer? Wie weit sind sie zu diesem Zeitpunkt vom Startpunkt entfernt?
			\end{itemize}
		\end{framed}
	\end{worksheet}
\end{document}