\documentclass[oneside,openany,headings=optiontotoc,11pt,numbers=noenddot]{scrreprt}

\usepackage[a4paper]{geometry}
\usepackage[utf8]{inputenc}
\usepackage[T1]{fontenc}
\usepackage{lmodern}
\usepackage[ngerman]{babel}
\usepackage{ngerman}

\usepackage[onehalfspacing]{setspace}

\usepackage{fancyhdr}
\usepackage{fancybox}

\usepackage{rotating}
\usepackage{varwidth}


\usepackage{pdflscape}
\usepackage{graphicx}
\usepackage{graphbox}
\graphicspath{
	{Pics/PDFs/}
	{Pics/JPGs/}
	{Pics/PNGs/}
}
\usepackage{caption}
\usepackage{tabularx}
\usepackage{dashrule}
\usepackage{hhline}
\usepackage{multirow}
\usepackage{enumerate}
\usepackage[hidelinks]{hyperref}
\usepackage{listings}

\usepackage[table]{xcolor}
\usepackage{array}
\usepackage{enumitem,amssymb,amsmath}
\usepackage{interval}
\usepackage{stmaryrd}
\usepackage{polynom}
\usepackage{diagbox}
\usepackage{dashrule}
\usepackage{framed}
\usepackage{mdframed}
\usepackage{karnaugh-map}

\usepackage{blindtext}

\usepackage{eso-pic}

\usepackage{amssymb}
\usepackage{eurosym}
\pagestyle{headings}
\renewcommand{\headrulewidth}{0.2pt}
\renewcommand{\footrulewidth}{0.2pt}
\newcommand*{\underdownarrow}[2]{\ensuremath{\underset{\overset{\Big\downarrow}{#2}}{#1}}}
\setlength{\fboxsep}{5pt}

% Codestyle defined
\definecolor{codegreen}{rgb}{0,0.6,0}
\definecolor{codegray}{rgb}{0.5,0.5,0.5}
\definecolor{codepurple}{rgb}{0.58,0,0.82}
\definecolor{backcolour}{rgb}{0.95,0.95,0.92}
\definecolor{deepgreen}{rgb}{0,0.5,0}
\definecolor{darkblue}{rgb}{0,0,0.65}
\definecolor{mauve}{rgb}{0.40, 0.19,0.28}
\colorlet{exceptioncolour}{yellow!50!red}
\colorlet{commandcolour}{blue!60!black}
\colorlet{numpycolour}{blue!60!green}
\colorlet{specmethodcolour}{violet}

%Neue Spaltendefinition
\newcolumntype{L}[1]{>{\raggedright\let\newline\\\arraybackslash\hspace{0pt}}m{#1}}
\newcolumntype{M}[1]{>{\centering\arraybackslash}X}
\newcommand{\cmnt}[1]{\ignorespaces}
%Textausrichtung ändern
\newcommand\tabrotate[1]{\rotatebox{90}{\raggedright#1\hspace{\tabcolsep}}}

%Intervall-Konfig
\intervalconfig {
	soft open fences
}

%Bash
\lstdefinestyle{BashInputStyle}{
	language=bash,
	basicstyle=\small\sffamily,
	backgroundcolor=\color{backcolour},
	columns=fullflexible,
	backgroundcolor=\color{backcolour},
	breaklines=true,
}
%Java
\lstdefinestyle{JavaInputStyle}{
	language=Java,
	backgroundcolor=\color{backcolour},
	aboveskip=1mm,
	belowskip=1mm,
	showstringspaces=false,
	columns=flexible,
	basicstyle={\footnotesize\ttfamily},
	numberstyle={\tiny},
	numbers=none,
	keywordstyle=\color{purple},,
	commentstyle=\color{deepgreen},
	stringstyle=\color{blue},
	emph={out},
	emphstyle=\color{darkblue},
	emph={[2]rand},
	emphstyle=[2]\color{specmethodcolour},
	breaklines=true,
	breakatwhitespace=true,
	tabsize=2,
}
%Python
\lstdefinestyle{PythonInputStyle}{
	language=Python,
	alsoletter={1234567890},
	aboveskip=1ex,
	basicstyle=\footnotesize,
	breaklines=true,
	breakatwhitespace= true,
	backgroundcolor=\color{backcolour},
	commentstyle=\color{red},
	otherkeywords={\ , \}, \{, \&,\|},
	emph={and,break,class,continue,def,yield,del,elif,else,%
		except,exec,finally,for,from,global,if,import,in,%
		lambda,not,or,pass,print,raise,return,try,while,assert},
	emphstyle=\color{exceptioncolour},
	emph={[2]True,False,None,min},
	emphstyle=[2]\color{specmethodcolour},
	emph={[3]object,type,isinstance,copy,deepcopy,zip,enumerate,reversed,list,len,dict,tuple,xrange,append,execfile,real,imag,reduce,str,repr},
	emphstyle=[3]\color{commandcolour},
	emph={[4]ode, fsolve, sqrt, exp, sin, cos, arccos, pi,  array, norm, solve, dot, arange, , isscalar, max, sum, flatten, shape, reshape, find, any, all, abs, plot, linspace, legend, quad, polyval,polyfit, hstack, concatenate,vstack,column_stack,empty,zeros,ones,rand,vander,grid,pcolor,eig,eigs,eigvals,svd,qr,tan,det,logspace,roll,mean,cumsum,cumprod,diff,vectorize,lstsq,cla,eye,xlabel,ylabel,squeeze},
	emphstyle=[4]\color{numpycolour},
	emph={[5]__init__,__add__,__mul__,__div__,__sub__,__call__,__getitem__,__setitem__,__eq__,__ne__,__nonzero__,__rmul__,__radd__,__repr__,__str__,__get__,__truediv__,__pow__,__name__,__future__,__all__},
	emphstyle=[5]\color{specmethodcolour},
	emph={[6]assert,range,yield},
	emphstyle=[6]\color{specmethodcolour}\bfseries,
	emph={[7]Exception,NameError,IndexError,SyntaxError,TypeError,ValueError,OverflowError,ZeroDivisionError,KeyboardInterrupt},
	emphstyle=[7]\color{specmethodcolour}\bfseries,
	emph={[8]taster,send,sendMail,capture,check,noMsg,go,move,switch,humTem,ventilate,buzz},
	emphstyle=[8]\color{blue},
	keywordstyle=\color{blue}\bfseries,
	rulecolor=\color{black!40},
	showstringspaces=false,
	stringstyle=\color{deepgreen}
}

\lstset{literate=%
	{Ö}{{\"O}}1
	{Ä}{{\"A}}1
	{Ü}{{\"U}}1
	{ß}{{\ss}}1
	{ü}{{\"u}}1
	{ä}{{\"a}}1
	{ö}{{\"o}}1
}

% Neue Klassenarbeits-Umgebung
\newenvironment{worksheet}[3]
% Begin-Bereich
{
	\newpage
	\sffamily
	\setcounter{page}{1}
	\ClearShipoutPicture
	\AddToShipoutPicture{
		\put(55,761){{
				\mbox{\parbox{385\unitlength}{\tiny \color{codegray}BBS I Mainz, #1 \newline #2
						\newline #3
					}
				}
			}
		}
		\put(455,761){{
				\mbox{\hspace{0.3cm}\includegraphics[width=0.2\textwidth]{../../logo.jpg}}
			}
		}
	}
}
% End-Bereich
{
	\clearpage
	\ClearShipoutPicture
}

\geometry{left=2.00cm,right=1.50cm,top=3.00cm,bottom=1.00cm,includeheadfoot}

\begin{document}
	\begin{test}{Vorbereitung}{Mathematik}{HBF IT 18A - V}
		\begin{framed}
			\noindent
			\scriptsize{Allgemeines
			\begin{itemize}[noitemsep,topsep=0pt]
				\item[-] Bei der Bearbeitung ist ein \textbf{nachvollziehbarer, vollständiger Rechenweg} aufzuschreiben.
				\item[-] Die Bewertung der Klassenarbeit ist nur bei \textbf{gut lesbarer Schrift} möglich.
				\item[-] Die Lösungen müssen mit dokumentenechtem Stift (\textbf{Kugelschreiber} oder \textbf{Fine-Liner} - keine rote Mine) erstellt werden.
				\item[-] Runden Sie ihre Ergebnisse auf \textbf{2 Nachkommastellen}. Wurzelausdrücke müssen nicht berechnet werden (z.B. \(\sqrt{10}\)).\\
				
				\item[-] \textbf{Zugelassene Hilfsmittel}: Taschenrechner (nicht graphikfähig / programmierbar)
				\item[-] \textbf{Bearbeitungszeit:} 90 Minuten
			\end{itemize}}
		\end{framed}
		\normalsize
			\noindent
		\begin{tabularx}{\textwidth}{Xl}\underline{\textbf{Aufgabe 1}}& \end{tabularx}\\
		\textbf{Ergänzen} Sie alle Eigenschaften, die Sie \underline{direkt aus der Funktionsgleichung} ableiten können.\\
		\renewcommand{\arraystretch}{1.5}
		\begin{tabularx}{\textwidth}{XX}
			(a) \( f(x) = 4(x+3)(x-6)\) & (c) \(f(x) = 0,2(x-5)^2 + 7\)\\
			(b) \(f(x) = -2x^2 + 8x -20\) & (d) \(f(x) = -(x+4)^2 + 1\)
		\end{tabularx}\\
		\par\noindent
		\renewcommand{\arraystretch}{1.3}
		\begin{tabularx}{\textwidth}{|X|X|X|X|X|X|}
			\hline
			\rowcolor{codegray!10} \textbf{Gleichung} & \textbf{Öffnungs-} & \textbf{Normalparabel/} & \textbf{Scheitelpunkt} & \textbf{Nullstellen} & \textbf{y-AAS}\\
			\rowcolor{codegray!10} & \textbf{richtung} & \textbf{gestreckte P./} & \textbf{SP(...|...)} &  \(x_1 = \ldots\), & \(y_s = \ldots\)\\
			\rowcolor{codegray!10} & \textbf{(oben/unten)} & \textbf{gestauchte P.} & & \(x_2 = \ldots\) & \\
			\hline
			\renewcommand{\arraystretch}{3}
			(a) & & & & & \\
			\hline
			(b) & & & & & \\
			\hline
			(c) & & & & & \\
			\hline
			(c) & & & & & \\
			\hline
			\multicolumn{6}{c}{}\\
		\end{tabularx}\\
		\par\noindent
		\begin{tabularx}{\textwidth}{Xl}\underline{\textbf{Aufgabe 2}}&\end{tabularx}\\
		Über die Entwicklung der Anzahl von Touristen in Rheinland-Pfalz sind folgende Daten bekannt:\\
		\par\noindent
		\renewcommand{\arraystretch}{1}
		\begin{tabularx}{\textwidth}{|X|X|X|}
			\hline
			\textbf{x} (eine Einheit = 10 Jahre, \(0 = 1980\)) & 0 & 3\\
			\hline
			\textbf{y} (in Millionen) & 1 & 8\\
			\hline
		\end{tabularx}\\
		\par\noindent
		\begin{itemize}
			\item[(a)] \textbf{Stellen} Sie die Funktionsgleichung einer quadratischen Funktion \textbf{auf}, die den Scheitelpunkt bei \((3|8)\) hat und durch den Punkt \((0|1)\) geht!			
			\item[(b)] \textbf{Bescheiben} Sie die Bedeutung des Scheitelpunkts für diese Entwicklung!
			\item[(c)] Eine andere Entwicklung wird mit \(f(x) = -\frac{1}{5}(x+1)(x-11)\) angegeben.\\
			\textbf{Vergleichen} Sie diese mit der Entwicklung aus (a)
			\item[(d)] \textbf{Geben} Sie die Funktionsgleichung aus (c) in allgemeiner Form an.
		\end{itemize}
	\end{test}
	\newgeometry{left=2.00cm,right=1.50cm,top=1.50cm,bottom=1.00cm,includeheadfoot}
	\noindent
	\begin{tabularx}{\textwidth}{Xl}\underline{\textbf{Aufgabe 3}}&\end{tabularx}\\
	Gegeben sind die nachfolgenden Funktionsgleichungen.\\
	\begin{tabularx}{\textwidth}{XXX}
		(a) \(f(x) = x(x-1)^2(x+5)\) & (b) \( f(x) = x^3 - 10x + 25x\) & (c) \(f(x) = x^3 -4x\)\\
		\\
		(d) \(f(x) = -2x^3 +8x^2\) & (e) \(f(x) = x^2 + 4\)
	\end{tabularx}\\
	\par\noindent
	\begin{itemize}
		\item[(1)] \textbf{Bestimmen} Sie \textit{jeweils} die Nullstellen! \textbf{Markieren} Sie doppelte Nullstellen\footnote{Kommt zweimal vor.} entsprechend.
		\item[(2)] \textbf{Geben} Sie das Verhalten der Funktionswerte für große x-Beträge der Funktionen (c) - (e) an.\\
		\small{Nutzen Sie dabei die formale Schreibweise: \(f(x)\xrightarrow{x\rightarrow{}-\infty}?\) bzw. \(f(x)\xrightarrow{x\rightarrow{}\infty}?\)}\normalsize
	\end{itemize}
	\begin{tabularx}{\textwidth}{Xl}\underline{\textbf{Aufgabe 4}}&\end{tabularx}\\
	\textbf{Ordnen} Sie die angegebenen Funktionsgleichungen den zugehörigen Graphen zu.\\
	Schreiben Sie die entsprechende Funktionsbezeichnung (\(f(x), g(x), \ldots\)) an den Graphen.\\
	\textbf{Begründen} Sie ihre Entscheidung kurz.\\
	\par\noindent
	\renewcommand{\arraystretch}{1.3}
	\begin{tabularx}{\textwidth}{XXX}
		\(f(x) = -x + 2\) & \(g(x) = -x^2 + 4\) & \(h(x) = -x^2 + 4x\)\\
		\(k(x) = -0,5x^4 + 2x^2\) & \(l(x) = 0,25x(x+2)^2(x-2)\) & \(m(x) = 0,2x^2(x+2)3(x-2)\)
	\end{tabularx}\\
	\par\noindent
	\begin{tabularx}{\textwidth}{M|M}
		\includegraphics[width=0.4\textwidth,align=c]{../99_Bilder/2VKA/Vl.png} & \includegraphics[width=0.4\textwidth,align=c]{../99_Bilder/2VKA/Vk.png}\\
		& \\
		\hline
		& \\
		\includegraphics[width=0.4\textwidth,align=c]{../99_Bilder/2VKA/Vf.png} & \includegraphics[width=0.4\textwidth,align=c]{../99_Bilder/2VKA/Vh.png}\\
		\\
	\end{tabularx}
	\begin{tabularx}{\textwidth}{M|M}
		& \\
		\includegraphics[width=0.4\textwidth,align=c]{../99_Bilder/2VKA/Vm.png} & \includegraphics[width=0.4\textwidth,align=c]{../99_Bilder/2VKA/Vg.png}\\
		\\
	\end{tabularx}\\
	\par\noindent
	\begin{tabularx}{\textwidth}{Xl}\underline{\textbf{Aufgabe 5}}&\end{tabularx}\\
	\textbf{Entscheiden} Sie für die folgenden Teilaufgaben, ob eine Lösung möglich ist.\\
	\indent
	Falls \textit{ja}, \textbf{geben} Sie eine Lösung an.\\
	\indent
	Falls \textit{nein}, \textbf{begründen} Sie, warum keine Lösung existiert.
	\begin{itemize}
		\item[(a)] Geben Sie den Funktionsterm einer ganzrationalen Funktion siebten Grades an, für den gilt:\\
		\(f(x) \xrightarrow{x\rightarrow\infty}-\infty\)
		\item[(b)] Geben Sie den Funktionsterm einer ganzrationalen Funktion zweiten Grades an, die an der Stelle \(x = 2\) eine dreifache Nullstelle hat.
		\item[(c)] Geben Sie den Funktionsterm einer ganzrationalen Funktion dritten Grades an, die an der Stelle \(x = 2\) eine zweifache Nullstelle hat.
	\end{itemize}
\end{document}