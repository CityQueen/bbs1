\documentclass[oneside,openany,headings=optiontotoc,12pt,numbers=noenddot]{scrreprt}

\usepackage[a4paper]{geometry}
\usepackage[utf8]{inputenc}
\usepackage[T1]{fontenc}
\usepackage{lmodern}
\usepackage[ngerman]{babel}
\usepackage{ngerman}

\usepackage[onehalfspacing]{setspace}

\usepackage{fancyhdr}
\usepackage{fancybox}

\usepackage{rotating}
\usepackage{varwidth}


\usepackage{pdflscape}
\usepackage{graphicx}
\usepackage{graphbox}
\graphicspath{
	{Pics/PDFs/}
	{Pics/JPGs/}
	{Pics/PNGs/}
}
\usepackage{caption}
\usepackage{tabularx}
\usepackage{dashrule}
\usepackage{hhline}
\usepackage{multirow}
\usepackage{enumerate}
\usepackage[hidelinks]{hyperref}
\usepackage{listings}

\usepackage[table]{xcolor}
\usepackage{array}
\usepackage{enumitem,amssymb,amsmath}
\usepackage{interval}
\usepackage{stmaryrd}
\usepackage{polynom}
\usepackage{diagbox}
\usepackage{dashrule}
\usepackage{framed}
\usepackage{mdframed}
\usepackage{karnaugh-map}

\usepackage{blindtext}

\usepackage{eso-pic}

\usepackage{amssymb}
\usepackage{eurosym}
\pagestyle{headings}
\renewcommand{\headrulewidth}{0.2pt}
\renewcommand{\footrulewidth}{0.2pt}
\newcommand*{\underdownarrow}[2]{\ensuremath{\underset{\overset{\Big\downarrow}{#2}}{#1}}}
\setlength{\fboxsep}{5pt}

% Codestyle defined
\definecolor{codegreen}{rgb}{0,0.6,0}
\definecolor{codegray}{rgb}{0.5,0.5,0.5}
\definecolor{codepurple}{rgb}{0.58,0,0.82}
\definecolor{backcolour}{rgb}{0.95,0.95,0.92}
\definecolor{deepgreen}{rgb}{0,0.5,0}
\definecolor{darkblue}{rgb}{0,0,0.65}
\definecolor{mauve}{rgb}{0.40, 0.19,0.28}
\colorlet{exceptioncolour}{yellow!50!red}
\colorlet{commandcolour}{blue!60!black}
\colorlet{numpycolour}{blue!60!green}
\colorlet{specmethodcolour}{violet}

%Neue Spaltendefinition
\newcolumntype{L}[1]{>{\raggedright\let\newline\\\arraybackslash\hspace{0pt}}m{#1}}
\newcolumntype{M}[1]{>{\centering\arraybackslash}X}
\newcommand{\cmnt}[1]{\ignorespaces}
%Textausrichtung ändern
\newcommand\tabrotate[1]{\rotatebox{90}{\raggedright#1\hspace{\tabcolsep}}}

%Intervall-Konfig
\intervalconfig {
	soft open fences
}

%Bash
\lstdefinestyle{BashInputStyle}{
	language=bash,
	basicstyle=\small\sffamily,
	backgroundcolor=\color{backcolour},
	columns=fullflexible,
	backgroundcolor=\color{backcolour},
	breaklines=true,
}
%Java
\lstdefinestyle{JavaInputStyle}{
	language=Java,
	backgroundcolor=\color{backcolour},
	aboveskip=1mm,
	belowskip=1mm,
	showstringspaces=false,
	columns=flexible,
	basicstyle={\footnotesize\ttfamily},
	numberstyle={\tiny},
	numbers=none,
	keywordstyle=\color{purple},,
	commentstyle=\color{deepgreen},
	stringstyle=\color{blue},
	emph={out},
	emphstyle=\color{darkblue},
	emph={[2]rand},
	emphstyle=[2]\color{specmethodcolour},
	breaklines=true,
	breakatwhitespace=true,
	tabsize=2,
}
%Python
\lstdefinestyle{PythonInputStyle}{
	language=Python,
	alsoletter={1234567890},
	aboveskip=1ex,
	basicstyle=\footnotesize,
	breaklines=true,
	breakatwhitespace= true,
	backgroundcolor=\color{backcolour},
	commentstyle=\color{red},
	otherkeywords={\ , \}, \{, \&,\|},
	emph={and,break,class,continue,def,yield,del,elif,else,%
		except,exec,finally,for,from,global,if,import,in,%
		lambda,not,or,pass,print,raise,return,try,while,assert},
	emphstyle=\color{exceptioncolour},
	emph={[2]True,False,None,min},
	emphstyle=[2]\color{specmethodcolour},
	emph={[3]object,type,isinstance,copy,deepcopy,zip,enumerate,reversed,list,len,dict,tuple,xrange,append,execfile,real,imag,reduce,str,repr},
	emphstyle=[3]\color{commandcolour},
	emph={[4]ode, fsolve, sqrt, exp, sin, cos, arccos, pi,  array, norm, solve, dot, arange, , isscalar, max, sum, flatten, shape, reshape, find, any, all, abs, plot, linspace, legend, quad, polyval,polyfit, hstack, concatenate,vstack,column_stack,empty,zeros,ones,rand,vander,grid,pcolor,eig,eigs,eigvals,svd,qr,tan,det,logspace,roll,mean,cumsum,cumprod,diff,vectorize,lstsq,cla,eye,xlabel,ylabel,squeeze},
	emphstyle=[4]\color{numpycolour},
	emph={[5]__init__,__add__,__mul__,__div__,__sub__,__call__,__getitem__,__setitem__,__eq__,__ne__,__nonzero__,__rmul__,__radd__,__repr__,__str__,__get__,__truediv__,__pow__,__name__,__future__,__all__},
	emphstyle=[5]\color{specmethodcolour},
	emph={[6]assert,range,yield},
	emphstyle=[6]\color{specmethodcolour}\bfseries,
	emph={[7]Exception,NameError,IndexError,SyntaxError,TypeError,ValueError,OverflowError,ZeroDivisionError,KeyboardInterrupt},
	emphstyle=[7]\color{specmethodcolour}\bfseries,
	emph={[8]taster,send,sendMail,capture,check,noMsg,go,move,switch,humTem,ventilate,buzz},
	emphstyle=[8]\color{blue},
	keywordstyle=\color{blue}\bfseries,
	rulecolor=\color{black!40},
	showstringspaces=false,
	stringstyle=\color{deepgreen}
}

\lstset{literate=%
	{Ö}{{\"O}}1
	{Ä}{{\"A}}1
	{Ü}{{\"U}}1
	{ß}{{\ss}}1
	{ü}{{\"u}}1
	{ä}{{\"a}}1
	{ö}{{\"o}}1
}

% Neue Klassenarbeits-Umgebung
\newenvironment{worksheet}[3]
% Begin-Bereich
{
	\newpage
	\sffamily
	\setcounter{page}{1}
	\ClearShipoutPicture
	\AddToShipoutPicture{
		\put(55,761){{
				\mbox{\parbox{385\unitlength}{\tiny \color{codegray}BBS I Mainz, #1 \newline #2
						\newline #3
					}
				}
			}
		}
		\put(455,761){{
				\mbox{\hspace{0.3cm}\includegraphics[width=0.2\textwidth]{../../logo.jpg}}
			}
		}
	}
}
% End-Bereich
{
	\clearpage
	\ClearShipoutPicture
}

\geometry{left=2.00cm,right=1.50cm,top=2.25cm,bottom=1.00cm,includeheadfoot}

\begin{document}
	\begin{worksheet}{HBF IT 18A}{Grundstufe}{Kompetenzraster - Ganzrationale Funktionen}
		\footnotesize
		\raggedright
		\renewcommand{\arraystretch}{1.5}
		\begin{tabularx}{\textwidth}{|L{2.25cm}|X|X|X|}
			\multicolumn{1}{|c|}{\small{}Kompetenzbezug} & \multicolumn{1}{c|}{\small{}B A S I C} & \multicolumn{1}{c|}{\small{}P R O F I} & \multicolumn{1}{c|}{\small{}E X P E R T E}\\
			\hline\hline
			\textbf{Prototypen von Funktionsgleichungen quadratischer Funktionen} & Ich erkenne \textbf{Prototypen quadratischer Funktionen}: Allgemeine Form, Scheitelpunktform, Linearfaktorform & Ich kann Charakteristika (Nullstellen und Scheitelpunkt) des Graphen, wenn möglich, anhand der Funktionsgleichung ablesen. & Ich kann eine Funktionsgleichung \textbf{von einer Form in andere Formen überführen}.\\
			& Ich kann mit einer Wertetabelle die \textbf{passende Parabel zu einer Funktion sichtbar machen}. & Ich kann eine \textbf{quadratische Ergänzung} durchführen & Ich kann die \textbf{Parabeln aus der Scheitelpunktform und der Linearfaktorform skizzieren}.\\
			& Ich kann bei einer Parabel den \textbf{Scheitelpunkt markieren}. & & \\
			& \textit{WP } & \textit{WP } & \textit{WP } \\
			\hline
			\textbf{Quadratische Gleichungen} & Ich kenne die \textbf{Ansätze zur Berechnung} von \textbf{Nullstellen} und des \textbf{y-Achsenabschnitts}. & Ich kann \textbf{quadratische Gleichungen} durch fehlerfreie Anwendung der pq-Formel \textbf{lösen}. & Ich kann den Wert unterhalb der Wurzel bei der pq-Formel (\textbf{Diskriminante}) im Hinblick auf das \textbf{Vorliegen von Nullstellen} interpretieren.\\
			& Ich weiß, wo ich die \textbf{pq-Formel nachschlagen} kann und kenne die \textbf{Voraussetzungen zum Einsatz der pq-Formel}. & Steht ein \textbf{anderer Faktor als \grqq{}1\grqq{} vor dem \(\mathbf{x^2}\)}, kann ich die \textbf{Gleichung} so \textbf{umformen}, dass ich die pq-Formel anwenden kann. & Ich kann \textbf{Schnittpunkte} zwischen Parabeln und Geraden \textbf{berechnen}, wenn die Gleichungen gegeben sind.\\
			& Ich kann bei jeder Form unter Beachtung der \textit{KlaPoPuStri} \textbf{zu einer Stelle x den entsprechenden Funktionswert y berechnen}. & Ich kann quadratische Gleichungen vom Typ \(ax^2 + bx  = 0\) und \(ax^2 + c = 0\) durch Ausklammern bzw. Umformen lösen. & \\
			& & Ich kann quadratische Gleichungen vom Typ \(a(x-x_1)(x-x_2) = 0\) \textbf{direkt lösen}.& \\
			& & & \\
			& \textit{WP } & \textit{WP } & \textit{WP } \\
			\hline
		\end{tabularx}
		\begin{tabularx}{\textwidth}{|L{2.25cm}|X|X|X|}
			\multicolumn{1}{|c|}{\small{}Kompetenzbezug} & \multicolumn{1}{c|}{\small{}B A S I C} & \multicolumn{1}{c|}{\small{}P R O F I} & \multicolumn{1}{c|}{\small{}E X P E R T E}\\
			\hline\hline
			& & \scriptsize{Modell Funktionsgleichung selbst aufstellen}\footnotesize & \\
			\textbf{Modellierung mit quadratischen Funktionen} & Ich kenne folgenden \textbf{Zusammehang: Extrempunkt} (Extremstelle und relatives Maximum bzw Minimum) und \textbf{Scheitelpunkt} (x-Koordinate des SP und y-Koordinate des SP). & Ich kann je nach gegebenen Informationen \textbf{entscheiden, welcher Prototyp am besten geeignet ist}, um die notwendigen Parameter für ein Modell zu bestimmen. & Ich kann die Eignung des Modells (Funktion) für die Situation \textbf{kritisch einschätzen}.\\
			& Ich erkenne aus der Aufgabenstellung, \textbf{welche charakteristischen Punkte gesucht} sind. & Ich kann aus den geforderten Punkten des Graphen und den Prototypen Gleichungen aufstellen, mit deren Hilfe ich die Parameter für das Modell berechnen kann. & \\
			& Ich kann den \textbf{Verlauf einer Parabel} mit den Vokabeln zur Beschreibung von Funktionsgraphen \textbf{beschreiben}. & Ich kann \textbf{im Modell} arbeiten (\textbf{Nullstellen und Funktionsstellen berechnen}) und die gewonnenen \textbf{Ergebnisse interpretieren}. & \\
			& & & \\
			& \textit{WP } & \textit{WP } & \textit{WP } \\
			\hline
		\end{tabularx}
		\newpage
		\begin{tabularx}{\textwidth}{|L{2.5cm}|X|X|X|}
			\multicolumn{1}{|c|}{\small{}Kompetenzbezug} & \multicolumn{1}{c|}{\small{}B A S I C} & \multicolumn{1}{c|}{\small{}P R O F I} & \multicolumn{1}{c|}{\small{}E X P E R T E}\\
			\hline\hline
			\textbf{Basics zu ganzrationalen Funktionen} & Ich erkenne eine \textbf{Funktionsgleichung} einer \textbf{ganzrationalen Funktion}. & Ich kenne das \textbf{Potenzgesetz} zur Multiplikation von Potenzen mit gleicher Basis (\(x^2\cdot{}x = x^3\)). & Ich kann den \textbf{Grad} einer Funktion auch bestimmen, wenn die Funktion in \textbf{Faktorform} gegeben ist.\\
			& Ich kann erläutern, ob eine Produktstruktur (\textbf{Faktorform}) oder eine Summenstruktur (\textbf{Polynomform}) vorliegt und kann entsprechend die Faktoren bzw. die Summanden markieren. & Ich kann eine in Faktorform gegebene Funktion durch \textbf{Ausmultiplizieren} in die \textbf{Polynomform} überführen. & Ich kann den \textbf{charakteristischen Summanden} einer Funktion auch bestimmen, wenn die Funktion in \textbf{Faktorform} gegeben ist.\\
			& Ich kann den \textbf{Grad}, den \textbf{charakteristischen Summanden} und die \textbf{Koeffizienten} einer ganzrationalen Funktion benennen, wenn sie in der \textbf{Polynomform} gegeben ist. & & \\
			& & & \\
			& \textit{WP } & \textit{WP } & \textit{WP } \\
			\hline
			\textbf{Verhalten für große x-Beträge} & Wenn ich einen Graphen einer ganzrationalen Funktion sehe, kann ich das \textbf{Verhalten der Funktionswerte} für große x-Beträge zum Ausdruck bringen. & Ich erkenne anhand des \textbf{charakteristischen Summanden}, woher der Graph \grqq{}kommt\grqq{} und wohin er \grqq{}geht\grqq{}. & Ich kann das \textbf{Verhalten} der Funktionswerte für große x-Beträge fehlerfrei durch die \textbf{Symbolik} (z.B. \(f(x) \xrightarrow{x\rightarrow{}\infty}\infty\)) ausdrücken.\\
			& & & \\
			& \textit{WP } & \textit{WP } & \textit{WP } \\
			\hline
			\textbf{Nullstellen} & Ich weiß, dass ich bei Vorliegen einer \textbf{Faktorform} die \textbf{Nullstellen faktorweise berechnen} kann. & Ich kenne die \textbf{Bedingung} zur Anwendung der \textbf{pq-Formel} und kann sie fehlerfrei anwenden. & Ich kann erläutern, warum der \textbf{Grad} einer ganzrationale Funktion die \textbf{maximal mögliche Anzahl an Nullstellen angibt}.\\
		\end{tabularx}
		\begin{tabularx}{\textwidth}{|L{2.5cm}|X|X|X|}
			& Ich kenne den \textbf{Ansatz \(\mathbf{f(x) = 0}\)} zur Berechnung der \textbf{Nullstellen} und kann ihn erläutern & Ich kann bei einer in Polynomform gegebenen ganzrationalen Funktion \textbf{x so oft wie möglich ausklammern} und die \textbf{Nullstellen faktorweise berechnen}. &  Ich kann eine \textbf{Polynomdivision fehlerfrei durchführen} und die restlichen Nullstellen mit der pq-Formel berechnen.\\
			& Ich weiß, dass sich Funktionen mit ausreichend Nullstellen durch \(\mathbf{f(x) = (x-N_1)(x-N_2)\ldots(\ldots{}x\ldots)}\) angeben lässt. & Ich kenne den Ansatz \textbf{Polynomdivision}, weiß wann ich ihn anwenden muss und kann ihn erläutern. & Ist ein \textbf{Graph mit n Nullstellen gegeben}, kann ich eine \textbf{Funktion von Grad n aufstellen}, die diese Nullstellen hat und durch einen vorgegebenen Punkt geht.\\
			& & & \\
			& \textit{WP } & \textit{WP } & \textit{WP } \\
			\hline
		\end{tabularx}
	\end{worksheet}
\end{document}