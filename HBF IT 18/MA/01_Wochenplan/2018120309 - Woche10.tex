\documentclass[oneside,openany,headings=optiontotoc,11pt,numbers=noenddot]{scrreprt}

\usepackage[a4paper]{geometry}
\usepackage[utf8]{inputenc}
\usepackage[T1]{fontenc}
\usepackage{lmodern}
\usepackage[ngerman]{babel}
\usepackage{ngerman}

\usepackage[onehalfspacing]{setspace}

\usepackage{fancyhdr}
\usepackage{fancybox}

\usepackage{rotating}
\usepackage{varwidth}


\usepackage{pdflscape}
\usepackage{graphicx}
\usepackage{graphbox}
\graphicspath{
	{Pics/PDFs/}
	{Pics/JPGs/}
	{Pics/PNGs/}
}
\usepackage{caption}
\usepackage{tabularx}
\usepackage{dashrule}
\usepackage{hhline}
\usepackage{multirow}
\usepackage{enumerate}
\usepackage[hidelinks]{hyperref}
\usepackage{listings}

\usepackage[table]{xcolor}
\usepackage{array}
\usepackage{enumitem,amssymb,amsmath}
\usepackage{interval}
\usepackage{stmaryrd}
\usepackage{polynom}
\usepackage{diagbox}
\usepackage{dashrule}
\usepackage{framed}
\usepackage{mdframed}
\usepackage{karnaugh-map}

\usepackage{blindtext}

\usepackage{eso-pic}

\usepackage{amssymb}
\usepackage{eurosym}
\pagestyle{headings}
\renewcommand{\headrulewidth}{0.2pt}
\renewcommand{\footrulewidth}{0.2pt}
\newcommand*{\underdownarrow}[2]{\ensuremath{\underset{\overset{\Big\downarrow}{#2}}{#1}}}
\setlength{\fboxsep}{5pt}

% Codestyle defined
\definecolor{codegreen}{rgb}{0,0.6,0}
\definecolor{codegray}{rgb}{0.5,0.5,0.5}
\definecolor{codepurple}{rgb}{0.58,0,0.82}
\definecolor{backcolour}{rgb}{0.95,0.95,0.92}
\definecolor{deepgreen}{rgb}{0,0.5,0}
\definecolor{darkblue}{rgb}{0,0,0.65}
\definecolor{mauve}{rgb}{0.40, 0.19,0.28}
\colorlet{exceptioncolour}{yellow!50!red}
\colorlet{commandcolour}{blue!60!black}
\colorlet{numpycolour}{blue!60!green}
\colorlet{specmethodcolour}{violet}

%Neue Spaltendefinition
\newcolumntype{L}[1]{>{\raggedright\let\newline\\\arraybackslash\hspace{0pt}}m{#1}}
\newcolumntype{M}[1]{>{\centering\arraybackslash}X}
\newcommand{\cmnt}[1]{\ignorespaces}
%Textausrichtung ändern
\newcommand\tabrotate[1]{\rotatebox{90}{\raggedright#1\hspace{\tabcolsep}}}

%Intervall-Konfig
\intervalconfig {
	soft open fences
}

%Bash
\lstdefinestyle{BashInputStyle}{
	language=bash,
	basicstyle=\small\sffamily,
	backgroundcolor=\color{backcolour},
	columns=fullflexible,
	backgroundcolor=\color{backcolour},
	breaklines=true,
}
%Java
\lstdefinestyle{JavaInputStyle}{
	language=Java,
	backgroundcolor=\color{backcolour},
	aboveskip=1mm,
	belowskip=1mm,
	showstringspaces=false,
	columns=flexible,
	basicstyle={\footnotesize\ttfamily},
	numberstyle={\tiny},
	numbers=none,
	keywordstyle=\color{purple},,
	commentstyle=\color{deepgreen},
	stringstyle=\color{blue},
	emph={out},
	emphstyle=\color{darkblue},
	emph={[2]rand},
	emphstyle=[2]\color{specmethodcolour},
	breaklines=true,
	breakatwhitespace=true,
	tabsize=2,
}
%Python
\lstdefinestyle{PythonInputStyle}{
	language=Python,
	alsoletter={1234567890},
	aboveskip=1ex,
	basicstyle=\footnotesize,
	breaklines=true,
	breakatwhitespace= true,
	backgroundcolor=\color{backcolour},
	commentstyle=\color{red},
	otherkeywords={\ , \}, \{, \&,\|},
	emph={and,break,class,continue,def,yield,del,elif,else,%
		except,exec,finally,for,from,global,if,import,in,%
		lambda,not,or,pass,print,raise,return,try,while,assert},
	emphstyle=\color{exceptioncolour},
	emph={[2]True,False,None,min},
	emphstyle=[2]\color{specmethodcolour},
	emph={[3]object,type,isinstance,copy,deepcopy,zip,enumerate,reversed,list,len,dict,tuple,xrange,append,execfile,real,imag,reduce,str,repr},
	emphstyle=[3]\color{commandcolour},
	emph={[4]ode, fsolve, sqrt, exp, sin, cos, arccos, pi,  array, norm, solve, dot, arange, , isscalar, max, sum, flatten, shape, reshape, find, any, all, abs, plot, linspace, legend, quad, polyval,polyfit, hstack, concatenate,vstack,column_stack,empty,zeros,ones,rand,vander,grid,pcolor,eig,eigs,eigvals,svd,qr,tan,det,logspace,roll,mean,cumsum,cumprod,diff,vectorize,lstsq,cla,eye,xlabel,ylabel,squeeze},
	emphstyle=[4]\color{numpycolour},
	emph={[5]__init__,__add__,__mul__,__div__,__sub__,__call__,__getitem__,__setitem__,__eq__,__ne__,__nonzero__,__rmul__,__radd__,__repr__,__str__,__get__,__truediv__,__pow__,__name__,__future__,__all__},
	emphstyle=[5]\color{specmethodcolour},
	emph={[6]assert,range,yield},
	emphstyle=[6]\color{specmethodcolour}\bfseries,
	emph={[7]Exception,NameError,IndexError,SyntaxError,TypeError,ValueError,OverflowError,ZeroDivisionError,KeyboardInterrupt},
	emphstyle=[7]\color{specmethodcolour}\bfseries,
	emph={[8]taster,send,sendMail,capture,check,noMsg,go,move,switch,humTem,ventilate,buzz},
	emphstyle=[8]\color{blue},
	keywordstyle=\color{blue}\bfseries,
	rulecolor=\color{black!40},
	showstringspaces=false,
	stringstyle=\color{deepgreen}
}

\lstset{literate=%
	{Ö}{{\"O}}1
	{Ä}{{\"A}}1
	{Ü}{{\"U}}1
	{ß}{{\ss}}1
	{ü}{{\"u}}1
	{ä}{{\"a}}1
	{ö}{{\"o}}1
}

% Neue Klassenarbeits-Umgebung
\newenvironment{worksheet}[3]
% Begin-Bereich
{
	\newpage
	\sffamily
	\setcounter{page}{1}
	\ClearShipoutPicture
	\AddToShipoutPicture{
		\put(55,761){{
				\mbox{\parbox{385\unitlength}{\tiny \color{codegray}BBS I Mainz, #1 \newline #2
						\newline #3
					}
				}
			}
		}
		\put(455,761){{
				\mbox{\hspace{0.3cm}\includegraphics[width=0.2\textwidth]{../../logo.jpg}}
			}
		}
	}
}
% End-Bereich
{
	\clearpage
	\ClearShipoutPicture
}

\geometry{left=2.50cm,right=2.50cm,top=3.00cm,bottom=1.00cm,includeheadfoot}

\begin{document}
	\begin{worksheet}{Höhere Berufsfachschule IT-Systeme}{Grundstufe - Mathematik}{Wochenplan Ganzrationale Funktionen}
		\noindent
		\begin{tabularx}{\textwidth}{XXl}
			Wochenplan Nr.: \rule{0.15\textwidth}{1pt} & Erledigt: & Zeitraum: \underline{03.12 - 09.12}
		\end{tabularx}\\
		\par\noindent
		Die Aufgaben gliedern sich nach folgender Schwierigkeitsstufe.\\
		\begin{tabularx}{\textwidth}{XXX}
			(I) Grundlagen & (II) Forstgeschritten & (III) Experte
		\end{tabularx}\\
		\par\noindent
		\textbf{\underline{Pflicht}}: Sie bearbeiten \textit{pro Teil} jeweils \textbf{eine Aufgabe} vom Schwierigkeitsgrad ihrer Wahl.\\
		\underline{\textbf{Wahl}}: Zur Vertiefung und Festigung stehen ihnen die übrigen Aufgaben zur Verfügung.
		\begin{framed}
			\noindent
			\textbf{Teil 1:} \textbf{Geben} Sie zu \underline{einer} der nachfolgenden ganzrationalen Funktionen den \textit{größten gemeinsamen Teiler} aller Summanden an. \textbf{Klammern} Sie diesen aus.\\
			\par\noindent
			\begin{tabularx}{\textwidth}{M|M}
				(I) \(f(x) = 2x^4 + 27x^3 + 15x\) & (II) \(f(x) = 7x^6 - 2x^4 + 3x^3 + 6,5x^2\)\\
				\\
				\hline
				\multicolumn{2}{c}{}\\
				\multicolumn{2}{c}{(III) \(f(x) = \frac{1}{4}x^5 + 2x^3 - \frac{1}{2}x^2\)}
			\end{tabularx}
		\end{framed}
		\begin{framed}
			\noindent
			\textbf{Teil 2:} \textbf{Entscheiden} Sie welches Verfahren für die Bestimmung der Nullstellen der nachfolgenden ganzrationalen Funktion geeignet ist.\\
			\textbf{Benennen} Sie es und \textbf{begründen} Sie ihre Entscheidung.\\
			\par\noindent
			\begin{tabularx}{\textwidth}{M|M}
				\(f(x) = 3x^2 + 4x -5\) & \(f(x) = 12x^4 - 8x^2 + 3\)\\
				\\
				\hline
				\multicolumn{2}{c}{}\\
				\multicolumn{2}{c}{\(f(x) = \frac{1}{3}x^3 + 2x^2 - 4x + 3\)}
			\end{tabularx}
		\end{framed}
		\normalsize
		\begin{framed}
			\noindent
			\textbf{Teil 3:} \textbf{Ermitteln} Sie die Nullstellen \underline{einer} der ganzrationalen Funktion durch \textit{Substitution}.\\
			\par\noindent
			\begin{tabularx}{\textwidth}{M|M}
				(I) \(f(x) = 2x^4 -30,5x^2 + 112,5\) & (II) \(f(x) = -x^4 + 7x^2 -12\)\\
				\\
				\hline
				\multicolumn{2}{c}{}\\
				\multicolumn{2}{c}{(III) \(f(x) = -\frac{1}{4}x^4 + x^2 + \frac{5}{4}\)}
			\end{tabularx}
		\end{framed}
		\newpage
		\begin{framed}
			\noindent
			\textbf{Teil 4:} \textbf{Führen} Sie eine \textit{Polynomdivision} mit gegebenem Teiler durch.\\
			\par\noindent
			\begin{tabularx}{\textwidth}{XX}
				(I) \(f(x) = -2x^3-4x^2+2x+4\) & \((x+2)\)\\
				\hline
				\\
				(II) \(f(x) = 4x^3+6x^2-12x-5\) & \((x+\frac{5}{2})\)\\
				\hline
				\\
				(I) \(f(x) = x^4-2x^3-3x^2+10x-8\) & \((x-2)\)\\
			\end{tabularx}
		\end{framed}
		\begin{framed}
			\noindent
			\textbf{Teil 5:} \textbf{Untersuchen} Sie eine der nachfolgenden ganzrationalen Funktionen auf \textit{Nullstellen}.\\
			\par\noindent
			\begin{tabularx}{\textwidth}{M|M}
				(I) \(f(x) = x^3- 1,5x^2 - 8,5x + 15\) & (II) \(f(x) = 0,5x^4 + 2x^2-4\)\\
				\\
				\hline
				\multicolumn{2}{c}{}\\
				\multicolumn{2}{c}{(III) \(f(x) = x^3 -0,25x^2 -8,88x +8,75\)}
			\end{tabularx}
		\end{framed}
	\end{worksheet}
\end{document}