\documentclass[oneside,openany,headings=optiontotoc,11pt,numbers=noenddot]{scrreprt}

\usepackage[a4paper]{geometry}
\usepackage[utf8]{inputenc}
\usepackage[T1]{fontenc}
\usepackage{lmodern}
\usepackage[ngerman]{babel}
\usepackage{ngerman}

\usepackage[onehalfspacing]{setspace}

\usepackage{fancyhdr}
\usepackage{fancybox}

\usepackage{rotating}
\usepackage{varwidth}


\usepackage{pdflscape}
\usepackage{graphicx}
\usepackage{graphbox}
\graphicspath{
	{Pics/PDFs/}
	{Pics/JPGs/}
	{Pics/PNGs/}
}
\usepackage{caption}
\usepackage{tabularx}
\usepackage{dashrule}
\usepackage{hhline}
\usepackage{multirow}
\usepackage{enumerate}
\usepackage[hidelinks]{hyperref}
\usepackage{listings}

\usepackage[table]{xcolor}
\usepackage{array}
\usepackage{enumitem,amssymb,amsmath}
\usepackage{interval}
\usepackage{stmaryrd}
\usepackage{polynom}
\usepackage{diagbox}
\usepackage{dashrule}
\usepackage{framed}
\usepackage{mdframed}
\usepackage{karnaugh-map}

\usepackage{blindtext}

\usepackage{eso-pic}

\usepackage{amssymb}
\usepackage{eurosym}
\pagestyle{headings}
\renewcommand{\headrulewidth}{0.2pt}
\renewcommand{\footrulewidth}{0.2pt}
\newcommand*{\underdownarrow}[2]{\ensuremath{\underset{\overset{\Big\downarrow}{#2}}{#1}}}
\setlength{\fboxsep}{5pt}

% Codestyle defined
\definecolor{codegreen}{rgb}{0,0.6,0}
\definecolor{codegray}{rgb}{0.5,0.5,0.5}
\definecolor{codepurple}{rgb}{0.58,0,0.82}
\definecolor{backcolour}{rgb}{0.95,0.95,0.92}
\definecolor{deepgreen}{rgb}{0,0.5,0}
\definecolor{darkblue}{rgb}{0,0,0.65}
\definecolor{mauve}{rgb}{0.40, 0.19,0.28}
\colorlet{exceptioncolour}{yellow!50!red}
\colorlet{commandcolour}{blue!60!black}
\colorlet{numpycolour}{blue!60!green}
\colorlet{specmethodcolour}{violet}

%Neue Spaltendefinition
\newcolumntype{L}[1]{>{\raggedright\let\newline\\\arraybackslash\hspace{0pt}}m{#1}}
\newcolumntype{M}[1]{>{\centering\arraybackslash}X}
\newcommand{\cmnt}[1]{\ignorespaces}
%Textausrichtung ändern
\newcommand\tabrotate[1]{\rotatebox{90}{\raggedright#1\hspace{\tabcolsep}}}

%Intervall-Konfig
\intervalconfig {
	soft open fences
}

%Bash
\lstdefinestyle{BashInputStyle}{
	language=bash,
	basicstyle=\small\sffamily,
	backgroundcolor=\color{backcolour},
	columns=fullflexible,
	backgroundcolor=\color{backcolour},
	breaklines=true,
}
%Java
\lstdefinestyle{JavaInputStyle}{
	language=Java,
	backgroundcolor=\color{backcolour},
	aboveskip=1mm,
	belowskip=1mm,
	showstringspaces=false,
	columns=flexible,
	basicstyle={\footnotesize\ttfamily},
	numberstyle={\tiny},
	numbers=none,
	keywordstyle=\color{purple},,
	commentstyle=\color{deepgreen},
	stringstyle=\color{blue},
	emph={out},
	emphstyle=\color{darkblue},
	emph={[2]rand},
	emphstyle=[2]\color{specmethodcolour},
	breaklines=true,
	breakatwhitespace=true,
	tabsize=2,
}
%Python
\lstdefinestyle{PythonInputStyle}{
	language=Python,
	alsoletter={1234567890},
	aboveskip=1ex,
	basicstyle=\footnotesize,
	breaklines=true,
	breakatwhitespace= true,
	backgroundcolor=\color{backcolour},
	commentstyle=\color{red},
	otherkeywords={\ , \}, \{, \&,\|},
	emph={and,break,class,continue,def,yield,del,elif,else,%
		except,exec,finally,for,from,global,if,import,in,%
		lambda,not,or,pass,print,raise,return,try,while,assert},
	emphstyle=\color{exceptioncolour},
	emph={[2]True,False,None,min},
	emphstyle=[2]\color{specmethodcolour},
	emph={[3]object,type,isinstance,copy,deepcopy,zip,enumerate,reversed,list,len,dict,tuple,xrange,append,execfile,real,imag,reduce,str,repr},
	emphstyle=[3]\color{commandcolour},
	emph={[4]ode, fsolve, sqrt, exp, sin, cos, arccos, pi,  array, norm, solve, dot, arange, , isscalar, max, sum, flatten, shape, reshape, find, any, all, abs, plot, linspace, legend, quad, polyval,polyfit, hstack, concatenate,vstack,column_stack,empty,zeros,ones,rand,vander,grid,pcolor,eig,eigs,eigvals,svd,qr,tan,det,logspace,roll,mean,cumsum,cumprod,diff,vectorize,lstsq,cla,eye,xlabel,ylabel,squeeze},
	emphstyle=[4]\color{numpycolour},
	emph={[5]__init__,__add__,__mul__,__div__,__sub__,__call__,__getitem__,__setitem__,__eq__,__ne__,__nonzero__,__rmul__,__radd__,__repr__,__str__,__get__,__truediv__,__pow__,__name__,__future__,__all__},
	emphstyle=[5]\color{specmethodcolour},
	emph={[6]assert,range,yield},
	emphstyle=[6]\color{specmethodcolour}\bfseries,
	emph={[7]Exception,NameError,IndexError,SyntaxError,TypeError,ValueError,OverflowError,ZeroDivisionError,KeyboardInterrupt},
	emphstyle=[7]\color{specmethodcolour}\bfseries,
	emph={[8]taster,send,sendMail,capture,check,noMsg,go,move,switch,humTem,ventilate,buzz},
	emphstyle=[8]\color{blue},
	keywordstyle=\color{blue}\bfseries,
	rulecolor=\color{black!40},
	showstringspaces=false,
	stringstyle=\color{deepgreen}
}

\lstset{literate=%
	{Ö}{{\"O}}1
	{Ä}{{\"A}}1
	{Ü}{{\"U}}1
	{ß}{{\ss}}1
	{ü}{{\"u}}1
	{ä}{{\"a}}1
	{ö}{{\"o}}1
}

% Neue Klassenarbeits-Umgebung
\newenvironment{worksheet}[3]
% Begin-Bereich
{
	\newpage
	\sffamily
	\setcounter{page}{1}
	\ClearShipoutPicture
	\AddToShipoutPicture{
		\put(55,761){{
				\mbox{\parbox{385\unitlength}{\tiny \color{codegray}BBS I Mainz, #1 \newline #2
						\newline #3
					}
				}
			}
		}
		\put(455,761){{
				\mbox{\hspace{0.3cm}\includegraphics[width=0.2\textwidth]{../../logo.jpg}}
			}
		}
	}
}
% End-Bereich
{
	\clearpage
	\ClearShipoutPicture
}

\geometry{left=2.50cm,right=2.50cm,top=3.00cm,bottom=1.00cm,includeheadfoot}

\begin{document}
	\begin{worksheet}{Höhere Berufsfachschule IT-Systeme}{Grundstufe - Mathematik}{Wochenplan Grundlagen - Lösungen}
		\noindent
		\begin{tabularx}{\textwidth}{XXl}
			Wochenplan Nr.: \rule{0.15\textwidth}{1pt} & Erledigt: & Zeitraum: \underline{13.08 - 17.08}
		\end{tabularx}
	
		\begin{framed}
			\noindent
			\textbf{Montag:} Lösen Sie die Klammern auf und vereinfachen Sie soweit wie möglich:\\
			\begin{tabularx}{\textwidth}{lX}
				(a) & \(5(a+b+c)\)\\
				& \(= 5a + 5b + 5c\) keine Vereinfachung möglich\\\\
				(b) & \((6x-5y+9z)\cdot{}(-2x)\)\\
				& \(= (-2x)\cdot{}6x (-2x)\cdot(-5y) + (-2x)\cdot{}9z\)\\
				& \(= -12x^2 +10xy -18xz\) keine Vereinfachung möglich\\\\
				(c) & \(5(3a+4b)+2(a-b)-3(2a-3b)\)\\
				& \(= 5\cdot{}3a + 5\cdot{}4b +2a -2b - (3\cdot{}2a 3\cdot(-3b))\)\\
				& \(= 15a +20b +2a -2b -6a \underbrace{+9b}_{- (-9b)}\)\\
				& \(= 15a + 2a - 6a +20b + 2b + 9b\)\\
				& \(= 11a + 31b\)\\\\
				(d) & \(-3(5a+2c) + 4(-3a+b)^2\)\\
				& \(= (-3)\cdot{}5a + (-3)\cdot{}2c + 4(\underbrace{9a^2-6ab+b^2}_{(-3a+b)(-3a+b)})\)\\
				& \(= -15a -6c + 4\cdot{}9a^2 + 4\cdot{}(-6ab) +4\cdot{}b^2\)\\
				& \(= -15a -6c + 36a^2 - 24ab + 4b^2\) keine Vereinfachung möglich
			\end{tabularx}
		\end{framed}
		\begin{framed}
			\noindent
			\textbf{Dienstag:} Berechnen Sie mit Hilfe der binomischen Formeln:\\
			\begin{tabularx}{\textwidth}{lX}
				(a) & \((x+y)^2\)\\
				& \((x+y)^2 = (x+y)\cdot(x+y)\)\\
				& \(= x^2 +xy + yx + y^2 = x^2 +2xy + y^2\)\\\\
				(b) & \((x+1)^2\)\\
				& \(= (x+1)\cdot(x+1) = x^2 +x + x + 1 = x^2 +2x +1\)\\\\
				(c) & \((2x-y)^2\)\\
				& \(= (2x-y)\cdot(2x-y) = 4x^2 -2xy - y2x + y^2 = 4x^2 -4xy + y^2\)\\\\
				(d) & \((x-y)(x+y)\)\\
				& \(= x^2 +xy -yx -y^2 = x^2 -y^2\)
			\end{tabularx}
		\end{framed}
		\begin{framed}
			\noindent
			\textbf{Mittwoch:} Faktorisieren Sie die folgenden Ausdrücke:\\
			\begin{tabularx}{\textwidth}{lXlX}
				(a) & \(16a^2 +20ab\) & (b) & \(ab +ab^2+a^2b\)\\
				& \(= 4a(4a +5b)\) & & \(= ab(1+b+a)\)\\\\
				(c) & \(12x^2-12y^2\) & (d) & \(3a^2+6a+3\)\\
				& \(= 12(x^2 - y^2)\) & & \(= 3(a^2 +2a+1)\)
			\end{tabularx}
		\end{framed}
		\begin{framed}
			\noindent
			\textbf{Donnerstag:} Löse Sie die folgende Gleichung nach x auf:\\
			\begin{tabularx}{\textwidth}{lll}
				(a) & \(5x+4=3x+10\) & |\(-4\)\\
				& \(5x = 3x + 7\) & |\(-3x\)\\
				& \(2x = 7\) & |\(:2\)\\\\
				& \(x = 3,5\)\\
				(b) & \(2(x-1)=3(2-x)\) & | AM\\
				& \(2x - 2 = 6 -3x \) & |\(+3x\)\\
				& \(5x -2 = 6\) & |\(+2\)\\
				& \(5x = 8\) & |\(:5\)\\
				& \(x = \frac{8}{5}\)\\\\
				(c) & \(2x+4 = 2-4x\) & |\(+4x\)\\
				& \(6x+4 = 2\) & |\(-4\)\\
				& \(2x = -2\) & |\(:6\)\\
				& \(x = -\frac{\cancelto{1}{2}}{\cancelto{3}{6}}\)\\\\
				(d) & \(x-10 = 4x+20\) & |\(-4x\)\\
				& \(-3x -10 = 20\) & |\(+10\)\\
				& \(-3x = 30\) & |\(:(-3)\)\\
				& \(x = -10\)\\\\
				(e) & \(-(5x-3) =-(-x+1)\) |MK\\
				& \(-5x+3 = x-1\) & |\(-x\)\\
				& \(-6x + 3 = -1\) & |\(-3\)\\
				& \(-6x = -4\) & |\(:(-6)\)\\
				& \(x = \frac{\cancelto{2}{4}}{\cancelto{3}{6}}\)\\\\
			\end{tabularx}
			\begin{tabularx}{\textwidth}{lll}
				(f) & \(\frac{1}{2}(x-1) = \frac{1}{4}(2x+12)\) & |AM\\
				& \(\frac{1\cdot{}x}{2} - \frac{1}{2} = \frac{1\cdot{}\cancelto{1}{2}x}{\cancelto{2}{4}} + \frac{\cancelto{3}{12}}{\cancelto{1}{4}}\)\\
				& \(\frac{x}{2} - \frac{1}{2} = \frac{x}{2} + 3\) & |\(+\frac{1}{2}\)\\
				& \(\frac{x}{2} = \frac{x}{2} + 3\frac{1}{2}\) & |\(-\frac{x}{2}\)\\
				& \(0 = 3\frac{1}{2}\) & \(\Rightarrow \lightning\) Also nicht lösbar.
			\end{tabularx}
		\end{framed}
		\begin{framed}
			\noindent
			\textbf{Freitag:} Kürzen Sie die folgenden Brüche soweit wie möglich:\\
			\begin{tabularx}{\textwidth}{lX}
				(a) & \(\frac{2a+2ab}{2a^2b}\)\\\\
				& \(= \frac{\cancel{2a}}{\cancel{2}a^{\cancel{2}}b} + \frac{\cancel{2ab}}{\cancel{2}a^{\cancel{2}}\cancel{b}} = \frac{1}{ab} + \frac{1}{b}\)\\\\
				(b) & \(\frac{24ab + 36ab^2}{12a^2b}\)\\\\
				& \(= \frac{24ab}{12a^2b} + \frac{36ab^2}{12a^2b}\)\\\\
				& \(= \frac{\cancelto{2}{24}\cancel{ab}}{\cancel{12}a^{\cancel{2}}\cancel{b}} + \frac{\cancelto{3}{36}\cancel{a}b^{\cancel{2}}}{\cancel{12}a^{\cancel{2}}\cancel{b}}\)\\\\
				& \(= \frac{2}{a} + \frac{3b}{a}\)\\\\
				
				(c) & \(\frac{x^2-1}{x+1} = \frac{(x-1)(x+1)}{x+1} = x-1\)
			\end{tabularx}
		\end{framed}
	\end{worksheet}
\end{document}