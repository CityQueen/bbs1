\documentclass[oneside,openany,headings=optiontotoc,11pt,numbers=noenddot]{scrreprt}

\usepackage[a4paper]{geometry}
\usepackage[utf8]{inputenc}
\usepackage[T1]{fontenc}
\usepackage{lmodern}
\usepackage[ngerman]{babel}
\usepackage{ngerman}

\usepackage[onehalfspacing]{setspace}

\usepackage{fancyhdr}
\usepackage{fancybox}

\usepackage{rotating}
\usepackage{varwidth}


\usepackage{pdflscape}
\usepackage{graphicx}
\usepackage{graphbox}
\graphicspath{
	{Pics/PDFs/}
	{Pics/JPGs/}
	{Pics/PNGs/}
}
\usepackage{caption}
\usepackage{tabularx}
\usepackage{dashrule}
\usepackage{hhline}
\usepackage{multirow}
\usepackage{enumerate}
\usepackage[hidelinks]{hyperref}
\usepackage{listings}

\usepackage[table]{xcolor}
\usepackage{array}
\usepackage{enumitem,amssymb,amsmath}
\usepackage{interval}
\usepackage{stmaryrd}
\usepackage{polynom}
\usepackage{diagbox}
\usepackage{dashrule}
\usepackage{framed}
\usepackage{mdframed}
\usepackage{karnaugh-map}

\usepackage{blindtext}

\usepackage{eso-pic}

\usepackage{amssymb}
\usepackage{eurosym}
\pagestyle{headings}
\renewcommand{\headrulewidth}{0.2pt}
\renewcommand{\footrulewidth}{0.2pt}
\newcommand*{\underdownarrow}[2]{\ensuremath{\underset{\overset{\Big\downarrow}{#2}}{#1}}}
\setlength{\fboxsep}{5pt}

% Codestyle defined
\definecolor{codegreen}{rgb}{0,0.6,0}
\definecolor{codegray}{rgb}{0.5,0.5,0.5}
\definecolor{codepurple}{rgb}{0.58,0,0.82}
\definecolor{backcolour}{rgb}{0.95,0.95,0.92}
\definecolor{deepgreen}{rgb}{0,0.5,0}
\definecolor{darkblue}{rgb}{0,0,0.65}
\definecolor{mauve}{rgb}{0.40, 0.19,0.28}
\colorlet{exceptioncolour}{yellow!50!red}
\colorlet{commandcolour}{blue!60!black}
\colorlet{numpycolour}{blue!60!green}
\colorlet{specmethodcolour}{violet}

%Neue Spaltendefinition
\newcolumntype{L}[1]{>{\raggedright\let\newline\\\arraybackslash\hspace{0pt}}m{#1}}
\newcolumntype{M}[1]{>{\centering\arraybackslash}X}
\newcommand{\cmnt}[1]{\ignorespaces}
%Textausrichtung ändern
\newcommand\tabrotate[1]{\rotatebox{90}{\raggedright#1\hspace{\tabcolsep}}}

%Intervall-Konfig
\intervalconfig {
	soft open fences
}

%Bash
\lstdefinestyle{BashInputStyle}{
	language=bash,
	basicstyle=\small\sffamily,
	backgroundcolor=\color{backcolour},
	columns=fullflexible,
	backgroundcolor=\color{backcolour},
	breaklines=true,
}
%Java
\lstdefinestyle{JavaInputStyle}{
	language=Java,
	backgroundcolor=\color{backcolour},
	aboveskip=1mm,
	belowskip=1mm,
	showstringspaces=false,
	columns=flexible,
	basicstyle={\footnotesize\ttfamily},
	numberstyle={\tiny},
	numbers=none,
	keywordstyle=\color{purple},,
	commentstyle=\color{deepgreen},
	stringstyle=\color{blue},
	emph={out},
	emphstyle=\color{darkblue},
	emph={[2]rand},
	emphstyle=[2]\color{specmethodcolour},
	breaklines=true,
	breakatwhitespace=true,
	tabsize=2,
}
%Python
\lstdefinestyle{PythonInputStyle}{
	language=Python,
	alsoletter={1234567890},
	aboveskip=1ex,
	basicstyle=\footnotesize,
	breaklines=true,
	breakatwhitespace= true,
	backgroundcolor=\color{backcolour},
	commentstyle=\color{red},
	otherkeywords={\ , \}, \{, \&,\|},
	emph={and,break,class,continue,def,yield,del,elif,else,%
		except,exec,finally,for,from,global,if,import,in,%
		lambda,not,or,pass,print,raise,return,try,while,assert},
	emphstyle=\color{exceptioncolour},
	emph={[2]True,False,None,min},
	emphstyle=[2]\color{specmethodcolour},
	emph={[3]object,type,isinstance,copy,deepcopy,zip,enumerate,reversed,list,len,dict,tuple,xrange,append,execfile,real,imag,reduce,str,repr},
	emphstyle=[3]\color{commandcolour},
	emph={[4]ode, fsolve, sqrt, exp, sin, cos, arccos, pi,  array, norm, solve, dot, arange, , isscalar, max, sum, flatten, shape, reshape, find, any, all, abs, plot, linspace, legend, quad, polyval,polyfit, hstack, concatenate,vstack,column_stack,empty,zeros,ones,rand,vander,grid,pcolor,eig,eigs,eigvals,svd,qr,tan,det,logspace,roll,mean,cumsum,cumprod,diff,vectorize,lstsq,cla,eye,xlabel,ylabel,squeeze},
	emphstyle=[4]\color{numpycolour},
	emph={[5]__init__,__add__,__mul__,__div__,__sub__,__call__,__getitem__,__setitem__,__eq__,__ne__,__nonzero__,__rmul__,__radd__,__repr__,__str__,__get__,__truediv__,__pow__,__name__,__future__,__all__},
	emphstyle=[5]\color{specmethodcolour},
	emph={[6]assert,range,yield},
	emphstyle=[6]\color{specmethodcolour}\bfseries,
	emph={[7]Exception,NameError,IndexError,SyntaxError,TypeError,ValueError,OverflowError,ZeroDivisionError,KeyboardInterrupt},
	emphstyle=[7]\color{specmethodcolour}\bfseries,
	emph={[8]taster,send,sendMail,capture,check,noMsg,go,move,switch,humTem,ventilate,buzz},
	emphstyle=[8]\color{blue},
	keywordstyle=\color{blue}\bfseries,
	rulecolor=\color{black!40},
	showstringspaces=false,
	stringstyle=\color{deepgreen}
}

\lstset{literate=%
	{Ö}{{\"O}}1
	{Ä}{{\"A}}1
	{Ü}{{\"U}}1
	{ß}{{\ss}}1
	{ü}{{\"u}}1
	{ä}{{\"a}}1
	{ö}{{\"o}}1
}

% Neue Klassenarbeits-Umgebung
\newenvironment{worksheet}[3]
% Begin-Bereich
{
	\newpage
	\sffamily
	\setcounter{page}{1}
	\ClearShipoutPicture
	\AddToShipoutPicture{
		\put(55,761){{
				\mbox{\parbox{385\unitlength}{\tiny \color{codegray}BBS I Mainz, #1 \newline #2
						\newline #3
					}
				}
			}
		}
		\put(455,761){{
				\mbox{\hspace{0.3cm}\includegraphics[width=0.2\textwidth]{../../logo.jpg}}
			}
		}
	}
}
% End-Bereich
{
	\clearpage
	\ClearShipoutPicture
}

\geometry{left=2.50cm,right=2.50cm,top=3.00cm,bottom=1.00cm,includeheadfoot}

\begin{document}
	\begin{worksheet}{Höhere Berufsfachschule IT-Systeme}{Grundstufe - Mathematik}{Wochenplan Grundlagen - Lösungen}
		\noindent
		\begin{framed}
			\noindent
			\textbf{Montag:} Bestimmen Sie die Lösungen:\\
			\begin{tabularx}{\textwidth}{ll}
				(a) \(\frac{1}{3}y -5 = -\frac{1}{3}y + 3\) & |\(+\frac{1}{3}y\)\\
				\(\frac{2}{3}y-5 = 3\) & |\(+5\)\\
				\(\frac{2}{3}y = 8\) & |\(:\frac{2}{3}\)\\
				\(y = 8:\frac{2}{3} = 8\cdot\frac{3}{2} = 12\)\\
				\\				
				(b) \(12 +5 = 3\cdot{}(z-8)\) & | AM\\
				\(17 = 3z - 24\) & |\(+24\)\\
				\(41 = 3z\) & |\(:3\)
				\(\frac{41}{3}\)\\
				\\
				(c) \(\frac{2}{5}+(-\frac{1}{5}z) + \frac{3}{5} = 9\) & | AM\\
				\(\frac{2}{5}+\frac{3}{5} - \frac{1}{5}z = 9\) & |\(+\frac{1}{5}z\)\\
				\(1 = 9 + \frac{1}{5}z\) & |\(-9\)\\
				\(-8 = \frac{1}{5}z\) & |\(:\frac{1}{5}\)\\
				\((-8):\frac{1}{5} = (-8)\cdot{}5 = -40 = z\)\\
				\\
				(d) \(3x -(-2x+15) = -35x\) & | AM\\
				\(3x +2x -15 = 5x-15 = -35x\) & |\(+15\)\\
				\(5x = -35x +15\) & |\(+35x\)\\
				\(40x = 15\) & |\(:40\)\\
				\(x = \frac{\cancelto{3}{15}}{\cancelto{8}{40}} = \frac{3}{8}\)
			\end{tabularx}
		\end{framed}
		\begin{framed}
			\noindent
			\textbf{Dienstag:} Berechnen Sie die Lösung mit Hilfe der binomischen Formeln:\\
			\begin{tabularx}{\textwidth}{ll}
				(a) \((x+1)^2 = x^2 + 10\)\\
				\(x^2 + 2x + 1 = x^2 +10\) & |\(-x^2\)\\
				\(2x+1 = 10\) & |\(-1\)\\
				\(2x = 9\) & |\(:2\)\\
				\(x = \frac{9}{2}\)\\
				\\
				(b) \((2x-5)^2 = 4x^2 -20\)\\
				\(4x^2 -20x +25 = 4x^2 -20\) &|\(-4x^2\)\\
				\(-20x+25 = -20\) & |\(-25\)\\
				\(-20x = -45\) & |\(:(-20)\)\\
				\(x = \frac{\cancelto{9}{45}}{\cancelto{4}{20}} = \frac{9}{4}\)\\
				\\
			\end{tabularx}
			\begin{tabularx}{\textwidth}{ll}
				(c) \((\frac{1}{2} + 2)^2 = \frac{1}{4}x^2 +16\)\\
				\((\frac{5}{2})^2 = \frac{25}{4} = \frac{1}{4}x^2 + 16\) & |\(-16\)\\
				\(\frac{25}{4} - \frac{64}{4} = -\frac{39}{4} = \frac{1}{4}x^2\) & |\(:\frac{1}{4}\)\\
				\(-\frac{39}{4}:\frac{1}{4} = -\frac{39}{4}\cdot{}4 = -39 = x^2\)\\
				\(-39 = x^2\) & \(\lightning\)\\
				\\
				(d) \((3x-6)^2 + x^2 = 5x^2 +2 +5x^2\)\\
				\(9x^2 -36x +36 + x^2 = 10x^2 + 2\)\\
				\(10x^2 -36x +36 = 10x^2 + 2\) & |\(-10x^2\)\\
				\(-36x + 36 = 2\) & |\(+36x\)\\
				\(36 = 36x + 2\) & |\(-2\)\\
				\(34 = 36x\) & |\(:36\)\\
				\(\frac{\cancelto{17}{34}}{\cancelto{18}{36}} = \frac{17}{18} = x\)\\
			\end{tabularx}
		\end{framed}
		\begin{framed}
			\noindent
			\textbf{Mittwoch:} Lösen Sie die Ungleichungen. Geben Sie die Lösungsmenge an!\\
			\begin{tabularx}{\textwidth}{ll}
				(a) \(2x -14 > 22\) & |\(+14\)\\
				\(2x > 36\) & \(:2\)\\
				\(x > 13\)\\
				\(\mathbb{L} = \{x>13\}\)\\
				\\
				(b) \(1,5x-9 < 7,5\) & |\(+9\)\\
				\(1,5x < 16,5\) & |\(:1,5\)\\
				\(x < 11\)\\
				\(\mathbb{L} = \{x<11\}\)\\
				\\
				(c) \((2x-1)(2x+5) > (-x-1)(-4x+6)\) & | AM\\
				\(4x^2 + 8x -5 > 4x^2 -2x -6\) & |\(-4x^2\)\\
				\(8x -5 > -2x-6\) & |\(+2x\)\\
				\(10x -5 > -6\) & |\(+5\)\\
				\(10x > -1\) & |\(:10\)\\
				\(x > -\frac{1}{10}\)\\
				\(\mathbb{L} = \{x > -\frac{1}{10}\}\)\\
				\\
			\end{tabularx}
			\begin{tabularx}{\textwidth}{ll}
				(d) \(12-(3x+2) < x-6\) & | AM\\
				\(12 -3x -2 = 10 -3x < x-6\) & |\(+3x\)\\
				\(10 < 4x -6\) & \(+6\)\\
				\(16 < 4x\) & \(:4\)\\
				\(4 < x\)\\
				\(\mathbb{L} = \{4 < x\}\)\\
			\end{tabularx}
		\end{framed}
		\begin{framed}
			\noindent
			\textbf{Donnerstag:} Geben Sie für die folgenden Gleichungen die Lösungsmenge an:\\
			\begin{tabularx}{\textwidth}{Xl|Xl}
				(a) \(3x+5 = 7x - 5\) & |\(-3x\) & (b) \(-x = -6x + \frac{25}{2}\) & |\(+6x\)\\
				\(5 = 10x -5\) & |\(+5\) & \(5x = \frac{25}{2}\) & |\(:5\)\\
				\(10 = 10x\) & |\(:10\) & \(x = \frac{\cancelto{5}{25}}{2\cdot\cancel{5}}\)\\
				\(x = 1\) & & \(x = \frac{5}{2}\) & \\
				\(\mathbb{L} = \{1\}\) & & \(\mathbb{L} = \{\frac{5}{2}\}\)\\
				\hline
				\hline
				\\
				(c) \(\frac{1}{2}x = 2,5x-60\) & |\(+60\) & (d) \(5\cdot(2x-4) = 26\) & | AM\\
				\(\frac{1}{2}x +60 = 2,5x\) & |\(-\frac{1}{2}x\) & \(10x -20 = 26\) & |\(+20\)\\
				\(60 = 2x\) & |\(:2\) & \(10x = 46\) & |\(:10\)\\
				\(30 = x\) & & \(x = \frac{\cancelto{23}{46}}{\cancelto{5}{10}}\) & \\
				\(\mathbb{L} = \{30\}\) & & \(\mathbb{L} = \{\frac{23}{5}\}\)\\
				\hline
				\hline
				\\
				(e) \(-3\cdot(x+15) = 5 +2x\) & | AM & (f) \(-\frac{1}{3}(x-1) = \frac{1}{6}(2x+12)\) & | AM\\
				\(-3x -35 = 5 +2x\) & |\(+3x\) & \(-\frac{1}{3}x + \frac{1}{3} = \frac{1}{3}x + 2\) & |\(+\frac{1}{3}x\)\\
				\(-35 = 5 + 5x\) & |\(-5\) & \(\frac{1}{3} = \frac{2}{3}x + 2\) & |\(-2\)\\
				\(-40 = 5x\) & |\(:5\) & \( -\frac{5}{3} = \frac{2}{3}x\) &|\(:\frac{2}{3}\)\\
				\(-8 = x\) & & \(-\frac{5}{2} = x\)\\
				\(\mathbb{L} = \{-8\}\) & & \(\mathbb{L} = \{-\frac{5}{2}\}\)
			\end{tabularx}
		\end{framed}
		\newpage
		\begin{framed}
			\noindent
			\textbf{Freitag:} Bestimmen Sie die Lösungsmenge der Gleichungen bzw. der Ungleichungen:\\
			\begin{tabularx}{\textwidth}{XlXl}
				(a) \(\frac{1}{3}x + 6 + \frac{5}{3}x -5 = 0\) & & (b) \(-6x-3 < 4x+7\) & |\(+6x\)\\
				\(2x +1 = 0\) & |\(-1\) & \(-3 < 10x +7\) & |\(-7\)\\
				\(2x = -1\) & |\(:2\) & \(-10 < 10x\) & |\(:10\)\\
				\( x = -\frac{1}{2}\) & & \(-1 < x\) & \\
				\(\mathbb{L} = \{-\frac{1}{2}\}\) & & \(\mathbb{L} = \{-1 < x\}\)\\
				\hline
				\hline
				\\
				(c) \((-2x-2)(3x-5) > -6x\cdot(x+3)\) & | AM & (d) \(-\frac{2}{3}x + \frac{1}{4} = (\frac{4}{3}x - \frac{2}{4})\) & | \(+\frac{2}{3}x\)\\
				\(-6x^2 + 4x +10 > -6x^2 - 18x\) & |\(+6x^2\) & \(\frac{1}{4} = 2x -\frac{2}{4}\) & |\(+\frac{2}{4}\)\\
				\(4x +10 > -18x\) & |\(+18x\) & \(\frac{3}{4} = 2x\) & |\(:2\)\\
				\(22x + 10 > 0\) & |\(-10\) & \(\frac{3}{8} = x\)\\
				\(22x > -10\) & \(:22\) & \(\mathbb{L} = \{\frac{3}{8}\}\)\\
				\(x > -\frac{\cancelto{5}{10}}{\cancelto{11}{22}}\)\\
				\(\mathbb{L} = \{x> -\frac{5}{11}\}\)\\
				\hline
				\hline
				\\
				(e) \(-(\frac{5}{2}x+3) = \frac{5}{2}x +6\) & |\(+\frac{5}{2}x\) & (f) \((-4x+4)(3x - 3) > (2x-5)(-6x +3)\) & | AM\\
				\(3 = 5x + 6\) & |\(-6\) & \(-12x^2 +24x - 12 > -12x^2 + 36x - 15\) & |\(+12x^2\)\\
				\(-3 = 5x\) & |\(:5\) & \(24x -12 > 36x -15\) & |\(-24x\)\\
				\(-\frac{3}{5} = x\) & & \(-12 > 12x - 15\) & |\(+15\)\\
				\(\mathbb{L} = \{-\frac{3}{5}\}\) & & \(3 > 12x\) & |\(:12\)\\
				& & \(\frac{1}{4} > x\)\\
				& & \(\mathbb{L} = \{\frac{1}{4} > x\}\)
				
			\end{tabularx}
		\end{framed}
	\end{worksheet}
\end{document}