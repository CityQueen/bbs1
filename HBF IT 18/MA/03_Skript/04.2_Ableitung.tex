\documentclass[11pt,twocolumn,oneside,openany,headings=optiontotoc,11pt,numbers=noenddot]{article}

\usepackage[a4paper]{geometry}
\usepackage[utf8]{inputenc}
\usepackage[T1]{fontenc}
\usepackage{lmodern}
\usepackage[ngerman]{babel}
\usepackage{ngerman}

\usepackage[onehalfspacing]{setspace}

\usepackage{fancyhdr}
\usepackage{fancybox}

\usepackage{rotating}
\usepackage{varwidth}


\usepackage{pdflscape}
\usepackage{graphicx}
\usepackage{graphbox}
\graphicspath{
	{Pics/PDFs/}
	{Pics/JPGs/}
	{Pics/PNGs/}
}
\usepackage{caption}
\usepackage{tabularx}
\usepackage{dashrule}
\usepackage{hhline}
\usepackage{multirow}
\usepackage{enumerate}
\usepackage[hidelinks]{hyperref}
\usepackage{listings}

\usepackage[table]{xcolor}
\usepackage{array}
\usepackage{enumitem,amssymb,amsmath}
\usepackage{interval}
\usepackage{stmaryrd}
\usepackage{polynom}
\usepackage{diagbox}
\usepackage{dashrule}
\usepackage{framed}
\usepackage{mdframed}
\usepackage{karnaugh-map}

\usepackage{blindtext}

\usepackage{eso-pic}

\usepackage{amssymb}
\usepackage{eurosym}
\pagestyle{headings}
\renewcommand{\headrulewidth}{0.2pt}
\renewcommand{\footrulewidth}{0.2pt}
\newcommand*{\underdownarrow}[2]{\ensuremath{\underset{\overset{\Big\downarrow}{#2}}{#1}}}
\setlength{\fboxsep}{5pt}

% Codestyle defined
\definecolor{codegreen}{rgb}{0,0.6,0}
\definecolor{codegray}{rgb}{0.5,0.5,0.5}
\definecolor{codepurple}{rgb}{0.58,0,0.82}
\definecolor{backcolour}{rgb}{0.95,0.95,0.92}
\definecolor{deepgreen}{rgb}{0,0.5,0}
\definecolor{darkblue}{rgb}{0,0,0.65}
\definecolor{mauve}{rgb}{0.40, 0.19,0.28}
\colorlet{exceptioncolour}{yellow!50!red}
\colorlet{commandcolour}{blue!60!black}
\colorlet{numpycolour}{blue!60!green}
\colorlet{specmethodcolour}{violet}

%Neue Spaltendefinition
\newcolumntype{L}[1]{>{\raggedright\let\newline\\\arraybackslash\hspace{0pt}}m{#1}}
\newcolumntype{M}[1]{>{\centering\arraybackslash}X}
\newcommand{\cmnt}[1]{\ignorespaces}
%Textausrichtung ändern
\newcommand\tabrotate[1]{\rotatebox{90}{\raggedright#1\hspace{\tabcolsep}}}

%Intervall-Konfig
\intervalconfig {
	soft open fences
}

%Bash
\lstdefinestyle{BashInputStyle}{
	language=bash,
	basicstyle=\small\sffamily,
	backgroundcolor=\color{backcolour},
	columns=fullflexible,
	backgroundcolor=\color{backcolour},
	breaklines=true,
}
%Java
\lstdefinestyle{JavaInputStyle}{
	language=Java,
	backgroundcolor=\color{backcolour},
	aboveskip=1mm,
	belowskip=1mm,
	showstringspaces=false,
	columns=flexible,
	basicstyle={\footnotesize\ttfamily},
	numberstyle={\tiny},
	numbers=none,
	keywordstyle=\color{purple},,
	commentstyle=\color{deepgreen},
	stringstyle=\color{blue},
	emph={out},
	emphstyle=\color{darkblue},
	emph={[2]rand},
	emphstyle=[2]\color{specmethodcolour},
	breaklines=true,
	breakatwhitespace=true,
	tabsize=2,
}
%Python
\lstdefinestyle{PythonInputStyle}{
	language=Python,
	alsoletter={1234567890},
	aboveskip=1ex,
	basicstyle=\footnotesize,
	breaklines=true,
	breakatwhitespace= true,
	backgroundcolor=\color{backcolour},
	commentstyle=\color{red},
	otherkeywords={\ , \}, \{, \&,\|},
	emph={and,break,class,continue,def,yield,del,elif,else,%
		except,exec,finally,for,from,global,if,import,in,%
		lambda,not,or,pass,print,raise,return,try,while,assert},
	emphstyle=\color{exceptioncolour},
	emph={[2]True,False,None,min},
	emphstyle=[2]\color{specmethodcolour},
	emph={[3]object,type,isinstance,copy,deepcopy,zip,enumerate,reversed,list,len,dict,tuple,xrange,append,execfile,real,imag,reduce,str,repr},
	emphstyle=[3]\color{commandcolour},
	emph={[4]ode, fsolve, sqrt, exp, sin, cos, arccos, pi,  array, norm, solve, dot, arange, , isscalar, max, sum, flatten, shape, reshape, find, any, all, abs, plot, linspace, legend, quad, polyval,polyfit, hstack, concatenate,vstack,column_stack,empty,zeros,ones,rand,vander,grid,pcolor,eig,eigs,eigvals,svd,qr,tan,det,logspace,roll,mean,cumsum,cumprod,diff,vectorize,lstsq,cla,eye,xlabel,ylabel,squeeze},
	emphstyle=[4]\color{numpycolour},
	emph={[5]__init__,__add__,__mul__,__div__,__sub__,__call__,__getitem__,__setitem__,__eq__,__ne__,__nonzero__,__rmul__,__radd__,__repr__,__str__,__get__,__truediv__,__pow__,__name__,__future__,__all__},
	emphstyle=[5]\color{specmethodcolour},
	emph={[6]assert,range,yield},
	emphstyle=[6]\color{specmethodcolour}\bfseries,
	emph={[7]Exception,NameError,IndexError,SyntaxError,TypeError,ValueError,OverflowError,ZeroDivisionError,KeyboardInterrupt},
	emphstyle=[7]\color{specmethodcolour}\bfseries,
	emph={[8]taster,send,sendMail,capture,check,noMsg,go,move,switch,humTem,ventilate,buzz},
	emphstyle=[8]\color{blue},
	keywordstyle=\color{blue}\bfseries,
	rulecolor=\color{black!40},
	showstringspaces=false,
	stringstyle=\color{deepgreen}
}

\lstset{literate=%
	{Ö}{{\"O}}1
	{Ä}{{\"A}}1
	{Ü}{{\"U}}1
	{ß}{{\ss}}1
	{ü}{{\"u}}1
	{ä}{{\"a}}1
	{ö}{{\"o}}1
}

% Neue Klassenarbeits-Umgebung
\newenvironment{worksheet}[3]
% Begin-Bereich
{
	\newpage
	\sffamily
	\setcounter{page}{1}
	\ClearShipoutPicture
	\AddToShipoutPicture{
		\put(55,761){{
				\mbox{\parbox{385\unitlength}{\tiny \color{codegray}BBS I Mainz, #1 \newline #2
						\newline #3
					}
				}
			}
		}
		\put(455,761){{
				\mbox{\hspace{0.3cm}\includegraphics[width=0.2\textwidth]{../../logo.jpg}}
			}
		}
	}
}
% End-Bereich
{
	\clearpage
	\ClearShipoutPicture
}

\setlength{\columnsep}{3em}
\setlength{\columnseprule}{0.5pt}

\geometry{left=2.50cm,right=2.50cm,top=3.00cm,bottom=1.00cm,includeheadfoot}
\pagenumbering{gobble}
\pagestyle{empty}

\begin{document}
	\begin{worksheet}{BBS I, Höhere Berufsfachschule IT-Systeme}{Grundstufe - Mathematik}{Lernabschnitt 4: Differenzialrechnung - Die Ableitungsfunktion}
		\setcounter{section}{7}
		\setcounter{subsection}{2}
		\subsection{Wie leite ich eigentlich ab?}
		Innerhalb der Untersuchung einer ganzrationalen Funktion, bin ich dazu gezwungen, diese abzuleiten.\\
		Welche Regeln wende ich dabei aber eigentlich an?
		\subsubsection{Konstantenregel}
		Habe ich eine konstante Funktion, also eine Funktion ohne \(x\), so ist die dazugehörige Ableitung \(0\).\\
		Die dazugehörige Ableitungsregel heißt auch \textbf{Konstantenregel}. Sie lautet wie folgt:
		\begin{framed}
			\noindent
			\begin{tabularx}{0.7\textwidth}{XX}
				\(f(x) = c\) & \(\Rightarrow f'(x) = \)\\
			\end{tabularx}
		\end{framed}
		\subsubsection*{Beispiel Konstantenregel}
		\includegraphics[scale=0.2]{../../empty.jpg}\\
		\hdashrule[0.5ex][x]{0.45\textwidth}{0.1mm}{4mm 2pt}
		\subsubsection{Potenzregel}
		Beinhaltet meine Funktion ein \(x^n\) mit einer Potenz \(n \geq 1\), so wird beim Ableiten der Exponent zum Koeffizienten und der Exponent wird um \(1\) verringert.\\
		Diese Regel heißt auch \textbf{Potenzregel} und kann wie folgt formalisiert werden:
		\begin{framed}
			\noindent
			\begin{tabularx}{0.7\textwidth}{lX}
				\(f(x) = x^n\) & \(\Rightarrow f'(x) = \)
			\end{tabularx}
		\end{framed}
		\subsubsection*{Beispiele Potenzregel}
		\includegraphics[scale=0.2]{../../empty.jpg}\\
		\hdashrule[0.5ex][x]{0.45\textwidth}{0.1mm}{8mm 2pt}
		\subsubsection{Faktorregel}
		Besteht unsere Funktion aus einem \(a\cdot x^n\) mit einem Koeffizienten, so wird dieser Koeffizient mit dem Exponenten multipliziert. Auch hier wird der Exponent um \(1\) verringert.\\
		Diese Regel trägt den Namen \textbf{Faktorregel} und kann im Allgemeinen auch so ausgedrückt werden:
		\begin{framed}
			\noindent
			\begin{tabularx}{0.8\textwidth}{XX}
				\(f(x) = a\cdot x^n\) & \(\Rightarrow f'(x) = \)
			\end{tabularx}
		\end{framed}
		\color{codegray} Beachten Sie, sollte ein \(x^n\) keinen expliziten Koeffizienzen besitzen, so ist dieser immer 1.\\
		Es gilt also: \colorbox{green!5}{\(f(x) = x = 1\cdot x\)}.
		\normalcolor
		\subsubsection*{Beispiele Faktorregel}
		\includegraphics[scale=0.2]{../../empty.jpg}\\
		\hdashrule[0.5ex][x]{0.45\textwidth}{0.1mm}{8mm 2pt}
		\subsubsection{Summenregel}
		Ist eine Funktion als Summe oder Differenz einzelner \(x\)-Terme gegeben \((x^n + a\cdot x^m)\), so leiten wir jeden Term einzeln ab. Die Rechenoperatoren zwischen den Termen bleiben erhalten.\\
		Die hier anzuwendende Regel wird \textbf{Summenregel} genannt und lässt sich so zusammenfassen:
		\begin{framed}
			\noindent
			\begin{tabularx}{0.8\textwidth}{X}
				\(f(x) = x^n + a\cdot x^m\)\\
				\(\Rightarrow f'(x) = \)
			\end{tabularx}
		\end{framed}
	
		\subsubsection*{Beispiel Summenregel}
		\includegraphics[scale=0.2]{../../empty.jpg}\\
		\hdashrule[0.5ex][x]{0.45\textwidth}{0.1mm}{4mm 2pt}\\
		\subsubsection{Zusammenfassend}
		Wir können also festhalten, für die Ableitung einer ganzrationalen Funktion gelten folgende \textit{Ableitungsregeln}:\\
		\par
		\textbf{Konstantenregel}\\
		\includegraphics[scale=0.1]{../../empty.jpg}\\
		\par
		\textbf{Potenzregel}\\
		\includegraphics[scale=0.1]{../../empty.jpg}\\
		\par
		\textbf{Faktorregel}\\
		\includegraphics[scale=0.1]{../../empty.jpg}\\
		\par
		\textbf{Summenregel}\\
		\includegraphics[scale=0.1]{../../empty.jpg}\\
		\subsection*{Ihre Aufgabe}
		Leiten sie die folgenden Funktionen ab und geben Sie jeweils die verwendete Ableitungsregel an:
		\begin{itemize}
			\item[(a)] \(f(x) = 3x^3 + 4x^2\)
			\item[(b)] \(f(x) = 0,5x^2 + 9x - 1\)
			\item[(c)] \(f(x) = \frac{1}{3}x\)
			\item[(d)] \(f(x) = 2,5\)
		\end{itemize}		
	\end{worksheet}
\end{document}