\documentclass[11pt,twocolumn,oneside,openany,headings=optiontotoc,11pt,numbers=noenddot]{article}

\usepackage[a4paper]{geometry}
\usepackage[utf8]{inputenc}
\usepackage[T1]{fontenc}
\usepackage{lmodern}
\usepackage[ngerman]{babel}
\usepackage{ngerman}

\usepackage[onehalfspacing]{setspace}

\usepackage{fancyhdr}
\usepackage{fancybox}

\usepackage{rotating}
\usepackage{varwidth}


\usepackage{pdflscape}
\usepackage{graphicx}
\usepackage{graphbox}
\graphicspath{
	{Pics/PDFs/}
	{Pics/JPGs/}
	{Pics/PNGs/}
}
\usepackage{caption}
\usepackage{tabularx}
\usepackage{dashrule}
\usepackage{hhline}
\usepackage{multirow}
\usepackage{enumerate}
\usepackage[hidelinks]{hyperref}
\usepackage{listings}

\usepackage[table]{xcolor}
\usepackage{array}
\usepackage{enumitem,amssymb,amsmath}
\usepackage{interval}
\usepackage{stmaryrd}
\usepackage{polynom}
\usepackage{diagbox}
\usepackage{dashrule}
\usepackage{framed}
\usepackage{mdframed}
\usepackage{karnaugh-map}

\usepackage{blindtext}

\usepackage{eso-pic}

\usepackage{amssymb}
\usepackage{eurosym}
\pagestyle{headings}
\renewcommand{\headrulewidth}{0.2pt}
\renewcommand{\footrulewidth}{0.2pt}
\newcommand*{\underdownarrow}[2]{\ensuremath{\underset{\overset{\Big\downarrow}{#2}}{#1}}}
\setlength{\fboxsep}{5pt}

% Codestyle defined
\definecolor{codegreen}{rgb}{0,0.6,0}
\definecolor{codegray}{rgb}{0.5,0.5,0.5}
\definecolor{codepurple}{rgb}{0.58,0,0.82}
\definecolor{backcolour}{rgb}{0.95,0.95,0.92}
\definecolor{deepgreen}{rgb}{0,0.5,0}
\definecolor{darkblue}{rgb}{0,0,0.65}
\definecolor{mauve}{rgb}{0.40, 0.19,0.28}
\colorlet{exceptioncolour}{yellow!50!red}
\colorlet{commandcolour}{blue!60!black}
\colorlet{numpycolour}{blue!60!green}
\colorlet{specmethodcolour}{violet}

%Neue Spaltendefinition
\newcolumntype{L}[1]{>{\raggedright\let\newline\\\arraybackslash\hspace{0pt}}m{#1}}
\newcolumntype{M}[1]{>{\centering\arraybackslash}X}
\newcommand{\cmnt}[1]{\ignorespaces}
%Textausrichtung ändern
\newcommand\tabrotate[1]{\rotatebox{90}{\raggedright#1\hspace{\tabcolsep}}}

%Intervall-Konfig
\intervalconfig {
	soft open fences
}

%Bash
\lstdefinestyle{BashInputStyle}{
	language=bash,
	basicstyle=\small\sffamily,
	backgroundcolor=\color{backcolour},
	columns=fullflexible,
	backgroundcolor=\color{backcolour},
	breaklines=true,
}
%Java
\lstdefinestyle{JavaInputStyle}{
	language=Java,
	backgroundcolor=\color{backcolour},
	aboveskip=1mm,
	belowskip=1mm,
	showstringspaces=false,
	columns=flexible,
	basicstyle={\footnotesize\ttfamily},
	numberstyle={\tiny},
	numbers=none,
	keywordstyle=\color{purple},,
	commentstyle=\color{deepgreen},
	stringstyle=\color{blue},
	emph={out},
	emphstyle=\color{darkblue},
	emph={[2]rand},
	emphstyle=[2]\color{specmethodcolour},
	breaklines=true,
	breakatwhitespace=true,
	tabsize=2,
}
%Python
\lstdefinestyle{PythonInputStyle}{
	language=Python,
	alsoletter={1234567890},
	aboveskip=1ex,
	basicstyle=\footnotesize,
	breaklines=true,
	breakatwhitespace= true,
	backgroundcolor=\color{backcolour},
	commentstyle=\color{red},
	otherkeywords={\ , \}, \{, \&,\|},
	emph={and,break,class,continue,def,yield,del,elif,else,%
		except,exec,finally,for,from,global,if,import,in,%
		lambda,not,or,pass,print,raise,return,try,while,assert},
	emphstyle=\color{exceptioncolour},
	emph={[2]True,False,None,min},
	emphstyle=[2]\color{specmethodcolour},
	emph={[3]object,type,isinstance,copy,deepcopy,zip,enumerate,reversed,list,len,dict,tuple,xrange,append,execfile,real,imag,reduce,str,repr},
	emphstyle=[3]\color{commandcolour},
	emph={[4]ode, fsolve, sqrt, exp, sin, cos, arccos, pi,  array, norm, solve, dot, arange, , isscalar, max, sum, flatten, shape, reshape, find, any, all, abs, plot, linspace, legend, quad, polyval,polyfit, hstack, concatenate,vstack,column_stack,empty,zeros,ones,rand,vander,grid,pcolor,eig,eigs,eigvals,svd,qr,tan,det,logspace,roll,mean,cumsum,cumprod,diff,vectorize,lstsq,cla,eye,xlabel,ylabel,squeeze},
	emphstyle=[4]\color{numpycolour},
	emph={[5]__init__,__add__,__mul__,__div__,__sub__,__call__,__getitem__,__setitem__,__eq__,__ne__,__nonzero__,__rmul__,__radd__,__repr__,__str__,__get__,__truediv__,__pow__,__name__,__future__,__all__},
	emphstyle=[5]\color{specmethodcolour},
	emph={[6]assert,range,yield},
	emphstyle=[6]\color{specmethodcolour}\bfseries,
	emph={[7]Exception,NameError,IndexError,SyntaxError,TypeError,ValueError,OverflowError,ZeroDivisionError,KeyboardInterrupt},
	emphstyle=[7]\color{specmethodcolour}\bfseries,
	emph={[8]taster,send,sendMail,capture,check,noMsg,go,move,switch,humTem,ventilate,buzz},
	emphstyle=[8]\color{blue},
	keywordstyle=\color{blue}\bfseries,
	rulecolor=\color{black!40},
	showstringspaces=false,
	stringstyle=\color{deepgreen}
}

\lstset{literate=%
	{Ö}{{\"O}}1
	{Ä}{{\"A}}1
	{Ü}{{\"U}}1
	{ß}{{\ss}}1
	{ü}{{\"u}}1
	{ä}{{\"a}}1
	{ö}{{\"o}}1
}

% Neue Klassenarbeits-Umgebung
\newenvironment{worksheet}[3]
% Begin-Bereich
{
	\newpage
	\sffamily
	\setcounter{page}{1}
	\ClearShipoutPicture
	\AddToShipoutPicture{
		\put(55,761){{
				\mbox{\parbox{385\unitlength}{\tiny \color{codegray}BBS I Mainz, #1 \newline #2
						\newline #3
					}
				}
			}
		}
		\put(455,761){{
				\mbox{\hspace{0.3cm}\includegraphics[width=0.2\textwidth]{../../logo.jpg}}
			}
		}
	}
}
% End-Bereich
{
	\clearpage
	\ClearShipoutPicture
}

\setlength{\columnsep}{3em}
\setlength{\columnseprule}{0.5pt}

\geometry{left=2.50cm,right=2.50cm,top=3.00cm,bottom=1.00cm,includeheadfoot}
\pagenumbering{gobble}
\pagestyle{empty}

\begin{document}
	\begin{worksheet}{HBF IT 17A}{Grundstufe}{Lernbaustein 3 - LB 2: Kurvendiskussion}
		\section{Grundwissen}
		Gegeben ist eine ganzrationale Funktion \[f(x) = a_nx^n + a_{n-1}x^{n-1} + \ldots + a_1x + a_0 \] mit \(x\in\mathbb{R}\).\\
		Um diese Funktion (oder auch \textit{Kurve}) untersuchen zu können, müssen wir auf Informationen aus vergangenen Lernabschnitten aber auch auf neues Wissen zurückgreifen. Nachfolgend werden die entsprechenden Informationen in Kurzform nochmal dargestellt.
		\subsection{Defintionsbereich} Als Definitionsbereich kann man die Menge von Zahlen (x-Werte) bezeichnen, welche man in die Funktion \(f(x)\) einsetzen darf.\\
		Man schreibt: \(\mathbb{D} = \)\\
		\subsection{Wertebereich} Zum Wertebereich zählen alle möglichen \(y\)-Werte, die die Funktion annehmen kann.\\
		Man schreibt auch \(\mathbb{W} = \)\\
		\subsection{Symmetrie} \(\cdot\) Eine Funktion heißt \underline{\textbf{achsensymmetrisch}}, wenn gilt: es kommen nur gerade Exponenten in dem Funktionsterm \(f(x)\) vor.\\
		\tiny{\(a_0\) gilt als Glied mit Geradem Exponenten.}\normalsize\\
		In diesem Fall bedeutet das außerdem, dass \(f(x) = f(-x)\ \text{für alle zulässigen } x\). Die Funktion nennt man dann auch \underline{gerade}.\\
		\par\noindent
		\(\cdot\) Eine Funktion heißt \underline{\textbf{punktsymmetrisch}}, wenn gilt: es kommen nur ungerade Exponenten in dem Funktionsterm \(f(x)\) vor.\\
		\tiny{Im Funktionsterm gibt es kein \(a_0\).}\normalsize\\
		In diesem Fall heißt das, dass \(f(x) = -f(x)\ \text{für alle zulässigen } x\). Eine solche Funktion wird auch \underline{ungerade} genannt.\\
		\par\noindent
		\(\cdot\) Enthält der Funktionsterm \(f(x)\) sowohl gerade als auch ungerade Exponenten, so ist sie \underline{\textbf{nicht symmetrisch}}.
		\subsection{Verhalten für große x-Werte} Um das Verhalten für große x-Werte zu bestimmen, betrachten wir und lediglich den \textit{charakteristischen Summanden}, also den Summanden, mit der höchsten Potenz (\(a_nx^n\)).\\
		Das Verhalten kann dann wie folgt angegeben werden:
		\begin{tabular}{|l|L{0.2\textwidth}|L{0.2\textwidth}|}
			\hline
			\diagbox{\(a_{n}\)}{n} & gerade & ungerade\\
			\hline
			positiv & \shortstack{\(f(x) \xrightarrow{x \rightarrow -\infty}\infty\)\\\(f(x) \xrightarrow{x \rightarrow \infty} \infty\)} & \shortstack{\(f(x) \xrightarrow{x \rightarrow -\infty}-\infty\)\\\(f(x) \xrightarrow{x \rightarrow \infty}\infty\)}\\
			\hline
			negativ & \shortstack{\(f(x) \xrightarrow{x \rightarrow -\infty}-\infty\)\\\(f(x) \xrightarrow{x \rightarrow \infty}-\infty\)} & \shortstack{\(f(x) \xrightarrow{x \rightarrow -\infty}\infty\)\\\(f(x) \xrightarrow{x \rightarrow \infty}-\infty\)}\\
			\hline
		\end{tabular}
		\subsection{Achsenschnitte}
		\subsubsection*{Nullstellen} Die Nullstellen sind die Stellen, an denen der Funktionswert Null ist, der Funktionsgraph also die x-Achse schneidet.\\
		Um die Nullstellen zu bestimmen, setzen wir \[f(x) = 0\]\\
		Wir merken uns:
		\begin{itemize}
			\item \(a_1x + a_0 \Rightarrow\) Null setzen und nach \(x\) umformen (\(a_1x + a_0 = 0 \Rightarrow x = \frac{a_0}{a_1}\))
			\item  \(a_2x^2 + a_1x + a_0 \Rightarrow\) Anwenden der \textbf{pq-Formel}\\
			\textbf{Beachte:} \(a_2\) muss den Wert 1 haben (also muss gegebenenfalls \(:a_2\) gerechnet werden).
			\item \(a_2x^2 + a_0 \Rightarrow\) Null setzen und nach \(x\) umformen (\(a_2x^2 + a_0 = 0 \Rightarrow x = \sqrt{\frac{a_0}{a_2}}\))
			\item Funktionen mit Grad 3 und Höher \(\Rightarrow\) NST ausprobieren (meist \(-2,-1,0,1,2\)); Dann mit \textbf{Polynomdivision} den Grad reduzieren\\
			\((a_nx^n + \ldots +a_1x +a_0):(x-NST) = p(x)\)\\
			NST des Ergebnisses \(p(x)\) bestimmen
		\end{itemize}
		\subsubsection{Schnittstelle mit der y-Achse} Um den y-Achsenabschnitt zu bestimmen, setzen wir für \(x\) Null (\(0\)) in den Funktionsterm ein. Wir bestimmen also \[f(0) = a_0\]
		So erhalten wir \(S(0|a_0)\). Ist \(f(0) = a_0 = 0\), so verläuft der Graph durch den Ursprung.
		\subsection{Extremstellen} \underline{\textit{\textbf{Notwendige Bedingung}}} Ist \(x_0\) eine Extremstelle, dann muss \(f'(x) = 0\) sein.\\
		\par\noindent
		\textbf{\textit{1. Hinreichende Bedingung}}\\
		\(\circ\) \(f'(x_0)=0\ \text{und}\ f''(x_0) > 0 \rightarrow\ f(x_0)\) ist \underline{lokales} Minimum\\
		\(\circ\) \(f'(x_0)=0\ \text{und}\ f''(x_0) < 0 \rightarrow\ f(x_0)\) ist \underline{lokales} Maximum\\
		\par\noindent
		Wenn \(f'(x_0)=0\ \text{und}\ f''(x_0)=0\), dann verwende nachfolgende Bedingung.\\
		\par\noindent
		\textit{\textbf{2. Hinreichende Bedingung}} \(f'(x)\) hat an der Stelle \(x_0\) einen \underline{Vorzeichenwechsel}.\\
		\(\circ\) Von \(+\ \text{nach}\ - \rightarrow\ f(x_0)\) ist \underline{lokales} Minimum\\
		\(\circ\) Von \(-\ \text{nach}\ + \rightarrow\ f(x_0)\) ist \underline{lokales} Maximum\\
		\subsection{Krümmungsverhalten und Wendestellen}
		Für das \textit{Krümmungsverhalten} eines Funktionsgraphen gilt folgendes:\\
		\(\circ\) Ist \(f'(x)\) streng monoton \textit{steigend}, so ist der Funktionsgraph von f \textbf{linksgekrümmt}.\\
		\(\circ\) Ist \(f'\) streng monoton \textit{fallend}, so ist der Funktionsgraph von \(f\) \textbf{rechtsgekrümmt}.\\
		\small{Das Krümmungsverhalten gibt man jeweils vor, zwischen und nach den Wendepunkten \([x_{W_1}, x_{W_2}]\) an.}\normalsize
		\par\noindent
		Das Krümmungsverhalten des Funktionsgraphen ändert sich an einer \textit{Wendestelle}. Für eine solche gilt:\\
		\textit{\underline{\textbf{Notwendige Bedingung}}} Ist \(x_0\) eine Wendestelle, dann muss \(f''(x_0)  = 0\) sein.\\
		\par\noindent
		\textbf{\textit{1. Hinreichende Bedingung}}\\
		\(f''(x_0)=0\) und \(f'''(x_0) \neq 0\), dann ist \(x_0\) eine Wendestelle.\\
		\textbf{\textit{2. Hinreichende Bedingung}}\\\(f''(x_0)=0\) und \(f''(x)\) hat an der Stelle \(x_0\) einen \textbf{Vorzeichenwechsel}.\\
		\par\noindent
		Wenn \(f'(x_0) = 0\) und \(f''(x_0)= 0\), aber \(f'''(x_0)\neq 0\), dann bezeichnet man \(x_0\) als Sattelstelle.\\
		Ein \textbf{Sattelpunkt} ist ein besonderer Wendepunkt, in dem die Tangente am Graphen waagerecht ist.\\
		\subsection{Schaubild} Zeichne die Nullstellen, die Extrempunkte und die Wendepunkte in das Koordinatensystem.\\
		Verbinde die Punkte entsprechen der Information, die du über die Symmetrie, das Randverhalten (Verhalten für große x-Werte), die Monotonie und das Krümmungsverhalten erhalten hast.\\
		So erhältst du den \underline{ungefähren} Verlauf des Funktionsgraphen.\\
		\hdashrule[0.5ex][x]{0.45\textwidth}{0.1mm}{8mm 2pt}\\
		\subsection*{Monotonieverhalten \tiny{(Zusatz)}\normalsize} Betrachtest du eine Funktion, so ist es bisweilen interessant zu wissen, wie die verschiedenen Funktionswerte zueinander stehen. Dies interessiert uns meist in einem bestimmten Intervall (z.B. zwischen \(x_0\) und \(x_1\)). Wir definieren also ein Intervall \(I = [x_0,x_1]\), als die Menge der x-Werte, für die gilt: \(x_0 \leq x \leq x_1\).\\
		Um eine Aussage über das Monotonieverhalten treffen zu können, ist es zwingend notwendig, dass die Funktion auf dem entsprechenden Intervall \(I\) definiert ist.\\
		\par\noindent
		\(f\) heißt \textbf{streng monoton steigend} auf \(I\), wenn für alle \(x_1,x_2 \in I\) mit \(x_1<x_2\) gilt:\\
		\(\circ f(x_1)<f(x_2)\).\\
		Entsprechend heißt \(f\) \textbf{streng monoton fallend} auf \(I\), wenn für alle \(x_1,x_2 \in I\) mit \(x_1<x_2\) gilt:\\
		\(\circ f(x_1) > f(x_2)\).
		\newpage
		\section{Kurvendiskussion 101}
		Wirst du mit einer ganzrationalen Funktion \(f(x) = a_nx^n + a_{n-1}x^{n-1} +\ldots + a_1x +a_0\) konfrontiert und sollst diese skizzieren, führst du die folgenden Schritte der \textbf{Kurvendiskussion} aus.\\
		\begin{enumerate}
			\item Definitions- und Wertebereich angeben
			\item Ableitungen bilden (\(f'(x), f''(x), f'''(x)\))
			\item Symmetrie bestimmen\\
			\(\underbrace{f(x) = f(-x)}_{achsensymmetrisch}\) oder \(\underbrace{f(x) = -f(-x)}_{punktsymmetrisch}\)
			\item Verhalten für große x-Werte (\tiny{Wo kommt die Funktion her, wo geht sie hin}\normalsize)
			\item Nullstellen und die dazugehörigen Punkte bestimmen\\
			\(f(x) = 0\)
			\item Extremstellen und die dazugehörigen Punkte\\
			\(f'(x) = 0\)
			\item Wendestellen (ggf. Sattelstelle) und die dazugehörigen Punkte\\
			\(f''(x) = 0\)
			\item Übertragen der Punkte in das Koordinatensystem.\\
			Skizzieren des Funktionsgraphen anhand der Schritte (2.), (3.), (6.) und (8.) und der eingetragenen Punkte.
		\end{enumerate}
	\end{worksheet}
\end{document}