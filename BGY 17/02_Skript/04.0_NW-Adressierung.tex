\documentclass[11pt,oneside,openany,headings=optiontotoc,11pt,numbers=noenddot]{article}

\usepackage[a4paper]{geometry}
\usepackage[utf8]{inputenc}
\usepackage[T1]{fontenc}
\usepackage{lmodern}
\usepackage[ngerman]{babel}
\usepackage{ngerman}

\usepackage[onehalfspacing]{setspace}

\usepackage{fancyhdr}
\usepackage{fancybox}

\usepackage{rotating}
\usepackage{varwidth}


\usepackage{pdflscape}
\usepackage{graphicx}
\usepackage{graphbox}
\graphicspath{
	{Pics/PDFs/}
	{Pics/JPGs/}
	{Pics/PNGs/}
}
\usepackage{caption}
\usepackage{tabularx}
\usepackage{dashrule}
\usepackage{hhline}
\usepackage{multirow}
\usepackage{enumerate}
\usepackage[hidelinks]{hyperref}
\usepackage{listings}

\usepackage[table]{xcolor}
\usepackage{array}
\usepackage{enumitem,amssymb,amsmath}
\usepackage{interval}
\usepackage{stmaryrd}
\usepackage{polynom}
\usepackage{diagbox}
\usepackage{dashrule}
\usepackage{framed}
\usepackage{mdframed}
\usepackage{karnaugh-map}

\usepackage{blindtext}

\usepackage{eso-pic}

\usepackage{amssymb}
\usepackage{eurosym}
\pagestyle{headings}
\renewcommand{\headrulewidth}{0.2pt}
\renewcommand{\footrulewidth}{0.2pt}
\newcommand*{\underdownarrow}[2]{\ensuremath{\underset{\overset{\Big\downarrow}{#2}}{#1}}}
\setlength{\fboxsep}{5pt}

% Codestyle defined
\definecolor{codegreen}{rgb}{0,0.6,0}
\definecolor{codegray}{rgb}{0.5,0.5,0.5}
\definecolor{codepurple}{rgb}{0.58,0,0.82}
\definecolor{backcolour}{rgb}{0.95,0.95,0.92}
\definecolor{deepgreen}{rgb}{0,0.5,0}
\definecolor{darkblue}{rgb}{0,0,0.65}
\definecolor{mauve}{rgb}{0.40, 0.19,0.28}
\colorlet{exceptioncolour}{yellow!50!red}
\colorlet{commandcolour}{blue!60!black}
\colorlet{numpycolour}{blue!60!green}
\colorlet{specmethodcolour}{violet}

%Neue Spaltendefinition
\newcolumntype{L}[1]{>{\raggedright\let\newline\\\arraybackslash\hspace{0pt}}m{#1}}
\newcolumntype{M}[1]{>{\centering\arraybackslash}X}
\newcommand{\cmnt}[1]{\ignorespaces}
%Textausrichtung ändern
\newcommand\tabrotate[1]{\rotatebox{90}{\raggedright#1\hspace{\tabcolsep}}}

%Intervall-Konfig
\intervalconfig {
	soft open fences
}

%Bash
\lstdefinestyle{BashInputStyle}{
	language=bash,
	basicstyle=\small\sffamily,
	backgroundcolor=\color{backcolour},
	columns=fullflexible,
	backgroundcolor=\color{backcolour},
	breaklines=true,
}
%Java
\lstdefinestyle{JavaInputStyle}{
	language=Java,
	backgroundcolor=\color{backcolour},
	aboveskip=1mm,
	belowskip=1mm,
	showstringspaces=false,
	columns=flexible,
	basicstyle={\footnotesize\ttfamily},
	numberstyle={\tiny},
	numbers=none,
	keywordstyle=\color{purple},,
	commentstyle=\color{deepgreen},
	stringstyle=\color{blue},
	emph={out},
	emphstyle=\color{darkblue},
	emph={[2]rand},
	emphstyle=[2]\color{specmethodcolour},
	breaklines=true,
	breakatwhitespace=true,
	tabsize=2,
}
%Python
\lstdefinestyle{PythonInputStyle}{
	language=Python,
	alsoletter={1234567890},
	aboveskip=1ex,
	basicstyle=\footnotesize,
	breaklines=true,
	breakatwhitespace= true,
	backgroundcolor=\color{backcolour},
	commentstyle=\color{red},
	otherkeywords={\ , \}, \{, \&,\|},
	emph={and,break,class,continue,def,yield,del,elif,else,%
		except,exec,finally,for,from,global,if,import,in,%
		lambda,not,or,pass,print,raise,return,try,while,assert},
	emphstyle=\color{exceptioncolour},
	emph={[2]True,False,None,min},
	emphstyle=[2]\color{specmethodcolour},
	emph={[3]object,type,isinstance,copy,deepcopy,zip,enumerate,reversed,list,len,dict,tuple,xrange,append,execfile,real,imag,reduce,str,repr},
	emphstyle=[3]\color{commandcolour},
	emph={[4]ode, fsolve, sqrt, exp, sin, cos, arccos, pi,  array, norm, solve, dot, arange, , isscalar, max, sum, flatten, shape, reshape, find, any, all, abs, plot, linspace, legend, quad, polyval,polyfit, hstack, concatenate,vstack,column_stack,empty,zeros,ones,rand,vander,grid,pcolor,eig,eigs,eigvals,svd,qr,tan,det,logspace,roll,mean,cumsum,cumprod,diff,vectorize,lstsq,cla,eye,xlabel,ylabel,squeeze},
	emphstyle=[4]\color{numpycolour},
	emph={[5]__init__,__add__,__mul__,__div__,__sub__,__call__,__getitem__,__setitem__,__eq__,__ne__,__nonzero__,__rmul__,__radd__,__repr__,__str__,__get__,__truediv__,__pow__,__name__,__future__,__all__},
	emphstyle=[5]\color{specmethodcolour},
	emph={[6]assert,range,yield},
	emphstyle=[6]\color{specmethodcolour}\bfseries,
	emph={[7]Exception,NameError,IndexError,SyntaxError,TypeError,ValueError,OverflowError,ZeroDivisionError,KeyboardInterrupt},
	emphstyle=[7]\color{specmethodcolour}\bfseries,
	emph={[8]taster,send,sendMail,capture,check,noMsg,go,move,switch,humTem,ventilate,buzz},
	emphstyle=[8]\color{blue},
	keywordstyle=\color{blue}\bfseries,
	rulecolor=\color{black!40},
	showstringspaces=false,
	stringstyle=\color{deepgreen}
}

\lstset{literate=%
	{Ö}{{\"O}}1
	{Ä}{{\"A}}1
	{Ü}{{\"U}}1
	{ß}{{\ss}}1
	{ü}{{\"u}}1
	{ä}{{\"a}}1
	{ö}{{\"o}}1
}

% Neue Klassenarbeits-Umgebung
\newenvironment{worksheet}[3]
% Begin-Bereich
{
	\newpage
	\sffamily
	\setcounter{page}{1}
	\ClearShipoutPicture
	\AddToShipoutPicture{
		\put(55,761){{
				\mbox{\parbox{385\unitlength}{\tiny \color{codegray}BBS I Mainz, #1 \newline #2
						\newline #3
					}
				}
			}
		}
		\put(455,761){{
				\mbox{\hspace{0.3cm}\includegraphics[width=0.2\textwidth]{../../logo.jpg}}
			}
		}
	}
}
% End-Bereich
{
	\clearpage
	\ClearShipoutPicture
}

\setlength{\columnsep}{3em}
\setlength{\columnseprule}{0.5pt}

\geometry{left=1.50cm,right=1.50cm,top=2.50cm,bottom=1.00cm,includeheadfoot}
\pagenumbering{gobble}
\pagestyle{empty}

\begin{document}
	\begin{worksheet}{Berufliches Gymnasium}{Klassenstufe 12 - Informationsverarbeitung}{Lernabschnitt 1: Netzwerke - Adressierung}
		\section{Adressierung}
		Wie auch im realen Leben benötigen die kommunizierenden Netzwerkkomponenten\footnote{Nachfolgend nennen wir diese Computer.} Adresse, die sie eindeutig identifizieren. Innerhalb der Netzwerktechnik unterscheidet man dabei zwischen zwei Arten von Adressen:\\
		\par\noindent
		\begin{tabularx}{\textwidth}{MM}
			\(\circ\) IP-Adresse & \(\circ\) MAC-Adresse
		\end{tabularx}
		\subsection{Die IP-Adresse}
		In einem \textbf{L}ocal \textbf{A}rea \textbf{N}etwork\footnote{LAN.} können viele Computer vorhanden sein. Damit jeder Knoten mit jedem anderen Knoten in Kontakt treten kann, erhält jeder Knoten eine \textbf{eindeutige} Adresse.\\
		Innerhalb der Netzwerktechnik bildet das \textit{Internet Protocol (\textbf{IP})} das Zentrum der Kommunikation. Daher wird für diese Adresse auch der Begriff \textbf{IP-Adresse} verwendet.
		\begin{framed}
			\noindent
			Eine \textbf{IP-Adresse} ist die im Internetprotokoll angewandte Kennzeichnung für ein Interface. Jedes benutzte Interface erhält eine eigene IP-Adresse. Einem Rechner können daher auch mehrere IP-Adressen zugeordnet sein.
		\end{framed}
		Man kann sich IP-Adressen ein bisschen wie Postleitzahlen vorstellen. Das bedeutet, aus der IP-Adresse lässt sich direkt auf den \textbf{Zielort} schließen, so kann die ungefähre Richtung festleget werden, in die das Datenpaket gesendet werden muss.\\
		\par
		\texttt{Wie kann man sich eine IP-Adresse vorstellen?}\\
		\noindent
		Im derzeit noch verwendeten Standard (IPv4) hat eine IP-Adresse \textbf{4 Bytes}, was 32 Bit entspricht.\\
		In der Regel wird die IP-Adresse aber in der Dezimalschreibweise angegeben.\\
		\par\noindent
		\renewcommand{\arraystretch}{1.5}
		\begin{tabularx}{0.8\textwidth}{|L{0.12cm}|L{0.12cm}|L{0.12cm}|L{0.12cm}|L{0.12cm}|L{0.12cm}|L{0.12cm}|L{0.12cm}|L{0.12cm}|L{0.12cm}|L{0.12cm}|L{0.12cm}|L{0.12cm}|L{0.12cm}|L{0.12cm}|L{0.12cm}|L{0.12cm}|L{0.12cm}|L{0.12cm}|L{0.12cm}|L{0.12cm}|L{0.12cm}|L{0.12cm}|L{0.12cm}|L{0.12cm}|L{0.12cm}|L{0.12cm}|L{0.12cm}|L{0.12cm}|L{0.12cm}|L{0.12cm}|L{0.12cm}|}
			\cline{1-32}
			\rowcolor{blue!5} \multicolumn{8}{|c|}{\textbf{1. Byte}} & \multicolumn{8}{|c|}{\textbf{2. Byte}} & \multicolumn{8}{|c|}{\textbf{3. Byte}} & \multicolumn{8}{|c|}{\textbf{4. Byte}}\\
			\cline{1-32}
			0 & 0 & 0 & 0 & 1 & 0 & 1 & 1 &
			0 & 1 & 0 & 1 & 1 & 0 & 0 & 0 &
			0 & 0 & 0 & 0 & 0 & 0 & 1 & 1 &
			0 & 1 & 1 & 0 & 0 & 1 & 0 & 0\\
			\cline{1-32}
			\multicolumn{8}{|c|}{11} & \multicolumn{8}{|c|}{88} & \multicolumn{8}{|c|}{3} & \multicolumn{8}{|c|}{100}\\
			\cline{1-32}
		\end{tabularx}\\
		\par\noindent
		Die einzelnen Bytes werden durch Punkte voneinander getrennt, so dass obiges Beispiel als \(\mathbf{11.88.3.100}\) geschrieben wird. Man bezeichnet diese Darstellungsform auch als \textit{dotted deimal notation}.\\
		Ähnlich wie bei einer Telefonnummer geben die ersten Stellen (\textit{prefix}) das Netz an, zu dem die IP-Adresse gehört, während die darauffolgenden das entsprechende Gerät (\textit{Host}) selbst.\\
		\par\noindent
		\begin{tabularx}{\textwidth}{|M|M|}
			\hline
			\rowcolor{blue!10} Netz-Adressteil (Prefix) & Host-Adressteil\\
			\hline
		\end{tabularx}\\
		\par\noindent
		Das bedeutet für die IP-Adressen aller in einem Netz befindlichen Computer, dass der Netz-Adressteil bei allen der gleiche ist. Die IP-Adressen unterscheiden sich lediglich im Host-Adressteil.\\
		Da der Netz-Adressanteil nicht zwangsläufig aus \underline{8}, \underline{16} oder \underline{24} Bit besteht, muss die Anzahl an Binärstellen des Netz-Adressteils immer mit angegeben werden:
		\begin{enumerate}
			\item Die (binäre) Stellenzahl wird mit einem Schrägstrich an die IP-Adresse angehängt, also z.B. 11.88.3.100/16.\\
			Hieraus lässt sich ableiten, dass der Host innerhalb des mit 11.88 ausgewiesenen \textbf{Netz-Adressteil} den \textbf{Host-Adressteil} 3.100 besitzt. \textit{Dies entspricht der modernen Schreibweise.}
			\item Alternativ lässt sich eine \textbf{Netzmaske} angeben. Diese besteht, wie die IP-Adresse selbst, aus 4 Byte. Davon sind die Bit des Netz-Adressteils mit Einsen belegt und die des Host-Adressteils mit Nullen. \textit{Diese ist die klassische Darstellung.}\\
			Für das obige Beispiel wäre \underline{255.255.0.0} die zugehörige Netzmaske.
		\end{enumerate}
		\begin{framed}
			\noindent
			Zur vollständigen Kennzeichnung eines Interfaces in einem IP-Netz gehören die \textbf{IP-Adresse} und die Länge der \textbf{Netzwerkmaske}.\\
			\par\noindent
			IP-Adressen können nicht beliebig gewählt werden. Globale (offizielle) Adressen vergibt der Provider, lokale Adressen vergibt der örtliche Administrator.
		\end{framed}
		\subsection{Die MAC-Adresse}
		Jedes in einem Netz aktiv kommunizierende Gerät besitzt zusätzlich zur zugewiesenen IP-Adresse eine physikalische Adresse (\textbf{MAC-Adresse}).\footnote{Das bedeutet, dass ein Computer, der zwei Ethernet-Anschlüsse besitzt, auch zwei MAC-Adressen besitzt.}\\
		Für die MAC-Adresse gilt, wie auch bei der IP-Adresse, dass sie innerhalb eines Netzabschnitts (LAN) eindeutig sein muss. Dennoch muss man hier zwischen drei Arten von MAC-Adressen unterscheiden:
		\begin{itemize}
			\item die \textbf{statische Adresse} (auch physikalische Adresse) ist eine vom Hersteller der Netzwerkkarte vorgegebene und fest eingestellte Adresse.
			\item die \textbf{konfigurierbare Adresse} wird vom Netzbetreiber festgelegt, muss aber innerhalb des genutzten LAN eindeutig sein.
			\item die \textbf{dynamische Adresse} wird bei Rechnerneustart automatisch zugeordnet.
		\end{itemize}
	\end{worksheet}
\end{document}