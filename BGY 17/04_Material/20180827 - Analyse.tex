\documentclass[oneside,openany,headings=optiontotoc,11pt,numbers=noenddot]{scrreprt}

\usepackage[a4paper]{geometry}
\usepackage[utf8]{inputenc}
\usepackage[T1]{fontenc}
\usepackage{lmodern}
\usepackage[ngerman]{babel}
\usepackage{ngerman}

\usepackage[onehalfspacing]{setspace}

\usepackage{fancyhdr}
\usepackage{fancybox}

\usepackage{rotating}
\usepackage{varwidth}


\usepackage{pdflscape}
\usepackage{graphicx}
\usepackage{graphbox}
\graphicspath{
	{Pics/PDFs/}
	{Pics/JPGs/}
	{Pics/PNGs/}
}
\usepackage{caption}
\usepackage{tabularx}
\usepackage{dashrule}
\usepackage{hhline}
\usepackage{multirow}
\usepackage{enumerate}
\usepackage[hidelinks]{hyperref}
\usepackage{listings}

\usepackage[table]{xcolor}
\usepackage{array}
\usepackage{enumitem,amssymb,amsmath}
\usepackage{interval}
\usepackage{stmaryrd}
\usepackage{polynom}
\usepackage{diagbox}
\usepackage{dashrule}
\usepackage{framed}
\usepackage{mdframed}
\usepackage{karnaugh-map}

\usepackage{blindtext}

\usepackage{eso-pic}

\usepackage{amssymb}
\usepackage{eurosym}
\pagestyle{headings}
\renewcommand{\headrulewidth}{0.2pt}
\renewcommand{\footrulewidth}{0.2pt}
\newcommand*{\underdownarrow}[2]{\ensuremath{\underset{\overset{\Big\downarrow}{#2}}{#1}}}
\setlength{\fboxsep}{5pt}

% Codestyle defined
\definecolor{codegreen}{rgb}{0,0.6,0}
\definecolor{codegray}{rgb}{0.5,0.5,0.5}
\definecolor{codepurple}{rgb}{0.58,0,0.82}
\definecolor{backcolour}{rgb}{0.95,0.95,0.92}
\definecolor{deepgreen}{rgb}{0,0.5,0}
\definecolor{darkblue}{rgb}{0,0,0.65}
\definecolor{mauve}{rgb}{0.40, 0.19,0.28}
\colorlet{exceptioncolour}{yellow!50!red}
\colorlet{commandcolour}{blue!60!black}
\colorlet{numpycolour}{blue!60!green}
\colorlet{specmethodcolour}{violet}

%Neue Spaltendefinition
\newcolumntype{L}[1]{>{\raggedright\let\newline\\\arraybackslash\hspace{0pt}}m{#1}}
\newcolumntype{M}[1]{>{\centering\arraybackslash}X}
\newcommand{\cmnt}[1]{\ignorespaces}
%Textausrichtung ändern
\newcommand\tabrotate[1]{\rotatebox{90}{\raggedright#1\hspace{\tabcolsep}}}

%Intervall-Konfig
\intervalconfig {
	soft open fences
}

%Bash
\lstdefinestyle{BashInputStyle}{
	language=bash,
	basicstyle=\small\sffamily,
	backgroundcolor=\color{backcolour},
	columns=fullflexible,
	backgroundcolor=\color{backcolour},
	breaklines=true,
}
%Java
\lstdefinestyle{JavaInputStyle}{
	language=Java,
	backgroundcolor=\color{backcolour},
	aboveskip=1mm,
	belowskip=1mm,
	showstringspaces=false,
	columns=flexible,
	basicstyle={\footnotesize\ttfamily},
	numberstyle={\tiny},
	numbers=none,
	keywordstyle=\color{purple},,
	commentstyle=\color{deepgreen},
	stringstyle=\color{blue},
	emph={out},
	emphstyle=\color{darkblue},
	emph={[2]rand},
	emphstyle=[2]\color{specmethodcolour},
	breaklines=true,
	breakatwhitespace=true,
	tabsize=2,
}
%Python
\lstdefinestyle{PythonInputStyle}{
	language=Python,
	alsoletter={1234567890},
	aboveskip=1ex,
	basicstyle=\footnotesize,
	breaklines=true,
	breakatwhitespace= true,
	backgroundcolor=\color{backcolour},
	commentstyle=\color{red},
	otherkeywords={\ , \}, \{, \&,\|},
	emph={and,break,class,continue,def,yield,del,elif,else,%
		except,exec,finally,for,from,global,if,import,in,%
		lambda,not,or,pass,print,raise,return,try,while,assert},
	emphstyle=\color{exceptioncolour},
	emph={[2]True,False,None,min},
	emphstyle=[2]\color{specmethodcolour},
	emph={[3]object,type,isinstance,copy,deepcopy,zip,enumerate,reversed,list,len,dict,tuple,xrange,append,execfile,real,imag,reduce,str,repr},
	emphstyle=[3]\color{commandcolour},
	emph={[4]ode, fsolve, sqrt, exp, sin, cos, arccos, pi,  array, norm, solve, dot, arange, , isscalar, max, sum, flatten, shape, reshape, find, any, all, abs, plot, linspace, legend, quad, polyval,polyfit, hstack, concatenate,vstack,column_stack,empty,zeros,ones,rand,vander,grid,pcolor,eig,eigs,eigvals,svd,qr,tan,det,logspace,roll,mean,cumsum,cumprod,diff,vectorize,lstsq,cla,eye,xlabel,ylabel,squeeze},
	emphstyle=[4]\color{numpycolour},
	emph={[5]__init__,__add__,__mul__,__div__,__sub__,__call__,__getitem__,__setitem__,__eq__,__ne__,__nonzero__,__rmul__,__radd__,__repr__,__str__,__get__,__truediv__,__pow__,__name__,__future__,__all__},
	emphstyle=[5]\color{specmethodcolour},
	emph={[6]assert,range,yield},
	emphstyle=[6]\color{specmethodcolour}\bfseries,
	emph={[7]Exception,NameError,IndexError,SyntaxError,TypeError,ValueError,OverflowError,ZeroDivisionError,KeyboardInterrupt},
	emphstyle=[7]\color{specmethodcolour}\bfseries,
	emph={[8]taster,send,sendMail,capture,check,noMsg,go,move,switch,humTem,ventilate,buzz},
	emphstyle=[8]\color{blue},
	keywordstyle=\color{blue}\bfseries,
	rulecolor=\color{black!40},
	showstringspaces=false,
	stringstyle=\color{deepgreen}
}

\lstset{literate=%
	{Ö}{{\"O}}1
	{Ä}{{\"A}}1
	{Ü}{{\"U}}1
	{ß}{{\ss}}1
	{ü}{{\"u}}1
	{ä}{{\"a}}1
	{ö}{{\"o}}1
}

% Neue Klassenarbeits-Umgebung
\newenvironment{worksheet}[3]
% Begin-Bereich
{
	\newpage
	\sffamily
	\setcounter{page}{1}
	\ClearShipoutPicture
	\AddToShipoutPicture{
		\put(55,761){{
				\mbox{\parbox{385\unitlength}{\tiny \color{codegray}BBS I Mainz, #1 \newline #2
						\newline #3
					}
				}
			}
		}
		\put(455,761){{
				\mbox{\hspace{0.3cm}\includegraphics[width=0.2\textwidth]{../../logo.jpg}}
			}
		}
	}
}
% End-Bereich
{
	\clearpage
	\ClearShipoutPicture
}

\geometry{left=2.50cm,right=2.50cm,top=3.00cm,bottom=1.00cm,includeheadfoot}

\begin{document}
	\begin{worksheet}{BGY 17}{Klassenstufe 12 - Informationsverarbeitung}{Lernabschnitt 1: Bedingte Anweisungen}
				
		\noindent
		\sffamily
		\pagestyle{empty}
		\textbf{Gruppe 1}
		\begin{lstlisting}[style=JavaInputStyle]
			//rand.nextInt() liefert eine zufällige Zahl zurück
			int zahl1 = rand.nextInt(), zahl2 = rand.nextInt();
			//Die Scanner-Klasse ermöglicht das Einlesen von Nutzereingaben.
			Scanner input = new Scanner(System.in);
			
			//Wenn zahl1 >= zahl2 muss der Nutzer eine ganzzahlige Eingabe tätigen. Diese wird mit zahl2 multipliziert. Ist hingegen zahl1 < zahl2 wird nach einer ganzzahlige Eingabe verlangt, die mit zahl1 multipliziert wird.
			//Danach wird erneut geprüft, ob zahl1 >= zahl2. Ist die Bedingung erfüllt wird die Summe der beiden Zahlen ausgegeben.
			if (zahl1 >= zahl2) {;
				System.out.print("Geben Sie einen ganzzahligen Wert ein:");
				int eingabe = input.nextInt();
				zahl2 = zahl2*eingabe;
			}
			else if (zahl1 < zahl2){
				System.out.print("Geben Sie bitte eine ganze Zahl ein:");
				int eingabe = input.nextInt();
				zahl1 = zahl1*eingabe;
				
			if (zahl1 >= zahl2)
				System.out.println(zahl1 + zahl2);
			}
		\end{lstlisting}
		\newpage
		\textbf{Gruppe 2}
		\begin{lstlisting}[style=JavaInputStyle]
			//rand.nextInt() liefert eine zufällige Zahl zurück
			int zahl1 = rand.nextInt(), zahl2 = rand.nextInt();
			
			//Wenn zahl1 < zahl2 wird Hallo ausgegeben, ansonsten wird eine neue zufällige Zahl an zahl2 zugewiesen.
			//Danach wird erneut geprüft, ob zahl1 < zahl2. Ist die Bedingung erfüllt wird die Summe der beiden Zahlen ausgegeben.
			//Ist hingegen zahl1 größer oder gleich zahl2, so wird eine dritte Zahlvariable, zahl3, erzeugt. Diese bekommt die Summe aus zahl1 und zahl2 zugewiesen.
			//Ist die Hälfte von zahl3 kleiner als zahl2, wird zahl2 ausgegeben.
			if (zahl1 < zahl2); {
				System.out.println("Hallo!");
			}
			
			zahl2 = rand.nextInt();
			
			if (zahl1 < zahl2)
				System.out.println(zahl1 + zahl2);
			}
			else if {zahl1 => zahl2) 
				int zahl3 = zahl1 + zahl2;
				if (zahl3/2 < zahl2});
					System.out.println(zahl2);
				}
			
		\end{lstlisting}
		\newpage
		\textbf{Gruppe 3}
		\begin{lstlisting}[style=JavaInputStyle]
			//rand.nextInt() liefert eine zufällige Zahl zurück
			int zahl1 = rand.nextInt(), zahl2 = rand.nextInt();
			
			//Wenn zahl1 < zahl2 und zahl1 den Wert 5 hat, wird nach dem Namen gefragt. Ist das nicht der Fall aber zahl1 + zahl2 ist negativ, werden beiden Zahlen neue Werte zugewiesen.
			//Ansonsten wird eine neue zufällige Zahl an zahl2 zugewiesen.
			if (zahl1 < zahl2, zahl1 == 5){
				System.out.println("Wie heißt du?");
			}
			if (zahl1 + zahl2 <0) }
				zahl1 = rand.nextInt();
				zahl2 = rand.nextInt();
				
			else {
				zahl2 = rand.nextInt();
			}
		\end{lstlisting}
		\newpage
		\textbf{Gruppe 4}
		\begin{lstlisting}[style=JavaInputStyle]
			//rand.nextInt() liefert eine zufällige Zahl zurück
			int zahl1 = rand.nextInt(), zahl2 = rand.nextInt();
			
			//Wenn zahl1 < zahl2 und zahl1 < 100000, wird zahl1 ausgegeben.
			//Sonst wird geprüft ob zahl2 < 200000, dann wird zahl2 eine neue zufällige Zahl zugewiesen.
			//Ist keine der beiden Bedingungen erfüllt, wird eine neue Variable zahl3 erzeugt.
			if (zahl1 < zahl2, zahl1 < 100000){
				System.out.println(zahl1);
				
			else {
				int zahl3;
			}
			else if{ (zahl2 < 200000)
				zahl2 = rand.nextInt();
			}
		\end{lstlisting}
		\newpage
		\textbf{Gruppe 5}
		\begin{lstlisting}[style=JavaInputStyle]
			//rand.nextInt() liefert eine zufällige Zahl zurück
			int zahl1 = rand.nextInt(), zahl2 = rand.nextInt(), zahl3 = rand.nextInt();
			
			//Wenn zahl1 + zahl2 < zahl3, wird zahl3 ausgegeben
			//Sonst wird geprüft, ob zahl1 + zahl3 < zahl2. Dann wird zahl2 ausgegeben.
			//Ansonsten prüfen wir, ob zahl2 + zahl3 < zahl1. Ist das der Fall, wird zahl2 und zahl3 ausgegeben
			if (zahl1 + zahl2 < zahl3){
				System.out.println(zahl3);
			}
			else if{ (zahl1 + zahl3 < zahl2)
				System.out.println(zahl2);
			};
			else if{ zahl2 + zahl3 < zahl1;
				System.out.println(zahl2 + " und " + zahl3);
			}
		\end{lstlisting}
		\newpage
		\textbf{Gruppe 6}
		\begin{lstlisting}[style=JavaInputStyle]
			//rand.nextInt() liefert eine zufällige Zahl zurück
			int zahl1 = rand.nextInt(), zahl2 = rand.nextInt(), zahl3 = rand.nextInt();
			
			//Ist zahl1 > zahl2 und zahl2 < zahl3, werden zahl2 und zahl3 addiert und anschließend ausgegeben.
			//Sonst wird geprüft ob zahl1 < zahl2, dann werden zahl1 und eine zufällige Zahl addiert und es wird der String "Nicht richtig." ausgegeben.
			//Ist hingegen zahl2 < zahl3, fragen wir nach dem Alter.
			//Ansonsten verabschiedet sich das Programm
			if zahl1 > zahl2 && zahl2 < zahl3 {
				int temp = zahl2 + zahl3;
				System.out.println(temp);
			}
			else if (zahl1 < zahl2)
				int temp = zahl1 + rand.nextInt();
				System.out.println("Nicht richtig");
			else zahl2 < zahl3 {
				System.out.println("Wie alt bist du?");
			}
			else
				System.out.println("Auf Wiedersehen.");
		\end{lstlisting}
		\newpage
		\textbf{Gruppe 7}
		\begin{lstlisting}[style=JavaInputStyle]
			//rand.nextInt() liefert eine zufällige Zahl zurück
			int zahl1 = rand.nextInt(), zahl2 = rand.nextInt();
			
			//Ist zahl1*5 < zahl2, wird solange zahl2 ausgegeben, bis zahl3 als Summe der Durchlaufnummern (Summe der ersten n Zahlen) den Wert von zahl1 erreicht hat
			//Sollte die erste Bedingung nicht erfüllt sein, wird geprüft ob zahl2/5 < zahl2 ist. Ist das der Fall, wird eine zufällige Zahl ausgegeben.
			if (zahl1*5 < zahl2) {
				int i = 0, counter = 1;
				while (i < zahl2); {
					System.out.println(zahl2);
					i = i + counter;
					counter++;
			
			}
			else (zahl2/5 < zahl1) }
				System.out.println(rand.nextInt());
			}
		\end{lstlisting}
	\end{worksheet}
\end{document}