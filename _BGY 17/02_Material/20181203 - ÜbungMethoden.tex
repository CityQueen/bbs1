\documentclass[oneside,openany,headings=optiontotoc,11pt,numbers=noenddot]{scrreprt}

\usepackage[a4paper]{geometry}
\usepackage[utf8]{inputenc}
\usepackage[T1]{fontenc}
\usepackage{lmodern}
\usepackage[ngerman]{babel}
\usepackage{ngerman}

\usepackage[onehalfspacing]{setspace}

\usepackage{fancyhdr}
\usepackage{fancybox}

\usepackage{rotating}
\usepackage{varwidth}


\usepackage{pdflscape}
\usepackage{graphicx}
\usepackage{graphbox}
\graphicspath{
	{Pics/PDFs/}
	{Pics/JPGs/}
	{Pics/PNGs/}
}
\usepackage{caption}
\usepackage{tabularx}
\usepackage{dashrule}
\usepackage{hhline}
\usepackage{multirow}
\usepackage{enumerate}
\usepackage[hidelinks]{hyperref}
\usepackage{listings}

\usepackage[table]{xcolor}
\usepackage{array}
\usepackage{enumitem,amssymb,amsmath}
\usepackage{interval}
\usepackage{stmaryrd}
\usepackage{polynom}
\usepackage{diagbox}
\usepackage{dashrule}
\usepackage{framed}
\usepackage{mdframed}
\usepackage{karnaugh-map}

\usepackage{blindtext}

\usepackage{eso-pic}

\usepackage{amssymb}
\usepackage{eurosym}
\pagestyle{headings}
\renewcommand{\headrulewidth}{0.2pt}
\renewcommand{\footrulewidth}{0.2pt}
\newcommand*{\underdownarrow}[2]{\ensuremath{\underset{\overset{\Big\downarrow}{#2}}{#1}}}
\setlength{\fboxsep}{5pt}

% Codestyle defined
\definecolor{codegreen}{rgb}{0,0.6,0}
\definecolor{codegray}{rgb}{0.5,0.5,0.5}
\definecolor{codepurple}{rgb}{0.58,0,0.82}
\definecolor{backcolour}{rgb}{0.95,0.95,0.92}
\definecolor{deepgreen}{rgb}{0,0.5,0}
\definecolor{darkblue}{rgb}{0,0,0.65}
\definecolor{mauve}{rgb}{0.40, 0.19,0.28}
\colorlet{exceptioncolour}{yellow!50!red}
\colorlet{commandcolour}{blue!60!black}
\colorlet{numpycolour}{blue!60!green}
\colorlet{specmethodcolour}{violet}

%Neue Spaltendefinition
\newcolumntype{L}[1]{>{\raggedright\let\newline\\\arraybackslash\hspace{0pt}}m{#1}}
\newcolumntype{M}[1]{>{\centering\arraybackslash}X}
\newcommand{\cmnt}[1]{\ignorespaces}
%Textausrichtung ändern
\newcommand\tabrotate[1]{\rotatebox{90}{\raggedright#1\hspace{\tabcolsep}}}

%Intervall-Konfig
\intervalconfig {
	soft open fences
}

%Bash
\lstdefinestyle{BashInputStyle}{
	language=bash,
	basicstyle=\small\sffamily,
	backgroundcolor=\color{backcolour},
	columns=fullflexible,
	backgroundcolor=\color{backcolour},
	breaklines=true,
}
%Java
\lstdefinestyle{JavaInputStyle}{
	language=Java,
	backgroundcolor=\color{backcolour},
	aboveskip=1mm,
	belowskip=1mm,
	showstringspaces=false,
	columns=flexible,
	basicstyle={\footnotesize\ttfamily},
	numberstyle={\tiny},
	numbers=none,
	keywordstyle=\color{purple},,
	commentstyle=\color{deepgreen},
	stringstyle=\color{blue},
	emph={out},
	emphstyle=\color{darkblue},
	emph={[2]rand},
	emphstyle=[2]\color{specmethodcolour},
	breaklines=true,
	breakatwhitespace=true,
	tabsize=2,
}
%Python
\lstdefinestyle{PythonInputStyle}{
	language=Python,
	alsoletter={1234567890},
	aboveskip=1ex,
	basicstyle=\footnotesize,
	breaklines=true,
	breakatwhitespace= true,
	backgroundcolor=\color{backcolour},
	commentstyle=\color{red},
	otherkeywords={\ , \}, \{, \&,\|},
	emph={and,break,class,continue,def,yield,del,elif,else,%
		except,exec,finally,for,from,global,if,import,in,%
		lambda,not,or,pass,print,raise,return,try,while,assert},
	emphstyle=\color{exceptioncolour},
	emph={[2]True,False,None,min},
	emphstyle=[2]\color{specmethodcolour},
	emph={[3]object,type,isinstance,copy,deepcopy,zip,enumerate,reversed,list,len,dict,tuple,xrange,append,execfile,real,imag,reduce,str,repr},
	emphstyle=[3]\color{commandcolour},
	emph={[4]ode, fsolve, sqrt, exp, sin, cos, arccos, pi,  array, norm, solve, dot, arange, , isscalar, max, sum, flatten, shape, reshape, find, any, all, abs, plot, linspace, legend, quad, polyval,polyfit, hstack, concatenate,vstack,column_stack,empty,zeros,ones,rand,vander,grid,pcolor,eig,eigs,eigvals,svd,qr,tan,det,logspace,roll,mean,cumsum,cumprod,diff,vectorize,lstsq,cla,eye,xlabel,ylabel,squeeze},
	emphstyle=[4]\color{numpycolour},
	emph={[5]__init__,__add__,__mul__,__div__,__sub__,__call__,__getitem__,__setitem__,__eq__,__ne__,__nonzero__,__rmul__,__radd__,__repr__,__str__,__get__,__truediv__,__pow__,__name__,__future__,__all__},
	emphstyle=[5]\color{specmethodcolour},
	emph={[6]assert,range,yield},
	emphstyle=[6]\color{specmethodcolour}\bfseries,
	emph={[7]Exception,NameError,IndexError,SyntaxError,TypeError,ValueError,OverflowError,ZeroDivisionError,KeyboardInterrupt},
	emphstyle=[7]\color{specmethodcolour}\bfseries,
	emph={[8]taster,send,sendMail,capture,check,noMsg,go,move,switch,humTem,ventilate,buzz},
	emphstyle=[8]\color{blue},
	keywordstyle=\color{blue}\bfseries,
	rulecolor=\color{black!40},
	showstringspaces=false,
	stringstyle=\color{deepgreen}
}

\lstset{literate=%
	{Ö}{{\"O}}1
	{Ä}{{\"A}}1
	{Ü}{{\"U}}1
	{ß}{{\ss}}1
	{ü}{{\"u}}1
	{ä}{{\"a}}1
	{ö}{{\"o}}1
}

% Neue Klassenarbeits-Umgebung
\newenvironment{worksheet}[3]
% Begin-Bereich
{
	\newpage
	\sffamily
	\setcounter{page}{1}
	\ClearShipoutPicture
	\AddToShipoutPicture{
		\put(55,761){{
				\mbox{\parbox{385\unitlength}{\tiny \color{codegray}BBS I Mainz, #1 \newline #2
						\newline #3
					}
				}
			}
		}
		\put(455,761){{
				\mbox{\hspace{0.3cm}\includegraphics[width=0.2\textwidth]{../../logo.jpg}}
			}
		}
	}
}
% End-Bereich
{
	\clearpage
	\ClearShipoutPicture
}

\geometry{left=2.50cm,right=2.50cm,top=3.00cm,bottom=1.00cm,includeheadfoot}

\begin{document}
	\begin{worksheet}{BGY 17}{Klassenstufe 12 - Informationsverarbeitung}{Lernabschnitt 1: Wiederholungs- und Bedingte Anweisung}
				
		\noindent
		\sffamily
		\begin{tabularx}{\textwidth}{Xr}
			\textbf{Datum:} \underline{03.12.2018} & \textbf{Abgabe bis:} \underline{10.12.2018 11:30}
		\end{tabularx}
		\par\noindent
		\rule{\textwidth}{0.1pt}\\
		Realisieren Sie die nachfolgenden Aufgaben unter Verwendung des Methoden- bzw. Prozedurkonzepts (siehe dazu 03-Methodenkonzept).\\
		\par\noindent
		\underline{\textit{Hinweis:}} Erfinden Sie das Rad nicht neu. Nutzen Sie, wenn Sie es möchten, ihre Lösungen der Vorstunden wenn zur Realisierung.
		\begin{framed}
			\noindent
			\textbf{Bevor Sie eine Aufgabe beginnen...}\\
			Erstellen Sie zunächst eine Klasse mit dem Namen \lstinline[style=JavaInputStyle]|IhrName_uebung|.\\
			Innerhalb dieser Klasse lösen Sie \textbf{zwei} der nachfolgenden Aufgaben.\\
			\hdashrule{\textwidth}{0.1pt}{4pt}\\
			\par\noindent
			\textbf{Am Ende...}\\
			Ergänzen Sie ihr Hauptprogramm (also die \lstinline[style=JavaInputStyle]|public static void main(String[] args)|-Methode) um eine entsprechende Benutzereingabe, die abhängig vom eingegeben Wert eine ihrer Methoden/Prozeduren ausführt.\\
			\textit{Hinweis:} Sollte die aufgerufene Methode Parameter erwarten, lassen Sie diese durch den Benutzer eingeben.\\
			\rule{\textwidth}{1mm}\\
			\par\noindent
			\textbf{Erste Primzahl}\\
			Übernehmen Sie die folgende Rohform der Methode \lstinline[style=JavaInputStyle]|produkt| in die Klasse und ergänzen die Auslassung (...) so, dass de gewünschte Funktionalität gegeben ist.
			\begin{lstlisting}[style=JavaInputStyle]
				public static int allePrim(int zahl1, int zahl2){
				// Die Methode gibt die erste Primzahlen zwischen zahl1 und zahl2 als Ergebnis zurück.
				...}
			\end{lstlisting}
			\par\noindent
			\rule{\textwidth}{0.1pt}\\
			\par\noindent
			\textbf{Alle Rechtwinkligen}\\
			Programmieren Sie eine \lstinline[style=JavaInputStyle]|public static|-Prozedur mit dem Namen \lstinline[style=JavaInputStyle]|alleRechtwinkligen|. Die Prozedur besitzt keine Parameter.\\
			Die Ausgabe der Prozedur sind alle Ganzzahl-Tripel (a,b,c) für die folgende Bedingungen erfüllt sind:
			\begin{itemize}
				\item Jeder der drei Ganzzahlen \(a,\ b\) und \(c\) liegt zwischen 1 und 100 (einschliesslich).
				\item \(a\) ist größer oder gleich \(b\) und \(b\) ist größer oder gleich \(c\).
				\item Die drei Ganzzahlen repräsentieren die Seiten eines \textit{\textbf{rechtwinkligen Dreiecks}}, d.h. es gilt der \grq{}Satz des Pythagoras\grq{}, nämlich: \(a^2 = b^2 + c^2\).
			\end{itemize}
			Die Ausgabe des Tripels soll Zeilenweise folgende Form haben:\\
			\lstinline[style=JavaInputStyle]|Nr. 19: 50, 40, 30|\\
			\par\noindent
			\rule{\textwidth}{0.1pt}\\
			\par\noindent
			\textbf{Größter gemeinsamer Teiler}\\
			Entwerfen Sie eine \lstinline[style=JavaInputStyle]|public static|-Methode mit dem Rückgabetyp \lstinline[style=JavaInputStyle]|int| namens \lstinline[style=JavaInputStyle]|ggT|. Die Methode erhält zwei \lstinline[style=JavaInputStyle]|int|-Parameter mit den Bezeichnungen \lstinline[style=JavaInputStyle]|zahlA| und \lstinline[style=JavaInputStyle]|zahlB|.\\
			Die Methode liefert den \textit{größten gemeinsamen Teiler} der Variablenwerte von \lstinline[style=JavaInputStyle]|zahlA| und \lstinline[style=JavaInputStyle]|zahlB| zurück.\\
			\par\noindent
			\rule{\textwidth}{0.1pt}\\
			\par\noindent
			\textbf{Ein vollständig gekürzter Bruch}\\
			Entwerfen Sie eine \lstinline[style=JavaInputStyle]|public static|-Prozedur mit der Bezeichnung \lstinline[style=JavaInputStyle]|Bruch|. Die Prozedur erhält zwei \lstinline[style=JavaInputStyle]|int|-Parameter mit den Namen \lstinline[style=JavaInputStyle]|zaehler| und \lstinline[style=JavaInputStyle]|nenner|.\\
			Die Prozedur soll zunächst den \textit{größten gemeinsamen Teiler} der beiden Zahlen bestimmen und diesen Verwenden, um den \textbf{vollständig gekürzten Bruch} auf der Konsole auszugeben.
			\rule{\textwidth}{0.1pt}\\
			\par\noindent
			\textbf{Prim-Pärchen}\\
			als Prim-Pärchen oder auch \textit{Prim-Doublette} bezeichnet werden zwei Primzahlen bezeichnet, deren Differenz gleich \(2\) ist. z.B. \(3\ \text{und}\ 5\), \(9\ \text{und}\ 11\) oder \(1019\ \text{und}\ 1021\).\\
			Programmieren Sie eine \lstinline[style=JavaInputStyle]|public static int|-Methode namens \lstinline[style=JavaInputStyle]|primPaar|. Die Methode erhält einen \lstinline[style=JavaInputStyle]|int|-Parameter mit der Bezeichnung \lstinline[style=JavaInputStyle]|min|.\\
			\lstinline[style=JavaInputStyle]|primPaar| sucht von \lstinline[style=JavaInputStyle]|min| aufwärts nach solchen Prim-Pärchen.\\
			Wird eines gefunden, soll die kleinere der beiden Primzahlen zurückgegeben werden.\\
			Wird zwischen \lstinline[style=JavaInputStyle]|min| und der größten Zahl vom Typ \lstinline[style=JavaInputStyle]|int| (\lstinline[style=JavaInputStyle]|Integer.MAX_VALUE|) kein Pärchen gefunden, liefert \lstinline[style=JavaInputStyle]|primPaar| den Wert \lstinline[style=JavaInputStyle]|0| als Ergebnis.\\
		\end{framed}
	\end{worksheet}
\end{document}