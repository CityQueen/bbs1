\documentclass[11pt,twocolumn,oneside,openany,headings=optiontotoc,11pt,numbers=noenddot]{article}

\usepackage[a4paper]{geometry}
\usepackage[utf8]{inputenc}
\usepackage[T1]{fontenc}
\usepackage{lmodern}
\usepackage[ngerman]{babel}
\usepackage{ngerman}

\usepackage[onehalfspacing]{setspace}

\usepackage{fancyhdr}
\usepackage{fancybox}

\usepackage{rotating}
\usepackage{varwidth}


\usepackage{pdflscape}
\usepackage{graphicx}
\usepackage{graphbox}
\graphicspath{
	{Pics/PDFs/}
	{Pics/JPGs/}
	{Pics/PNGs/}
}
\usepackage{caption}
\usepackage{tabularx}
\usepackage{dashrule}
\usepackage{hhline}
\usepackage{multirow}
\usepackage{enumerate}
\usepackage[hidelinks]{hyperref}
\usepackage{listings}

\usepackage[table]{xcolor}
\usepackage{array}
\usepackage{enumitem,amssymb,amsmath}
\usepackage{interval}
\usepackage{stmaryrd}
\usepackage{polynom}
\usepackage{diagbox}
\usepackage{dashrule}
\usepackage{framed}
\usepackage{mdframed}
\usepackage{karnaugh-map}

\usepackage{blindtext}

\usepackage{eso-pic}

\usepackage{amssymb}
\usepackage{eurosym}
\pagestyle{headings}
\renewcommand{\headrulewidth}{0.2pt}
\renewcommand{\footrulewidth}{0.2pt}
\newcommand*{\underdownarrow}[2]{\ensuremath{\underset{\overset{\Big\downarrow}{#2}}{#1}}}
\setlength{\fboxsep}{5pt}

% Codestyle defined
\definecolor{codegreen}{rgb}{0,0.6,0}
\definecolor{codegray}{rgb}{0.5,0.5,0.5}
\definecolor{codepurple}{rgb}{0.58,0,0.82}
\definecolor{backcolour}{rgb}{0.95,0.95,0.92}
\definecolor{deepgreen}{rgb}{0,0.5,0}
\definecolor{darkblue}{rgb}{0,0,0.65}
\definecolor{mauve}{rgb}{0.40, 0.19,0.28}
\colorlet{exceptioncolour}{yellow!50!red}
\colorlet{commandcolour}{blue!60!black}
\colorlet{numpycolour}{blue!60!green}
\colorlet{specmethodcolour}{violet}

%Neue Spaltendefinition
\newcolumntype{L}[1]{>{\raggedright\let\newline\\\arraybackslash\hspace{0pt}}m{#1}}
\newcolumntype{M}[1]{>{\centering\arraybackslash}X}
\newcommand{\cmnt}[1]{\ignorespaces}
%Textausrichtung ändern
\newcommand\tabrotate[1]{\rotatebox{90}{\raggedright#1\hspace{\tabcolsep}}}

%Intervall-Konfig
\intervalconfig {
	soft open fences
}

%Bash
\lstdefinestyle{BashInputStyle}{
	language=bash,
	basicstyle=\small\sffamily,
	backgroundcolor=\color{backcolour},
	columns=fullflexible,
	backgroundcolor=\color{backcolour},
	breaklines=true,
}
%Java
\lstdefinestyle{JavaInputStyle}{
	language=Java,
	backgroundcolor=\color{backcolour},
	aboveskip=1mm,
	belowskip=1mm,
	showstringspaces=false,
	columns=flexible,
	basicstyle={\footnotesize\ttfamily},
	numberstyle={\tiny},
	numbers=none,
	keywordstyle=\color{purple},,
	commentstyle=\color{deepgreen},
	stringstyle=\color{blue},
	emph={out},
	emphstyle=\color{darkblue},
	emph={[2]rand},
	emphstyle=[2]\color{specmethodcolour},
	breaklines=true,
	breakatwhitespace=true,
	tabsize=2,
}
%Python
\lstdefinestyle{PythonInputStyle}{
	language=Python,
	alsoletter={1234567890},
	aboveskip=1ex,
	basicstyle=\footnotesize,
	breaklines=true,
	breakatwhitespace= true,
	backgroundcolor=\color{backcolour},
	commentstyle=\color{red},
	otherkeywords={\ , \}, \{, \&,\|},
	emph={and,break,class,continue,def,yield,del,elif,else,%
		except,exec,finally,for,from,global,if,import,in,%
		lambda,not,or,pass,print,raise,return,try,while,assert},
	emphstyle=\color{exceptioncolour},
	emph={[2]True,False,None,min},
	emphstyle=[2]\color{specmethodcolour},
	emph={[3]object,type,isinstance,copy,deepcopy,zip,enumerate,reversed,list,len,dict,tuple,xrange,append,execfile,real,imag,reduce,str,repr},
	emphstyle=[3]\color{commandcolour},
	emph={[4]ode, fsolve, sqrt, exp, sin, cos, arccos, pi,  array, norm, solve, dot, arange, , isscalar, max, sum, flatten, shape, reshape, find, any, all, abs, plot, linspace, legend, quad, polyval,polyfit, hstack, concatenate,vstack,column_stack,empty,zeros,ones,rand,vander,grid,pcolor,eig,eigs,eigvals,svd,qr,tan,det,logspace,roll,mean,cumsum,cumprod,diff,vectorize,lstsq,cla,eye,xlabel,ylabel,squeeze},
	emphstyle=[4]\color{numpycolour},
	emph={[5]__init__,__add__,__mul__,__div__,__sub__,__call__,__getitem__,__setitem__,__eq__,__ne__,__nonzero__,__rmul__,__radd__,__repr__,__str__,__get__,__truediv__,__pow__,__name__,__future__,__all__},
	emphstyle=[5]\color{specmethodcolour},
	emph={[6]assert,range,yield},
	emphstyle=[6]\color{specmethodcolour}\bfseries,
	emph={[7]Exception,NameError,IndexError,SyntaxError,TypeError,ValueError,OverflowError,ZeroDivisionError,KeyboardInterrupt},
	emphstyle=[7]\color{specmethodcolour}\bfseries,
	emph={[8]taster,send,sendMail,capture,check,noMsg,go,move,switch,humTem,ventilate,buzz},
	emphstyle=[8]\color{blue},
	keywordstyle=\color{blue}\bfseries,
	rulecolor=\color{black!40},
	showstringspaces=false,
	stringstyle=\color{deepgreen}
}

\lstset{literate=%
	{Ö}{{\"O}}1
	{Ä}{{\"A}}1
	{Ü}{{\"U}}1
	{ß}{{\ss}}1
	{ü}{{\"u}}1
	{ä}{{\"a}}1
	{ö}{{\"o}}1
}

% Neue Klassenarbeits-Umgebung
\newenvironment{worksheet}[3]
% Begin-Bereich
{
	\newpage
	\sffamily
	\setcounter{page}{1}
	\ClearShipoutPicture
	\AddToShipoutPicture{
		\put(55,761){{
				\mbox{\parbox{385\unitlength}{\tiny \color{codegray}BBS I Mainz, #1 \newline #2
						\newline #3
					}
				}
			}
		}
		\put(455,761){{
				\mbox{\hspace{0.3cm}\includegraphics[width=0.2\textwidth]{../../logo.jpg}}
			}
		}
	}
}
% End-Bereich
{
	\clearpage
	\ClearShipoutPicture
}

\setlength{\columnsep}{3em}
\setlength{\columnseprule}{0.5pt}

\geometry{left=1.00cm,right=1.00cm,top=3.00cm,bottom=1.00cm,includeheadfoot}
\pagenumbering{gobble}
\pagestyle{empty}

\begin{document}
	\begin{landscape}
		\renewcommand{\arraystretch}{1.5}
		\begin{tabularx}{45\baselineskip}{|M|X|X|X|}
			\hline
			\rowcolor{codegray!20} \textbf{Kompetenzbezug} & \textbf{B A S I C S} & \textbf{P R O F I} & \textbf{E X P E R T E}\\
			\hline
			\multirow{6}{*}{\tabrotate{Grundlagen}} & Ich kann zu einer Zufallssituation ein einstufiges bzw. mehrstufiges Baumdiagramm anlegen und beschriften. & Ich kann zu einem Ereignis die zugehörigen Ergebnisse (Pfade) nennen (markieren) bzw. Ergebnisse sinnvoll zu einem Ereignis zusammenfassen. & Ich kann den Erwartungswert im Hinblick auf die Situation interpretieren und Entscheidungsempfehlungen bezüglich des Einlassens auf das Zufallsexperiment aus mathematischer Sicht abgeben.\\
			& Ich kenne die Pfad- und Additionsregel. & \multirow{2}{\hsize}{Ich kann die Wahrscheinlichkeit eines Ereignisses durch Aufsummieren aller Wahrscheinlichkeiten der zugehörigen Ergebnisse berechnen} &  Ich kann einem Ereignis eine für die Anwendungssituation sinnvolle Zahl zuordnen.\\
			& Ich kann erläutern, was man unter einer Wahrscheinlichkeit für eine Ergebnis bzw. Ereignis versteht und wie Werte für eine Wahrscheinlichkeit zustande kommen. & & \\
			& & & \\
			& Ich kann die Begriffe Ereignis und Ergebnis anhand eines Baumdiagramms erläutern. & Ist die Verteilung einer Zufallsvariable tabellarisch dargestellt, kann ich den Erwartungswert berechnen. & \\
			& & & \\
			& Anstatt eines Baumdiagramms kann ich die Ergebnisse auch durch Klammern darstellen - z.B. (\(E, E, \overline{E}, \overline{E}, E\)). & \multirow{2}{\hsize}{Ich kann zu einem Ereignis ein Gegenereignis formulieren  insbesondere kenne ich das Gegenereignis zu \glqq{}mindestens ein Mal...\grqq{}.} & \\
			& Ich kann erläutern, was man unter einer Zufallsvariable und einer Wahrscheinlichkeitsverteilung versteht. & & \\
			\hline
		\end{tabularx}
	\end{landscape}
	\begin{landscape}
		\renewcommand{\arraystretch}{1.5}
		\begin{tabularx}{45\baselineskip}{|M|X|X|X|}
			\hline
			\rowcolor{codegray!20} \textbf{Kompetenzbezug} & \textbf{B A S I C S} & \textbf{P R O F I} & \textbf{E X P E R T E}\\
			\hline
			\multirow{6}{*}{\tabrotate{einstufige / mehrstufige Zufallsexperiments}} & Ich kann eine zufallsbehaftete Zituation als Zufallsexperiment auffassen und alle möglichen Ergebnisse angeben. & Ich kann eine zufallsbehaftete Situation als mehrstufiges Zufallsexperiment auffassen und die Stufen korrekt definieren. & Ich kann die Wahrscheinlichkeit eines Ereignisses durch Aufsummieren aller Wahrscheinlichkeiten der zugehörigen Ergebnisse berechnen.\\
			& Ich kann die Voraussetzungen für ein Zufallsexperiment nennen und ihr Vorliegen für eine gegebene Situation überprüfen. & Ich kann zu einem Ereignis die zugehörigen Ereignisse (Pfade) nennen (markieren). & Ich kann Wahrscheinlichkeitsverteilungen tabellarische darstellen.\\
			& Ich kann erläutern, was man unter einer Wahrscheinlichkeit für ein Ergebnis bzw. Ereignis versteht und wie Werte für eine Wahrscheinlichkeit zustande kommen können. & Ich kann auch ohne ein komplettes Baumdiagramm die Wahrscheinlichkeit der Ergebnisse \glqq{}kein Mal\grqq{} oder \glqq{}jedes Mal\grqq{} berechnen. & \\
			& Ich kann die Begriffe Ereignis und Ergebnis anhand eines Beispiels erläutern. & Ich kann zu einem Ereignis ein Gegenereignis formulieren. & \\
			& Ich kenne die Pfadregel und die Additionsregeln und kann sie anhand eines Baumdiagramms anwenden. & Ich kann durch das Gegenereignis die Wahrscheinlichkeit zu Ereignissen wie \glqq{}mindestens ein Mal \ldots\grqq{} berechnen. & \\
			& Ich kann einfache Laplace-Wahrscheinlichkeiten berechnen und begründen, dass man dieses Verfahren anwenden kann. & & \\
			\hline
		\end{tabularx}
	\end{landscape}
	\begin{landscape}
		\renewcommand{\arraystretch}{1.5}
		\begin{tabularx}{45\baselineskip}{|M|X|X|X|}
			\hline
			\rowcolor{codegray!20} \textbf{Kompetenzbezug} & \textbf{B A S I C S} & \textbf{P R O F I} & \textbf{E X P E R T E}\\
			\hline
			\multirow{1}{*}{\tabrotate{Zufallsvariable}} & Ich kann einem Ereignis eine zur Situation passende, sinnvolle Zahl zuordnen & Ich kann den Erwartungswert und die Varianz für eine gegebeen Wahrscheinlichkeitsverteilung einer Zufallsvariable berechnen. & Ich kann den Erwartungswert und die Varianz im Hinblick auf die Situation interpretieren und Entscheidungen - z.B. bezüglich Fairness eines Spiels - treffen.\\
			\hline
			\multirow{4}{*}{\tabrotate{Binomialverteilung}} & Ich kann eine Situation als Bernoulli-Experiment auffassen, indem ich definiere, was ein Treffer ist und den passenden Wert für n und p herauslese. & Ich kann Werte \(P(X=x_{ii})\) mit der Formel berechnen und die Bestandteile der Formel erläutern. & Ich kann ein 95\%-Vertrauensintervall für die Anzahl von Treffern durch eine Formel bestimmen.\\
			& Ich kenne die Bedingungen zum Vorliegen eines Bernoulli-Experimentes. & Ich kann Werte aus einer Tabelle zur Binomialverteilung ablesen und die Bedeutung erläutern. & Ich kann die Wahrscheinlichkeit, dass die Anzahl der Treffer in einem bestimmten Intervall liegt durch Subtraktion von Tabellenwerten bestimmen.\\
			& Ich kann den Zusammenhang zwischen einem Bernoulli-Experiment und der Binomialverteilung erläutern. & Ich kann beurteilen, ob die Bedingungen für ein Bernoulli-Experiment vorliegen und ggf. Kritikpunkte äußern. & Ich kann der Anzahl der Treffer sinnvolle Zahlen zuordnen, um einen zur Situation passenden Erwartungswert bestimmen.\\
			& & Ich kann die Wahrscheinlichkeit, dass die Anzahl der Treffer einen bestimmten Wert nicht übersteigt / unterschreitet mithilfe einer Tabelle bestimmen. & Ich kann die Anzahl der Stufe \glqq{}n\grqq{} bestimmen, damit mit einer bestimmten Wahrscheinlichkeit eine Mindesttrefferzahl vorliegt (Gegenereignis / In \ldots)\\
			\hline
		\end{tabularx}
	\end{landscape}
\end{document}