\documentclass[oneside,openany,headings=optiontotoc,11pt,numbers=noenddot]{scrreprt}

\usepackage[a4paper]{geometry}
\usepackage[utf8]{inputenc}
\usepackage[T1]{fontenc}
\usepackage{lmodern}
\usepackage[ngerman]{babel}
\usepackage{ngerman}

\usepackage[onehalfspacing]{setspace}

\usepackage{fancyhdr}
\usepackage{fancybox}

\usepackage{rotating}
\usepackage{varwidth}


\usepackage{pdflscape}
\usepackage{graphicx}
\usepackage{graphbox}
\graphicspath{
	{Pics/PDFs/}
	{Pics/JPGs/}
	{Pics/PNGs/}
}
\usepackage{caption}
\usepackage{tabularx}
\usepackage{dashrule}
\usepackage{hhline}
\usepackage{multirow}
\usepackage{enumerate}
\usepackage[hidelinks]{hyperref}
\usepackage{listings}

\usepackage[table]{xcolor}
\usepackage{array}
\usepackage{enumitem,amssymb,amsmath}
\usepackage{interval}
\usepackage{stmaryrd}
\usepackage{polynom}
\usepackage{diagbox}
\usepackage{dashrule}
\usepackage{framed}
\usepackage{mdframed}
\usepackage{karnaugh-map}

\usepackage{blindtext}

\usepackage{eso-pic}

\usepackage{amssymb}
\usepackage{eurosym}
\pagestyle{headings}
\renewcommand{\headrulewidth}{0.2pt}
\renewcommand{\footrulewidth}{0.2pt}
\newcommand*{\underdownarrow}[2]{\ensuremath{\underset{\overset{\Big\downarrow}{#2}}{#1}}}
\setlength{\fboxsep}{5pt}

% Codestyle defined
\definecolor{codegreen}{rgb}{0,0.6,0}
\definecolor{codegray}{rgb}{0.5,0.5,0.5}
\definecolor{codepurple}{rgb}{0.58,0,0.82}
\definecolor{backcolour}{rgb}{0.95,0.95,0.92}
\definecolor{deepgreen}{rgb}{0,0.5,0}
\definecolor{darkblue}{rgb}{0,0,0.65}
\definecolor{mauve}{rgb}{0.40, 0.19,0.28}
\colorlet{exceptioncolour}{yellow!50!red}
\colorlet{commandcolour}{blue!60!black}
\colorlet{numpycolour}{blue!60!green}
\colorlet{specmethodcolour}{violet}

%Neue Spaltendefinition
\newcolumntype{L}[1]{>{\raggedright\let\newline\\\arraybackslash\hspace{0pt}}m{#1}}
\newcolumntype{M}[1]{>{\centering\arraybackslash}X}
\newcommand{\cmnt}[1]{\ignorespaces}
%Textausrichtung ändern
\newcommand\tabrotate[1]{\rotatebox{90}{\raggedright#1\hspace{\tabcolsep}}}

%Intervall-Konfig
\intervalconfig {
	soft open fences
}

%Bash
\lstdefinestyle{BashInputStyle}{
	language=bash,
	basicstyle=\small\sffamily,
	backgroundcolor=\color{backcolour},
	columns=fullflexible,
	backgroundcolor=\color{backcolour},
	breaklines=true,
}
%Java
\lstdefinestyle{JavaInputStyle}{
	language=Java,
	backgroundcolor=\color{backcolour},
	aboveskip=1mm,
	belowskip=1mm,
	showstringspaces=false,
	columns=flexible,
	basicstyle={\footnotesize\ttfamily},
	numberstyle={\tiny},
	numbers=none,
	keywordstyle=\color{purple},,
	commentstyle=\color{deepgreen},
	stringstyle=\color{blue},
	emph={out},
	emphstyle=\color{darkblue},
	emph={[2]rand},
	emphstyle=[2]\color{specmethodcolour},
	breaklines=true,
	breakatwhitespace=true,
	tabsize=2,
}
%Python
\lstdefinestyle{PythonInputStyle}{
	language=Python,
	alsoletter={1234567890},
	aboveskip=1ex,
	basicstyle=\footnotesize,
	breaklines=true,
	breakatwhitespace= true,
	backgroundcolor=\color{backcolour},
	commentstyle=\color{red},
	otherkeywords={\ , \}, \{, \&,\|},
	emph={and,break,class,continue,def,yield,del,elif,else,%
		except,exec,finally,for,from,global,if,import,in,%
		lambda,not,or,pass,print,raise,return,try,while,assert},
	emphstyle=\color{exceptioncolour},
	emph={[2]True,False,None,min},
	emphstyle=[2]\color{specmethodcolour},
	emph={[3]object,type,isinstance,copy,deepcopy,zip,enumerate,reversed,list,len,dict,tuple,xrange,append,execfile,real,imag,reduce,str,repr},
	emphstyle=[3]\color{commandcolour},
	emph={[4]ode, fsolve, sqrt, exp, sin, cos, arccos, pi,  array, norm, solve, dot, arange, , isscalar, max, sum, flatten, shape, reshape, find, any, all, abs, plot, linspace, legend, quad, polyval,polyfit, hstack, concatenate,vstack,column_stack,empty,zeros,ones,rand,vander,grid,pcolor,eig,eigs,eigvals,svd,qr,tan,det,logspace,roll,mean,cumsum,cumprod,diff,vectorize,lstsq,cla,eye,xlabel,ylabel,squeeze},
	emphstyle=[4]\color{numpycolour},
	emph={[5]__init__,__add__,__mul__,__div__,__sub__,__call__,__getitem__,__setitem__,__eq__,__ne__,__nonzero__,__rmul__,__radd__,__repr__,__str__,__get__,__truediv__,__pow__,__name__,__future__,__all__},
	emphstyle=[5]\color{specmethodcolour},
	emph={[6]assert,range,yield},
	emphstyle=[6]\color{specmethodcolour}\bfseries,
	emph={[7]Exception,NameError,IndexError,SyntaxError,TypeError,ValueError,OverflowError,ZeroDivisionError,KeyboardInterrupt},
	emphstyle=[7]\color{specmethodcolour}\bfseries,
	emph={[8]taster,send,sendMail,capture,check,noMsg,go,move,switch,humTem,ventilate,buzz},
	emphstyle=[8]\color{blue},
	keywordstyle=\color{blue}\bfseries,
	rulecolor=\color{black!40},
	showstringspaces=false,
	stringstyle=\color{deepgreen}
}

\lstset{literate=%
	{Ö}{{\"O}}1
	{Ä}{{\"A}}1
	{Ü}{{\"U}}1
	{ß}{{\ss}}1
	{ü}{{\"u}}1
	{ä}{{\"a}}1
	{ö}{{\"o}}1
}

% Neue Klassenarbeits-Umgebung
\newenvironment{worksheet}[3]
% Begin-Bereich
{
	\newpage
	\sffamily
	\setcounter{page}{1}
	\ClearShipoutPicture
	\AddToShipoutPicture{
		\put(55,761){{
				\mbox{\parbox{385\unitlength}{\tiny \color{codegray}BBS I Mainz, #1 \newline #2
						\newline #3
					}
				}
			}
		}
		\put(455,761){{
				\mbox{\hspace{0.3cm}\includegraphics[width=0.2\textwidth]{../../logo.jpg}}
			}
		}
	}
}
% End-Bereich
{
	\clearpage
	\ClearShipoutPicture
}

\geometry{left=2.50cm,right=2.50cm,top=3.00cm,bottom=1.00cm,includeheadfoot}

\begin{document}
	\begin{worksheet}{BGY 16}{Mathematik - Lernbereich 3, Algebraisierung}{Übersicht Kursarbeit}
		
		\begin{framed}
			\noindent
			\tiny{\color{codegray} Vektorgrundlagen}\\
			\normalcolor\normalsize
			Wir haben zwei Vektoren \(\vec{a} = \left(\begin{matrix}{c}a_1\\a_2\\a_3\end{matrix}\right)\) und \(\vec{b} = \left(\begin{matrix}{c}b_1\\b_2\\b_3\end{matrix}\right)\) gegeben und erinnern uns an die verschiedenen Operationen, die wir mit diesen Vektoren ausführen können:
			\begin{itemize}
				\item[] \textbf{Länge eines Vektors}: \(|\vec{a}| = \sqrt{a_1^2 + a_2^2 + a_3^2}\)
				\item[] \textbf{Skalarprodukt zweier Vektoren}: \(\vec{a}*\vec{b} = a_1*b_1 + a_2*b_2 + a_3*b_3\)
				\item[] \textbf{Winkel \(\alpha\) zwischen zwei Vektoren}: Hier ist zu beachten, dass die Formel immer den Winkel \(\alpha < 90^\circ\) angibt. Außerdem müssen beide Vektoren von dem selben Punkt ausgehen.\\
				z.B. Das Ergebnis für den Winkel zwischen \(\vec{AB}\) und \(\vec{AC}\) wäre sinnvoll, wohingegen der Winkel zwischen \(\vec{AB}\) und \(\vec{CA}\) falsch wäre.\\
				\[\cos(\gamma) = \frac{\vec{a}*\vec{b}}{|\vec{a}|*|\vec{b}|} = \frac{a_1*b_1 + a_2*b_2 + a_3*b_3}{\sqrt{a_1^2 + a_2^2 + a_3^2}*\sqrt{b_1^2 + b_2^2 + b_3^2}}\]
				Da wir den Winkel benötigen, muss noch die Umkehrfunktion (\(\arccos\)) angewendet werden.
				\[\gamma = \arccos{\frac{\vec{a}*\vec{b}}{|\vec{a}|*|\vec{b}|}} = \arccos{\frac{a_1*b_1 + a_2*b_2 + a_3*b_3}{\sqrt{a_1^2 + a_2^2 + a_3^2}*\sqrt{b_1^2 + b_2^2 + b_3^2}}}\]
			\end{itemize}
		\end{framed}
		\begin{framed}
			\noindent
			\tiny{\color{codegray}Aufstellen von Geraden}\\
			\normalcolor\normalsize
			Um eine Gerade (\(g: \vec{x} = \underbrace{\vec{p}}_{Stützvektor} + r*\underbrace{\vec{u}}_{Richtungsvektor}\)) aufzustellen benötigt man zwei Informationen:
			\begin{itemize}
				\item[\tiny{(a)}] \textbf{Zwei} Punkte (A und B)
				\item[\tiny{(b)}] den \textbf{Stützvektor} \(\vec{p}\) sowie einen \textbf{weiteren Punkt} B
				\item[\tiny{(c)}] den \textbf{Richtungsvektor} \(\vec{u}\) sowie einen \textbf{weiteren Punkt} A
				\item[\tiny{(d)}] den \textbf{Stützvektor} \(\vec{p}\) und den \textbf{Richtungsvektor} \(\vec{u}\)
			\end{itemize}
			\textbf{(a)} Hat man \textbf{zwei} Punkte \(\mathbf{A (a_1|a_2|a_3)}\) und \(\mathbf{B (b_1|b_2|b_3)}\), müssen wir aus diesen den \textit{Stütz-} sowie den \textit{Richtungsvektor} aufstellen. Dies tun wir wie folgt:\\
			\par\noindent
			\begin{tabularx}{\textwidth}{ccc}
				\(\vec{p} = \underbrace{\left(\begin{array}{c}a_1\\a_2\\a_3\end{array}\right)}_{Stützvektor}\) & \(\vec{u} = \left(\begin{array}{c}b_1-a_1\\b_2-a_2\\b_3-a_3\end{array}\right) = \underbrace{\left(\begin{array}{c}u_1\\u_2\\u_3\end{array}\right)}_{Richtungsvektor}\) & \(\Rightarrow \mathbf{g:} \vec{x} \mathbf{= \left(\begin{array}{c}a_1\\a_2\\a_3\end{array}\right) + r*\left(\begin{array}{c}u_1\\u_2\\u_3\end{array}\right)}\)
			\end{tabularx}\\
			\par\bigskip\noindent
			\textbf{(b)} Hat man den \textbf{Stützvektor} \(\vec{p} = \left(\begin{array}{c}p_1\\p_2\\p_3\end{array}\right)\) und \textbf{einen} Punkt \(\mathbf{B (b_1|b_2|b_3)}\) müssen wir daraus den \textit{Richtungsvektor} bestimmen. Dies tun wir wie folgt:\\
			\par\noindent
			\begin{tabularx}{\textwidth}{cc}
				\(\vec{u} = \left(\begin{array}{c}b_1-p_1\\b_2-p_2\\b_3-p_3\end{array}\right) = \underbrace{\left(\begin{array}{c}u_1\\u_2\\u_3\end{array}\right)}_{Richtungsvektor}\) & \(\Rightarrow \mathbf{g:} \vec{x} \mathbf{= \left(\begin{array}{c}p_1\\p_2\\p_3\end{array}\right) + r*\left(\begin{array}{c}u_1\\u_2\\u_3\end{array}\right)}\)
			\end{tabularx}\\
			\par\bigskip\noindent
			\textbf{(c)} Hat man den \textbf{Richtungsvektor} \(\vec{u} = \left(\begin{array}{c}u_1\\u_2\\u_3\end{array}\right)\) und \textbf{einen} Punkt \(\mathbf{a (a_1|a_2|a_3)}\) benötigt man nur noch den \textit{Stützvektor}. Diesen erhält man wie folgt:\\
			\par\noindent
			\begin{tabularx}{\textwidth}{cc}
				\(\vec{p} = \underbrace{\left(\begin{array}{c}a_1\\a_2\\a_3\end{array}\right)}_{Stützvektor}\) & \(\Rightarrow \mathbf{g:} \vec{x} \mathbf{= \left(\begin{array}{c}a_1\\a_2\\a_3\end{array}\right) + r*\left(\begin{array}{c}u_1\\u_2\\u_3\end{array}\right)}\)
			\end{tabularx}\\
			\par\bigskip\noindent
			\textbf{(d)} Hat man den \textbf{Stützvektor} \(\vec{p} = \left(\begin{array}{c}p_1\\p_2\\p_3\end{array}\right)\) und den \textbf{Richtungsvektor} \(\vec{u} = \left(\begin{array}{c}u_1\\u_2\\u_3\end{array}\right)\) muss man diese nur noch in die Geradengleichung einsetzen:\\
			\par\noindent
			\[\Rightarrow \mathbf{g:} \vec{x} \mathbf{= \left(\begin{array}{c}p_1\\p_2\\p_3\end{array}\right) + r*\left(\begin{array}{c}u_1\\u_2\\u_3\end{array}\right)}\]
		\end{framed}
		\begin{framed}
			\noindent
			\tiny{\color{codegray}Ein weiterer Punkt auf der Geraden}\normalcolor\normalsize\\
			Gegeben ist eine Gerade der Form \(g: \vec{x} = \left(\begin{array}{c}p_1\\p_2\\p_3\end{array}\right) + r*\left(\begin{array}{c}u_1\\u_2\\u_3\end{array}\right)\).\\
			\par\bigskip\noindent
			Um einen weiteren Punkt auf dieser Geraden zu bestimmen, wählt man für den Parameter \(r \in \mathbb{R}\) einen Wert und bestimmt den entsprechenden Punkt P.\\
			\par\bigskip\noindent
			\textit{Beispiel}: \[g: \vec{x} = \left(\begin{array}{c}4\\-3\\2\end{array}\right) + r*\left(\begin{array}{c}2\\3\\-5\end{array}\right)\]
			Wir wählen \(r = 2\) und erhalten
			\[\vec{P} = \left(\begin{array}{c}4\\-3\\2\end{array}\right) + 2*\left(\begin{array}{c}2\\3\\-5\end{array}\right) = \left(\begin{array}{c}8\\3\\-8\end{array}\right)\]
			Da aber nach einem Punkt gesucht ist, müssen wir zu dem erhaltenen Vektor \(\vec{P}\) noch den Punkt P (8|3|-8) angeben.
		\end{framed}
		\begin{framed}
			\noindent
			\tiny{\color{codegray}Punktprüfung}\normalcolor\normalsize\\
			Gegeben ist eine Gerade der Form \(g: \vec{x} = \left(\begin{array}{c}p_1\\p_2\\p_3\end{array}\right) + r*\left(\begin{array}{c}u_1\\u_2\\u_3\end{array}\right)\) sowie ein Punkt P(\(P_1|P_2|P_3\)).\\
			Um zu prüfen, ob der Punkt P auf der gegebenen Gerade \(g\) liegt, setzen wir \(\vec{P} = g\) und bestimmen die jeweiligen Werte für \(r\).\\
			\par\bigskip\noindent
			\textit{Beispiel}: Gegeben ist \(g: \vec{x} = \left(\begin{array}{c}4\\-3\\2\end{array}\right) + r*\left(\begin{array}{c}2\\3\\-5\end{array}\right)\) und der Punkt P (\(6|0|-2\)).\\
			\begin{tabularx}{\textwidth}{Xl}
				\(\left(\begin{array}{c}6\\0\\-2\end{array}\right) = \left(\begin{array}{c}4\\-3\\2\end{array}\right) + r*\left(\begin{array}{c}2\\3\\-5\end{array}\right)\) & \(\Rightarrow \begin{array}{c}6 = 4+2r\\0 = -3+3r\\-2 = 2 -5r\end{array}\)\\
			\end{tabularx}
			\begin{tabularx}{\textwidth}{lll}
				\hline
				\(6 = 4+2r\) & |\(-4\)\\
				\(2 = 2r\)& | \(:2\)& \(\Rightarrow r = 1\)\\
				\\
				\(0 = -3 +3r\) & | \(+3\)\\
				\(3 = 3r\) & | \(:3\) & \(\Rightarrow r = 1\)\\
				\\
				\(-2 = 2-5r\) & |\(-2\)\\
				\(-4 = -5r\) & |\(:5\) & \(\Rightarrow r = \frac{4}{5}\)
			\end{tabularx}\\
			Für \(r\) haben wir zweimal den Wert \(1\) und einmal den Wert \(\frac{4}{5}\). Daher können wir folgern, dass der Punkt P \underline{nicht} auf der Geraden \(g\) liegt.\\
			\par\bigskip\noindent
			Erhalten wir hingegen \textbf{dreimal den gleichen Wert} für \(r\) berechnet, so liegt der Punkt P auf der Geraden \(g\).
		\end{framed}
		\begin{framed}
			\noindent
			\tiny{\color{codegray}Lage von Geraden}\normalcolor\normalsize\\
			Wir haben die zwei Geradengleichungen gegeben.
			\begin{center}
				\begin{tabularx}{\textwidth}{XlX}
					\(g: \vec{x} = \vec{p} + r\vec{u}\) & und & \(h: \vec{x} = \vec{q} + t\vec{v}\)
				\end{tabularx}
			\end{center}
			Für die gegenseitige Lage dieser zwei Geraden gilt folgendes: \(g\) und \(h\) \(\ldots\)
			\begin{framed}
				\noindent
				\begin{itemize}
					\item[+] \(\ldots\) haben \color{codegreen}genau einen\normalcolor{} Schnittpunkt, wenn die Vektorgleichung bzw. das dazugehörige Gleichungssystem \(\vec{p} + r\vec{u} = \vec{q} + t\vec{v}\) \underline{\textbf{eine}} Lösung besitzt. 
					\item[+] \(\ldots\) sind \color{codegreen}gleich\normalcolor{}, wenn die Vektorgleichung bzw. das dazugehörige Gleichungssystem \(\vec{p} + r\vec{u} = \vec{q} + t\vec{v}\) \underline{\textbf{unendlich viele}} Lösungen besitzt.
				\end{itemize}
			\end{framed}
			\begin{framed}
				\begin{itemize}
					\item[+] \(\ldots\) haben \color{red}keinen\normalcolor{} Schnittpunkt, wenn die Vektorgleichung bzw. das dazugehörige Gleichungssystem \(\vec{p} + r\vec{u} = \vec{q} + t\vec{v}\) \underline{\textbf{keine}} Lösungen besitzt.\\
					Sind ferner die Richtungsvektoren \(\vec{u}\) und \(\vec{v}\) \(\ldots\)
					\begin{itemize}
						\item[\(\circ_1\)] \(\ldots\) linear \color{blue}abhängig\normalcolor{}, so sind \(g\) und \(h\) \color{red}parallel\normalcolor{}
						\item[\(\circ_2\)] \(\ldots\) linear \color{blue}unabhängig\normalcolor{}, so sind \(g\) und \(h\) zueinander \color{red}windschief\normalcolor{}
					\end{itemize}
				\end{itemize}
			\end{framed}
			\noindent
			\textbf{Beispiel}: Gegeben sind die Geraden\\ \(g: \vec{x} = \left(\begin{array}{c}-6\\6\\10\end{array}\right) + r\left(\begin{array}{c}3.5\\-1.5\\0\end{array}\right)\) und \(h: \vec{x} = \left(\begin{array}{c}2\\-3\\8\end{array}\right) + t\left(\begin{array}{c}-0.5\\3\\1\end{array}\right)\).\\
			\par\noindent
			Um die gegenseitige Lage der beiden Geraden zu prüfen, müssen wir \(g=h\) setzen und das Gleichungssystem lösen.\\
			\begin{tabularx}{\textwidth}{llll}
				\(-6 +3.5r = 2 -0.5t\) & & I\\
				\(6 -1.5r = -3 +3t\) & & II\\
				\(10 = 8 +r\) & |\(-8\) & \(\Rightarrow r= 2\) & (*)\\
			\end{tabularx}\\
			\par\noindent
			Wir setzen nun (*) in die Gleichungen \textbf{I und II} ein und bestimmen jeweils \(t\).\\
			\par\noindent
			\begin{tabularx}{\textwidth}{lXll|lXll}
				I \(\xrightarrow{mit (*)}\) & \( -6+3.5*2 = 2-0.5t\) & & & II \(\xrightarrow{mit (*)}\) & \(6 -1.5*2 = -3 +3t\)\\
				&\(\Leftrightarrow -6+7 = 2-0.5t\) & |\(-2\) & & & \(\Leftrightarrow 6-3 = -3 +3t\) & | \(+3\)\\
				&\(\Leftrightarrow -1 = -0.5t\) & | \(:(-0.5)\) & & & \(\Leftrightarrow 6 = 3t\) & | \(:3\)\\
				&\(\Rightarrow t = 2\) & & & & \(\Rightarrow t = 2\)\\				
			\end{tabularx}\\
			\par\noindent
			Wir erhalten \underline{\textbf{eine}} Lösung. Um \color{codegreen}den\normalcolor{} Schnittpunkt zu berechnen, setzen wir die Werte von \(r\) bzw. \(t\) in die entsprechenden Geradengleichungen ein.\\
			\begin{center}
				\(-6 + 3.5*2= \underbrace{1}_{= p_1} = 2 - 0.5*2 \)\\
				\par\bigskip
				\(6 - 1.5*2 = \underbrace{3}_{= p_2} = -3 + 3*2\)\\
				\par\bigskip
				\(10 + 0*2 = \underbrace{10}_{= p_3} = 8+1*2\)\\
			\end{center}
			So ergibt sich ein Schnittpunkt bei SP \((1|3|10)\)\\
			Erhalten wir hingegen \underline{\textbf{keine}} eindeutige Lösungen für \(r\) und \(t\), können wir daraus schließen, dass sich \(g\) und \(h\) \color{red}keinen\normalcolor{} Schnittpunkt haben. Für eine genauere Aussage über die Lage müssen wir uns das Verhältnis der Richtungsvektoren anschauen.\\
			Dies tun wir anhand eines \textbf{Beispiels}.\\
			Gegeben sind die Geraden\\ \(g: \vec{x} = \left(\begin{array}{c}1\\2\\1\end{array}\right) + r\left(\begin{array}{c}2\\0\\1\end{array}\right)\) und \(h: \vec{x} = \left(\begin{array}{c}2\\3\\4\end{array}\right) + t\left(\begin{array}{c}0\\1\\-1\end{array}\right)\).\\
			\par\noindent
			Um die gegenseitige Lage der beiden Geraden zu prüfen, müssen wir \(g=h\) setzen und das Gleichungssystem lösen.\\
			\begin{tabularx}{\textwidth}{lll}
				\(1 +2r = 2 +0t\) & \(\Rightarrow r = \frac{1}{2}\) & (*)\\
				\(2 + 0r = 3 + 1t\) & \(\Rightarrow t = -1\) & (**)\\
				\(1 +r = 4 -1t\) & & I\\
			\end{tabularx}\\
			\par\noindent
			Wir prüfen mit I, ob unsere berechneten Werte von \(r\) (*) und \(t\) (**) korrekt sind.\\
			\par\noindent
			\begin{tabularx}{\textwidth}{lX}
				I \(\xrightarrow{mit (*)/(**)}\) & \(1 + \overbrace{\frac{1}{2}}^{r} = 1.5 \neq  3 = 4 - 1*\overbrace{1}^{t}\)				
			\end{tabularx}\\
			\par\noindent
			Wir erhalten also \underline{\textbf{keine}} Lösung für \(r\) und \(t\).\\
			Daher betrachten wir nun die Richtungsvektoren \(\vec{u} = \left(\begin{array}{c}2\\0\\1\end{array}\right)\) und \(\vec{v} = \left(\begin{array}{c}0\\1\\-1\end{array}\right)\).\\
			Zu klären ist die Frage: \textit{Existiert eine Zahl \(a\in\mathbb{R}\), so dass \(\mathbf{a*\vec{u} = \vec{v}}\)?}
			\begin{tabularx}{\textwidth}{ll}
				\(2a = 0\) & \(\Rightarrow a = 0\)\\
				\(0a = 1\) & \(\lightning\)\\
				\(1a = -1\) & \(\Rightarrow a = -1\)
			\end{tabularx}
			Es ergeben sich unterschiedliche bzw. widersrpüchliche Werte für \(r\), damit können wir folgern, dass es \underline{\textbf{kein}} solches \(a\) gibt. \(\vec{u}\) und \(\vec{v}\) also linear \color{blue}unabhängig\normalcolor{} und somit \(g\) und \(h\) \color{red}windschief\normalcolor{} sind.\\
		\end{framed}
		\begin{framed}
			\noindent
			\tiny{\color{codegray}Sich schneidende/Gleiche/Parallele/Windschiefe Geraden aufstellen}\\\normalsize
			\underline{Gegeben} ist eine Gerade mit folgender Form \(g: \vec{x} = \left(\begin{array}{c}p_1\\p_2\\p_3\end{array}\right) + r*\left(\begin{array}{c}u_1\\u_2\\u_3\end{array}\right)\).\\
			\par\noindent
			\underline{Gesucht} ist jeweils eine Gerade ...
			\begin{itemize}
				\item[+] ...\(h\), die \(g\) \color{codegreen}einen\normalcolor{} gemeinsamen Schnittpunkt hat
				\item[+] ...\(i\), die \color{codegreen}gleich\normalcolor{} \(g\) ist
				\item[+] ...\(j\), die \color{red}parallel\normalcolor{} zu \(g\) verläuft
				\item[+] ...\(k\), die zur Geraden \(g\) \color{red}windschief\normalcolor{} ist
			\end{itemize}
			\par\bigskip\noindent
			\underline{\textit{\(h\) hat \color{codegreen}einen\normalcolor{} gemeinsamen Schnittpunkt mit \(g\)}}:\\
			Als gemeinsamen Schnittpunkt wählen wir den Stützvektor von \(g\). Dieser wird auch Stützvektor von \(h\). Nun müssen wir noch gewährleisten, dass die Richtungsvektoren linear \color{blue}unabhängig\normalcolor{} sind. Dafür ändern wir lediglich ein Vorzeichen im Richtungsvektor \(\vec{u}\) von \(g\) und erhalten so den Richtungsvektor \(\vec{v}\) von \(h\).\\
			\(\Rightarrow h: \vec{x} = \left(\begin{array}{c}p_1\\p_2\\p_3\end{array}\right) + s*\left(\begin{array}{c}u_1\\-u_2\\u_3\end{array}\right)\)\\
			\par\bigskip\noindent
			\underline{\textit{\(i\) ist \color{codegreen}gleich\normalcolor{} \(g\)}}:\\
			Als einen gemeinsamen Punkt wählen wir den Stützvektor von \(g\). Dieser wird auch Stützvektor von \(i\). Nun müssen wir noch gewährleisten, dass die Richtungsvektoren linear \color{blue}abhängig\normalcolor{} sind. Dafür ändern wir die einzelnen Komponenten des Richtungsvektors \(\vec{u}\) von \(g\) im gleichen Verhältnis und erhalten so den Richtungsvektor \(\vec{v}\) von \(i\).\\
			\(\Rightarrow i: \vec{x} = \left(\begin{array}{c}p_1\\p_2\\p_3\end{array}\right) + t*\left(\begin{array}{c}2*u_1\\2*u_2\\2*u_3\end{array}\right)\)\\
			\par\bigskip\noindent
			\underline{\textit{\(j\) ist \color{red}parallel\normalcolor{} zu \(g\)}}:\\
			Für den Stützvektor von \(j\) benötigen wir einen Punkt, der \underline{nicht} auf \(g\) liegt. Hierfür können wir eine Komponente des Stützvektors \(\vec{p}\) von \(g\) ändern.\\ Da die Richtungsvektoren bei \color{red}parallelen\normalcolor{} Geraden linear \color{blue}abhängig\normalcolor{} sind, können wir den Richtungsvektor \(\vec{u}\) von \(g\) oder ein Vielfaches davon als Richtungsvektor \(\vec{v}\) von \(j\) verwenden.\\
			\(\Rightarrow i: \vec{x} = \left(\begin{array}{c}p_1+2\\p_2\\p_3\end{array}\right) + w*\left(\begin{array}{c}u_1\\u_2\\u_3\end{array}\right)\)\\
			\par\bigskip\noindent
			\underline{\textit{\(j\) ist \color{red}windschief\normalcolor{} zu \(g\)}}:\\
			Für den Stützvektor von \(j\) benötigen wir einen Punkt, der \underline{nicht} auf \(g\) liegt. Hierfür können wir eine Komponente des Stützvektors \(\vec{p}\) von \(g\) ändern.\\ Da die Richtungsvektoren bei \color{red}windschiefen\normalcolor{} Geraden linear \color{blue}unabhängig\normalcolor{} sind, ändern wir eine Komponente des Richtungsvektors \(\vec{u}\) von \(g\) und erhalten den Richtungsvektor \(\vec{v}\) von \(i\) verwenden.\\
			\(\Rightarrow j: \vec{x} = \left(\begin{array}{c}p_1\\p_2\\p_3-1\end{array}\right) + t*\left(\begin{array}{c}u_1\\u_2\\4*u_3\end{array}\right)\)\\
		\end{framed}
	\end{worksheet}
\end{document}