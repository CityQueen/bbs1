\documentclass[11pt,twocolumn,oneside,openany,headings=optiontotoc,11pt,numbers=noenddot]{article}

\usepackage[a4paper]{geometry}
\usepackage[utf8]{inputenc}
\usepackage[T1]{fontenc}
\usepackage{lmodern}
\usepackage[ngerman]{babel}
\usepackage{ngerman}

\usepackage[onehalfspacing]{setspace}

\usepackage{fancyhdr}
\usepackage{fancybox}

\usepackage{rotating}
\usepackage{varwidth}


\usepackage{pdflscape}
\usepackage{graphicx}
\usepackage{graphbox}
\graphicspath{
	{Pics/PDFs/}
	{Pics/JPGs/}
	{Pics/PNGs/}
}
\usepackage{caption}
\usepackage{tabularx}
\usepackage{dashrule}
\usepackage{hhline}
\usepackage{multirow}
\usepackage{enumerate}
\usepackage[hidelinks]{hyperref}
\usepackage{listings}

\usepackage[table]{xcolor}
\usepackage{array}
\usepackage{enumitem,amssymb,amsmath}
\usepackage{interval}
\usepackage{stmaryrd}
\usepackage{polynom}
\usepackage{diagbox}
\usepackage{dashrule}
\usepackage{framed}
\usepackage{mdframed}
\usepackage{karnaugh-map}

\usepackage{blindtext}

\usepackage{eso-pic}

\usepackage{amssymb}
\usepackage{eurosym}
\pagestyle{headings}
\renewcommand{\headrulewidth}{0.2pt}
\renewcommand{\footrulewidth}{0.2pt}
\newcommand*{\underdownarrow}[2]{\ensuremath{\underset{\overset{\Big\downarrow}{#2}}{#1}}}
\setlength{\fboxsep}{5pt}

% Codestyle defined
\definecolor{codegreen}{rgb}{0,0.6,0}
\definecolor{codegray}{rgb}{0.5,0.5,0.5}
\definecolor{codepurple}{rgb}{0.58,0,0.82}
\definecolor{backcolour}{rgb}{0.95,0.95,0.92}
\definecolor{deepgreen}{rgb}{0,0.5,0}
\definecolor{darkblue}{rgb}{0,0,0.65}
\definecolor{mauve}{rgb}{0.40, 0.19,0.28}
\colorlet{exceptioncolour}{yellow!50!red}
\colorlet{commandcolour}{blue!60!black}
\colorlet{numpycolour}{blue!60!green}
\colorlet{specmethodcolour}{violet}

%Neue Spaltendefinition
\newcolumntype{L}[1]{>{\raggedright\let\newline\\\arraybackslash\hspace{0pt}}m{#1}}
\newcolumntype{M}[1]{>{\centering\arraybackslash}X}
\newcommand{\cmnt}[1]{\ignorespaces}
%Textausrichtung ändern
\newcommand\tabrotate[1]{\rotatebox{90}{\raggedright#1\hspace{\tabcolsep}}}

%Intervall-Konfig
\intervalconfig {
	soft open fences
}

%Bash
\lstdefinestyle{BashInputStyle}{
	language=bash,
	basicstyle=\small\sffamily,
	backgroundcolor=\color{backcolour},
	columns=fullflexible,
	backgroundcolor=\color{backcolour},
	breaklines=true,
}
%Java
\lstdefinestyle{JavaInputStyle}{
	language=Java,
	backgroundcolor=\color{backcolour},
	aboveskip=1mm,
	belowskip=1mm,
	showstringspaces=false,
	columns=flexible,
	basicstyle={\footnotesize\ttfamily},
	numberstyle={\tiny},
	numbers=none,
	keywordstyle=\color{purple},,
	commentstyle=\color{deepgreen},
	stringstyle=\color{blue},
	emph={out},
	emphstyle=\color{darkblue},
	emph={[2]rand},
	emphstyle=[2]\color{specmethodcolour},
	breaklines=true,
	breakatwhitespace=true,
	tabsize=2,
}
%Python
\lstdefinestyle{PythonInputStyle}{
	language=Python,
	alsoletter={1234567890},
	aboveskip=1ex,
	basicstyle=\footnotesize,
	breaklines=true,
	breakatwhitespace= true,
	backgroundcolor=\color{backcolour},
	commentstyle=\color{red},
	otherkeywords={\ , \}, \{, \&,\|},
	emph={and,break,class,continue,def,yield,del,elif,else,%
		except,exec,finally,for,from,global,if,import,in,%
		lambda,not,or,pass,print,raise,return,try,while,assert},
	emphstyle=\color{exceptioncolour},
	emph={[2]True,False,None,min},
	emphstyle=[2]\color{specmethodcolour},
	emph={[3]object,type,isinstance,copy,deepcopy,zip,enumerate,reversed,list,len,dict,tuple,xrange,append,execfile,real,imag,reduce,str,repr},
	emphstyle=[3]\color{commandcolour},
	emph={[4]ode, fsolve, sqrt, exp, sin, cos, arccos, pi,  array, norm, solve, dot, arange, , isscalar, max, sum, flatten, shape, reshape, find, any, all, abs, plot, linspace, legend, quad, polyval,polyfit, hstack, concatenate,vstack,column_stack,empty,zeros,ones,rand,vander,grid,pcolor,eig,eigs,eigvals,svd,qr,tan,det,logspace,roll,mean,cumsum,cumprod,diff,vectorize,lstsq,cla,eye,xlabel,ylabel,squeeze},
	emphstyle=[4]\color{numpycolour},
	emph={[5]__init__,__add__,__mul__,__div__,__sub__,__call__,__getitem__,__setitem__,__eq__,__ne__,__nonzero__,__rmul__,__radd__,__repr__,__str__,__get__,__truediv__,__pow__,__name__,__future__,__all__},
	emphstyle=[5]\color{specmethodcolour},
	emph={[6]assert,range,yield},
	emphstyle=[6]\color{specmethodcolour}\bfseries,
	emph={[7]Exception,NameError,IndexError,SyntaxError,TypeError,ValueError,OverflowError,ZeroDivisionError,KeyboardInterrupt},
	emphstyle=[7]\color{specmethodcolour}\bfseries,
	emph={[8]taster,send,sendMail,capture,check,noMsg,go,move,switch,humTem,ventilate,buzz},
	emphstyle=[8]\color{blue},
	keywordstyle=\color{blue}\bfseries,
	rulecolor=\color{black!40},
	showstringspaces=false,
	stringstyle=\color{deepgreen}
}

\lstset{literate=%
	{Ö}{{\"O}}1
	{Ä}{{\"A}}1
	{Ü}{{\"U}}1
	{ß}{{\ss}}1
	{ü}{{\"u}}1
	{ä}{{\"a}}1
	{ö}{{\"o}}1
}

% Neue Klassenarbeits-Umgebung
\newenvironment{worksheet}[3]
% Begin-Bereich
{
	\newpage
	\sffamily
	\setcounter{page}{1}
	\ClearShipoutPicture
	\AddToShipoutPicture{
		\put(55,761){{
				\mbox{\parbox{385\unitlength}{\tiny \color{codegray}BBS I Mainz, #1 \newline #2
						\newline #3
					}
				}
			}
		}
		\put(455,761){{
				\mbox{\hspace{0.3cm}\includegraphics[width=0.2\textwidth]{../../logo.jpg}}
			}
		}
	}
}
% End-Bereich
{
	\clearpage
	\ClearShipoutPicture
}

\setlength{\columnsep}{3em}
\setlength{\columnseprule}{0.5pt}

\geometry{left=2.00cm,right=2.00cm,top=3.00cm,bottom=1.00cm,includeheadfoot}
\pagenumbering{gobble}
\pagestyle{empty}

\begin{document}
	\begin{worksheet}{BS FI}{1. Lehrjahr, LF 4 - Einfache IT-Systeme}{Digitaltechnik}
		\setcounter{section}{7}
		\subsubsection*{\underline{Begrifflichkeiten}}
		\textbf{Einschlägiger Index} bezeichnet die Zeilennummer in der Funktionstabelle. Dabei ist zu beachten, dass die Variablenbelegung die Binärcodierung der Zeilennummer ist.\\
		\par\noindent
		Geben wir die Variablenbelegung für die \(1\)-Zeilen mit \(\wedge\) an, so nennt man diese \textbf{Minterme}.\\
		\par\noindent
		Die \textbf{disjunktiven Normalform} (DNF) entspricht der durch Disjunktion (\(\vee\)) miteinander verknüpften Minterme.
		\section{Vereinfachen von Darstellungsformen}
		Wir wollen die Frage diskutieren, welche Möglichkeiten es gibt, gegebene Schaltfunktionen zu vereinfachen um so wenig Gatter wie nötig zu verwenden und ein möglichst schnelles Schaltnetz aufzubauen.\\
		Hierzu benötigen wir die Begriffe \textbf{DNF} (disjunktive Normalform)  die man auch als \textit{SOP} (Sum of Products) bezeichnet.\\
		\par\noindent
		Schauen wir uns zunächst ein Beispiel an und vereinfachen dieses mit Hilfe der Boolschen Algebra.
		\begin{align*}
			F(x_1,x_2,x_3) &= \overline{x_1}x_2x_3 + x_1x_2x_3\\
			&= \underbrace{(\overline{x_1} + x_1)}_{=1}x_2x_3\\
			&= x_2x_3
		\end{align*}
		\texttt{Irgendwie muss man auch ohne die boolesche Algebra die vereinfachte Funktion aufstellen können, oder?}\\
		Betrachten wir dazu die nachfolgende Funktionstabelle mit drei Eingabevariablen \(x_1, x_2\) und \(x_3\).
		\begin{center}
			\begin{tabular}{|c|ccc|c|}
				\hline
				\textit{i} & \(x_1\) & \(x_2\) & \(x_3\) & \(f(x_1,x_2,x_3)\)\\
				\hline
				0 & 0 & 0 & 0 & 0\\
				\hline
				1 & 0 & 0 & 1 & 0\\
				\hline
				2 & 0 & 1 & 0 & 0\\
				\hline
				3 & 0 & 1 & 1 & 1\\
				\hline
				4 & 1 & 0 & 0 & 0\\
				\hline
				5 & 1 & 0 & 1 & 1\\
				\hline
				6 & 1 & 1 & 0 & 0\\
				\hline
				7 & 1 & 1 & 1 & 1\\
				\hline
			\end{tabular}
		\end{center}
		\subsection*{Ihre Aufgabe} Versuchen Sie die dazugehörige Schaltfunktion \[F(x_1,x_2,x_3) = \overline{x_1}x_2x_3 + x_1\overline{x_2}x_3 + x_1x_2x_3\] zu vereinfachen.\\
		Eine Regel, der man ganz intuitiv Beachtung schenkt lautet wie folgt:
		\begin{framed} \textbf{Resolutionsregel} Sind in einer \textit{SOP} zwei Summanden vorhanden, bei der sich nur eine Eingabevariable komplementär (\(x_1 \Leftrightarrow \overline{x_1}\)) verhält, kann man beide Summanden streichen und durch den gemeinsamen Teil der Summanden ersetzen.\end{framed}
		\noindent
		Ist die Schaltfunktion, die zu vereinfachen ist, komplexer, reicht ein einfaches Betrachten manchmal nicht aus.\\
		Mit Hilfe der sogenannten \textbf{K-Map} kann man mögliche Zusammenfassungen schnell erkennen.\\
		Prinzipiell ist die \textbf{K-Map} einfach nur eine andere Darstellungsform der Funktionstafel unserer Schaltfunktion \(F\). Diese ist wie auch die Funktionstafel abhängig von der Anzahl der Eingabevariablen.\\
		\newpage\noindent
		Bei \underline{drei Eingabevariablen}: (2 x 4)\\
		\par\noindent
		\begin{karnaugh-map}[4][2][1][$x_2x_3$][$x_1$]
			
		\end{karnaugh-map}
		\par\noindent
		Bei \underline{vier Eingabevariablen}: (4 x 4)\\
		\par\noindent
		\begin{karnaugh-map}[4][4][1][$x_3x_4$][$x_1x_2$]
			
		\end{karnaugh-map}\\
		Bei der Zeilen- bzw. Spaltenzählung ist zu beachten, dass sich von einer zur nächsten Zeile bzw. Spalte jeweils nur \underline{\textbf{eine}} Variable ändern darf.\\
		Innerhalb der \textit{K-Map} werden die Zellen mit einer \(1\) befüllt, deren Belegung bei der Funktion auch eine \(1\) als Ausgabe haben.
		\subsection*{Ihre Aufgabe}	Erstellen sie zu vorne angegebener Funktionstabelle die entsprechende \textbf{K-Map}.
		
		\subsection{Überdeckung der Einsen} Betrachten wir die Funktion\\
		\(f(x_1,x_2,x_3,x_4) = x_1\ \overline{x_2}\ x_3\ x_4 + x_1\ \overline{x_2}\ \overline{x_3}\ x_4 + x_1\ x_2\ x_3\ x_4 + \overline{x_1}\ \overline{x_2}\ \overline{x_3}\ x_4 + \overline{x_1}\ \overline{x_2}\ x_3\ x_4\)
		\subsection*{Ihre Aufgabe} Befüllen Sie die dazugehörige \textbf{K-Map}.
		\begin{karnaugh-map}[4][4][1][$x_3x_4$][$x_1x_2$]
			
		\end{karnaugh-map}\\
		Unser Ziel ist es, die Schaltfunktion zu vereinfachen, dafür versuchen wir nun, so viele \(1\)en wie möglich\footnote{Die Anzahl muss immer einer \underline{2er Potenz} entsprechen.} zu überdecken. Dabei sind folgende Überdeckungen zulässig:
		\begin{itemize}
			\item[+] zwei nebeneinander/übereinander liegende \(1\)en
			\item[+] vier zusammenhängende \(1\)en
			\item[+] zwei bzw. vier über die Außenkanten nebeneinander/übereinander liegende \(1\)en
			\item[+] zwei oder vier in den Ecken befindliche \(1\)en
		\end{itemize}
		\begin{karnaugh-map}[4][4][1][$x_3x_4$][$x_1x_2$]
			\minterms{0,1,7,15}
			\implicant{0}{1}
			\implicant{7}{15}
		\end{karnaugh-map}
		\begin{karnaugh-map}[4][4][1][$x_3x_4$][$x_1x_2$]
			\minterms{0,1,4,5,9,11,13,15}
			\implicant{0}{5}
			\implicant{13}{11}
		\end{karnaugh-map}
		\begin{karnaugh-map}[4][4][1][$x_3x_4$][$x_1x_2$]
			\minterms{1,2,4,6,9}
			\implicantedge{1}{1}{9}{9}
			\implicantedge{0}{4}{2}{6}
		\end{karnaugh-map}
		\begin{karnaugh-map}[4][4][1][$x_3x_4$][$x_1x_2$]
			\minterms{0,2,8,10}
			\implicantcorner
		\end{karnaugh-map}
		\subsection*{Ihre Aufgabe} Versuchen Sie für die zu Beginn von 8.1 aufgestellt \textbf{K-Map} eine entsprechende Überdeckung der \(1\)en zu finden.\\
		\subsection{Vereinfachen mit Hilfe der Überdeckung}
		Mit Hilfe dieser Überdeckung können wir nun die Schaltfunktion aufstellen. Dabei verwenden wir nur die Belegung der Eingabevariablen, die für den überdeckten \(1\)er-Block konstant bleibt.
		\begin{karnaugh-map}[4][2][1][$x_2x_3$][$x_1$]
			\minterms{3,5,7}
			\implicant{5}{7}
			\implicant{3}{7}
		\end{karnaugh-map}\\
		Hieraus ergibt sich \(f(x_1,x_2,x_3) = x_2x_3 + x_1x_3\).\\
		\subsection*{Ihre Aufgabe} Erstellen Sie mit Hilfe der zuletzt aufgestellten K-Map die vereinfachte Funktion zu der zu Beginn von 8.1 genannten Funktion.\\
		\par\bigskip\noindent
		\tiny{\color{codegray}\textit{W. Oberschelp/G.Vossen} Rechneraufbau und Rechnerstrukturen 10. Auflage (72-75)}
	\end{worksheet}
\end{document}