\documentclass[11pt,oneside,openany,headings=optiontotoc,11pt,numbers=noenddot]{article}

\usepackage[a4paper]{geometry}
\usepackage[utf8]{inputenc}
\usepackage[T1]{fontenc}
\usepackage{lmodern}
\usepackage[ngerman]{babel}
\usepackage{ngerman}

\usepackage[onehalfspacing]{setspace}

\usepackage{fancyhdr}
\usepackage{fancybox}

\usepackage{rotating}
\usepackage{varwidth}


\usepackage{pdflscape}
\usepackage{graphicx}
\usepackage{graphbox}
\graphicspath{
	{Pics/PDFs/}
	{Pics/JPGs/}
	{Pics/PNGs/}
}
\usepackage{caption}
\usepackage{tabularx}
\usepackage{dashrule}
\usepackage{hhline}
\usepackage{multirow}
\usepackage{enumerate}
\usepackage[hidelinks]{hyperref}
\usepackage{listings}

\usepackage[table]{xcolor}
\usepackage{array}
\usepackage{enumitem,amssymb,amsmath}
\usepackage{interval}
\usepackage{stmaryrd}
\usepackage{polynom}
\usepackage{diagbox}
\usepackage{dashrule}
\usepackage{framed}
\usepackage{mdframed}
\usepackage{karnaugh-map}

\usepackage{blindtext}

\usepackage{eso-pic}

\usepackage{amssymb}
\usepackage{eurosym}
\pagestyle{headings}
\renewcommand{\headrulewidth}{0.2pt}
\renewcommand{\footrulewidth}{0.2pt}
\newcommand*{\underdownarrow}[2]{\ensuremath{\underset{\overset{\Big\downarrow}{#2}}{#1}}}
\setlength{\fboxsep}{5pt}

% Codestyle defined
\definecolor{codegreen}{rgb}{0,0.6,0}
\definecolor{codegray}{rgb}{0.5,0.5,0.5}
\definecolor{codepurple}{rgb}{0.58,0,0.82}
\definecolor{backcolour}{rgb}{0.95,0.95,0.92}
\definecolor{deepgreen}{rgb}{0,0.5,0}
\definecolor{darkblue}{rgb}{0,0,0.65}
\definecolor{mauve}{rgb}{0.40, 0.19,0.28}
\colorlet{exceptioncolour}{yellow!50!red}
\colorlet{commandcolour}{blue!60!black}
\colorlet{numpycolour}{blue!60!green}
\colorlet{specmethodcolour}{violet}

%Neue Spaltendefinition
\newcolumntype{L}[1]{>{\raggedright\let\newline\\\arraybackslash\hspace{0pt}}m{#1}}
\newcolumntype{M}[1]{>{\centering\arraybackslash}X}
\newcommand{\cmnt}[1]{\ignorespaces}
%Textausrichtung ändern
\newcommand\tabrotate[1]{\rotatebox{90}{\raggedright#1\hspace{\tabcolsep}}}

%Intervall-Konfig
\intervalconfig {
	soft open fences
}

%Bash
\lstdefinestyle{BashInputStyle}{
	language=bash,
	basicstyle=\small\sffamily,
	backgroundcolor=\color{backcolour},
	columns=fullflexible,
	backgroundcolor=\color{backcolour},
	breaklines=true,
}
%Java
\lstdefinestyle{JavaInputStyle}{
	language=Java,
	backgroundcolor=\color{backcolour},
	aboveskip=1mm,
	belowskip=1mm,
	showstringspaces=false,
	columns=flexible,
	basicstyle={\footnotesize\ttfamily},
	numberstyle={\tiny},
	numbers=none,
	keywordstyle=\color{purple},,
	commentstyle=\color{deepgreen},
	stringstyle=\color{blue},
	emph={out},
	emphstyle=\color{darkblue},
	emph={[2]rand},
	emphstyle=[2]\color{specmethodcolour},
	breaklines=true,
	breakatwhitespace=true,
	tabsize=2,
}
%Python
\lstdefinestyle{PythonInputStyle}{
	language=Python,
	alsoletter={1234567890},
	aboveskip=1ex,
	basicstyle=\footnotesize,
	breaklines=true,
	breakatwhitespace= true,
	backgroundcolor=\color{backcolour},
	commentstyle=\color{red},
	otherkeywords={\ , \}, \{, \&,\|},
	emph={and,break,class,continue,def,yield,del,elif,else,%
		except,exec,finally,for,from,global,if,import,in,%
		lambda,not,or,pass,print,raise,return,try,while,assert},
	emphstyle=\color{exceptioncolour},
	emph={[2]True,False,None,min},
	emphstyle=[2]\color{specmethodcolour},
	emph={[3]object,type,isinstance,copy,deepcopy,zip,enumerate,reversed,list,len,dict,tuple,xrange,append,execfile,real,imag,reduce,str,repr},
	emphstyle=[3]\color{commandcolour},
	emph={[4]ode, fsolve, sqrt, exp, sin, cos, arccos, pi,  array, norm, solve, dot, arange, , isscalar, max, sum, flatten, shape, reshape, find, any, all, abs, plot, linspace, legend, quad, polyval,polyfit, hstack, concatenate,vstack,column_stack,empty,zeros,ones,rand,vander,grid,pcolor,eig,eigs,eigvals,svd,qr,tan,det,logspace,roll,mean,cumsum,cumprod,diff,vectorize,lstsq,cla,eye,xlabel,ylabel,squeeze},
	emphstyle=[4]\color{numpycolour},
	emph={[5]__init__,__add__,__mul__,__div__,__sub__,__call__,__getitem__,__setitem__,__eq__,__ne__,__nonzero__,__rmul__,__radd__,__repr__,__str__,__get__,__truediv__,__pow__,__name__,__future__,__all__},
	emphstyle=[5]\color{specmethodcolour},
	emph={[6]assert,range,yield},
	emphstyle=[6]\color{specmethodcolour}\bfseries,
	emph={[7]Exception,NameError,IndexError,SyntaxError,TypeError,ValueError,OverflowError,ZeroDivisionError,KeyboardInterrupt},
	emphstyle=[7]\color{specmethodcolour}\bfseries,
	emph={[8]taster,send,sendMail,capture,check,noMsg,go,move,switch,humTem,ventilate,buzz},
	emphstyle=[8]\color{blue},
	keywordstyle=\color{blue}\bfseries,
	rulecolor=\color{black!40},
	showstringspaces=false,
	stringstyle=\color{deepgreen}
}

\lstset{literate=%
	{Ö}{{\"O}}1
	{Ä}{{\"A}}1
	{Ü}{{\"U}}1
	{ß}{{\ss}}1
	{ü}{{\"u}}1
	{ä}{{\"a}}1
	{ö}{{\"o}}1
}

% Neue Klassenarbeits-Umgebung
\newenvironment{worksheet}[3]
% Begin-Bereich
{
	\newpage
	\sffamily
	\setcounter{page}{1}
	\ClearShipoutPicture
	\AddToShipoutPicture{
		\put(55,761){{
				\mbox{\parbox{385\unitlength}{\tiny \color{codegray}BBS I Mainz, #1 \newline #2
						\newline #3
					}
				}
			}
		}
		\put(455,761){{
				\mbox{\hspace{0.3cm}\includegraphics[width=0.2\textwidth]{../../logo.jpg}}
			}
		}
	}
}
% End-Bereich
{
	\clearpage
	\ClearShipoutPicture
}

\setlength{\columnsep}{3em}
\setlength{\columnseprule}{0.5pt}

\geometry{left=2.00cm,right=2.00cm,top=3.00cm,bottom=1.00cm,includeheadfoot}
\pagenumbering{gobble}
\pagestyle{empty}

\begin{document}
	\begin{worksheet}{BS FI 16}{2. Lehrjahr, LF 9 - Öffentliche Netze}{Qualitätskriterien für eine gute WLAN-Abdeckung}
		\begin{framed}
			\noindent
			Im Zuge der Sanierung des A-Gebäudes soll das dort bestehende WLAN des gesamten Schulgebäudes durch die Anbringung neuer Access Points (AP) erneuert.\\
			In den übrigen Gebäudeteilen (B-, C- und W-Gebäude) wurden auf Grund eines Kooperationsvertrags mit der Firma Sophos entsprechende AP verwendet.
			\par\noindent
			Für die Erneuerung stehen zwei Sophos Geräte zur Auswahl:\\
			\par\noindent
			\begin{tabularx}{\textwidth}{lX|lX}
				\hline
				\hline
				\multicolumn{2}{c|}{\textbf{Sophos AP 100 Access Point}} & \multicolumn{2}{c}{\textbf{Sophos AP 55 Access Point}}\\
				\hline
				\multicolumn{2}{l|}{Dual-Band/Dual-Radio} &\multicolumn{2}{l}{Dualband/Doppelsender für Großunternehmen}\\
				\hline
				max. Durchsatz: & 1,3 GBit/s + 450MBit/s & max. Durchsatz: & 867 MBit/s + 300 MBit/s\\
				\hline
				mehrere SSIDs: & 8 pro Radio (insg. 16) & mehrere SSIDs: & 8 pro Radio (insg. 16)\\
				\hline
				LAN-Schnittstelle: & 1x 10/100/1000 BaseTX & LAN-Schnittstelle: & 1x 10/100/1000 BaseTX\\
				\hline
				Standards: & 802.11 a/b/g/n/ac & Standards: & 802.11 a/b/g/n/ac\\
				& 2,4 GHz und 5 GHz & & 2,4 GHz und 5 GHz\\
				\hline
				PoE: & 802.3at & PoE: & 802.3at\\
				\hline
				Antennen: & 3 extern & Antennen: & 2 extern\\
				\hline
				Anzahl Sender: & 2 & Anzahl Sender: & 2\\
				\hline
				MIMO-Funktionen: & 3x3:3 & MIMO-Funktion: & 2x2:2\\
				\hline
				\hline
			\end{tabularx}\\
			\par\noindent
			\section*{Ihre Aufgabe}
			\begin{itemize}
				\item \textbf{Entscheiden} Sie sich für ein \textit{AP-Modell}. \textbf{Begründen} Sie ihre Entscheidung.
				\item \textbf{Entscheiden} Sie sich für einen \textit{Standard}. \textbf{Begründen} Sie diese.
				\item \textbf{Planen} Sie unter Verwendung ihres ausgewählten Modells eine \textit{möglichst optimale Positionierung der APs} zur bestmöglichen Ausleuchtung!\\
				\textbf{Vermerken} Sie an jedem positionierten AP:
				\begin{itemize}
					\item Signalausleuchtung
					\item Antennenausrichtung
					\item gewählter Kanal
					\item Anzahl der bedienbaren Clients
				\end{itemize}
				\item[] Beachten Sie bei ihrer Planung die folgende Anforderung:
				\begin{itemize}
					\item Clients/AP im Schnitt: \(30\)
					\item APs dürfen sichtbar montiert sein
					\item bestmögliche Signalabdeckung in den Unterrichtsräumen
				\end{itemize}
				\item \textbf{Definieren und begründen} Sie zudem, für welche Form des \textit{AP-Management} Sie sich entscheiden.
			\end{itemize}
		\end{framed}
		\begin{framed}
			Für die Beurteilung ihrer eigenen Planung können Sie sich an den nachfolgenden Qualitätskriterien orientieren:\\
			\begin{enumerate}[label*=\protect\fbox{}]
				\item Abstand der AP
				\item Flächendeckende Signalausleuchtung
				\item Power over Ethernet
				\item Controllereinsatz möglich
				\item Funkkanalwahl
				\item Signalzuführung zum AP
				\item Access Control
			\end{enumerate}
		\end{framed}
	\end{worksheet}
\end{document}